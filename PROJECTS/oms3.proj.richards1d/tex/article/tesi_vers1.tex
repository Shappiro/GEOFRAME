\documentclass[11pt]{amsart}
%Preamble for language and encoding
\usepackage[italian]{babel}
\usepackage[utf8]{inputenc}
\usepackage[T1]{fontenc}


\usepackage[dvips]{epsfig}
\usepackage{graphics}
\usepackage{latexsym}
\usepackage{verbatim}
\usepackage{amsmath}
\usepackage{amsthm}
\usepackage{amssymb}


\usepackage[]{hyperref}
\hypersetup{
    colorlinks=true,%
 %   citecolor=black,%
    filecolor=black,%
    linkcolor=black,%
    urlcolor=BrickRed
}

\usepackage [table,dvipsnames]{xcolor}
\usepackage{listings} % For complex verbatim environments


\lstset{%
basicstyle=\ttfamily\small, % This changes typesetting character for code environment
numbers=left, % Activate line numbers
numbersep=6pt, % Separation of line numbers from code
numberstyle=\tiny\color{black}\ttfamily, % Style of line numbers
xleftmargin=\parindent, % Distance of whole code box from left margin
frame=l, % Do we want a frame around our code block? l=single left line, L=double left line, r= [...]
showstringspaces=false,
commentstyle=\color{Gray},
keywordstyle=\color{Black},
stringstyle=\color{Gray},
literate=%
         {è}{{\`e}}1
}

%%%%%%%%%%% plus plus %%%%%%%%%%%%%
\usepackage{float}
\usepackage{tikz}
\usetikzlibrary{calc}
\usetikzlibrary{decorations.pathmorphing}
\usepackage{pgfplots}
\usepackage{subfig}
%For bibliography
\usepackage[backend=bibtex,style=numeric]{biblatex} 
\addbibresource{tesi_bibliography.bib}

%Extra mathematical symbols
\usepackage{amssymb}

\usepackage{epsfig}  		% For postscript
\usepackage{epic,eepic}       % For epic and eepic output from xfig

%\usepackage{showkeys}


\newtheorem{thm}{Theorem}[section]
\newtheorem{prop}[thm]{Proposition}
\newtheorem{lem}[thm]{Lemma}
\newtheorem{cor}[thm]{Corollary}

%\usepackage[labelfont={footnotesize,sf,bf},textfont={footnotesize,sf}]{caption}

\theoremstyle{definition}
\newtheorem{definition}[thm]{Definition}
\newtheorem{example}[thm]{Example}

\theoremstyle{remark}
\newtheorem{remark}[thm]{Remark}

\numberwithin{equation}{section}

%%%REDEFINED COMMANDS AND NEW COMMANDS
\newcommand*\rfrac[2]{{}^{#1}\!/_{#2}}
\definecolor{SkyBlue}{RGB}{26 100 191}
\definecolor{GreenYellow}{RGB}{154 205 50}
\newcommand*\circled[1]{\tikz[baseline=(char.base)]{
            \node[shape=circle,draw,inner sep=.7pt,color=Red] (char) {\textcolor{red}{#1}};}}

\begin{document}

\title[Metodo iterativo per l'equazione di Richards]{Un problema di convergenza: implementazione di un metodo iterativo per la soluzione dell'equazione di Richards}
\author{Aaron Iemma}
\address{DICAM, Università degli Studi di Trento, 
Italy}
\email{aaron.iemma@studenti.unitn.it}
\urladdr{https://github.com/Shappiro}

%\author{Second Author}
%\address{Department of Mathematics, University of South Carolina,
%Columbia, SC 29208}
%\email{second@math.sc.edu}
%\urladdr{www.math.sc.edu/$\sim$second}

%%%

\begin{abstract}
I flussi liquidi attraverso i mezzi porosi sono ben descritti e caratterizzati dall'equazione di Richards, un'equazione differenziale 
fortemente non lineare non risolubile in forma chiusa, se non con alcune semplificazioni. Si propone di seguito l'implementazione di un innovativo metodo 
di risoluzione ai volumi finiti dell'equazione, che riscritta nella forma mista in termini di contenuto d'umidità volumetrico $\theta$ e di prevalenza più 
altezza potenziale $\psi + z$, fornisce una risoluzione ai volumi finiti \emph{mass-conservative} anche in condizioni di saturazione. L'imlementazione 
viene presentata all'interno del \emph{framework} \emph{Java} \texttt{OMS3} (\emph{Object Modelling System}), garanzia della costruzione di un modello basato 
sulle componenti facilmente riusabile e reimplementabile.
\end{abstract}

%%
%%  LaTeX will not make the title for the paper unless told to do so.
%%  This is done by uncommenting the following.
%%

\maketitle
\tableofcontents

\newpage
%%%%%%%%%%%%%%%%%%%%%%%%%%%%%%%%%%%%%%%%%%%%%%%%%%%%%%%%%%%%%%%%%%%%%%
\section{Introduzione}
\label{sec:intro}
%%%%%%%%%%%%%%%%%%%%%%%%%%%%%%%%%%%%%%%%%%%%%%%%%%%%%%%%%%%%%%%%%%%%%%

Il presente lavoro deve quasi tutto alla chiarezza espositiva di un articolo di Vincenzo Casulli e Paola Zanolli, 
apparso per la prima volta nel 2010 sul \emph{Journal} del \texttt{SIAM} \cite{art:casulli}. 
In esso è ampiamente discussa, in uno stile chiaro e conciso, la creazione di un nuovo modello ai volumi finiti per la forma mista 
$\psi - \theta$ dell'equazione di Richards:

\begin{equation}
\frac{\partial \theta(\psi)}{\partial t} = \nabla\cdot\lbrack \textbf{K}(\psi)\nabla(\psi + z) \rbrack + S
\end{equation}
dove $\theta$ è il contenuto volumetrico d'acqua ($\theta=\rfrac{V_{w}}{V_t}$), $\psi$ la prevalenza, $z$ la coordinata verticale, $t$ 
il tempo, $\mathbf{K}$ la matrice di diffusivit\`a e $S$ rappresenta le eventuali sorgenti interne al volume di controllo 
(il concetto di 'volume di controllo' sarà presto più chiaro,:eventualmente, per una derivazione dell'equazione di cui sopra si faccia riferimento 
alle Appendici \ref{appendix:richards}). 

Purtroppo, l'equazione include un termine, il contenuto d'acqua $\theta(\psi)$, che è in generale variabile \emph{non linearmente} con la prevalenza (per un 
approfondimento, si vedano le Appendici, in particolare, le sezioni dedicate alle \emph{Soil Water Retention Curve} \ref{appendix:swrc}). Proprio questo 
termine rende la risoluzione numerica delle equazioni di Richards difficile, in particolare in presenza di particolari \emph{spikes} in uno dei termini che la 
compongono (ad esempio, una improvvisa variazione di \textbf{K} per un cambio di tipologia di suolo). È in questo punto che entrano in gioco degli opportuni schemi
di linearizzazione: seguendo le orme dell'articolo di Casulli, adotteremo di seguito uno schema monodimensionale ai volumi finiti utilizzando come schemi di 
linearizzazione delle iterazioni ''alla Newton'' derivate linearizzando la differenza di contenuto d'acqua di due volumi combacianti ($\theta_{1}(\psi)$ e 
$\theta_{2}(\psi)$), contemporaneamente garantendo la conservazione della massa innestando nel ciclo principale una \emph{iterazione di Picard} (si vedano a questo proposito le appendici \ref{appendix:picard})


%%%%%%%%%%%%%%%%%%%%%%%%%%%%%%%%%%%%%%%%%%%%%%%%%%%%%%%%%%%%%%%%%%%%%%
\section{Il metodo}
\label{sec:method}
%%%%%%%%%%%%%%%%%%%%%%%%%%%%%%%%%%%%%%%%%%%%%%%%%%%%%%%%%%%%%%%%%%%%%%
Il primo passo per la costruzione del metodo è la derivaizone di una formulazione implicita ai volumi finiti per l'equazione di Richards in forma mista, che tenga conto della variabilità sia spaziale (del mezzo, di $K$, \dots) che temporale (variazione di $\theta$, di $K$, \dots).


A questo punto, il metodo dei volumi finiti è esplicitato, cassando per semplicità il termina di generazioe interna $S$ (che, essendo assunto indipendente dal tempo e dalle proprietà del mezzo, è sempre reintroducibile in un secondo momento, essendo semplicemente un flusso che va a sommarsi/sottrarsi a quello computato ad ogni tempo) con i seguenti passi:

\begin{itemize}
  \item Prima di tutto, integrando l'equazione di Richards sul dominio V:
  \begin{equation}
  \int_{V}\frac{\partial \theta(\psi)}{\partial t} = \int_{V}\nabla\cdot[\textbf{K}(\psi)\nabla(\psi+z) + S]
  \end{equation}
  Applicando il teorema della divergenza al secondo membro, ottengo:
  \begin{equation}
  \int_{V}\frac{\partial \theta(\psi)}{\partial t} = \int_{\partial V}([\textbf{K}(\psi)\nabla(\psi+z)]\cdot \hat{n} + S)dS
  \end{equation}
  Dove l'integrale esteso al secondo membro è esteto alla superficie $\partial V$ del volume di controllo $V$
  \item Infine, discretizzando la precedente usando una formulazione implicita (\emph{metodo di Eulero}) sia nel tempo che nello spazio. Definisco quindi le variabili di interesse (nella formulazione monodimensionale, non tutte verranno usate) su un dominio spaziale $\Omega$\footnote{Volendo ricondurre i dettagli il più possibile a \cite{art:casulli}, se ne manterrà la nomenclatura delle variabili.} partizionato da una griglia ortogonale non strutturata costituita da $\Omega_i, i=1,2,\dots,N_{v}$ volumi di controllo senza sovrapposizioni e separati da $M$facce interne $F_{\emph{j}}, j = 1,2,\dots,M$ ciascuna con area $A_{\emph{j}}$, identifico:
    \begin{itemize}
    \item un set $F_{\emph{i}}$ di facce interne (escludendo le eventuali facce di confine) per ogni \emph{i}-esimo volume di controllo in numero arbitrario;
    \item il volume di controllo $\wp(\emph{i},\emph{j})$ "vicino" del volume di controllo \emph{i} che condivide la faccia \emph{j} con lo stesso volume di controllo \emph{i}, cosicchè $1\leq\wp(\emph{i},\emph{j})\leq N_{v}$ per ogni $\emph{j} \in F_{i}$;
    \item la distanza $\delta_{j}$ tra i centri di due volumi adiacenti che condividono la stessa \emph{j}-esima faccia interna;
    \item le variabili discrete $\theta_{\emph{i}}$ e $\psi_{\emph{i}}$, localizzate nel centro dell'\emph{i}-esimo volume di controllo;
    \item una suddivisione temporale in \emph{n} istanti di lunghezza $\Delta t$;
    \item l'indice \emph{i} ($1,2,\dots N$)a \emph{pedice} per indicare il volume di controllo, e l'indice \emph{n} ad apice per indicare il passo temporale.
    \end{itemize}  
\end{itemize}
  
\begin{equation}
\label{eq:richards_first}
\frac{\theta_{\emph{i}}(\psi_{\emph{i}}^{n}) - \theta_{\emph{i}}(\psi_{\emph{i}}^{n})}{\Delta t} = \sum_{j \in F_{\emph{i}}}K_{\emph{}}^{n}\frac{(\psi + z)_{\wp(i,j)}^{n}-(\psi + z)^{n}_{\emph{i}}}{\delta_{\emph{j}}}+\sum_{j \in F_{\emph{i}}}K_{\emph{}}^{n}n_{i,j}^{z}+S_{\emph{i}}^{n}
\end{equation}

Dove:
\begin{itemize}
  \item $ K_{\emph{j}}^{n} = A_{\emph{j}}max[K_{\emph{i}}(\psi_{\emph{i}^{\emph{n}}},K_{\wp(\emph{i},\emph{j})}(\psi_{\emph{i}^{\emph{n}}_{\wp(\emph{i},\emph{j})}})] $ , ovvero, l'area della \emph{j}-esima faccia moltiplicata per la maggior conduttività idraulica scelta in un insieme che comprende quella dello stesso volume \emph{n} e quelle di ogni volume adiacente: si noti come questa condizione specifica un flusso favorito nella direzione della massima conduttività idraulica;
  \item $ n_{\emph{i},\emph{j}}^{n} $ è la proiezione, nella direzione della congiungente il centro del volume \emph{n} con il \emph{j}-esimo volume adiacente, della normale (diretta verso l'esterno) alle facce di confine dello stesso volume \emph{n};
  \item $ S_i = \int_{\Omega_{i}}Sd\Omega$ è la sorgente \emph{totale} all'interno dell'\emph{i-esimo volume di controllo}, compresi eventuali flussi al contorno;
  \item $ \theta_{i}(\psi) = \int_{\Omega_{i}}\theta(\psi)d\Omega$ è l'\emph{i}-esimo volume d'acqua.
\end{itemize}


L'equazione è attaccabile prima di tutto individuando una parametrizzazione di $\theta(\psi)$ e di $K(\psi)$, ovvero, una relazione costitutiva che, in base ai parametri del terreno considerato, leghi biunivocamente $\theta$ e $K$ a $\psi$. Il modello generalmente adoperato è quello di \emph{van Genuchten}, dettagliato nell'appendice \ref{appendix:swrc}. Ci si limita qua a ricordare che il contenuto d'acqua di un generico volume di controllo \emph{i} si ottiene dal modello di van Genuchten integrando la capacità idrica specifica:

\begin{equation}
\theta(\psi) = \theta_{r} + \int_{-\infty}^{\psi}c(\xi)d\xi = \theta_{s} - \int_{\psi}^{+\infty}c(\xi)d\xi
\end{equation} 

ove con $\theta_{r}$ e con $\theta_s$ si sono indicati rispettivamente il contenuto idrico residuo (la parte di umidità che non può essere eliminata tramite la semplice suzione) del suolo e la porosità dello stesso. 


Il modello di van Genuchten esplicita bene il carattere non lineare dell'equazione di Richards: le derivate di $\theta(\psi)$, in particolare in prossimità delle condizioni di saturazione, esibiscono cambiamenti repentini che rendono la soluzione numerica del problema computazionalmente onerosa se si desidera arrivare a convergenza entro piccole tolleranze. 

È tuttavia possibile attaccare il problema dividendolo in due parti risolvibili separatamente, in un unico algoritmo:
\begin{enumerate}
 \item la continuità della funzione che descrive il contenuto d'acqua all'interno del dominio: è necessario minimizzare il carattere discontinuo del solutore numerico, obbligandolo in qualche modo a convergere ad una soluzione che sia sotto una certa tolleranza prefissata;
 \item la risoluzione di un sistema linearizzato che assegni i contenuti d'acqua ad ogni volume di controllo.
\end{enumerate}


Di conseguenza, è necessario settare: 


\begin{enumerate}
 \item un \emph{ciclo iterativo esterno} che controlli la convergenza della suzione $\psi$ (e quindi, indirettamente, nel contenuto d'acqua $\theta(\psi)$) in ogni volume di controllo ad un certo istante di tempo \emph{i}, e che venga eventualmente ripetuto, sempre nell'istante di tempo \emph{i}, fino a quando la differenza (errore) tra la suzione al passo $\emph{m}$ rispetto al contenuto d'acqua al passo $\emph{m}-1$ non sia sotto una certa soglia;
 \item un \emph{ciclo iterativo interno} che si occupi di risolvere il sistema linearizzato per ottenere il contenuto d'acqua ad un certo passo temporale {i} per tutto il dominio di definizione, fino a raggiungere una tolleranza prefissata       
\end{enumerate}

In particolare, è fondamentale implementare un metodo \emph{efficiente} che dovrà risolvere $\emph{m}*\emph{l}$ sistemi linearizzati, ove $m$ ed $l$ sono rispettivamente il numero di iterazioni esterne ed interne.

 \subsection{Iterazioni esterne: il metodo di Picard}
 \label{subsec:extern}

 \subsection{Iterazioni interne: lineraizzazione con il metodo di Newton e risoluzione con il metodo del gradiente coniugato}
 \label{subsec:intern}

L'equazione \ref{eq:richards_first} può essere facilmente riscritta in forma matriciale più compatta:









 



%%%%%%%%%%%%%%%%%%%%%%%%%%%%%%%%%%%%%%%%%%%%%%%%%%%%%%%%%%%%%%%%%%%%%%
\section{Implementazione: Java e \texttt{OMS3}}
\label{sec:implementation}
%%%%%%%%%%%%%%%%%%%%%%%%%%%%%%%%%%%%%%%%%%%%%%%%%%%%%%%%%%%%%%%%%%%%%%



\newpage
%%%%%%%%%%%%%%%%%%%%%%%%%%%%%%%%%%%%%%%%%%%%%%%%%%%%%%%%%%%%%%%%%%%%%%
\appendix
%%%%%%%%%%%%%%%%%%%%%%%%%%%%%%%%%%%%%%%%%%%%%%%%%%%%%%%%%%%%%%%%%%%%%%
\setcounter{equation}{0}
\section{Derivazione della forma $\psi-\theta$ dell'equazione di Richards}
\label{appendix:richards}
L'equazione di Richards è sostanzialmente una forma della generica \emph{equazione di diffusione}, nella quale viene descritta la variazione nel tempo e nello spazio di una quantità vettoriale (nel nostro caso, il flusso d'acqua $\vec{q}$\,) soggetta a delle limitazioni imposte sul flusso dalle condizioni al contorno (direttamente sul flusso o sulle sue derivate), e condizionata dalle proprietà del mezzo in cui varia (qui riassunte nel tensore $\mathbf{K}$).


In generale quindi, l'equazione di Richards non è altro che l'applicazione del teorema della divergenza di Gauss: ragionando per un volume di controllo arbitrario, ho che:

\begin{equation} 
\label{eqn:gauss_int}
\frac{\partial }{\partial t} \iiint_{V} \theta dV = \iint_{S} \vec{q}\cdot\mathbf{\hat{n}}dS
\end{equation}

Ovvero, la variazione nel tempo del contenuto d'acqua volumetrico $\theta$ equivale al flusso netto uscente $\vec{q}$ attraverso le facce del volume: in questa forma
è particolarmente evidente la relazione tra il teorema di Gauss ed un bilancio di massa!

Da semplici considerazioni geometriche (in sostanza si effettua un bilancio dei flussi su un volume di controllo cubico infinitesimo per ogni faccia), in forma differenziale lo stesso teorema risulta:

\begin{equation}
\label{eqn:gauss_dif}
\frac{\partial \theta}{\partial t} = - \nabla\cdot\vec{q} 
\end{equation}

Ove l'operatore divergenza $\nabla \cdot $ fornisce la divergenza di un campo vettoriale (in uno spazio euclideo, 
$\sum_{i=1}^{3}\frac{\partial}{\partial x_{i}}f(x_{1},x_{2},x_{3})\hat{x}_{i} $ ), ovvero la tendenza del campo a formare pozzi (punti
di attrazione) e sorgenti (punti di fuga) nel suo dominio di definizione. 



Esplorata la relazione tra flusso e contenuto d'acqua, ora serve una relazione costitutiva che leghi il primo con le forze che ne condizionano il moto: ovvero,
è necessaria una relazione \emph{costitutiva} tra il flusso ed il mezzo in cui esso è libero di variare. 
Questa relazione è data dalla legge empirica di Darcy, che si esplica nella forma:

\begin{equation}
\label{eqn:darcy_int}
\frac{Q}{A}=\frac{-k}{\mu} \frac{(P_2 - P_1)}{L}' = q
\end{equation}

\dots ovvero, il flusso per unità di superficie attraverso un'area $A \lbrack m^{2} \rbrack$ è dato dal prodotto tra la permeabilità intrinseca del mezzo $k 
\lbrack m^{2} \rbrack$ e il salto di pressione per unità di lunghezza $\rfrac{(P_2 - P_1)}{L} \lbrack \rfrac{N}{m^{2}} \rbrack$, il tutto diviso per la lunghezza 
della colonna sulla quale avviene il salto di pressione $L \lbrack m \rbrack$ e per la viscosità $\mu \lbrack \rfrac{N}{sm^{2}} \rbrack$. 

Facendo tendere a zero la lunghezza del salto di pressione, ottengo la stessa equazione in forma differenziale:

\begin{equation}
\label{eqn_darcy_dif}
q= -\mathbf{K} \nabla \vec{P}  
\end{equation}
dove in \textbf{K}, la diffusività idraulica, si sono riassunte le opposte proprietà di permeabilità intrinseca del mezzo (nel caso generale, variabile da
punto a punto nel materiale) e di viscosità del fluente (costante per composizioni del liquidi e per temperature poco variabili).
Generalmente, la natura segue dei percorsi di minimo (di dissipazione, di energia, \dots), quindi tende a passare da uno stato a potenziale maggiore ad uno a 
potenziale maggiore: da qua la necessità del segno negativo di fronte all'equazione, ad indicare che l'acqua fluisce nel verso di \emph{massimo decremento} del gradiente 
delle pressioni.


Combinando il teorema di Gauss e l'equazione di Darcy nelle loro forme differenziali, si ottiene:
\begin{equation}
\begin{align}
\frac{\partial \theta}{\partial t} & = & - \nabla\cdot({-\mathbf{K} \nabla \vec{P}\,)} \\
   & = & \nabla\cdot({\mathbf{K} \nabla \vec{P}\,)} \\
   & = & \nabla\cdot\lbrack{\mathbf{K}(\psi) \nabla (\vec{P}+\vec{z}\,)\rbrack} \\
\end{align}
 \end{equation}
 
 dove si è usata la scomposizione del gradiente delle pressioni nel gradiente del carico più la prevalenza, e si è esplicitata la dipendenza del contenuto 
 d'acqua da $\psi$ in condizioni di suolo insaturo. Questa ulteriore dipendenza, non evidente dalla deduzione di cui sopra, è uno dei parametri del modello \texttt{OMS3}
 implementato, ed è esplicitabile attraverso una descrizione parametrica delle SWRC, \emph{Soil Water Retention Curves}: si veda la prossima sezione per ulteriori 
 dettagli sui modelli parametrici utilizzati, in particolare, sul modello di van Genuchten. 

\section{Dall'infinitamente piccolo al finitamente piccolo: il metodo iterativo}
\label{appendix:iterations}
Come è realizzato il metodo iterativo, partendo dall'equazione di Richards in forma mista? In questa sezione si vuole porre una base 
molto semplificata (il ''\emph{building block}'' del resto del metodo) che aiuti a capire come il problema della creazione del metodo iterativo possa essere
affrontato, e allo stesso tempo, fornisce una buona intuizione di come il contenuto d'acqua possa variare, date certe condizioni al contorno, nel tempo e nello spazio.

Esaminiamo quindi nel dettaglio una forte semplificazione dell'equazione di Richards, in questo caso quella che si ottiene:

\begin{itemize}
 \item Supponendo la colonna di suolo omogenea ($\mathbf{K(\psi)} = const = K$) e monodimensionale;
 \item Supponendo che l'effetto della gravità sia completamente ininfluente sulla dinamica del problema, e quindi, che la colonna di suolo sia ribaltata
 sull'orizzontale;
 \item Supponendo che non vi siano sorgenti o pozzi all'interno del dominio;
 \item Supponendo infine che $\theta(\psi) = \theta$.
\end{itemize}

Di conseguenza, la nostra equazione di Richards prende la forma, molto evocativa e probabilmente altrettanto conosciuta:

\begin{equation}
\label{eq:rich_sim}
\begin{align}
\frac{\partial \theta(\psi)}{\partial t} & = & \nabla\cdot\lbrack \textbf{K}(\psi)\nabla(\psi + z) \rbrack + S \\
		& = & K \frac{\partial^{2} \theta}{\partial x^{2}}  \\
\end{align}   
\end{equation}

\dots che è una equazione differenziale ordinaria di secondo grado a coefficienti costanti (\emph{equazione di diffusione})! In questa forma, come vedremo più avanti,
l'equazione di Richards è pure risolvibile in forma esatta: questa consapevolezza servirà come test di validità del metodo di discretizzazione utilizzato.

Ora, discretizzando il dominio spaziale e temporale tramite differenze finite (sostanzialmente, costruendo l'espansione di Taylor dei vari termini), ottengo per il problema in esame la discretizzazione:

\begin{equation}
\label{eq:rich_sim_discr}
\begin{align}
\frac{\theta_{i,j+1}-\theta_{i,j}}{\Delta t} & = & K \frac{\theta_{i-1,j} - 2\theta_{i,j} + \theta_{i+1,j}}{\Delta x^{2}} \\
\end{align}
\end{equation}


\begin{figure}[h]
 \centering 
   \begin{tikzpicture}[point/.style={circle,inner sep=0pt,minimum size=2pt,fill=black}]
      \begin{scope}[>=latex]          
        \draw[-] (-5,0) --(3,0)node[midway,right] {};
        \draw[-] (-5,1.5) --(3,1.5)node[midway,right] {};
      \end{scope}

       % LOWER: j, i-2<=i=<i+2 
       \draw   (2,-0.3) node{$\theta_{i+2,j}$};
       \draw   (-4,-0.3) node{$\theta_{i-2,j}$};       
       \draw   (-4,0) node{$\bullet$};
       \draw   (-2.5,0) node{$\bullet$};
       \draw   (-2.5,-0.3) node{$\textcolor{red}{\theta_{i-1,j}}$};       
       \draw   (-1,0) node{$\bullet$};
       \draw   (-1,-0.3) node{$\textcolor{red}{\theta_{i,j}}$};       
       \draw   (0.5,0) node{$\bullet$};
       \draw   (0.5,-0.3) node{$\textcolor{red}{\theta_{i+1,j}}$};       
       \draw   (2,0) node{$\bullet$};

       % UPPER: j+1, i-2<=i=<i+2
       \draw   (-4,1.2) node{$\theta_{i-2,j+1}$};       
       \draw   (-4,1.5) node{{$\bullet$}};
       \draw   (-2.5,1.2) node{$\theta_{i-1,j+1}$};         
       \draw   (-2.5,1.5) node{{$\bullet$}};
       \draw   (-1,1.2) node{$\textcolor{red}{\theta_{i,j+1}}$};         
       \draw   (-1,1.5) node{$\bullet$};
       \draw   (0.5,1.2) node{$\theta_{i+1,j+1}$};         
       \draw   (0.5,1.5) node{$\bullet$};
       \draw   (2,1.2) node{$\theta_{i+2,j+1}$};         
       \draw   (2,1.5) node{$\bullet$};      
    \end{tikzpicture}
    
\caption{\label{fig:mesh_1D} Una porzione della \emph{mesh} usata per il calcolo della diffusione di contenuto d'acqua nella colonna di suolo. 
Le quantità utili vengono calcolate direttamente nei punti della \emph{mesh}, indicizzati con una $\theta_{i,j}$, ove l'indice \textbf{i} indica il 
passo spaziale, mentre l'indice \textbf{j} quello temporale. In rosso sono evidenziati i punti utilizzati per il calcolo di $\theta$ ad un generico 
passo i,j.}
\end{figure}

Un minimo riarrangiamento dei termini dell'equazione porta alle seguenti semplificazioni:

\begin{equation}
\label{eq:rich_sim_discr_lambda}
\begin{align}
\theta_{i,j+1}-\theta_{i,j} & = & (K\frac{\Delta t}{\Delta x^{2}})(\theta_{i-1,j} - 2\theta_{i,j} + \theta_{i+1,j}) \\
			    & = &  \lambda (\theta_{i-1,j} - 2\theta_{i,j} + \theta_{i+1,j}) \\
\theta_{i,j+1}		    & = &  \lambda\theta_{i-1,j}  +(1-2\lambda)\theta_{i,j} +\lambda\theta_{i+1,j}\\    
\theta_{i,j}		    & = &  \lambda\theta_{i-1,j-1}  +(1-2\lambda)\theta_{i,j-1} +\lambda\theta_{i+1,j-1} \mathbf{*}\\    
\end{align}
\end{equation}
\footnotesize{\textbf{*} Nell'ultimo passaggio, si è fatto scorrere indietro l'indice temporale della mesh di una unità}
\normalsize
\newline

Il parametro $\lambda$ è caratteristico della discretizzazione usata e del problema fisico (tipo di terreno) trattato: in esso sono infatti riassunti 
sia i parametri di cella della \emph{mesh} (la distanza sia spaziale che temporale, quindi le dimensioni della cella) sia il coefficiente di diffusività.
Da notare poi che, con questa espressione, il contenuto d'acqua è immediatamente calcolabile per ogni cella del dominio ad ogni tempo \emph{t}: infatti,
$\theta_{i,j}$ non dipende altro che dai valori dei punti di griglia nel suo intorno al tempo immediatamente precedente a quello per cui se ne vuole calcolare il
valore.

Sulla colonna di suolo scegliamo di imporre due semplici condizioni al contorno costanti nel tempo:
\begin{description}
\item[$\theta=\theta_{0}$] per $x=0, t\geq0$;
\item[$\theta=\theta_{1}$] per $x=L, t\geq0$; 
\item[$\theta=\theta_{1}$] per $t=0, 0\leq x\leq L$; 
\end{description}

Si riporta di seguito un esempio di semplice \emph{script} \texttt{MATLAB} che implementa efficacemente quanto discusso sopra:

\lstinputlisting[language=Matlab,escapechar=@,firstline=14,lastline=91]{diffusione_richards.m}

Di seguito si specificano i vari punti del codiche che meritano di essere, brevemente, approfonditi:

\begin{description}
 \item[\circled{1}\,] qui è possibile impostare i parametri dimensionali del problema, come la lunghezza della colonna, il tempo computazionale, 
 la grandezza della \emph{mesh}, ecc\dots;
 \item[\circled{2}\,] per approfondimenti sulla misura della permeabilità intrinseca dei suoli con metodi semplici, ma fattivamente applicabili anche
 in contesti difficili senza alcun mezzo meccanico tranne un orologio, si veda \url{ftp://ftp.fao.org/fi/CDrom/FAO_Training/FAO_Training/General/x6706e/x6706e09.htm};
 \item[\circled{3}\,] un test sulla convergenza è necessario: da come è stata scritta l'equazione \ref{eq:rich_sim_discr}, è chiaro (da $(1-2\lambda)\theta_{i,j-1}$)
 come la definizione di $\lambda$ non possa coprire qualsiasi numero, ma anzi attestarsi su un campo piuttosto ristretto affinchè la convergenza del metodo sia 
 garantita (al limite inferiore di $\lambda$, un termine della mesh scompare del tutto, mentre invece al limite superiore inizia ad entrare in gioco un termine 
 \emph{negativo} non desiderabile;
 \item[\circled{4}\,] a questo punto, la matrice che terrà traccia delle variazioni di $\theta$ nel tempo e nello spazio viene inizializzata, con le opportune
 condizioni iniziali;
 \item[\circled{5}\,] qui il calcolo del valore di contenuto d'acqua in un generico punto i,j della \emph{mesh}, come descritto dall'equazione \ref{eq:rich_sim_discr};
 \item[\circled{6}\,] da questo punto in poi, il codice non fa altro che produrre i grafici sottoriportati. 
\end{description}

Il grafico che rappresenta l'infiltrazione all'interno della colonna di suolo nel tempo è il seguente:

\newpage
\begin{figure}[H]
% This file was created by matlab2tikz v0.4.4 running on MATLAB 8.2.
% Copyright (c) 2008--2013, Nico Schlömer <nico.schloemer@gmail.com>
% All rights reserved.
% 
% The latest updates can be retrieved from
%   http://www.mathworks.com/matlabcentral/fileexchange/22022-matlab2tikz
% where you can also make suggestions and rate matlab2tikz.
% 
\begin{tikzpicture}
\tikzstyle{every node}=[font=\footnotesize]
\begin{axis}[%
width=4.40888888888889in,
height=3.47733333333333in,
scale only axis,
xmin=0,
xmax=0.1,
xlabel={Distanza dalla cima della colonna (m)},
ymin=0,
ymax=0.45,
ylabel={$\text{Contenuto d'acqua (}\Theta\text{)}$},
axis x line*=bottom,
axis y line*=left
]
\addplot [
color=blue,
solid,
forget plot
]
table[row sep=crcr]{
0 0.3\\
0.00101010101010101 0.1\\
0.00202020202020202 0.1\\
0.00303030303030303 0.1\\
0.00404040404040404 0.1\\
0.00505050505050505 0.1\\
0.00606060606060606 0.1\\
0.00707070707070707 0.1\\
0.00808080808080808 0.1\\
0.00909090909090909 0.1\\
0.0101010101010101 0.1\\
0.0111111111111111 0.1\\
0.0121212121212121 0.1\\
0.0131313131313131 0.1\\
0.0141414141414141 0.1\\
0.0151515151515152 0.1\\
0.0161616161616162 0.1\\
0.0171717171717172 0.1\\
0.0181818181818182 0.1\\
0.0191919191919192 0.1\\
0.0202020202020202 0.1\\
0.0212121212121212 0.1\\
0.0222222222222222 0.1\\
0.0232323232323232 0.1\\
0.0242424242424242 0.1\\
0.0252525252525253 0.1\\
0.0262626262626263 0.1\\
0.0272727272727273 0.1\\
0.0282828282828283 0.1\\
0.0292929292929293 0.1\\
0.0303030303030303 0.1\\
0.0313131313131313 0.1\\
0.0323232323232323 0.1\\
0.0333333333333333 0.1\\
0.0343434343434343 0.1\\
0.0353535353535354 0.1\\
0.0363636363636364 0.1\\
0.0373737373737374 0.1\\
0.0383838383838384 0.1\\
0.0393939393939394 0.1\\
0.0404040404040404 0.1\\
0.0414141414141414 0.1\\
0.0424242424242424 0.1\\
0.0434343434343434 0.1\\
0.0444444444444444 0.1\\
0.0454545454545455 0.1\\
0.0464646464646465 0.1\\
0.0474747474747475 0.1\\
0.0484848484848485 0.1\\
0.0494949494949495 0.1\\
0.0505050505050505 0.1\\
0.0515151515151515 0.1\\
0.0525252525252525 0.1\\
0.0535353535353535 0.1\\
0.0545454545454545 0.1\\
0.0555555555555556 0.1\\
0.0565656565656566 0.1\\
0.0575757575757576 0.1\\
0.0585858585858586 0.1\\
0.0595959595959596 0.1\\
0.0606060606060606 0.1\\
0.0616161616161616 0.1\\
0.0626262626262626 0.1\\
0.0636363636363636 0.1\\
0.0646464646464646 0.1\\
0.0656565656565657 0.1\\
0.0666666666666667 0.1\\
0.0676767676767677 0.1\\
0.0686868686868687 0.1\\
0.0696969696969697 0.1\\
0.0707070707070707 0.1\\
0.0717171717171717 0.1\\
0.0727272727272727 0.1\\
0.0737373737373737 0.1\\
0.0747474747474748 0.1\\
0.0757575757575758 0.1\\
0.0767676767676768 0.1\\
0.0777777777777778 0.1\\
0.0787878787878788 0.1\\
0.0797979797979798 0.1\\
0.0808080808080808 0.1\\
0.0818181818181818 0.1\\
0.0828282828282828 0.1\\
0.0838383838383839 0.1\\
0.0848484848484849 0.1\\
0.0858585858585859 0.1\\
0.0868686868686869 0.1\\
0.0878787878787879 0.1\\
0.0888888888888889 0.1\\
0.0898989898989899 0.1\\
0.0909090909090909 0.1\\
0.0919191919191919 0.1\\
0.0929292929292929 0.1\\
0.0939393939393939 0.1\\
0.094949494949495 0.1\\
0.095959595959596 0.1\\
0.096969696969697 0.1\\
0.097979797979798 0.1\\
0.098989898989899 0.1\\
0.1 0.1\\
};
\node[right, inner sep=0mm, text=black]
at (axis cs:0.025,0.1,0) {T =0 s};
\addplot [
color=blue,
solid,
forget plot
]
table[row sep=crcr]{
0 0.3\\
0.00101010101010101 0.284729050508919\\
0.00202020202020202 0.269597844226155\\
0.00303030303030303 0.254742291874003\\
0.00404040404040404 0.240290814204931\\
0.00505050505050505 0.226361023345884\\
0.00606060606060606 0.213056885360688\\
0.00707070707070707 0.200466476466858\\
0.00808080808080808 0.188660409265484\\
0.00909090909090909 0.177690965953928\\
0.0101010101010101 0.167591935754613\\
0.0111111111111111 0.158379116553197\\
0.0121212121212121 0.150051408473872\\
0.0131313131313131 0.142592401744706\\
0.0141414141414141 0.135972343948156\\
0.0151515151515152 0.130150363095629\\
0.0161616161616162 0.125076822676554\\
0.0171717171717172 0.120695692050317\\
0.0181818181818182 0.116946828924518\\
0.0191919191919192 0.113768088530389\\
0.0202020202020202 0.111097194665696\\
0.0212121212121212 0.108873329262368\\
0.0222222222222222 0.107038417965694\\
0.0232323232323232 0.105538108086718\\
0.0242424242424242 0.104322451263527\\
0.0252525252525253 0.103346315668391\\
0.0262626262626263 0.102569561411997\\
0.0272727272727273 0.101957018021516\\
0.0282828282828283 0.101478304850649\\
0.0292929292929293 0.101107534532983\\
0.0303030303030303 0.100822936725744\\
0.0313131313131313 0.100606435046074\\
0.0323232323232323 0.100443204881957\\
0.0333333333333333 0.100321234197787\\
0.0343434343434343 0.100230903984658\\
0.0353535353535354 0.100164599951858\\
0.0363636363636364 0.100116362634052\\
0.0373737373737374 0.100081579414558\\
0.0383838383838384 0.100056719072587\\
0.0393939393939394 0.100039107321407\\
0.0404040404040404 0.100026740340806\\
0.0414141414141414 0.100018132420617\\
0.0424242424242424 0.100012193410639\\
0.0434343434343434 0.100008131604947\\
0.0444444444444444 0.100005377873239\\
0.0454545454545455 0.100003527199956\\
0.0464646464646465 0.100002294231644\\
0.0474747474747475 0.100001479908819\\
0.0484848484848485 0.100000946730531\\
0.0494949494949495 0.100000600641216\\
0.0505050505050505 0.100000377924577\\
0.0515151515151515 0.100000235830876\\
0.0525252525252525 0.100000145950737\\
0.0535353535353535 0.100000089583267\\
0.0545454545454545 0.100000054534035\\
0.0555555555555556 0.100000032925618\\
0.0565656565656566 0.100000019716574\\
0.0575757575757576 0.100000011710249\\
0.0585858585858586 0.100000006898333\\
0.0595959595959596 0.100000004030619\\
0.0606060606060606 0.100000002335907\\
0.0616161616161616 0.100000001342772\\
0.0626262626262626 0.10000000076563\\
0.0636363636363636 0.100000000433024\\
0.0646464646464646 0.100000000242934\\
0.0656565656565657 0.100000000135194\\
0.0666666666666667 0.100000000074631\\
0.0676767676767677 0.100000000040868\\
0.0686868686868687 0.1000000000222\\
0.0696969696969697 0.100000000011963\\
0.0707070707070707 0.100000000006395\\
0.0717171717171717 0.100000000003391\\
0.0727272727272727 0.100000000001783\\
0.0737373737373737 0.10000000000093\\
0.0747474747474748 0.100000000000481\\
0.0757575757575758 0.100000000000246\\
0.0767676767676768 0.100000000000125\\
0.0777777777777778 0.100000000000063\\
0.0787878787878788 0.100000000000031\\
0.0797979797979798 0.100000000000015\\
0.0808080808080808 0.100000000000007\\
0.0818181818181818 0.100000000000003\\
0.0828282828282828 0.100000000000001\\
0.0838383838383839 0.1\\
0.0848484848484849 0.0999999999999999\\
0.0858585858585859 0.0999999999999999\\
0.0868686868686869 0.0999999999999999\\
0.0878787878787879 0.0999999999999999\\
0.0888888888888889 0.0999999999999999\\
0.0898989898989899 0.0999999999999999\\
0.0909090909090909 0.0999999999999999\\
0.0919191919191919 0.0999999999999999\\
0.0929292929292929 0.0999999999999999\\
0.0939393939393939 0.0999999999999999\\
0.094949494949495 0.0999999999999999\\
0.095959595959596 0.0999999999999999\\
0.096969696969697 0.0999999999999999\\
0.097979797979798 0.0999999999999999\\
0.098989898989899 0.0999999999999999\\
0.1 0.1\\
};
\addplot [
color=blue,
solid,
forget plot
]
table[row sep=crcr]{
0 0.3\\
0.00101010101010101 0.289193574443107\\
0.00202020202020202 0.278436687383833\\
0.00303030303030303 0.267778196382268\\
0.00404040404040404 0.257265612715489\\
0.00505050505050505 0.246944466716572\\
0.00606060606060606 0.236857717911955\\
0.00707070707070707 0.227045222653515\\
0.00808080808080808 0.217543270142902\\
0.00909090909090909 0.208384195637969\\
0.0101010101010101 0.199596077297757\\
0.0111111111111111 0.191202520654186\\
0.0121212121212121 0.183222532188431\\
0.0131313131313131 0.175670481029932\\
0.0141414141414141 0.168556145472294\\
0.0151515151515152 0.161884838889967\\
0.0161616161616162 0.155657607807537\\
0.0171717171717172 0.149871493369844\\
0.0181818181818182 0.144519846320434\\
0.0191919191919192 0.139592684835719\\
0.0202020202020202 0.135077084184137\\
0.0212121212121212 0.130957587169742\\
0.0222222222222222 0.127216624650511\\
0.0232323232323232 0.123834936054535\\
0.0242424242424242 0.120791980704508\\
0.0252525252525253 0.118066331849198\\
0.0262626262626263 0.11563604653325\\
0.0272727272727273 0.113479005756945\\
0.0282828282828283 0.111573220730705\\
0.0292929292929293 0.109897102364888\\
0.0303030303030303 0.108429692409502\\
0.0313131313131313 0.107150855833822\\
0.0323232323232323 0.106041435083805\\
0.0333333333333333 0.105083367754953\\
0.0343434343434343 0.104259769957739\\
0.0353535353535354 0.103554988227262\\
0.0363636363636364 0.102954623240814\\
0.0373737373737374 0.102445528864842\\
0.0383838383838384 0.102015790169289\\
0.0393939393939394 0.101654684039157\\
0.0404040404040404 0.101352625899067\\
0.0414141414141414 0.10110110586669\\
0.0424242424242424 0.10089261738535\\
0.0434343434343434 0.100720581074197\\
0.0444444444444444 0.100579266194313\\
0.0454545454545455 0.100463711776835\\
0.0464646464646465 0.100369649108655\\
0.0474747474747475 0.100293426933657\\
0.0484848484848485 0.100231940411751\\
0.0494949494949495 0.100182564590532\\
0.0505050505050505 0.100143092889388\\
0.0515151515151515 0.1001116808756\\
0.0525252525252525 0.100086795426762\\
0.0535353535353535 0.100067169223004\\
0.0545454545454545 0.100051760393792\\
0.0555555555555556 0.100039717055038\\
0.0565656565656566 0.100030346409331\\
0.0575757575757576 0.100023088041988\\
0.0585858585858586 0.100017491024583\\
0.0595959595959596 0.100013194432157\\
0.0606060606060606 0.10000991088718\\
0.0616161616161616 0.100007412759487\\
0.0626262626262626 0.100005520674163\\
0.0636363636363636 0.100004094006535\\
0.0646464646464646 0.100003023073017\\
0.0656565656565657 0.100002222757091\\
0.0666666666666667 0.100001627339993\\
0.0676767676767677 0.100001186334793\\
0.0686868686868687 0.100000861149891\\
0.0696969696969697 0.100000622433078\\
0.0707070707070707 0.100000447970076\\
0.0717171717171717 0.100000321031645\\
0.0727272727272727 0.100000229081137\\
0.0737373737373737 0.100000162769744\\
0.0747474747474748 0.100000115159874\\
0.0757575757575758 0.100000081128265\\
0.0767676767676768 0.100000056909807\\
0.0777777777777778 0.100000039750852\\
0.0787878787878788 0.100000027647172\\
0.0797979797979798 0.10000001914699\\
0.0808080808080808 0.100000013203726\\
0.0818181818181818 0.100000009066494\\
0.0828282828282828 0.100000006199118\\
0.0838383838383839 0.100000004220547\\
0.0848484848484849 0.100000002861256\\
0.0858585858585859 0.100000001931499\\
0.0868686868686869 0.100000001298321\\
0.0878787878787879 0.100000000868999\\
0.0888888888888889 0.100000000579164\\
0.0898989898989899 0.100000000384345\\
0.0909090909090909 0.100000000253952\\
0.0919191919191919 0.100000000167044\\
0.0929292929292929 0.100000000109345\\
0.0939393939393939 0.100000000071164\\
0.094949494949495 0.100000000045942\\
0.095959595959596 0.100000000029251\\
0.096969696969697 0.100000000018089\\
0.097979797979798 0.1000000000104\\
0.098989898989899 0.100000000004736\\
0.1 0.1\\
};
\addplot [
color=blue,
solid,
forget plot
]
table[row sep=crcr]{
0 0.3\\
0.00101010101010101 0.2911743492219\\
0.00202020202020202 0.28237568772727\\
0.00303030303030303 0.273630757278456\\
0.00404040404040404 0.26496580837769\\
0.00505050505050505 0.256406364011495\\
0.00606060606060606 0.247976994420183\\
0.00707070707070707 0.239701106200364\\
0.00808080808080808 0.231600748746594\\
0.00909090909090909 0.223696440676849\\
0.0101010101010101 0.216007018474928\\
0.0111111111111111 0.208549509132179\\
0.0121212121212121 0.201339028092745\\
0.0131313131313131 0.194388703313158\\
0.0141414141414141 0.187709625750777\\
0.0151515151515152 0.181310826108391\\
0.0161616161616162 0.175199277195685\\
0.0171717171717172 0.169379920832604\\
0.0181818181818182 0.163855717824246\\
0.0191919191919192 0.158627719189484\\
0.0202020202020202 0.153695156532236\\
0.0212121212121212 0.149055549209542\\
0.0222222222222222 0.144704825776986\\
0.0232323232323232 0.140637457080411\\
0.0242424242424242 0.136846598312329\\
0.0252525252525253 0.133324237359644\\
0.0262626262626263 0.130061346832262\\
0.0272727272727273 0.12704803727494\\
0.0282828282828283 0.124273709221115\\
0.0292929292929293 0.121727201940822\\
0.0303030303030303 0.11939693695775\\
0.0313131313131313 0.117271054655578\\
0.0323232323232323 0.115337542553398\\
0.0333333333333333 0.113584354097067\\
0.0343434343434343 0.111999517080792\\
0.0353535353535354 0.110571231074887\\
0.0363636363636364 0.109287953485935\\
0.0373737373737374 0.108138474109652\\
0.0383838383838384 0.107111978250853\\
0.0393939393939394 0.106198098675983\\
0.0404040404040404 0.105386956829652\\
0.0414141414141414 0.104669193886261\\
0.0424242424242424 0.10403599232081\\
0.0434343434343434 0.103479088769532\\
0.0444444444444444 0.102990779012243\\
0.0454545454545455 0.102563915945611\\
0.0464646464646465 0.102191901432023\\
0.0474747474747475 0.101868672904555\\
0.0484848484848485 0.101588685587298\\
0.0494949494949495 0.10134689115464\\
0.0505050505050505 0.101138713605596\\
0.0515151515151515 0.100960023072718\\
0.0525252525252525 0.100807108221863\\
0.0535353535353535 0.100676647831594\\
0.0545454545454545 0.100565682071316\\
0.0555555555555556 0.100471583927277\\
0.0565656565656566 0.100392031156985\\
0.0575757575757576 0.100324979086632\\
0.0585858585858586 0.100268634504008\\
0.0595959595959596 0.100221430841805\\
0.0606060606060606 0.100182004793832\\
0.0616161616161616 0.100149174459808\\
0.0626262626262626 0.100121919073252\\
0.0636363636363636 0.100099360331468\\
0.0646464646464646 0.100080745316746\\
0.0656565656565657 0.100065430973145\\
0.0666666666666667 0.100052870083575\\
0.0676767676767677 0.100042598676561\\
0.0686868686868687 0.100034224780971\\
0.0696969696969697 0.100027418439344\\
0.0707070707070707 0.100021902885985\\
0.0717171717171717 0.100017446794119\\
0.0727272727272727 0.100013857496749\\
0.0737373737373737 0.100010975087935\\
0.0747474747474748 0.100008667314728\\
0.0757575757575758 0.100006825174497\\
0.0767676767676768 0.100005359137687\\
0.0777777777777778 0.100004195921762\\
0.0787878787878788 0.100003275748138\\
0.0797979797979798 0.100002550019957\\
0.0808080808080808 0.10000197936459\\
0.0818181818181818 0.100001531990572\\
0.0828282828282828 0.100001182314216\\
0.0838383838383839 0.100000909816377\\
0.0848484848484849 0.100000698094653\\
0.0858585858585859 0.100000534080769\\
0.0868686868686869 0.100000407396887\\
0.0878787878787879 0.100000309828235\\
0.0888888888888889 0.100000234892644\\
0.0898989898989899 0.100000177490467\\
0.0909090909090909 0.100000133620828\\
0.0919191919191919 0.100000100152351\\
0.0929292929292929 0.100000074638365\\
0.0939393939393939 0.100000055168236\\
0.094949494949495 0.100000040247782\\
0.095959595959596 0.100000028702925\\
0.096969696969697 0.100000019601652\\
0.097979797979798 0.100000012190132\\
0.098989898989899 0.100000005839463\\
0.1 0.1\\
};
\node[right, inner sep=0mm, text=black]
at (axis cs:0.025,0.136846598312329,0) {T =25900 s};
\addplot [
color=blue,
solid,
forget plot
]
table[row sep=crcr]{
0 0.3\\
0.00101010101010101 0.292355791669729\\
0.00202020202020202 0.284729121171288\\
0.00303030303030303 0.277137405657463\\
0.00404040404040404 0.269597822306607\\
0.00505050505050505 0.262127191772357\\
0.00606060606060606 0.25474186569596\\
0.00707070707070707 0.2474576195339\\
0.00808080808080808 0.240289551869253\\
0.00909090909090909 0.233251991273032\\
0.0101010101010101 0.226358411664017\\
0.0111111111111111 0.219621356984325\\
0.0121212121212121 0.21305237586603\\
0.0131313131313131 0.206661966814196\\
0.0141414141414141 0.20045953427664\\
0.0151515151515152 0.194453355813504\\
0.0161616161616162 0.188650560423269\\
0.0171717171717172 0.183057117928833\\
0.0181818181818182 0.177677839180607\\
0.0191919191919192 0.172516386695431\\
0.0202020202020202 0.167575295222915\\
0.0212121212121212 0.162856001616308\\
0.0222222222222222 0.158358883284841\\
0.0232323232323232 0.154083304419856\\
0.0242424242424242 0.150027669118819\\
0.0252525252525253 0.146189480479935\\
0.0262626262626263 0.142565404705772\\
0.0272727272727273 0.139151339236685\\
0.0282828282828283 0.135942483933512\\
0.0292929292929293 0.13293341434304\\
0.0303030303030303 0.130118156108011\\
0.0313131313131313 0.127490259624718\\
0.0323232323232323 0.125042874103824\\
0.0333333333333333 0.122768820252548\\
0.0343434343434343 0.120660660866742\\
0.0353535353535354 0.11871076869816\\
0.0363636363636364 0.116911391043354\\
0.0373737373737374 0.115254710584549\\
0.0383838383838384 0.113732902097729\\
0.0393939393939394 0.112338184727583\\
0.0404040404040404 0.111062869611311\\
0.0414141414141414 0.109899402712394\\
0.0424242424242424 0.108840402800169\\
0.0434343434343434 0.107878694580363\\
0.0444444444444444 0.107007337045018\\
0.0454545454545455 0.106219647166818\\
0.0464646464646465 0.10550921911234\\
0.0474747474747475 0.104869939190993\\
0.0484848484848485 0.104295996791319\\
0.0494949494949495 0.103781891584018\\
0.0505050505050505 0.103322437291713\\
0.0515151515151515 0.102912762339508\\
0.0525252525252525 0.102548307708174\\
0.0535353535353535 0.102224822313905\\
0.0545454545454545 0.101938356235507\\
0.0555555555555556 0.101685252102358\\
0.0565656565656566 0.101462134944983\\
0.0575757575757576 0.101265900795391\\
0.0585858585858586 0.101093704306962\\
0.0595959595959596 0.100942945644327\\
0.0606060606060606 0.100811256872831\\
0.0616161616161616 0.100696488055479\\
0.0626262626262626 0.10059669324306\\
0.0636363636363636 0.100510116521006\\
0.0646464646464646 0.100435178254765\\
0.0656565656565657 0.100370461654363\\
0.0666666666666667 0.100314699758771\\
0.0676767676767677 0.100266762921735\\
0.0686868686868687 0.10022564686319\\
0.0696969696969697 0.100190461334256\\
0.0707070707070707 0.100160419429264\\
0.0717171717171717 0.100134827565237\\
0.0727272727272727 0.100113076137815\\
0.0737373737373737 0.100094630852714\\
0.0747474747474748 0.100079024723354\\
0.0757575757575758 0.100065850718256\\
0.0767676767676768 0.10005475503608\\
0.0777777777777778 0.100045430981621\\
0.0787878787878788 0.100037613412658\\
0.0797979797979798 0.100031073725073\\
0.0808080808080808 0.100025615342062\\
0.0818181818181818 0.100021069672407\\
0.0828282828282828 0.100017292502567\\
0.0838383838383839 0.100014160787677\\
0.0848484848484849 0.100011569807317\\
0.0858585858585859 0.100009430653007\\
0.0868686868686869 0.100007668015816\\
0.0878787878787879 0.10000621824399\\
0.0888888888888889 0.100005027642279\\
0.0898989898989899 0.100004050986365\\
0.0909090909090909 0.100003250227643\\
0.0919191919191919 0.10000259336534\\
0.0929292929292929 0.100002053464721\\
0.0939393939393939 0.100001607801743\\
0.094949494949495 0.100001237116044\\
0.095959595959596 0.100000924955559\\
0.096969696969697 0.100000657097283\\
0.097979797979798 0.100000421029798\\
0.098989898989899 0.100000205484102\\
0.1 0.1\\
};
\addplot [
color=blue,
solid,
forget plot
]
table[row sep=crcr]{
0 0.3\\
0.00101010101010101 0.293162291346467\\
0.00202020202020202 0.286337135083833\\
0.00303030303030303 0.279537014487306\\
0.00404040404040404 0.27277427523486\\
0.00505050505050505 0.266061058185837\\
0.00606060606060606 0.259409234029543\\
0.00707070707070707 0.252830340389786\\
0.00808080808080808 0.246335521940062\\
0.00909090909090909 0.239935474045999\\
0.0101010101010101 0.23364039040737\\
0.0111111111111111 0.227459915122164\\
0.0121212121212121 0.221403099540683\\
0.0131313131313131 0.215478364219279\\
0.0141414141414141 0.209693466222025\\
0.0151515151515152 0.204055471955358\\
0.0161616161616162 0.198570735656394\\
0.0171717171717172 0.193244883591254\\
0.0181818181818182 0.18808280395622\\
0.0191919191919192 0.183088642412771\\
0.0202020202020202 0.178265803128443\\
0.0212121212121212 0.173616955139668\\
0.0222222222222222 0.169144043801111\\
0.0232323232323232 0.164848307039045\\
0.0242424242424242 0.160730296084606\\
0.0252525252525253 0.156789900326607\\
0.0262626262626263 0.153026375893477\\
0.0272727272727273 0.149438377549844\\
0.0282828282828283 0.146023993475449\\
0.0292929292929293 0.142780782482558\\
0.0303030303030303 0.13970581322248\\
0.0313131313131313 0.136795704932264\\
0.0323232323232323 0.134046669278629\\
0.0333333333333333 0.131454552867441\\
0.0343434343434343 0.129014880003099\\
0.0353535353535354 0.126722895302554\\
0.0363636363636364 0.124573605792848\\
0.0373737373737374 0.12256182214841\\
0.0383838383838384 0.120682198754408\\
0.0393939393939394 0.118929272314483\\
0.0404040404040404 0.117297498754797\\
0.0414141414141414 0.115781288210728\\
0.0424242424242424 0.1143750379174\\
0.0434343434343434 0.113073162859843\\
0.0444444444444444 0.1118701240726\\
0.0454545454545455 0.110760454511524\\
0.0464646464646465 0.109738782451949\\
0.0474747474747475 0.108799852397105\\
0.0484848484848485 0.107938543508197\\
0.0494949494949495 0.107149885592863\\
0.0505050505050505 0.106429072711482\\
0.0515151515151515 0.10577147448104\\
0.0525252525252525 0.105172645173723\\
0.0535353535353535 0.104628330722286\\
0.0545454545454545 0.104134473756387\\
0.0555555555555556 0.103687216803639\\
0.0565656565656566 0.103282903796217\\
0.0575757575757576 0.102918080028504\\
0.0585858585858586 0.102589490713778\\
0.0595959595959596 0.10229407828833\\
0.0606060606060606 0.102028978609989\\
0.0616161616161616 0.101791516194972\\
0.0626262626262626 0.101579198632412\\
0.0636363636363636 0.101389710310198\\
0.0646464646464646 0.101220905578953\\
0.0656565656565657 0.101070801473361\\
0.0666666666666667 0.100937570101814\\
0.0676767676767677 0.100819530806626\\
0.0686868686868687 0.100715142188103\\
0.0696969696969697 0.100622994076616\\
0.0707070707070707 0.100541799527736\\
0.0717171717171717 0.100470386906524\\
0.0727272727272727 0.100407692118318\\
0.0737373737373737 0.100352751035004\\
0.0747474747474748 0.100304692157719\\
0.0757575757575758 0.100262729549468\\
0.0767676767676768 0.100226156064059\\
0.0777777777777778 0.100194336891355\\
0.0787878787878788 0.10016670343286\\
0.0797979797979798 0.100142747516366\\
0.0808080808080808 0.100122015953565\\
0.0818181818181818 0.100104105440292\\
0.0828282828282828 0.100088657795391\\
0.0838383838383839 0.100075355530966\\
0.0848484848484849 0.100063917744061\\
0.0858585858585859 0.100054096317543\\
0.0868686868686869 0.100045672416064\\
0.0878787878787879 0.100038453261489\\
0.0888888888888889 0.100032269170955\\
0.0898989898989899 0.100026970839865\\
0.0909090909090909 0.100022426851431\\
0.0919191919191919 0.100018521393968\\
0.0929292929292929 0.100015152166839\\
0.0939393939393939 0.100012228455845\\
0.094949494949495 0.100009669358807\\
0.095959595959596 0.100007402142167\\
0.096969696969697 0.100005360709503\\
0.097979797979798 0.100003484163022\\
0.098989898989899 0.100001715439216\\
0.1 0.1\\
};
\addplot [
color=blue,
solid,
forget plot
]
table[row sep=crcr]{
0 0.3\\
0.00101010101010101 0.293757737550715\\
0.00202020202020202 0.287525025722799\\
0.00303030303030303 0.281311371307766\\
0.00404040404040404 0.275126193772597\\
0.00505050505050505 0.268978782431856\\
0.00606060606060606 0.262878254611051\\
0.00707070707070707 0.256833515115154\\
0.00808080808080808 0.250853217302312\\
0.00909090909090909 0.244945726045832\\
0.0101010101010101 0.239119082847649\\
0.0111111111111111 0.233380973344054\\
0.0121212121212121 0.227738697419693\\
0.0131313131313131 0.222199142119102\\
0.0141414141414141 0.216768757516698\\
0.0151515151515152 0.211453535676496\\
0.0161616161616162 0.206258992802357\\
0.0171717171717172 0.201190154648575\\
0.0181818181818182 0.196251545229511\\
0.0191919191919192 0.191447178836175\\
0.0202020202020202 0.186780555337483\\
0.0212121212121212 0.182254658714726\\
0.0222222222222222 0.177871958749897\\
0.0232323232323232 0.173634415762317\\
0.0242424242424242 0.169543488263575\\
0.0252525252525253 0.165600143378604\\
0.0262626262626263 0.161804869860763\\
0.0272727272727273 0.158157693511386\\
0.0282828282828283 0.15465819479943\\
0.0292929292929293 0.151305528464711\\
0.0303030303030303 0.148098444878865\\
0.0313131313131313 0.14503531293149\\
0.0323232323232323 0.142114144205007\\
0.0333333333333333 0.139332618200499\\
0.0343434343434343 0.136688108378024\\
0.0353535353535354 0.134177708778594\\
0.0363636363636364 0.131798261000931\\
0.0373737373737374 0.129546381314133\\
0.0383838383838384 0.127418487697279\\
0.0393939393939394 0.12541082660857\\
0.0404040404040404 0.12351949929961\\
0.0414141414141414 0.121740487504664\\
0.0424242424242424 0.120069678349908\\
0.0434343434343434 0.118502888343646\\
0.0444444444444444 0.117035886324901\\
0.0454545454545455 0.115664415264523\\
0.0464646464646465 0.114384212829729\\
0.0474747474747475 0.113191030639691\\
0.0484848484848485 0.112080652156068\\
0.0494949494949495 0.111048909168257\\
0.0505050505050505 0.110091696848297\\
0.0515151515151515 0.109204987364763\\
0.0525252525252525 0.108384842058491\\
0.0535353535353535 0.107627422195463\\
0.0545454545454545 0.106928998323615\\
0.0555555555555556 0.106285958270623\\
0.0565656565656566 0.105694813828866\\
0.0575757575757576 0.105152206181704\\
0.0585858585858586 0.10465491013198\\
0.0595959595959596 0.104199837199262\\
0.0606060606060606 0.103784037656772\\
0.0616161616161616 0.103404701582327\\
0.0626262626262626 0.103059158999904\\
0.0636363636363636 0.102744879189777\\
0.0646464646464646 0.102459469245554\\
0.0656565656565657 0.102200671956025\\
0.0666666666666667 0.101966363088499\\
0.0676767676767677 0.101754548148445\\
0.0686868686868687 0.101563358687753\\
0.0696969696969697 0.101391048230944\\
0.0707070707070707 0.101235987885208\\
0.0717171717171717 0.101096661696378\\
0.0727272727272727 0.100971661808891\\
0.0737373737373737 0.100859683483507\\
0.0747474747474748 0.100759520022159\\
0.0757575757575758 0.100670057644843\\
0.0767676767676768 0.100590270358946\\
0.0777777777777778 0.100519214856956\\
0.0787878787878788 0.100456025474116\\
0.0797979797979798 0.100399909233316\\
0.0808080808080808 0.100350141000383\\
0.0818181818181818 0.100306058769022\\
0.0828282828282828 0.10026705909087\\
0.0838383838383839 0.100232592662636\\
0.0848484848484849 0.10020216007896\\
0.0858585858585859 0.100175307756571\\
0.0868686868686869 0.100151624032479\\
0.0878787878787879 0.10013073543633\\
0.0888888888888889 0.100112303134678\\
0.0898989898989899 0.1000960195428\\
0.0909090909090909 0.100081605097722\\
0.0919191919191919 0.100068805184434\\
0.0929292929292929 0.100057387205725\\
0.0939393939393939 0.100047137784756\\
0.094949494949495 0.100037860088315\\
0.095959595959596 0.100029371257697\\
0.096969696969697 0.100021499933327\\
0.097979797979798 0.100014083858514\\
0.098989898989899 0.100006967547191\\
0.1 0.1\\
};
\node[right, inner sep=0mm, text=black]
at (axis cs:0.025,0.169543488263575,0) {T =51800 s};
\addplot [
color=blue,
solid,
forget plot
]
table[row sep=crcr]{
0 0.3\\
0.00101010101010101 0.294220578323082\\
0.00202020202020202 0.288448736598409\\
0.00303030303030303 0.282692024958219\\
0.00404040404040404 0.276957934090231\\
0.00505050505050505 0.271253866002329\\
0.00606060606060606 0.265587105366577\\
0.00707070707070707 0.259964791627398\\
0.00808080808080808 0.254393892051793\\
0.00909090909090909 0.248881175890907\\
0.0101010101010101 0.243433189812234\\
0.0111111111111111 0.238056234750305\\
0.0121212121212121 0.232756344311103\\
0.0131313131313131 0.227539264851644\\
0.0141414141414141 0.222410437341547\\
0.0151515151515152 0.21737498109791\\
0.0161616161616162 0.212437679468787\\
0.0171717171717172 0.207602967524064\\
0.0181818181818182 0.202874921795823\\
0.0191919191919192 0.198257252093488\\
0.0202020202020202 0.193753295402354\\
0.0212121212121212 0.189366011857726\\
0.0222222222222222 0.185097982770901\\
0.0232323232323232 0.180951410667893\\
0.0242424242424242 0.176928121287181\\
0.0252525252525253 0.173029567469001\\
0.0262626262626263 0.169256834855976\\
0.0272727272727273 0.165610649313208\\
0.0282828282828283 0.162091385965511\\
0.0292929292929293 0.158699079740226\\
0.0303030303030303 0.155433437296177\\
0.0313131313131313 0.152293850212778\\
0.0323232323232323 0.149279409308084\\
0.0333333333333333 0.146388919950822\\
0.0343434343434343 0.143620918228922\\
0.0353535353535354 0.140973687836001\\
0.0363636363636364 0.138445277537388\\
0.0373737373737374 0.136033519078717\\
0.0383838383838384 0.133736045402651\\
0.0393939393939394 0.131550309043019\\
0.0404040404040404 0.129473600570269\\
0.0414141414141414 0.127503066967764\\
0.0424242424242424 0.125635729824817\\
0.0434343434343434 0.123868503239465\\
0.0444444444444444 0.122198211331694\\
0.0454545454545455 0.120621605275975\\
0.0464646464646465 0.119135379770567\\
0.0474747474747475 0.117736188869863\\
0.0484848484848485 0.116420661115044\\
0.0494949494949495 0.115185413907384\\
0.0505050505050505 0.114027067077566\\
0.0515151515151515 0.112942255613291\\
0.0525252525252525 0.111927641516153\\
0.0535353535353535 0.11097992476718\\
0.0545454545454545 0.110095853388505\\
0.0555555555555556 0.109272232596309\\
0.0565656565656566 0.10850593304735\\
0.0575757575757576 0.107793898188102\\
0.0585858585858586 0.107133150721689\\
0.0595959595959596 0.106520798213358\\
0.0606060606060606 0.105954037860279\\
0.0616161616161616 0.105430160455841\\
0.0626262626262626 0.104946553582437\\
0.0636363636363636 0.104500704069967\\
0.0646464646464646 0.104090199759922\\
0.0656565656565657 0.103712730616972\\
0.0666666666666667 0.103366089231558\\
0.0676767676767677 0.10304817075795\\
0.0686868686868687 0.102756972332817\\
0.0696969696969697 0.102490592019368\\
0.0707070707070707 0.102247227321823\\
0.0717171717171717 0.102025173314206\\
0.0727272727272727 0.101822820426386\\
0.0737373737373737 0.101638651928896\\
0.0747474747474748 0.101471241156405\\
0.0757575757575758 0.101319248507819\\
0.0767676767676768 0.101181418258926\\
0.0777777777777778 0.101056575221209\\
0.0787878787878788 0.100943621278154\\
0.0797979797979798 0.100841531827852\\
0.0808080808080808 0.100749352158231\\
0.0818181818181818 0.100666193778679\\
0.0828282828282828 0.100591230729272\\
0.0838383838383839 0.100523695886294\\
0.0848484848484849 0.100462877280235\\
0.0858585858585859 0.100408114440028\\
0.0868686868686869 0.100358794774926\\
0.0878787878787879 0.100314350003156\\
0.0888888888888889 0.100274252634338\\
0.0898989898989899 0.1002380125106\\
0.0909090909090909 0.100205173409404\\
0.0919191919191919 0.100175309709336\\
0.0929292929292929 0.10014802311841\\
0.0939393939393939 0.100122939462993\\
0.094949494949495 0.100099705534033\\
0.095959595959596 0.100077985986118\\
0.096969696969697 0.100057460283775\\
0.097979797979798 0.100037819688546\\
0.098989898989899 0.100018764279597\\
0.1 0.1\\
};
\addplot [
color=blue,
solid,
forget plot
]
table[row sep=crcr]{
0 0.3\\
0.00101010101010101 0.294593698974027\\
0.00202020202020202 0.289193602634086\\
0.00303030303030303 0.283805894305954\\
0.00404040404040404 0.278436714717437\\
0.00505050505050505 0.273092141004765\\
0.00606060606060606 0.267778166082672\\
0.00707070707070707 0.262500678494883\\
0.00808080808080808 0.257265442857828\\
0.00909090909090909 0.252078081005746\\
0.0101010101010101 0.246944053939727\\
0.0111111111111111 0.241868644676956\\
0.0121212121212121 0.236856942089338\\
0.0131313131313131 0.231913825812995\\
0.0141414141414141 0.227043952301878\\
0.0151515151515152 0.222251742089959\\
0.0161616161616162 0.217541368317356\\
0.0171717171717172 0.212916746566254\\
0.0181818181818182 0.208381526042814\\
0.0191919191919192 0.203939082131469\\
0.0202020202020202 0.199592510338147\\
0.0212121212121212 0.195344621629164\\
0.0222222222222222 0.191197939162901\\
0.0232323232323232 0.187154696401937\\
0.0242424242424242 0.183216836584195\\
0.0252525252525253 0.179386013522938\\
0.0262626262626263 0.175663593697141\\
0.0272727272727273 0.172050659586038\\
0.0282828282828283 0.168548014194406\\
0.0292929292929293 0.165156186708611\\
0.0303030303030303 0.161875439217524\\
0.0313131313131313 0.158705774427202\\
0.0323232323232323 0.155646944293751\\
0.0333333333333333 0.152698459495043\\
0.0343434343434343 0.149859599658969\\
0.0353535353535354 0.147129424263667\\
0.0363636363636364 0.144506784123676\\
0.0373737373737374 0.141990333375196\\
0.0383838383838384 0.1395785418736\\
0.0393939393939394 0.137269707916963\\
0.0404040404040404 0.135061971210672\\
0.0414141414141414 0.132953325990082\\
0.0424242424242424 0.130941634220652\\
0.0434343434343434 0.129024638797994\\
0.0444444444444444 0.127199976673741\\
0.0454545454545455 0.125465191837031\\
0.0464646464646465 0.123817748085669\\
0.0474747474747475 0.122255041525595\\
0.0484848484848485 0.120774412742117\\
0.0494949494949495 0.119373158591385\\
0.0505050505050505 0.118048543565772\\
0.0515151515151515 0.116797810692053\\
0.0525252525252525 0.115618191926612\\
0.0535353535353535 0.114506918017146\\
0.0545454545454545 0.113461227805597\\
0.0555555555555556 0.112478376952142\\
0.0565656565656566 0.111555646065064\\
0.0575757575757576 0.110690348226102\\
0.0585858585858586 0.109879835905502\\
0.0595959595959596 0.109121507265287\\
0.0606060606060606 0.108412811853387\\
0.0616161616161616 0.107751255695023\\
0.0626262626262626 0.107134405791248\\
0.0636363636363636 0.106559894037724\\
0.0646464646464646 0.106025420579633\\
0.0656565656565657 0.105528756621186\\
0.0666666666666667 0.105067746710333\\
0.0676767676767677 0.104640310521185\\
0.0686868686868687 0.104244444158171\\
0.0696969696969697 0.103878221007184\\
0.0707070707070707 0.103539792159918\\
0.0717171717171717 0.103227386438188\\
0.0727272727272727 0.10293931004542\\
0.0737373737373737 0.102673945872584\\
0.0747474747474748 0.102429752485689\\
0.0757575757575758 0.102205262821618\\
0.0767676767676768 0.101999082618486\\
0.0777777777777778 0.101809888605986\\
0.0787878787878788 0.101636426480248\\
0.0797979797979798 0.101477508686703\\
0.0808080808080808 0.10133201203328\\
0.0818181818181818 0.101198875154968\\
0.0828282828282828 0.101077095849458\\
0.0838383838383839 0.10096572830214\\
0.0848484848484849 0.100863880217312\\
0.0858585858585859 0.100770709870951\\
0.0868686868686869 0.100685423098937\\
0.0878787878787879 0.10060727023314\\
0.0888888888888889 0.100535542996333\\
0.0898989898989899 0.100469571365483\\
0.0909090909090909 0.100408720411608\\
0.0919191919191919 0.100352387123097\\
0.0929292929292929 0.100299997218156\\
0.0939393939393939 0.100251001950909\\
0.094949494949495 0.100204874914619\\
0.095959595959596 0.100161108844549\\
0.096969696969697 0.100119212422134\\
0.097979797979798 0.10007870708139\\
0.098989898989899 0.100039123817875\\
0.1 0.1\\
};
\addplot [
color=blue,
solid,
forget plot
]
table[row sep=crcr]{
0 0.3\\
0.00101010101010101 0.294902782643043\\
0.00202020202020202 0.289810765522128\\
0.00303030303030303 0.284729132959023\\
0.00404040404040404 0.279663037528097\\
0.00505050505050505 0.274617584384953\\
0.00606060606060606 0.269597815836213\\
0.00707070707070707 0.264608696228199\\
0.00808080808080808 0.259655097229924\\
0.00909090909090909 0.254741783583068\\
0.0101010101010101 0.249873399388291\\
0.0111111111111111 0.245054454993476\\
0.0121212121212121 0.240289314545288\\
0.0131313131313131 0.235582184260851\\
0.0141414141414141 0.230937101471338\\
0.0151515151515152 0.226357924484055\\
0.0161616161616162 0.221848323303978\\
0.0171717171717172 0.217411771250007\\
0.0181818181818182 0.213051537495207\\
0.0191919191919192 0.208770680554275\\
0.0202020202020202 0.204572042735338\\
0.0212121212121212 0.200458245567044\\
0.0222222222222222 0.196431686205764\\
0.0232323232323232 0.192494534821742\\
0.0242424242424242 0.188648732957058\\
0.0252525252525253 0.184895992842581\\
0.0262626262626263 0.181237797655545\\
0.0272727272727273 0.177675402694105\\
0.0282828282828283 0.174209837440294\\
0.0292929292929293 0.170841908478096\\
0.0303030303030303 0.167572203229097\\
0.0313131313131313 0.16440109446424\\
0.0323232323232323 0.161328745546673\\
0.0333333333333333 0.158355116357621\\
0.0343434343434343 0.15547996985449\\
0.0353535353535354 0.152702879208208\\
0.0363636363636364 0.150023235465005\\
0.0373737373737374 0.147440255676483\\
0.0383838383838384 0.144952991440916\\
0.0393939393939394 0.14256033779823\\
0.0404040404040404 0.140261042421068\\
0.0414141414141414 0.138053715044652\\
0.0424242424242424 0.13593683707893\\
0.0434343434343434 0.133908771347531\\
0.0444444444444444 0.131967771899536\\
0.0454545454545455 0.130111993841788\\
0.0464646464646465 0.128339503141519\\
0.0474747474747475 0.12664828635139\\
0.0484848484848485 0.125036260211521\\
0.0494949494949495 0.123501281085887\\
0.0505050505050505 0.122041154193301\\
0.0515151515151515 0.120653642596272\\
0.0525252525252525 0.119336475914163\\
0.0535353535353535 0.118087358730298\\
0.0545454545454545 0.116903978665934\\
0.0555555555555556 0.115784014097287\\
0.0565656565656566 0.114725141495096\\
0.0575757575757576 0.113725042369432\\
0.0585858585858586 0.112781409805637\\
0.0595959595959596 0.111891954580387\\
0.0606060606060606 0.111054410849837\\
0.0616161616161616 0.110266541404712\\
0.0626262626262626 0.109526142489899\\
0.0636363636363636 0.108831048188699\\
0.0646464646464646 0.108179134374308\\
0.0656565656565657 0.107568322233317\\
0.0666666666666667 0.106996581368102\\
0.0676767676767677 0.106461932486812\\
0.0686868686868687 0.105962449691377\\
0.0696969696969697 0.105496262375406\\
0.0707070707070707 0.105061556745159\\
0.0717171717171717 0.104656576977866\\
0.0727272727272727 0.104279626032591\\
0.0737373737373737 0.103929066129556\\
0.0747474747474748 0.10360331891442\\
0.0757575757575758 0.10330086532438\\
0.0767676767676768 0.103020245173196\\
0.0777777777777778 0.102760056472355\\
0.0787878787878788 0.102518954505491\\
0.0797979797979798 0.102295650673056\\
0.0808080808080808 0.102088911123885\\
0.0818181818181818 0.101897555189974\\
0.0828282828282828 0.101720453640244\\
0.0838383838383839 0.101556526768553\\
0.0848484848484849 0.101404742330568\\
0.0858585858585859 0.101264113343457\\
0.0868686868686869 0.101133695761628\\
0.0878787878787879 0.101012586041022\\
0.0888888888888889 0.100899918603699\\
0.0898989898989899 0.100794863213711\\
0.0909090909090909 0.100696622274483\\
0.0919191919191919 0.100604428057224\\
0.0929292929292929 0.100517539869144\\
0.0939393939393939 0.100435241169598\\
0.094949494949495 0.100356836641663\\
0.095959595959596 0.100281649226045\\
0.096969696969697 0.100209017123712\\
0.097979797979798 0.100138290773195\\
0.098989898989899 0.100068829808089\\
0.1 0.1\\
};
\node[right, inner sep=0mm, text=black]
at (axis cs:0.025,0.188648732957058,0) {T =77800 s};
\end{axis}
\end{tikzpicture}% 
\caption{Infiltrazione nella colonna d'acqua diagrammata per vari tempi}
\label{fig:easy_infiltrazione_colonna}
\end{figure}


Da notare la somiglianza delle curve d'infiltrazione con quelle che descrivono l'andamento della temperatura nel tempo in una sbarra rigida innestata tra due 
capi mantenuti a temperature $T_{1}$ e $T_{2}$ costanti nel tempo.

\newpage
%\begin{figure}[H]
%% This file was created by matlab2tikz v0.4.4 running on MATLAB 8.2.
% Copyright (c) 2008--2013, Nico Schlömer <nico.schloemer@gmail.com>
% All rights reserved.
% 
% The latest updates can be retrieved from
%   http://www.mathworks.com/matlabcentral/fileexchange/22022-matlab2tikz
% where you can also make suggestions and rate matlab2tikz.
% 
\begin{tikzpicture}

\begin{axis}[%
width=4.40888888888889in,
height=3.41111111111111in,
scale only axis,
xmin=0,
xmax=90000,
xlabel={Tempo (s)},
ymin=0,
ymax=0.006,
ylabel={Infiltrazione cumulata (m)},
%title={Infiltrazione cumulata (m) VS tempo (s) sulla cima della colonna di suolo},
axis x line*=bottom,
axis y line*=left
]
\addplot [
color=red,
solid,
forget plot
]
table[row sep=crcr]{
0 0\\
8.64086408640864 1.08864e-05\\
17.2817281728173 2.11802314752e-05\\
25.9225922592259 3.09460038054838e-05\\
34.5634563456346 4.02394479343207e-05\\
43.2043204320432 4.91088542867479e-05\\
51.8451845184518 5.75961922106982e-05\\
60.4860486048605 6.57380483169114e-05\\
69.1269126912691 7.35664144808376e-05\\
77.7677767776778 8.11093508577639e-05\\
86.4086408640864 8.83915448278477e-05\\
95.049504950495 9.54347831505561e-05\\
103.690369036904 0.000102258351621715\\
112.331233123312 0.000108879374071852\\
120.972097209721 0.000115313100524883\\
129.61296129613 0.000121573152672461\\
138.253825382538 0.000127671733447235\\
146.894689468947 0.000133619806345323\\
155.535553555356 0.00013942724921167\\
164.176417641764 0.000145102986426678\\
172.817281728173 0.000150655102789921\\
181.458145814581 0.000156090941863458\\
190.09900990099 0.000161417191094068\\
198.739873987399 0.000166639955664941\\
207.380738073807 0.000171764822719957\\
216.021602160216 0.000176796917347208\\
224.662466246625 0.000181740951493989\\
233.303330333033 0.000186601266806036\\
241.944194419442 0.000191381872233314\\
250.585058505851 0.000196086477118329\\
259.225922592259 0.000200718520376685\\
267.866786678668 0.000205281196290115\\
276.507650765077 0.000209777477356665\\
285.148514851485 0.000214210134578929\\
293.789378937894 0.000218581755517157\\
302.430243024302 0.000222894760388262\\
311.071107110711 0.000227151416452802\\
319.71197119712 0.000231353850898905\\
328.352835283528 0.000235504062403854\\
336.993699369937 0.000239603931529924\\
345.634563456346 0.00024365523009044\\
354.275427542754 0.000247659629604339\\
362.916291629163 0.000251618708942321\\
371.557155715572 0.000255533961254631\\
380.19801980198 0.00025940680025924\\
388.838883888389 0.00026323856595949\\
397.479747974797 0.00026703052985185\\
406.120612061206 0.000270783899677169\\
414.761476147615 0.000274499823762462\\
423.402340234023 0.000278179394994799\\
432.043204320432 0.000281823654464054\\
440.684068406841 0.000285433594807118\\
449.324932493249 0.000289010163282495\\
457.965796579658 0.000292554264601038\\
466.606660666067 0.000296066763535751\\
475.247524752475 0.000299548487331114\\
483.888388838884 0.000303000227930246\\
492.529252925293 0.00030642274403627\\
501.170117011701 0.000309816763022587\\
509.81098109811 0.000313182982705241\\
518.451845184518 0.000316522072989258\\
527.092709270927 0.000319834677399642\\
535.733573357336 0.000323121414506689\\
544.374437443744 0.000326382879254332\\
553.015301530153 0.000329619644199401\\
561.656165616562 0.000332832260668957\\
570.29702970297 0.000336021259842171\\
578.937893789379 0.000339187153762658\\
587.578757875788 0.000342330436286598\\
596.219621962196 0.000345451583971563\\
604.860486048605 0.000348551056910456\\
613.501350135013 0.000351629299514671\\
622.142214221422 0.000354686741250154\\
630.783078307831 0.000357723797329785\\
639.423942394239 0.000360740869365188\\
648.064806480648 0.000363738345980831\\
656.705670567057 0.000366716603393032\\
665.346534653465 0.00036967600595629\\
673.987398739874 0.000372616906679159\\
682.628262826283 0.000375539647711698\\
691.269126912691 0.000378444560806398\\
699.9099909991 0.000381331967754311\\
708.550855085509 0.000384202180797997\\
717.191719171917 0.000387055503022761\\
725.832583258326 0.000389892228727572\\
734.473447344734 0.000392712643776922\\
743.114311431143 0.00039551702593481\\
751.755175517552 0.000398305645181949\\
760.39603960396 0.000401078764017208\\
769.036903690369 0.000403836637744237\\
777.677767776778 0.000406579514744152\\
786.318631863186 0.000409307636735098\\
794.959495949595 0.000412021239019454\\
803.600360036004 0.000414720550719388\\
812.241224122412 0.000417405795001428\\
820.882088208821 0.000420077189290659\\
829.52295229523 0.000422734945475142\\
838.163816381638 0.000425379270101072\\
846.804680468047 0.0004280103645592\\
855.445544554455 0.000430628425262981\\
864.086408640864 0.000433233643818898\\
872.727272727273 0.000435826207189376\\
881.368136813681 0.000438406297848664\\
890.00900090009 0.000440974093932078\\
898.649864986499 0.000443529769378921\\
907.290729072907 0.000446073494069412\\
915.931593159316 0.000448605433955937\\
924.572457245725 0.000451125751188899\\
933.213321332133 0.000453634604237426\\
941.854185418542 0.000456132148005218\\
950.495049504951 0.000458618533941737\\
959.135913591359 0.000461093910148995\\
967.776777677768 0.000463558421484132\\
976.417641764176 0.000466012209657994\\
985.058505850585 0.000468455413329897\\
993.699369936994 0.000470888168198759\\
1002.3402340234 0.000473310607090757\\
1010.98109810981 0.00047572286004369\\
1019.62196219622 0.000478125054388177\\
1028.26282628263 0.000480517314825844\\
1036.90369036904 0.00048289976350464\\
1045.54455445545 0.000485272520091395\\
1054.18541854185 0.000487635701841754\\
1062.82628262826 0.000489989423667607\\
1071.46714671467 0.000492333798202098\\
1080.10801080108 0.000494668935862357\\
1088.74887488749 0.000496994944910012\\
1097.3897389739 0.000499311931509604\\
1106.03060306031 0.000501619999784987\\
1114.67146714671 0.00050391925187378\\
1123.31233123312 0.000506209787979984\\
1131.95319531953 0.000508491706424809\\
1140.59405940594 0.000510765103695802\\
1149.23492349235 0.000513030074494338\\
1157.87578757876 0.000515286711781546\\
1166.51665166517 0.000517535106822726\\
1175.15751575158 0.000519775349230316\\
1183.79837983798 0.000522007527005483\\
1192.43924392439 0.000524231726578362\\
1201.0801080108 0.000526448032847025\\
1209.72097209721 0.000528656529215214\\
1218.36183618362 0.000530857297628884\\
1227.00270027003 0.000533050418611612\\
1235.64356435644 0.000535235971298903\\
1244.28442844284 0.000537414033471443\\
1252.92529252925 0.000539584681587336\\
1261.56615661566 0.000541747990813357\\
1270.20702070207 0.00054390403505527\\
1278.84788478848 0.00054605288698723\\
1287.48874887489 0.000548194618080315\\
1296.1296129613 0.000550329298630211\\
1304.7704770477 0.000552456997784088\\
1313.41134113411 0.000554577783566684\\
1322.05220522052 0.000556691722905638\\
1330.69306930693 0.000558798881656095\\
1339.33393339334 0.000560899324624602\\
1347.97479747975 0.00056299311559232\\
1356.61566156616 0.000565080317337595\\
1365.25652565257 0.000567160991657876\\
1373.89738973897 0.000569235199391033\\
1382.53825382538 0.000571303000436082\\
1391.17911791179 0.000573364453773334\\
1399.8199819982 0.000575419617483993\\
1408.46084608461 0.000577468548769224\\
1417.10171017102 0.000579511303968699\\
1425.74257425743 0.000581547938578647\\
1434.38343834383 0.000583578507269421\\
1443.02430243024 0.000585603063902592\\
1451.66516651665 0.000587621661547601\\
1460.30603060306 0.000589634352497964\\
1468.94689468947 0.000591641188287059\\
1477.58775877588 0.000593642219703496\\
1486.22862286229 0.0005956374968061\\
1494.86948694869 0.000597627068938499\\
1503.5103510351 0.000599610984743343\\
1512.15121512151 0.000601589292176164\\
1520.79207920792 0.000603562038518886\\
1529.43294329433 0.000605529270392985\\
1538.07380738074 0.000607491033772339\\
1546.71467146715 0.000609447373995741\\
1555.35553555356 0.000611398335779113\\
1563.99639963996 0.000613343963227414\\
1572.63726372637 0.000615284299846264\\
1581.27812781278 0.000617219388553279\\
1589.91899189919 0.000619149271689132\\
1598.5598559856 0.000621073991028354\\
1607.20072007201 0.000622993587789868\\
1615.84158415842 0.000624908102647282\\
1624.48244824482 0.000626817575738927\\
1633.12331233123 0.00062872204667767\\
1641.76417641764 0.000630621554560491\\
1650.40504050405 0.000632516137977835\\
1659.04590459046 0.00063440583502275\\
1667.68676867687 0.000636290683299818\\
1676.32763276328 0.000638170719933874\\
1684.96849684968 0.000640045981578526\\
1693.60936093609 0.000641916504424492\\
1702.2502250225 0.000643782324207733\\
1710.89108910891 0.000645643476217416\\
1719.53195319532 0.000647499995303691\\
1728.17281728173 0.0006493519158853\\
1736.81368136814 0.000651199271957013\\
1745.45454545455 0.000653042097096902\\
1754.09540954095 0.000654880424473454\\
1762.73627362736 0.000656714286852535\\
1771.37713771377 0.000658543716604192\\
1780.01800180018 0.000660368745709315\\
1788.65886588659 0.000662189405766157\\
1797.299729973 0.000664005727996707\\
1805.94059405941 0.000665817743252934\\
1814.58145814581 0.000667625482022896\\
1823.22232223222 0.000669428974436719\\
1831.86318631863 0.000671228250272452\\
1840.50405040504 0.000673023338961797\\
1849.14491449145 0.000674814269595725\\
1857.78577857786 0.000676601070929968\\
1866.42664266427 0.000678383771390403\\
1875.06750675067 0.000680162399078325\\
1883.70837083708 0.00068193698177561\\
1892.34923492349 0.000683707546949774\\
1900.9900990099 0.000685474121758931\\
1909.63096309631 0.000687236733056647\\
1918.27182718272 0.000688995407396699\\
1926.91269126913 0.000690750171037742\\
1935.55355535554 0.000692501049947876\\
1944.19441944194 0.000694248069809131\\
1952.83528352835 0.000695991256021852\\
1961.47614761476 0.00069773063370901\\
1970.11701170117 0.000699466227720424\\
1978.75787578758 0.000701198062636895\\
1987.39873987399 0.000702926162774268\\
1996.0396039604 0.000704650552187415\\
2004.6804680468 0.000706371254674133\\
2013.32133213321 0.000708088293778978\\
2021.96219621962 0.000709801692797019\\
2030.60306030603 0.000711511474777524\\
2039.24392439244 0.000713217662527574\\
2047.88478847885 0.00071492027861561\\
2056.52565256526 0.000716619345374917\\
2065.16651665167 0.000718314884907032\\
2073.80738073807 0.000720006919085107\\
2082.44824482448 0.000721695469557187\\
2091.08910891089 0.00072338055774945\\
2099.7299729973 0.00072506220486937\\
2108.37083708371 0.000726740431908833\\
2117.01170117012 0.000728415259647189\\
2125.65256525653 0.000730086708654256\\
2134.29342934293 0.000731754799293261\\
2142.93429342934 0.000733419551723738\\
2151.57515751575 0.000735080985904363\\
2160.21602160216 0.00073673912159575\\
2168.85688568857 0.000738393978363186\\
2177.49774977498 0.000740045575579328\\
2186.13861386139 0.000741693932426842\\
2194.77947794779 0.000743339067901007\\
2203.4203420342 0.000744981000812263\\
2212.06120612061 0.000746619749788718\\
2220.70207020702 0.000748255333278618\\
2229.34293429343 0.000749887769552763\\
2237.98379837984 0.00075151707670689\\
2246.62466246625 0.000753143272664012\\
2255.26552655266 0.000754766375176715\\
2263.90639063906 0.000756386401829419\\
2272.54725472547 0.0007580033700406\\
2281.18811881188 0.000759617297064975\\
2289.82898289829 0.000761228199995645\\
2298.4698469847 0.00076283609576621\\
2307.11071107111 0.000764441001152843\\
2315.75157515752 0.000766042932776332\\
2324.39243924392 0.000767641907104086\\
2333.03330333033 0.000769237940452108\\
2341.67416741674 0.000770831048986942\\
2350.31503150315 0.000772421248727576\\
2358.95589558956 0.000774008555547322\\
2367.59675967597 0.000775592985175668\\
2376.23762376238 0.000777174553200089\\
2384.87848784878 0.000778753275067842\\
2393.51935193519 0.000780329166087718\\
2402.1602160216 0.000781902241431781\\
2410.80108010801 0.000783472516137065\\
2419.44194419442 0.000785040005107255\\
2428.08280828083 0.000786604723114332\\
2436.72367236724 0.000788166684800201\\
2445.36453645365 0.000789725904678285\\
2454.00540054005 0.0007912823971351\\
2462.64626462646 0.000792836176431802\\
2471.28712871287 0.000794387256705712\\
2479.92799279928 0.000795935651971814\\
2488.56885688569 0.000797481376124233\\
2497.2097209721 0.000799024442937688\\
2505.85058505851 0.000800564866068928\\
2514.49144914491 0.000802102659058133\\
2523.13231323132 0.000803637835330311\\
2531.77317731773 0.000805170408196658\\
2540.41404140414 0.00080670039085591\\
2549.05490549055 0.000808227796395661\\
2557.69576957696 0.000809752637793676\\
2566.33663366337 0.000811274927919173\\
2574.97749774978 0.000812794679534092\\
2583.61836183618 0.000814311905294339\\
2592.25922592259 0.000815826617751019\\
2600.900090009 0.000817338829351646\\
2609.54095409541 0.000818848552441332\\
2618.18181818182 0.00082035579926397\\
2626.82268226823 0.000821860581963384\\
2635.46354635464 0.000823362912584476\\
2644.10441044104 0.000824862803074347\\
2652.74527452745 0.000826360265283407\\
2661.38613861386 0.000827855310966465\\
2670.02700270027 0.000829347951783807\\
2678.66786678668 0.000830838199302257\\
2687.30873087309 0.000832326064996219\\
2695.9495949595 0.000833811560248711\\
2704.5904590459 0.000835294696352379\\
2713.23132313231 0.000836775484510498\\
2721.87218721872 0.00083825393583796\\
2730.51305130513 0.000839730061362247\\
2739.15391539154 0.000841203872024388\\
2747.79477947795 0.000842675378679908\\
2756.43564356436 0.000844144592099759\\
2765.07650765077 0.000845611522971239\\
2773.71737173717 0.000847076181898899\\
2782.35823582358 0.000848538579405438\\
2790.99909990999 0.000849998725932584\\
2799.6399639964 0.000851456631841963\\
2808.28082808281 0.000852912307415959\\
2816.92169216922 0.000854365762858554\\
2825.56255625563 0.000855817008296171\\
2834.20342034203 0.000857266053778488\\
2842.84428442844 0.000858712909279256\\
2851.48514851485 0.000860157584697097\\
2860.12601260126 0.000861600089856294\\
2868.76687668767 0.000863040434507569\\
2877.40774077408 0.000864478628328855\\
2886.04860486049 0.000865914680926052\\
2894.68946894689 0.000867348601833775\\
2903.3303330333 0.000868780400516095\\
2911.97119711971 0.000870210086367263\\
2920.61206120612 0.000871637668712435\\
2929.25292529253 0.000873063156808375\\
2937.89378937894 0.000874486559844159\\
2946.53465346535 0.000875907886941866\\
2955.17551755176 0.000877327147157256\\
2963.81638163816 0.00087874434948045\\
2972.45724572457 0.000880159502836585\\
2981.09810981098 0.000881572616086479\\
2989.73897389739 0.000882983698027271\\
2998.3798379838 0.000884392757393062\\
3007.02070207021 0.000885799802855547\\
3015.66156615662 0.000887204843024637\\
3024.30243024302 0.00088860788644907\\
3032.94329432943 0.000890008941617025\\
3041.58415841584 0.000891408016956711\\
3050.22502250225 0.000892805120836968\\
3058.86588658866 0.000894200261567845\\
3067.50675067507 0.000895593447401178\\
3076.14761476148 0.000896984686531159\\
3084.78847884788 0.000898373987094901\\
3093.42934293429 0.000899761357172988\\
3102.0702070207 0.000901146804790027\\
3110.71107110711 0.000902530337915187\\
3119.35193519352 0.000903911964462738\\
3127.99279927993 0.000905291692292571\\
3136.63366336634 0.000906669529210729\\
3145.27452745275 0.000908045482969915\\
3153.91539153915 0.000909419561270005\\
3162.55625562556 0.000910791771758547\\
3171.19711971197 0.00091216212203126\\
3179.83798379838 0.000913530619632525\\
3188.47884788479 0.000914897272055867\\
3197.1197119712 0.000916262086744435\\
3205.76057605761 0.000917625071091476\\
3214.40144014401 0.000918986232440798\\
3223.04230423042 0.000920345578087236\\
3231.68316831683 0.000921703115277107\\
3240.32403240324 0.000923058851208659\\
3248.96489648965 0.000924412793032518\\
3257.60576057606 0.000925764947852124\\
3266.24662466247 0.000927115322724172\\
3274.88748874888 0.000928463924659037\\
3283.52835283528 0.000929810760621199\\
3292.16921692169 0.000931155837529665\\
3300.8100810081 0.000932499162258381\\
3309.45094509451 0.00093384074163664\\
3318.09180918092 0.000935180582449494\\
3326.73267326733 0.000936518691438146\\
3335.37353735374 0.000937855075300353\\
3344.01440144014 0.000939189740690809\\
3352.65526552655 0.00094052269422154\\
3361.29612961296 0.00094185394246228\\
3369.93699369937 0.000943183491940851\\
3378.57785778578 0.000944511349143536\\
3387.21872187219 0.000945837520515449\\
3395.8595859586 0.000947162012460898\\
3404.500450045 0.000948484831343747\\
3413.14131413141 0.000949805983487773\\
3421.78217821782 0.000951125475177018\\
3430.42304230423 0.000952443312656137\\
3439.06390639064 0.000953759502130745\\
3447.70477047705 0.000955074049767754\\
3456.34563456346 0.000956386961695715\\
3464.98649864987 0.000957698244005146\\
3473.62736273627 0.000959007902748866\\
3482.26822682268 0.000960315943942319\\
3490.90909090909 0.000961622373563894\\
3499.5499549955 0.00096292719755525\\
3508.19081908191 0.000964230421821624\\
3516.83168316832 0.000965532052232147\\
3525.47254725473 0.000966832094620152\\
3534.11341134113 0.000968130554783478\\
3542.75427542754 0.00096942743848477\\
3551.39513951395 0.000970722751451782\\
3560.03600360036 0.000972016499377668\\
3568.67686768677 0.000973308687921274\\
3577.31773177318 0.00097459932270743\\
3585.95859585959 0.000975888409327231\\
3594.59945994599 0.000977175953338324\\
3603.2403240324 0.000978461960265185\\
3611.88118811881 0.000979746435599394\\
3620.52205220522 0.000981029384799914\\
3629.16291629163 0.000982310813293355\\
3637.80378037804 0.000983590726474247\\
3646.44464446445 0.0009848691297053\\
3655.08550855086 0.000986146028317672\\
3663.72637263726 0.000987421427611223\\
3672.36723672367 0.000988695332854774\\
3681.00810081008 0.000989967749286361\\
3689.64896489649 0.000991238682113487\\
3698.2898289829 0.000992508136513367\\
3706.93069306931 0.000993776117633178\\
3715.57155715572 0.000995042630590302\\
3724.21242124212 0.000996307680472563\\
3732.85328532853 0.000997571272338469\\
3741.49414941494 0.000998833411217449\\
3750.13501350135 0.00100009410211008\\
3758.77587758776 0.00100135334998834\\
3767.41674167417 0.00100261115979579\\
3776.05760576058 0.00100386753644784\\
3784.69846984698 0.00100512248483199\\
3793.33933393339 0.00100637600980799\\
3801.9801980198 0.00100762811620811\\
3810.62106210621 0.00100887880883734\\
3819.26192619262 0.00101012809247361\\
3827.90279027903 0.00101137597186798\\
3836.54365436544 0.0010126224517449\\
3845.18451845185 0.00101386753680238\\
3853.82538253825 0.00101511123171218\\
3862.46624662466 0.00101635354112007\\
3871.10711071107 0.00101759446964599\\
3879.74797479748 0.00101883402188427\\
3888.38883888389 0.00102007220240381\\
3897.0297029703 0.0010213090157483\\
3905.67056705671 0.0010225444664364\\
3914.31143114311 0.00102377855896191\\
3922.95229522952 0.00102501129779403\\
3931.59315931593 0.00102624268737748\\
3940.23402340234 0.00102747273213269\\
3948.87488748875 0.00102870143645605\\
3957.51575157516 0.00102992880472001\\
3966.15661566157 0.00103115484127333\\
3974.79747974797 0.00103237955044122\\
3983.43834383438 0.00103360293652552\\
3992.07920792079 0.00103482500380489\\
4000.7200720072 0.001036045756535\\
4009.36093609361 0.00103726519894865\\
4018.00180018002 0.00103848333525599\\
4026.64266426643 0.00103970016964468\\
4035.28352835284 0.00104091570628004\\
4043.92439243924 0.00104212994930522\\
4052.56525652565 0.0010433429028414\\
4061.20612061206 0.00104455457098789\\
4069.84698469847 0.00104576495782236\\
4078.48784878488 0.00104697406740095\\
4087.12871287129 0.00104818190375843\\
4095.7695769577 0.00104938847090841\\
4104.4104410441 0.00105059377284343\\
4113.05130513051 0.00105179781353516\\
4121.69216921692 0.00105300059693451\\
4130.33303330333 0.00105420212697183\\
4138.97389738974 0.00105540240755703\\
4147.61476147615 0.00105660144257973\\
4156.25562556256 0.00105779923590942\\
4164.89648964897 0.00105899579139558\\
4173.53735373537 0.00106019111286786\\
4182.17821782178 0.00106138520413619\\
4190.81908190819 0.00106257806899095\\
4199.4599459946 0.00106376971120309\\
4208.10081008101 0.00106496013452427\\
4216.74167416742 0.00106614934268701\\
4225.38253825383 0.00106733733940483\\
4234.02340234023 0.00106852412837234\\
4242.66426642664 0.00106970971326545\\
4251.30513051305 0.00107089409774144\\
4259.94599459946 0.00107207728543912\\
4268.58685868587 0.00107325927997895\\
4277.22772277228 0.00107444008496317\\
4285.86858685869 0.00107561970397594\\
4294.50945094509 0.00107679814058346\\
4303.1503150315 0.00107797539833407\\
4311.79117911791 0.00107915148075844\\
4320.43204320432 0.00108032639136962\\
4329.07290729073 0.0010815001336632\\
4337.71377137714 0.00108267271111744\\
4346.35463546355 0.00108384412719334\\
4354.99549954995 0.00108501438533485\\
4363.63636363636 0.00108618348896889\\
4372.27722772277 0.00108735144150552\\
4380.91809180918 0.00108851824633806\\
4389.55895589559 0.00108968390684317\\
4398.199819982 0.00109084842638101\\
4406.84068406841 0.00109201180829531\\
4415.48154815482 0.00109317405591351\\
4424.12241224122 0.00109433517254685\\
4432.76327632763 0.00109549516149051\\
4441.40414041404 0.00109665402602367\\
4450.04500450045 0.00109781176940969\\
4458.68586858686 0.00109896839489614\\
4467.32673267327 0.00110012390571497\\
4475.96759675968 0.00110127830508255\\
4484.60846084608 0.00110243159619986\\
4493.24932493249 0.00110358378225251\\
4501.8901890189 0.00110473486641089\\
4510.53105310531 0.00110588485183026\\
4519.17191719172 0.00110703374165085\\
4527.81278127813 0.00110818153899795\\
4536.45364536454 0.00110932824698205\\
4545.09450945095 0.00111047386869888\\
4553.73537353735 0.00111161840722954\\
4562.37623762376 0.0011127618656406\\
4571.01710171017 0.00111390424698419\\
4579.65796579658 0.00111504555429808\\
4588.29882988299 0.00111618579060581\\
4596.9396939694 0.00111732495891673\\
4605.58055805581 0.00111846306222615\\
4614.22142214221 0.0011196001035154\\
4622.86228622862 0.00112073608575192\\
4631.50315031503 0.00112187101188935\\
4640.14401440144 0.00112300488486766\\
4648.78487848785 0.00112413770761316\\
4657.42574257426 0.00112526948303866\\
4666.06660666067 0.00112640021404354\\
4674.70747074707 0.0011275299035138\\
4683.34833483348 0.00112865855432219\\
4691.98919891989 0.00112978616932828\\
4700.6300630063 0.00113091275137853\\
4709.27092709271 0.00113203830330641\\
4717.91179117912 0.00113316282793244\\
4726.55265526553 0.0011342863280643\\
4735.19351935193 0.0011354088064969\\
4743.83438343834 0.00113653026601247\\
4752.47524752475 0.00113765070938063\\
4761.11611161116 0.0011387701393585\\
4769.75697569757 0.00113988855869071\\
4778.39783978398 0.00114100597010956\\
4787.03870387039 0.00114212237633506\\
4795.6795679568 0.00114323778007498\\
4804.3204320432 0.00114435218402499\\
4812.96129612961 0.00114546559086867\\
4821.60216021602 0.00114657800327763\\
4830.24302430243 0.00114768942391159\\
4838.88388838884 0.00114879985541841\\
4847.52475247525 0.00114990930043419\\
4856.16561656166 0.00115101776158335\\
4864.80648064807 0.00115212524147869\\
4873.44734473447 0.00115323174272148\\
4882.08820882088 0.00115433726790151\\
4890.72907290729 0.00115544181959714\\
4899.3699369937 0.00115654540037545\\
4908.01080108011 0.00115764801279223\\
4916.65166516652 0.00115874965939206\\
4925.29252925293 0.00115985034270843\\
4933.93339333933 0.00116095006526374\\
4942.57425742574 0.00116204882956945\\
4951.21512151215 0.00116314663812603\\
4959.85598559856 0.00116424349342316\\
4968.49684968497 0.00116533939793969\\
4977.13771377138 0.00116643435414375\\
4985.77857785779 0.00116752836449282\\
4994.41944194419 0.00116862143143379\\
5003.0603060306 0.001169713557403\\
5011.70117011701 0.00117080474482634\\
5020.34203420342 0.00117189499611928\\
5028.98289828983 0.00117298431368695\\
5037.62376237624 0.00117407269992421\\
5046.26462646265 0.0011751601572157\\
5054.90549054905 0.00117624668793587\\
5063.54635463546 0.00117733229444912\\
5072.18721872187 0.00117841697910978\\
5080.82808280828 0.0011795007442622\\
5089.46894689469 0.00118058359224083\\
5098.1098109811 0.00118166552537025\\
5106.75067506751 0.00118274654596523\\
5115.39153915392 0.0011838266563308\\
5124.03240324032 0.00118490585876231\\
5132.67326732673 0.00118598415554547\\
5141.31413141314 0.00118706154895643\\
5149.95499549955 0.00118813804126179\\
5158.59585958596 0.00118921363471873\\
5167.23672367237 0.00119028833157499\\
5175.87758775878 0.00119136213406897\\
5184.51845184518 0.00119243504442977\\
5193.15931593159 0.00119350706487725\\
5201.800180018 0.00119457819762207\\
5210.44104410441 0.00119564844486577\\
5219.08190819082 0.00119671780880079\\
5227.72277227723 0.00119778629161055\\
5236.36363636364 0.00119885389546946\\
5245.00450045005 0.00119992062254304\\
5253.64536453645 0.00120098647498792\\
5262.28622862286 0.00120205145495189\\
5270.92709270927 0.00120311556457398\\
5279.56795679568 0.00120417880598448\\
5288.20882088209 0.00120524118130503\\
5296.8496849685 0.00120630269264861\\
5305.49054905491 0.00120736334211964\\
5314.13141314131 0.00120842313181402\\
5322.77227722772 0.00120948206381915\\
5331.41314131413 0.001210540140214\\
5340.05400540054 0.00121159736306917\\
5348.69486948695 0.00121265373444691\\
5357.33573357336 0.00121370925640119\\
5365.97659765977 0.00121476393097771\\
5374.61746174617 0.001215817760214\\
5383.25832583258 0.00121687074613944\\
5391.89918991899 0.00121792289077527\\
5400.5400540054 0.00121897419613469\\
5409.18091809181 0.0012200246642229\\
5417.82178217822 0.00122107429703711\\
5426.46264626463 0.00122212309656659\\
5435.10351035103 0.00122317106479275\\
5443.74437443744 0.00122421820368916\\
5452.38523852385 0.00122526451522156\\
5461.02610261026 0.00122631000134799\\
5469.66696669667 0.00122735466401873\\
5478.30783078308 0.00122839850517642\\
5486.94869486949 0.00122944152675607\\
5495.5895589559 0.0012304837306851\\
5504.2304230423 0.00123152511888339\\
5512.87128712871 0.00123256569326332\\
5521.51215121512 0.0012336054557298\\
5530.15301530153 0.00123464440818034\\
5538.79387938794 0.00123568255250505\\
5547.43474347435 0.00123671989058672\\
5556.07560756076 0.00123775642430081\\
5564.71647164716 0.00123879215551557\\
5573.35733573357 0.00123982708609197\\
5581.99819981998 0.00124086121788385\\
5590.63906390639 0.00124189455273788\\
5599.2799279928 0.00124292709249363\\
5607.92079207921 0.00124395883898361\\
5616.56165616562 0.00124498979403329\\
5625.20252025203 0.00124601995946115\\
5633.84338433843 0.00124704933707874\\
5642.48424842484 0.00124807792869067\\
5651.12511251125 0.00124910573609469\\
5659.76597659766 0.00125013276108167\\
5668.40684068407 0.00125115900543572\\
5677.04770477048 0.00125218447093416\\
5685.68856885689 0.00125320915934758\\
5694.32943294329 0.00125423307243985\\
5702.9702970297 0.00125525621196821\\
5711.61116111611 0.00125627857968326\\
5720.25202520252 0.00125730017732899\\
5728.89288928893 0.00125832100664287\\
5737.53375337534 0.0012593410693558\\
5746.17461746175 0.00126036036719224\\
5754.81548154815 0.00126137890187014\\
5763.45634563456 0.00126239667510108\\
5772.09720972097 0.00126341368859023\\
5780.73807380738 0.00126442994403639\\
5789.37893789379 0.00126544544313207\\
5798.0198019802 0.00126646018756346\\
5806.66066606661 0.00126747417901052\\
5815.30153015301 0.00126848741914697\\
5823.94239423942 0.00126949990964035\\
5832.58325832583 0.00127051165215203\\
5841.22412241224 0.00127152264833726\\
5849.86498649865 0.00127253289984519\\
5858.50585058506 0.00127354240831891\\
5867.14671467147 0.00127455117539547\\
5875.78757875788 0.00127555920270593\\
5884.42844284428 0.00127656649187536\\
5893.06930693069 0.00127757304452291\\
5901.7101710171 0.0012785788622618\\
5910.35103510351 0.00127958394669938\\
5918.99189918992 0.00128058829943715\\
5927.63276327633 0.00128159192207078\\
5936.27362736274 0.00128259481619017\\
5944.91449144915 0.00128359698337943\\
5953.55535553555 0.00128459842521695\\
5962.19621962196 0.00128559914327543\\
5970.83708370837 0.00128659913912187\\
5979.47794779478 0.00128759841431764\\
5988.11881188119 0.0012885969704185\\
5996.7596759676 0.00128959480897459\\
6005.40054005401 0.00129059193153053\\
6014.04140414041 0.00129158833962536\\
6022.68226822682 0.00129258403479266\\
6031.32313231323 0.00129357901856051\\
6039.96399639964 0.00129457329245152\\
6048.60486048605 0.00129556685798291\\
6057.24572457246 0.00129655971666649\\
6065.88658865887 0.00129755187000869\\
6074.52745274527 0.00129854331951062\\
6083.16831683168 0.00129953406666805\\
6091.80918091809 0.00130052411297146\\
6100.4500450045 0.0013015134599061\\
6109.09090909091 0.00130250210895193\\
6117.73177317732 0.00130349006158374\\
6126.37263726373 0.00130447731927112\\
6135.01350135013 0.00130546388347848\\
6143.65436543654 0.00130644975566512\\
6152.29522952295 0.00130743493728523\\
6160.93609360936 0.00130841942978788\\
6169.57695769577 0.00130940323461712\\
6178.21782178218 0.00131038635321194\\
6186.85868586859 0.00131136878700633\\
6195.499549955 0.00131235053742929\\
6204.1404140414 0.00131333160590485\\
6212.78127812781 0.0013143119938521\\
6221.42214221422 0.00131529170268524\\
6230.06300630063 0.00131627073381354\\
6238.70387038704 0.00131724908864143\\
6247.34473447345 0.0013182267685685\\
6255.98559855986 0.0013192037749895\\
6264.62646264626 0.00132018010929439\\
6273.26732673267 0.00132115577286835\\
6281.90819081908 0.00132213076709183\\
6290.54905490549 0.00132310509334054\\
6299.1899189919 0.00132407875298546\\
6307.83078307831 0.00132505174739292\\
6316.47164716472 0.00132602407792457\\
6325.11251125113 0.00132699574593744\\
6333.75337533753 0.00132796675278393\\
6342.39423942394 0.00132893709981183\\
6351.03510351035 0.00132990678836439\\
6359.67596759676 0.00133087581978028\\
6368.31683168317 0.00133184419539367\\
6376.95769576958 0.00133281191653418\\
6385.59855985599 0.00133377898452698\\
6394.23942394239 0.00133474540069276\\
6402.8802880288 0.00133571116634777\\
6411.52115211521 0.00133667628280383\\
6420.16201620162 0.00133764075136836\\
6428.80288028803 0.00133860457334439\\
6437.44374437444 0.00133956775003061\\
6446.08460846085 0.00134053028272134\\
6454.72547254725 0.0013414921727066\\
6463.36633663366 0.0013424534212721\\
6472.00720072007 0.00134341402969927\\
6480.64806480648 0.00134437399926528\\
6489.28892889289 0.00134533333124306\\
6497.9297929793 0.00134629202690132\\
6506.57065706571 0.00134725008750455\\
6515.21152115211 0.0013482075143131\\
6523.85238523852 0.0013491643085831\\
6532.49324932493 0.00135012047156659\\
6541.13411341134 0.00135107600451145\\
6549.77497749775 0.00135203090866146\\
6558.41584158416 0.00135298518525632\\
6567.05670567057 0.00135393883553167\\
6575.69756975698 0.00135489186071909\\
6584.33843384338 0.00135584426204611\\
6592.97929792979 0.00135679604073629\\
6601.6201620162 0.00135774719800917\\
6610.26102610261 0.0013586977350803\\
6618.90189018902 0.00135964765316131\\
6627.54275427543 0.00136059695345985\\
6636.18361836184 0.00136154563717969\\
6644.82448244824 0.00136249370552066\\
6653.46534653465 0.00136344115967872\\
6662.10621062106 0.00136438800084595\\
6670.74707470747 0.0013653342302106\\
6679.38793879388 0.00136627984895707\\
6688.02880288029 0.00136722485826595\\
6696.6696669667 0.00136816925931402\\
6705.31053105311 0.0013691130532743\\
6713.95139513951 0.00137005624131601\\
6722.59225922592 0.00137099882460465\\
6731.23312331233 0.00137194080430199\\
6739.87398739874 0.00137288218156606\\
6748.51485148515 0.00137382295755122\\
6757.15571557156 0.00137476313340813\\
6765.79657965797 0.00137570271028379\\
6774.43744374437 0.00137664168932155\\
6783.07830783078 0.00137758007166113\\
6791.71917191719 0.00137851785843863\\
6800.3600360036 0.00137945505078656\\
6809.00090009001 0.00138039164983382\\
6817.64176417642 0.00138132765670577\\
6826.28262826283 0.00138226307252421\\
6834.92349234923 0.00138319789840738\\
6843.56435643564 0.00138413213547003\\
6852.20522052205 0.00138506578482339\\
6860.84608460846 0.00138599884757518\\
6869.48694869487 0.00138693132482969\\
6878.12781278128 0.00138786321768771\\
6886.76867686769 0.0013887945272466\\
6895.4095409541 0.00138972525460028\\
6904.0504050405 0.00139065540083928\\
6912.69126912691 0.0013915849670507\\
6921.33213321332 0.00139251395431827\\
6929.97299729973 0.00139344236372235\\
6938.61386138614 0.00139437019633994\\
6947.25472547255 0.0013952974532447\\
6955.89558955896 0.00139622413550697\\
6964.53645364536 0.00139715024419376\\
6973.17731773177 0.00139807578036881\\
6981.81818181818 0.00139900074509254\\
6990.45904590459 0.00139992513942213\\
6999.099909991 0.0014008489644115\\
7007.74077407741 0.00140177222111131\\
7016.38163816382 0.00140269491056901\\
7025.02250225023 0.00140361703382884\\
7033.66336633663 0.00140453859193182\\
7042.30423042304 0.0014054595859158\\
7050.94509450945 0.00140638001681546\\
7059.58595859586 0.0014072998856623\\
7068.22682268227 0.0014082191934847\\
7076.86768676868 0.00140913794130789\\
7085.50855085509 0.00141005613015399\\
7094.14941494149 0.00141097376104201\\
7102.7902790279 0.00141189083498788\\
7111.43114311431 0.00141280735300443\\
7120.07200720072 0.00141372331610143\\
7128.71287128713 0.00141463872528562\\
7137.35373537354 0.00141555358156066\\
7145.99459945995 0.00141646788592722\\
7154.63546354635 0.00141738163938294\\
7163.27632763276 0.00141829484292244\\
7171.91719171917 0.00141920749753738\\
7180.55805580558 0.00142011960421643\\
7189.19891989199 0.0014210311639453\\
7197.8397839784 0.00142194217770674\\
7206.48064806481 0.00142285264648057\\
7215.12151215122 0.00142376257124368\\
7223.76237623762 0.00142467195297003\\
7232.40324032403 0.00142558079263071\\
7241.04410441044 0.00142648909119388\\
7249.68496849685 0.00142739684962486\\
7258.32583258326 0.00142830406888608\\
7266.96669666967 0.00142921074993711\\
7275.60756075608 0.0014301168937347\\
7284.24842484248 0.00143102250123276\\
7292.88928892889 0.00143192757338235\\
7301.5301530153 0.00143283211113178\\
7310.17101710171 0.00143373611542651\\
7318.81188118812 0.00143463958720925\\
7327.45274527453 0.00143554252741993\\
7336.09360936094 0.0014364449369957\\
7344.73447344734 0.00143734681687097\\
7353.37533753375 0.00143824816797743\\
7362.01620162016 0.00143914899124401\\
7370.65706570657 0.00144004928759695\\
7379.29792979298 0.00144094905795977\\
7387.93879387939 0.00144184830325329\\
7396.5796579658 0.00144274702439567\\
7405.22052205221 0.00144364522230237\\
7413.86138613861 0.00144454289788621\\
7422.50225022502 0.00144544005205734\\
7431.14311431143 0.00144633668572328\\
7439.78397839784 0.00144723279978893\\
7448.42484248425 0.00144812839515655\\
7457.06570657066 0.00144902347272581\\
7465.70657065707 0.00144991803339377\\
7474.34743474347 0.00145081207805491\\
7482.98829882988 0.00145170560760113\\
7491.62916291629 0.00145259862292177\\
7500.2700270027 0.00145349112490361\\
7508.91089108911 0.00145438311443087\\
7517.55175517552 0.00145527459238526\\
7526.19261926193 0.00145616555964594\\
7534.83348334833 0.00145705601708958\\
7543.47434743474 0.00145794596559033\\
7552.11521152115 0.00145883540601983\\
7560.75607560756 0.00145972433924727\\
7569.39693969397 0.00146061276613934\\
7578.03780378038 0.00146150068756027\\
7586.67866786679 0.00146238810437183\\
7595.3195319532 0.00146327501743335\\
7603.9603960396 0.00146416142760172\\
7612.60126012601 0.00146504733573142\\
7621.24212421242 0.00146593274267448\\
7629.88298829883 0.00146681764928056\\
7638.52385238524 0.00146770205639688\\
7647.16471647165 0.00146858596486832\\
7655.80558055806 0.00146946937553735\\
7664.44644464446 0.00147035228924406\\
7673.08730873087 0.00147123470682621\\
7681.72817281728 0.0014721166291192\\
7690.36903690369 0.00147299805695608\\
7699.0099009901 0.00147387899116756\\
7707.65076507651 0.00147475943258205\\
7716.29162916292 0.00147563938202563\\
7724.93249324932 0.00147651884032207\\
7733.57335733573 0.00147739780829286\\
7742.21422142214 0.00147827628675719\\
7750.85508550855 0.00147915427653198\\
7759.49594959496 0.00148003177843187\\
7768.13681368137 0.00148090879326925\\
7776.77767776778 0.00148178532185425\\
7785.41854185419 0.00148266136499476\\
7794.05940594059 0.00148353692349643\\
7802.700270027 0.0014844119981627\\
7811.34113411341 0.00148528658979477\\
7819.98199819982 0.00148616069919165\\
7828.62286228623 0.00148703432715014\\
7837.26372637264 0.00148790747446485\\
7845.90459045905 0.00148878014192821\\
7854.54545454545 0.00148965233033046\\
7863.18631863186 0.00149052404045971\\
7871.82718271827 0.00149139527310186\\
7880.46804680468 0.00149226602904071\\
7889.10891089109 0.00149313630905788\\
7897.7497749775 0.00149400611393287\\
7906.39063906391 0.00149487544444307\\
7915.03150315032 0.00149574430136373\\
7923.67236723672 0.001496612685468\\
7932.31323132313 0.00149748059752692\\
7940.95409540954 0.00149834803830946\\
7949.59495949595 0.00149921500858248\\
7958.23582358236 0.00150008150911077\\
7966.87668766877 0.00150094754065706\\
7975.51755175518 0.00150181310398199\\
7984.15841584158 0.00150267819984419\\
7992.79927992799 0.00150354282900022\\
8001.4401440144 0.00150440699220458\\
8010.08100810081 0.00150527069020977\\
8018.72187218722 0.00150613392376627\\
8027.36273627363 0.00150699669362251\\
8036.00360036004 0.00150785900052495\\
8044.64446444644 0.00150872084521803\\
8053.28532853285 0.00150958222844419\\
8061.92619261926 0.00151044315094391\\
8070.56705670567 0.00151130361345567\\
8079.20792079208 0.00151216361671599\\
8087.84878487849 0.00151302316145941\\
8096.4896489649 0.00151388224841854\\
8105.13051305131 0.00151474087832402\\
8113.77137713771 0.00151559905190456\\
8122.41224122412 0.00151645676988692\\
8131.05310531053 0.00151731403299597\\
8139.69396939694 0.00151817084195461\\
8148.33483348335 0.00151902719748387\\
8156.97569756976 0.00151988310030284\\
8165.61656165617 0.00152073855112874\\
8174.25742574257 0.00152159355067688\\
8182.89828982898 0.0015224480996607\\
8191.53915391539 0.00152330219879174\\
8200.1800180018 0.00152415584877969\\
8208.82088208821 0.00152500905033237\\
8217.46174617462 0.00152586180415573\\
8226.10261026103 0.0015267141109539\\
8234.74347434744 0.00152756597142913\\
8243.38433843384 0.00152841738628185\\
8252.02520252025 0.00152926835621068\\
8260.66606660666 0.00153011888191238\\
8269.30693069307 0.00153096896408192\\
8277.94779477948 0.00153181860341244\\
8286.58865886589 0.00153266780059529\\
8295.2295229523 0.00153351655632002\\
8303.8703870387 0.0015343648712744\\
8312.51125112511 0.00153521274614439\\
8321.15211521152 0.0015360601816142\\
8329.79297929793 0.00153690717836624\\
8338.43384338434 0.00153775373708119\\
8347.07470747075 0.00153859985843795\\
8355.71557155716 0.00153944554311367\\
8364.35643564356 0.00154029079178375\\
8372.99729972997 0.00154113560512187\\
8381.63816381638 0.00154197998379996\\
8390.27902790279 0.00154282392848822\\
8398.9198919892 0.00154366743985514\\
8407.56075607561 0.0015445105185675\\
8416.20162016202 0.00154535316529036\\
8424.84248424843 0.00154619538068709\\
8433.48334833483 0.00154703716541934\\
8442.12421242124 0.0015478785201471\\
8450.76507650765 0.00154871944552865\\
8459.40594059406 0.00154955994222062\\
8468.04680468047 0.00155040001087794\\
8476.68766876688 0.00155123965215389\\
8485.32853285329 0.0015520788667001\\
8493.96939693969 0.00155291765516651\\
8502.6102610261 0.00155375601820145\\
8511.25112511251 0.00155459395645159\\
8519.89198919892 0.00155543147056196\\
8528.53285328533 0.00155626856117596\\
8537.17371737174 0.00155710522893538\\
8545.81458145815 0.00155794147448038\\
8554.45544554455 0.0015587772984495\\
8563.09630963096 0.00155961270147969\\
8571.73717371737 0.00156044768420627\\
8580.37803780378 0.00156128224726299\\
8589.01890189019 0.00156211639128199\\
8597.6597659766 0.00156295011689384\\
8606.30063006301 0.00156378342472753\\
8614.94149414941 0.00156461631541046\\
8623.58235823582 0.00156544878956848\\
8632.22322232223 0.00156628084782585\\
8640.86408640864 0.0015671124908053\\
8649.50495049505 0.001567943719128\\
8658.14581458146 0.00156877453341356\\
8666.78667866787 0.00156960493428007\\
8675.42754275428 0.00157043492234407\\
8684.06840684068 0.00157126449822056\\
8692.70927092709 0.00157209366252304\\
8701.3501350135 0.00157292241586346\\
8709.99099909991 0.00157375075885228\\
8718.63186318632 0.00157457869209844\\
8727.27272727273 0.00157540621620937\\
8735.91359135914 0.00157623333179102\\
8744.55445544554 0.00157706003944782\\
8753.19531953195 0.00157788633978272\\
8761.83618361836 0.0015787122333972\\
8770.47704770477 0.00157953772089125\\
8779.11791179118 0.00158036280286338\\
8787.75877587759 0.00158118747991064\\
8796.399639964 0.00158201175262862\\
8805.0405040504 0.00158283562161144\\
8813.68136813681 0.00158365908745178\\
8822.32223222322 0.00158448215074085\\
8830.96309630963 0.00158530481206844\\
8839.60396039604 0.00158612707202289\\
8848.24482448245 0.00158694893119111\\
8856.88568856886 0.00158777039015858\\
8865.52655265526 0.00158859144950934\\
8874.16741674167 0.00158941210982605\\
8882.80828082808 0.00159023237168991\\
8891.44914491449 0.00159105223568074\\
8900.0900090009 0.00159187170237694\\
8908.73087308731 0.00159269077235552\\
8917.37173717372 0.0015935094461921\\
8926.01260126013 0.00159432772446089\\
8934.65346534654 0.00159514560773473\\
8943.29432943294 0.00159596309658507\\
8951.93519351935 0.00159678019158199\\
8960.57605760576 0.00159759689329419\\
8969.21692169217 0.00159841320228902\\
8977.85778577858 0.00159922911913244\\
8986.49864986499 0.00160004464438907\\
8995.1395139514 0.00160085977862217\\
9003.7803780378 0.00160167452239367\\
9012.42124212421 0.00160248887626413\\
9021.06210621062 0.00160330284079277\\
9029.70297029703 0.0016041164165375\\
9038.34383438344 0.00160492960405488\\
9046.98469846985 0.00160574240390014\\
9055.62556255626 0.00160655481662719\\
9064.26642664266 0.00160736684278864\\
9072.90729072907 0.00160817848293577\\
9081.54815481548 0.00160898973761854\\
9090.18901890189 0.00160980060738564\\
9098.8298829883 0.00161061109278442\\
9107.47074707471 0.00161142119436096\\
9116.11161116112 0.00161223091266005\\
9124.75247524753 0.00161304024822517\\
9133.39333933393 0.00161384920159854\\
9142.03420342034 0.00161465777332108\\
9150.67506750675 0.00161546596393245\\
9159.31593159316 0.00161627377397105\\
9167.95679567957 0.00161708120397398\\
9176.59765976598 0.00161788825447711\\
9185.23852385239 0.00161869492601503\\
9193.87938793879 0.00161950121912109\\
9202.5202520252 0.00162030713432739\\
9211.16111611161 0.00162111267216477\\
9219.80198019802 0.00162191783316285\\
9228.44284428443 0.00162272261784998\\
9237.08370837084 0.00162352702675332\\
9245.72457245725 0.00162433106039876\\
9254.36543654365 0.00162513471931099\\
9263.00630063006 0.00162593800401347\\
9271.64716471647 0.00162674091502844\\
9280.28802880288 0.00162754345287693\\
9288.92889288929 0.00162834561807876\\
9297.5697569757 0.00162914741115254\\
9306.21062106211 0.00162994883261569\\
9314.85148514851 0.00163074988298441\\
9323.49234923492 0.00163155056277374\\
9332.13321332133 0.00163235087249749\\
9340.77407740774 0.00163315081266831\\
9349.41494149415 0.00163395038379766\\
9358.05580558056 0.00163474958639583\\
9366.69666966697 0.00163554842097193\\
9375.33753375338 0.00163634688803388\\
9383.97839783978 0.00163714498808845\\
9392.61926192619 0.00163794272164126\\
9401.2601260126 0.00163874008919675\\
9409.90099009901 0.00163953709125821\\
9418.54185418542 0.00164033372832777\\
9427.18271827183 0.00164113000090642\\
9435.82358235824 0.00164192590949401\\
9444.46444644464 0.00164272145458924\\
9453.10531053105 0.00164351663668967\\
9461.74617461746 0.00164431145629173\\
9470.38703870387 0.00164510591389073\\
9479.02790279028 0.00164590000998083\\
9487.66876687669 0.00164669374505508\\
9496.3096309631 0.00164748711960542\\
9504.9504950495 0.00164828013412266\\
9513.59135913591 0.00164907278909649\\
9522.23222322232 0.00164986508501551\\
9530.87308730873 0.00165065702236721\\
9539.51395139514 0.00165144860163797\\
9548.15481548155 0.00165223982331307\\
9556.79567956796 0.00165303068787672\\
9565.43654365436 0.00165382119581199\\
9574.07740774077 0.00165461134760091\\
9582.71827182718 0.0016554011437244\\
9591.35913591359 0.00165619058466231\\
9600 0.00165697967089339\\
9608.64086408641 0.00165776840289535\\
9617.28172817282 0.00165855678114481\\
9625.92259225923 0.00165934480611731\\
9634.56345634564 0.00166013247828734\\
9643.20432043204 0.00166091979812833\\
9651.84518451845 0.00166170676611266\\
9660.48604860486 0.00166249338271162\\
9669.12691269127 0.0016632796483955\\
9677.76777677768 0.00166406556363349\\
9686.40864086409 0.00166485112889378\\
9695.0495049505 0.0016656363446435\\
9703.6903690369 0.00166642121134872\\
9712.33123312331 0.00166720572947451\\
9720.97209720972 0.0016679898994849\\
9729.61296129613 0.00166877372184287\\
9738.25382538254 0.0016695571970104\\
9746.89468946895 0.00167034032544843\\
9755.53555355536 0.0016711231076169\\
9764.17641764176 0.00167190554397472\\
9772.81728172817 0.00167268763497978\\
9781.45814581458 0.00167346938108898\\
9790.09900990099 0.0016742507827582\\
9798.7398739874 0.00167503184044232\\
9807.38073807381 0.00167581255459522\\
9816.02160216022 0.00167659292566979\\
9824.66246624662 0.00167737295411791\\
9833.30333033303 0.00167815264039049\\
9841.94419441944 0.00167893198493742\\
9850.58505850585 0.00167971098820763\\
9859.22592259226 0.00168048965064908\\
9867.86678667867 0.00168126797270871\\
9876.50765076508 0.00168204595483252\\
9885.14851485149 0.00168282359746552\\
9893.78937893789 0.00168360090105176\\
9902.4302430243 0.00168437786603431\\
9911.07110711071 0.00168515449285529\\
9919.71197119712 0.00168593078195584\\
9928.35283528353 0.00168670673377616\\
9936.99369936994 0.00168748234875549\\
9945.63456345635 0.00168825762733211\\
9954.27542754275 0.00168903256994336\\
9962.91629162916 0.00168980717702562\\
9971.55715571557 0.00169058144901435\\
9980.19801980198 0.00169135538634404\\
9988.83888388839 0.00169212898944826\\
9997.4797479748 0.00169290225875965\\
10006.1206120612 0.00169367519470989\\
10014.7614761476 0.00169444779772976\\
10023.402340234 0.0016952200682491\\
10032.0432043204 0.00169599200669682\\
10040.6840684068 0.00169676361350092\\
10049.3249324932 0.00169753488908848\\
10057.9657965797 0.00169830583388566\\
10066.6066606661 0.00169907644831771\\
10075.2475247525 0.00169984673280896\\
10083.8883888389 0.00170061668778284\\
10092.5292529253 0.00170138631366188\\
10101.1701170117 0.00170215561086771\\
10109.8109810981 0.00170292457982104\\
10118.4518451845 0.00170369322094171\\
10127.0927092709 0.00170446153464864\\
10135.7335733573 0.00170522952135988\\
10144.3744374437 0.00170599718149258\\
10153.0153015302 0.001706764515463\\
10161.6561656166 0.00170753152368653\\
10170.297029703 0.00170829820657767\\
10178.9378937894 0.00170906456455004\\
10187.5787578758 0.00170983059801639\\
10196.2196219622 0.00171059630738859\\
10204.8604860486 0.00171136169307765\\
10213.501350135 0.00171212675549371\\
10222.1422142214 0.00171289149504603\\
10230.7830783078 0.00171365591214303\\
10239.4239423942 0.00171442000719224\\
10248.0648064806 0.00171518378060037\\
10256.7056705671 0.00171594723277323\\
10265.3465346535 0.00171671036411583\\
10273.9873987399 0.00171747317503227\\
10282.6282628263 0.00171823566592587\\
10291.2691269127 0.00171899783719904\\
10299.9099909991 0.00171975968925338\\
10308.5508550855 0.00172052122248966\\
10317.1917191719 0.00172128243730778\\
10325.8325832583 0.00172204333410683\\
10334.4734473447 0.00172280391328507\\
10343.1143114311 0.0017235641752399\\
10351.7551755176 0.00172432412036792\\
10360.396039604 0.0017250837490649\\
10369.0369036904 0.00172584306172577\\
10377.6777677768 0.00172660205874466\\
10386.3186318632 0.00172736074051487\\
10394.9594959496 0.00172811910742889\\
10403.600360036 0.00172887715987839\\
10412.2412241224 0.00172963489825422\\
10420.8820882088 0.00173039232294645\\
10429.5229522952 0.00173114943434432\\
10438.1638163816 0.00173190623283627\\
10446.804680468 0.00173266271880993\\
10455.4455445545 0.00173341889265215\\
10464.0864086409 0.00173417475474898\\
10472.7272727273 0.00173493030548565\\
10481.3681368137 0.00173568554524662\\
10490.0090009001 0.00173644047441556\\
10498.6498649865 0.00173719509337533\\
10507.2907290729 0.00173794940250804\\
10515.9315931593 0.00173870340219498\\
10524.5724572457 0.00173945709281669\\
10533.2133213321 0.0017402104747529\\
10541.8541854185 0.00174096354838258\\
10550.495049505 0.00174171631408393\\
10559.1359135914 0.00174246877223436\\
10567.7767776778 0.00174322092321053\\
10576.4176417642 0.00174397276738833\\
10585.0585058506 0.00174472430514285\\
10593.699369937 0.00174547553684847\\
10602.3402340234 0.00174622646287876\\
10610.9810981098 0.00174697708360657\\
10619.6219621962 0.00174772739940395\\
10628.2628262826 0.00174847741064225\\
10636.903690369 0.00174922711769201\\
10645.5445544554 0.00174997652092305\\
10654.1854185419 0.00175072562070445\\
10662.8262826283 0.00175147441740451\\
10671.4671467147 0.00175222291139082\\
10680.1080108011 0.00175297110303021\\
10688.7488748875 0.00175371899268877\\
10697.3897389739 0.00175446658073185\\
10706.0306030603 0.00175521386752408\\
10714.6714671467 0.00175596085342934\\
10723.3123312331 0.00175670753881078\\
10731.9531953195 0.00175745392403082\\
10740.5940594059 0.00175820000945115\\
10749.2349234923 0.00175894579543276\\
10757.8757875788 0.00175969128233587\\
10766.5166516652 0.00176043647052002\\
10775.1575157516 0.001761181360344\\
10783.798379838 0.00176192595216591\\
10792.4392439244 0.00176267024634311\\
10801.0801080108 0.00176341424323226\\
10809.7209720972 0.0017641579431893\\
10818.3618361836 0.00176490134656947\\
10827.00270027 0.0017656444537273\\
10835.6435643564 0.0017663872650166\\
10844.2844284428 0.0017671297807905\\
10852.9252925293 0.00176787200140141\\
10861.5661566157 0.00176861392720105\\
10870.2070207021 0.00176935555854043\\
10878.8478847885 0.00177009689576988\\
10887.4887488749 0.00177083793923902\\
10896.1296129613 0.00177157868929679\\
10904.7704770477 0.00177231914629144\\
10913.4113411341 0.00177305931057052\\
10922.0522052205 0.0017737991824809\\
10930.6930693069 0.00177453876236877\\
10939.3339333933 0.00177527805057963\\
10947.9747974797 0.00177601704745831\\
10956.6156615662 0.00177675575334895\\
10965.2565256526 0.00177749416859501\\
10973.897389739 0.00177823229353929\\
10982.5382538254 0.00177897012852391\\
10991.1791179118 0.00177970767389031\\
10999.8199819982 0.00178044492997927\\
11008.4608460846 0.0017811818971309\\
11017.101710171 0.00178191857568464\\
11025.7425742574 0.00178265496597927\\
11034.3834383438 0.00178339106835292\\
11043.0243024302 0.00178412688314303\\
11051.6651665167 0.00178486241068641\\
11060.3060306031 0.00178559765131919\\
11068.9468946895 0.00178633260537687\\
11077.5877587759 0.00178706727319427\\
11086.2286228623 0.00178780165510558\\
11094.8694869487 0.00178853575144432\\
11103.5103510351 0.00178926956254339\\
11112.1512151215 0.00179000308873502\\
11120.7920792079 0.0017907363303508\\
11129.4329432943 0.00179146928772168\\
11138.0738073807 0.00179220196117797\\
11146.7146714671 0.00179293435104935\\
11155.3555355536 0.00179366645766484\\
11163.99639964 0.00179439828135284\\
11172.6372637264 0.00179512982244111\\
11181.2781278128 0.00179586108125679\\
11189.9189918992 0.00179659205812637\\
11198.5598559856 0.00179732275337572\\
11207.200720072 0.0017980531673301\\
11215.8415841584 0.0017987833003141\\
11224.4824482448 0.00179951315265174\\
11233.1233123312 0.00180024272466638\\
11241.7641764176 0.00180097201668078\\
11250.4050405041 0.00180170102901706\\
11259.0459045905 0.00180242976199674\\
11267.6867686769 0.00180315821594072\\
11276.3276327633 0.00180388639116928\\
11284.9684968497 0.00180461428800211\\
11293.6093609361 0.00180534190675826\\
11302.2502250225 0.00180606924775619\\
11310.8910891089 0.00180679631131374\\
11319.5319531953 0.00180752309774816\\
11328.1728172817 0.00180824960737607\\
11336.8136813681 0.00180897584051353\\
11345.4545454545 0.00180970179747595\\
11354.095409541 0.00181042747857818\\
11362.7362736274 0.00181115288413444\\
11371.3771377138 0.00181187801445839\\
11380.0180018002 0.00181260286986306\\
11388.6588658866 0.00181332745066091\\
11397.299729973 0.0018140517571638\\
11405.9405940594 0.001814775789683\\
11414.5814581458 0.0018154995485292\\
11423.2223222322 0.00181622303401248\\
11431.8631863186 0.00181694624644237\\
11440.504050405 0.00181766918612779\\
11449.1449144914 0.00181839185337708\\
11457.7857785779 0.00181911424849801\\
11466.4266426643 0.00181983637179777\\
11475.0675067507 0.00182055822358297\\
11483.7083708371 0.00182127980415963\\
11492.3492349235 0.00182200111383322\\
11500.9900990099 0.00182272215290861\\
11509.6309630963 0.00182344292169014\\
11518.2718271827 0.00182416342048153\\
11526.9126912691 0.00182488364958596\\
11535.5535553555 0.00182560360930605\\
11544.1944194419 0.00182632329994383\\
11552.8352835284 0.00182704272180079\\
11561.4761476148 0.00182776187517783\\
11570.1170117012 0.00182848076037532\\
11578.7578757876 0.00182919937769305\\
11587.398739874 0.00182991772743026\\
11596.0396039604 0.00183063580988563\\
11604.6804680468 0.00183135362535728\\
11613.3213321332 0.00183207117414279\\
11621.9621962196 0.00183278845653917\\
11630.603060306 0.0018335054728429\\
11639.2439243924 0.00183422222334989\\
11647.8847884788 0.00183493870835551\\
11656.5256525653 0.00183565492815459\\
11665.1665166517 0.00183637088304141\\
11673.8073807381 0.0018370865733097\\
11682.4482448245 0.00183780199925265\\
11691.0891089109 0.00183851716116292\\
11699.7299729973 0.00183923205933262\\
11708.3708370837 0.00183994669405332\\
11717.0117011701 0.00184066106561606\\
11725.6525652565 0.00184137517431135\\
11734.2934293429 0.00184208902042914\\
11742.9342934293 0.00184280260425887\\
11751.5751575158 0.00184351592608945\\
11760.2160216022 0.00184422898620925\\
11768.8568856886 0.00184494178490612\\
11777.497749775 0.00184565432246736\\
11786.1386138614 0.00184636659917978\\
11794.7794779478 0.00184707861532965\\
11803.4203420342 0.00184779037120269\\
11812.0612061206 0.00184850186708415\\
11820.702070207 0.00184921310325871\\
11829.3429342934 0.00184992408001057\\
11837.9837983798 0.00185063479762338\\
11846.6246624662 0.00185134525638029\\
11855.2655265527 0.00185205545656394\\
11863.9063906391 0.00185276539845644\\
11872.5472547255 0.0018534750823394\\
11881.1881188119 0.00185418450849392\\
11889.8289828983 0.00185489367720057\\
11898.4698469847 0.00185560258873945\\
11907.1107110711 0.0018563112433901\\
11915.7515751575 0.0018570196414316\\
11924.3924392439 0.0018577277831425\\
11933.0333033303 0.00185843566880086\\
11941.6741674167 0.00185914329868423\\
11950.3150315032 0.00185985067306966\\
11958.9558955896 0.00186055779223369\\
11967.596759676 0.00186126465645239\\
11976.2376237624 0.0018619712660013\\
11984.8784878488 0.00186267762115549\\
11993.5193519352 0.00186338372218951\\
12002.1602160216 0.00186408956937744\\
12010.801080108 0.00186479516299286\\
12019.4419441944 0.00186550050330886\\
12028.0828082808 0.00186620559059802\\
12036.7236723672 0.00186691042513247\\
12045.3645364536 0.00186761500718381\\
12054.0054005401 0.00186831933702319\\
12062.6462646265 0.00186902341492126\\
12071.2871287129 0.00186972724114818\\
12079.9279927993 0.00187043081597364\\
12088.5688568857 0.00187113413966684\\
12097.2097209721 0.0018718372124965\\
12105.8505850585 0.00187254003473087\\
12114.4914491449 0.0018732426066377\\
12123.1323132313 0.0018739449284843\\
12131.7731773177 0.00187464700053748\\
12140.4140414041 0.00187534882306357\\
12149.0549054905 0.00187605039632844\\
12157.695769577 0.00187675172059749\\
12166.3366336634 0.00187745279613564\\
12174.9774977498 0.00187815362320735\\
12183.6183618362 0.00187885420207661\\
12192.2592259226 0.00187955453300693\\
12200.900090009 0.00188025461626137\\
12209.5409540954 0.00188095445210251\\
12218.1818181818 0.00188165404079249\\
12226.8226822682 0.00188235338259296\\
12235.4635463546 0.00188305247776513\\
12244.104410441 0.00188375132656974\\
12252.7452745275 0.00188444992926707\\
12261.3861386139 0.00188514828611694\\
12270.0270027003 0.00188584639737872\\
12278.6678667867 0.00188654426331132\\
12287.3087308731 0.00188724188417321\\
12295.9495949595 0.00188793926022238\\
12304.5904590459 0.00188863639171638\\
12313.2313231323 0.00188933327891231\\
12321.8721872187 0.00189002992206683\\
12330.5130513051 0.00189072632143613\\
12339.1539153915 0.00189142247727597\\
12347.7947794779 0.00189211838984165\\
12356.4356435644 0.00189281405938803\\
12365.0765076508 0.00189350948616953\\
12373.7173717372 0.00189420467044012\\
12382.3582358236 0.00189489961245332\\
12390.99909991 0.00189559431246223\\
12399.6399639964 0.0018962887707195\\
12408.2808280828 0.00189698298747732\\
12416.9216921692 0.00189767696298748\\
12425.5625562556 0.0018983706975013\\
12434.203420342 0.00189906419126967\\
12442.8442844284 0.00189975744454307\\
12451.4851485149 0.00190045045757152\\
12460.1260126013 0.00190114323060462\\
12468.7668766877 0.00190183576389152\\
12477.4077407741 0.00190252805768095\\
12486.0486048605 0.00190322011222123\\
12494.6894689469 0.00190391192776022\\
12503.3303330333 0.00190460350454538\\
12511.9711971197 0.0019052948428237\\
12520.6120612061 0.00190598594284181\\
12529.2529252925 0.00190667680484585\\
12537.8937893789 0.00190736742908158\\
12546.5346534653 0.00190805781579431\\
12555.1755175518 0.00190874796522895\\
12563.8163816382 0.00190943787762999\\
12572.4572457246 0.00191012755324147\\
12581.098109811 0.00191081699230704\\
12589.7389738974 0.00191150619506994\\
12598.3798379838 0.00191219516177295\\
12607.0207020702 0.00191288389265849\\
12615.6615661566 0.00191357238796852\\
12624.302430243 0.00191426064794461\\
12632.9432943294 0.00191494867282792\\
12641.5841584158 0.00191563646285918\\
12650.2250225023 0.00191632401827872\\
12658.8658865887 0.00191701133932646\\
12667.5067506751 0.00191769842624192\\
12676.1476147615 0.0019183852792642\\
12684.7884788479 0.00191907189863199\\
12693.4293429343 0.00191975828458359\\
12702.0702070207 0.00192044443735689\\
12710.7110711071 0.00192113035718936\\
12719.3519351935 0.00192181604431809\\
12727.9927992799 0.00192250149897975\\
12736.6336633663 0.00192318672141063\\
12745.2745274527 0.0019238717118466\\
12753.9153915392 0.00192455647052314\\
12762.5562556256 0.00192524099767533\\
12771.197119712 0.00192592529353785\\
12779.8379837984 0.00192660935834498\\
12788.4788478848 0.00192729319233063\\
12797.1197119712 0.00192797679572829\\
12805.7605760576 0.00192866016877106\\
12814.401440144 0.00192934331169165\\
12823.0423042304 0.00193002622472239\\
12831.6831683168 0.00193070890809519\\
12840.3240324032 0.00193139136204161\\
12848.9648964896 0.00193207358679279\\
12857.6057605761 0.0019327555825795\\
12866.2466246625 0.00193343734963211\\
12874.8874887489 0.00193411888818061\\
12883.5283528353 0.0019348001984546\\
12892.1692169217 0.00193548128068331\\
12900.8100810081 0.00193616213509558\\
12909.4509450945 0.00193684276191985\\
12918.0918091809 0.0019375231613842\\
12926.7326732673 0.00193820333371632\\
12935.3735373537 0.00193888327914353\\
12944.0144014401 0.00193956299789276\\
12952.6552655266 0.00194024249019056\\
12961.296129613 0.00194092175626312\\
12969.9369936994 0.00194160079633623\\
12978.5778577858 0.00194227961063532\\
12987.2187218722 0.00194295819938545\\
12995.8595859586 0.00194363656281129\\
13004.500450045 0.00194431470113715\\
13013.1413141314 0.00194499261458696\\
13021.7821782178 0.00194567030338429\\
13030.4230423042 0.00194634776775232\\
13039.0639063906 0.00194702500791389\\
13047.704770477 0.00194770202409144\\
13056.3456345635 0.00194837881650707\\
13064.9864986499 0.0019490553853825\\
13073.6273627363 0.00194973173093907\\
13082.2682268227 0.00195040785339778\\
13090.9090909091 0.00195108375297926\\
13099.5499549955 0.00195175942990376\\
13108.1908190819 0.00195243488439119\\
13116.8316831683 0.00195311011666108\\
13125.4725472547 0.00195378512693262\\
13134.1134113411 0.00195445991542461\\
13142.7542754275 0.00195513448235551\\
13151.395139514 0.00195580882794343\\
13160.0360036004 0.0019564829524061\\
13168.6768676868 0.00195715685596092\\
13177.3177317732 0.0019578305388249\\
13185.9585958596 0.00195850400121472\\
13194.599459946 0.00195917724334671\\
13203.2403240324 0.00195985026543682\\
13211.8811881188 0.00196052306770068\\
13220.5220522052 0.00196119565035354\\
13229.1629162916 0.00196186801361031\\
13237.803780378 0.00196254015768555\\
13246.4446444644 0.00196321208279348\\
13255.0855085509 0.00196388378914795\\
13263.7263726373 0.00196455527696247\\
13272.3672367237 0.00196522654645023\\
13281.0081008101 0.00196589759782402\\
13289.6489648965 0.00196656843129634\\
13298.2898289829 0.00196723904707931\\
13306.9306930693 0.00196790944538472\\
13315.5715571557 0.001968579626424\\
13324.2124212421 0.00196924959040827\\
13332.8532853285 0.00196991933754827\\
13341.4941494149 0.00197058886805443\\
13350.1350135014 0.00197125818213683\\
13358.7758775878 0.0019719272800052\\
13367.4167416742 0.00197259616186895\\
13376.0576057606 0.00197326482793713\\
13384.698469847 0.00197393327841847\\
13393.3393339334 0.00197460151352136\\
13401.9801980198 0.00197526953345386\\
13410.6210621062 0.00197593733842368\\
13419.2619261926 0.00197660492863821\\
13427.902790279 0.0019772723043045\\
13436.5436543654 0.00197793946562927\\
13445.1845184518 0.00197860641281891\\
13453.8253825383 0.00197927314607947\\
13462.4662466247 0.00197993966561667\\
13471.1071107111 0.00198060597163593\\
13479.7479747975 0.00198127206434229\\
13488.3888388839 0.00198193794394051\\
13497.0297029703 0.00198260361063499\\
13505.6705670567 0.00198326906462983\\
13514.3114311431 0.00198393430612878\\
13522.9522952295 0.00198459933533527\\
13531.5931593159 0.00198526415245242\\
13540.2340234023 0.00198592875768301\\
13548.8748874887 0.00198659315122952\\
13557.5157515752 0.00198725733329407\\
13566.1566156616 0.00198792130407849\\
13574.797479748 0.00198858506378428\\
13583.4383438344 0.00198924861261262\\
13592.0792079208 0.00198991195076436\\
13600.7200720072 0.00199057507844006\\
13609.3609360936 0.00199123799583993\\
13618.00180018 0.00199190070316388\\
13626.6426642664 0.00199256320061149\\
13635.2835283528 0.00199322548838205\\
13643.9243924392 0.00199388756667451\\
13652.5652565257 0.00199454943568751\\
13661.2061206121 0.00199521109561939\\
13669.8469846985 0.00199587254666816\\
13678.4878487849 0.00199653378903153\\
13687.1287128713 0.00199719482290689\\
13695.7695769577 0.00199785564849132\\
13704.4104410441 0.0019985162659816\\
13713.0513051305 0.00199917667557419\\
13721.6921692169 0.00199983687746523\\
13730.3330333033 0.00200049687185058\\
13738.9738973897 0.00200115665892578\\
13747.6147614761 0.00200181623888605\\
13756.2556255626 0.00200247561192631\\
13764.896489649 0.0020031347782412\\
13773.5373537354 0.00200379373802501\\
13782.1782178218 0.00200445249147177\\
13790.8190819082 0.00200511103877517\\
13799.4599459946 0.00200576938012862\\
13808.100810081 0.00200642751572523\\
13816.7416741674 0.00200708544575778\\
13825.3825382538 0.00200774317041879\\
13834.0234023402 0.00200840068990044\\
13842.6642664266 0.00200905800439464\\
13851.3051305131 0.00200971511409299\\
13859.9459945995 0.00201037201918678\\
13868.5868586859 0.00201102871986703\\
13877.2277227723 0.00201168521632443\\
13885.8685868587 0.00201234150874941\\
13894.5094509451 0.00201299759733207\\
13903.1503150315 0.00201365348226223\\
13911.7911791179 0.00201430916372942\\
13920.4320432043 0.00201496464192287\\
13929.0729072907 0.00201561991703152\\
13937.7137713771 0.00201627498924402\\
13946.3546354635 0.00201692985874872\\
13954.99549955 0.00201758452573368\\
13963.6363636364 0.00201823899038667\\
13972.2772277228 0.00201889325289518\\
13980.9180918092 0.0020195473134464\\
13989.5589558956 0.00202020117222723\\
13998.199819982 0.00202085482942429\\
14006.8406840684 0.00202150828522391\\
14015.4815481548 0.00202216153981213\\
14024.1224122412 0.0020228145933747\\
14032.7632763276 0.0020234674460971\\
14041.404140414 0.00202412009816451\\
14050.0450045005 0.00202477254976183\\
14058.6858685869 0.00202542480107367\\
14067.3267326733 0.00202607685228438\\
14075.9675967597 0.00202672870357799\\
14084.6084608461 0.00202738035513829\\
14093.2493249325 0.00202803180714875\\
14101.8901890189 0.00202868305979259\\
14110.5310531053 0.00202933411325272\\
14119.1719171917 0.0020299849677118\\
14127.8127812781 0.0020306356233522\\
14136.4536453645 0.002031286080356\\
14145.0945094509 0.00203193633890502\\
14153.7353735374 0.00203258639918079\\
14162.3762376238 0.00203323626136457\\
14171.0171017102 0.00203388592563734\\
14179.6579657966 0.0020345353921798\\
14188.298829883 0.0020351846611724\\
14196.9396939694 0.00203583373279528\\
14205.5805580558 0.00203648260722834\\
14214.2214221422 0.00203713128465118\\
14222.8622862286 0.00203777976524313\\
14231.503150315 0.00203842804918328\\
14240.1440144014 0.0020390761366504\\
14248.7848784878 0.00203972402782304\\
14257.4257425743 0.00204037172287943\\
14266.0666066607 0.00204101922199756\\
14274.7074707471 0.00204166652535516\\
14283.3483348335 0.00204231363312966\\
14291.9891989199 0.00204296054549825\\
14300.6300630063 0.00204360726263784\\
14309.2709270927 0.00204425378472507\\
14317.9117911791 0.00204490011193632\\
14326.5526552655 0.00204554624444771\\
14335.1935193519 0.00204619218243507\\
14343.8343834383 0.00204683792607401\\
14352.4752475248 0.00204748347553983\\
14361.1161116112 0.00204812883100758\\
14369.7569756976 0.00204877399265208\\
14378.397839784 0.00204941896064784\\
14387.0387038704 0.00205006373516913\\
14395.6795679568 0.00205070831638996\\
14404.3204320432 0.00205135270448408\\
14412.9612961296 0.00205199689962498\\
14421.602160216 0.00205264090198588\\
14430.2430243024 0.00205328471173975\\
14438.8838883888 0.0020539283290593\\
14447.5247524752 0.00205457175411699\\
14456.1656165617 0.002055214987085\\
14464.8064806481 0.00205585802813528\\
14473.4473447345 0.0020565008774395\\
14482.0882088209 0.0020571435351691\\
14490.7290729073 0.00205778600149524\\
14499.3699369937 0.00205842827658883\\
14508.0108010801 0.00205907036062055\\
14516.6516651665 0.00205971225376079\\
14525.2925292529 0.0020603539561797\\
14533.9333933393 0.00206099546804721\\
14542.5742574257 0.00206163678953294\\
14551.2151215122 0.0020622779208063\\
14559.8559855986 0.00206291886203644\\
14568.496849685 0.00206355961339225\\
14577.1377137714 0.00206420017504238\\
14585.7785778578 0.00206484054715523\\
14594.4194419442 0.00206548072989894\\
14603.0603060306 0.00206612072344141\\
14611.701170117 0.0020667605279503\\
14620.3420342034 0.00206740014359301\\
14628.9828982898 0.00206803957053669\\
14637.6237623762 0.00206867880894826\\
14646.2646264626 0.00206931785899439\\
14654.9054905491 0.00206995672084148\\
14663.5463546355 0.00207059539465572\\
14672.1872187219 0.00207123388060303\\
14680.8280828083 0.0020718721788491\\
14689.4689468947 0.00207251028955938\\
14698.1098109811 0.00207314821289906\\
14706.7506750675 0.00207378594903311\\
14715.3915391539 0.00207442349812623\\
14724.0324032403 0.0020750608603429\\
14732.6732673267 0.00207569803584735\\
14741.3141314131 0.00207633502480358\\
14749.9549954995 0.00207697182737534\\
14758.595859586 0.00207760844372615\\
14767.2367236724 0.00207824487401927\\
14775.8775877588 0.00207888111841774\\
14784.5184518452 0.00207951717708437\\
14793.1593159316 0.00208015305018171\\
14801.800180018 0.00208078873787209\\
14810.4410441044 0.00208142424031759\\
14819.0819081908 0.00208205955768006\\
14827.7227722772 0.00208269469012112\\
14836.3636363636 0.00208332963780216\\
14845.00450045 0.00208396440088431\\
14853.6453645365 0.00208459897952849\\
14862.2862286229 0.00208523337389537\\
14870.9270927093 0.0020858675841454\\
14879.5679567957 0.0020865016104388\\
14888.2088208821 0.00208713545293554\\
14896.8496849685 0.00208776911179538\\
14905.4905490549 0.00208840258717782\\
14914.1314131413 0.00208903587924215\\
14922.7722772277 0.00208966898814744\\
14931.4131413141 0.0020903019140525\\
14940.0540054005 0.00209093465711594\\
14948.6948694869 0.00209156721749612\\
14957.3357335734 0.00209219959535118\\
14965.9765976598 0.00209283179083903\\
14974.6174617462 0.00209346380411735\\
14983.2583258326 0.00209409563534361\\
14991.899189919 0.00209472728467502\\
15000.5400540054 0.0020953587522686\\
15009.1809180918 0.00209599003828113\\
15017.8217821782 0.00209662114286915\\
15026.4626462646 0.00209725206618899\\
15035.103510351 0.00209788280839676\\
15043.7443744374 0.00209851336964834\\
15052.3852385239 0.00209914375009937\\
15061.0261026103 0.00209977394990531\\
15069.6669666967 0.00210040396922135\\
15078.3078307831 0.00210103380820248\\
15086.9486948695 0.00210166346700347\\
15095.5895589559 0.00210229294577887\\
15104.2304230423 0.002102922244683\\
15112.8712871287 0.00210355136386996\\
15121.5121512151 0.00210418030349365\\
15130.1530153015 0.00210480906370771\\
15138.7938793879 0.00210543764466561\\
15147.4347434743 0.00210606604652056\\
15156.0756075608 0.00210669426942558\\
15164.7164716472 0.00210732231353345\\
15173.3573357336 0.00210795017899676\\
15181.99819982 0.00210857786596785\\
15190.6390639064 0.00210920537459887\\
15199.2799279928 0.00210983270504175\\
15207.9207920792 0.00211045985744818\\
15216.5616561656 0.00211108683196968\\
15225.202520252 0.0021117136287575\\
15233.8433843384 0.00211234024796273\\
15242.4842484248 0.00211296668973621\\
15251.1251125113 0.00211359295422858\\
15259.7659765977 0.00211421904159025\\
15268.4068406841 0.00211484495197146\\
15277.0477047705 0.00211547068552219\\
15285.6885688569 0.00211609624239223\\
15294.3294329433 0.00211672162273116\\
15302.9702970297 0.00211734682668834\\
15311.6111611161 0.00211797185441294\\
15320.2520252025 0.00211859670605388\\
15328.8928892889 0.00211922138175992\\
15337.5337533753 0.00211984588167958\\
15346.1746174617 0.00212047020596117\\
15354.8154815482 0.0021210943547528\\
15363.4563456346 0.00212171832820238\\
15372.097209721 0.00212234212645759\\
15380.7380738074 0.00212296574966592\\
15389.3789378938 0.00212358919797465\\
15398.0198019802 0.00212421247153086\\
15406.6606660666 0.00212483557048141\\
15415.301530153 0.00212545849497296\\
15423.9423942394 0.00212608124515197\\
15432.5832583258 0.00212670382116468\\
15441.2241224122 0.00212732622315715\\
15449.8649864986 0.00212794845127522\\
15458.5058505851 0.00212857050566453\\
15467.1467146715 0.0021291923864705\\
15475.7875787579 0.00212981409383839\\
15484.4284428443 0.00213043562791321\\
15493.0693069307 0.00213105698883979\\
15501.7101710171 0.00213167817676277\\
15510.3510351035 0.00213229919182656\\
15518.9918991899 0.0021329200341754\\
15527.6327632763 0.0021335407039533\\
15536.2736273627 0.0021341612013041\\
15544.9144914491 0.00213478152637141\\
15553.5553555356 0.00213540167929866\\
15562.196219622 0.00213602166022909\\
15570.8370837084 0.00213664146930571\\
15579.4779477948 0.00213726110667135\\
15588.1188118812 0.00213788057246866\\
15596.7596759676 0.00213849986684006\\
15605.400540054 0.00213911898992779\\
15614.0414041404 0.00213973794187389\\
15622.6822682268 0.00214035672282021\\
15631.3231323132 0.0021409753329084\\
15639.9639963996 0.00214159377227991\\
15648.604860486 0.002142212041076\\
15657.2457245725 0.00214283013943774\\
15665.8865886589 0.00214344806750599\\
15674.5274527453 0.00214406582542143\\
15683.1683168317 0.00214468341332454\\
15691.8091809181 0.00214530083135561\\
15700.4500450045 0.00214591807965474\\
15709.0909090909 0.00214653515836183\\
15717.7317731773 0.00214715206761659\\
15726.3726372637 0.00214776880755855\\
15735.0135013501 0.00214838537832702\\
15743.6543654365 0.00214900178006116\\
15752.295229523 0.0021496180128999\\
15760.9360936094 0.002150234076982\\
15769.5769576958 0.00215084997244603\\
15778.2178217822 0.00215146569943037\\
15786.8586858686 0.0021520812580732\\
15795.499549955 0.00215269664851253\\
15804.1404140414 0.00215331187088616\\
15812.7812781278 0.00215392692533172\\
15821.4221422142 0.00215454181198664\\
15830.0630063006 0.00215515653098818\\
15838.703870387 0.00215577108247338\\
15847.3447344734 0.00215638546657913\\
15855.9855985599 0.00215699968344211\\
15864.6264626463 0.00215761373319883\\
15873.2673267327 0.0021582276159856\\
15881.9081908191 0.00215884133193855\\
15890.5490549055 0.00215945488119362\\
15899.1899189919 0.00216006826388659\\
15907.8307830783 0.00216068148015303\\
15916.4716471647 0.00216129453012833\\
15925.1125112511 0.00216190741394769\\
15933.7533753375 0.00216252013174616\\
15942.3942394239 0.00216313268365857\\
15951.0351035104 0.00216374506981958\\
15959.6759675968 0.00216435729036368\\
15968.3168316832 0.00216496934542516\\
15976.9576957696 0.00216558123513813\\
15985.598559856 0.00216619295963654\\
15994.2394239424 0.00216680451905414\\
16002.8802880288 0.00216741591352451\\
16011.5211521152 0.00216802714318103\\
16020.1620162016 0.00216863820815692\\
16028.802880288 0.00216924910858522\\
16037.4437443744 0.00216985984459878\\
16046.0846084608 0.00217047041633028\\
16054.7254725473 0.00217108082391223\\
16063.3663366337 0.00217169106747693\\
16072.0072007201 0.00217230114715654\\
16080.6480648065 0.00217291106308302\\
16089.2889288929 0.00217352081538816\\
16097.9297929793 0.00217413040420358\\
16106.5706570657 0.0021747398296607\\
16115.2115211521 0.0021753490918908\\
16123.8523852385 0.00217595819102495\\
16132.4932493249 0.00217656712719406\\
16141.1341134113 0.00217717590052886\\
16149.7749774977 0.00217778451115993\\
16158.4158415842 0.00217839295921763\\
16167.0567056706 0.00217900124483219\\
16175.697569757 0.00217960936813364\\
16184.3384338434 0.00218021732925184\\
16192.9792979298 0.00218082512831648\\
16201.6201620162 0.00218143276545709\\
16210.2610261026 0.002182040240803\\
16218.901890189 0.0021826475544834\\
16227.5427542754 0.00218325470662727\\
16236.1836183618 0.00218386169736346\\
16244.8244824482 0.00218446852682063\\
16253.4653465347 0.00218507519512725\\
16262.1062106211 0.00218568170241165\\
16270.7470747075 0.00218628804880198\\
16279.3879387939 0.0021868942344262\\
16288.0288028803 0.00218750025941214\\
16296.6696669667 0.00218810612388743\\
16305.3105310531 0.00218871182797954\\
16313.9513951395 0.00218931737181577\\
16322.5922592259 0.00218992275552325\\
16331.2331233123 0.00219052797922895\\
16339.8739873987 0.00219113304305967\\
16348.5148514851 0.00219173794714203\\
16357.1557155716 0.0021923426916025\\
16365.796579658 0.00219294727656738\\
16374.4374437444 0.00219355170216279\\
16383.0783078308 0.00219415596851469\\
16391.7191719172 0.00219476007574889\\
16400.3600360036 0.00219536402399102\\
16409.00090009 0.00219596781336654\\
16417.6417641764 0.00219657144400076\\
16426.2826282628 0.0021971749160188\\
16434.9234923492 0.00219777822954566\\
16443.5643564356 0.00219838138470612\\
16452.2052205221 0.00219898438162485\\
16460.8460846085 0.00219958722042632\\
16469.4869486949 0.00220018990123484\\
16478.1278127813 0.00220079242417459\\
16486.7686768677 0.00220139478936954\\
16495.4095409541 0.00220199699694352\\
16504.0504050405 0.00220259904702022\\
16512.6912691269 0.00220320093972314\\
16521.3321332133 0.00220380267517561\\
16529.9729972997 0.00220440425350083\\
16538.6138613861 0.00220500567482183\\
16547.2547254725 0.00220560693926146\\
16555.895589559 0.00220620804694243\\
16564.5364536454 0.00220680899798728\\
16573.1773177318 0.00220740979251839\\
16581.8181818182 0.002208010430658\\
16590.4590459046 0.00220861091252817\\
16599.099909991 0.00220921123825081\\
16607.7407740774 0.00220981140794765\\
16616.3816381638 0.00221041142174031\\
16625.0225022502 0.00221101127975019\\
16633.6633663366 0.0022116109820986\\
16642.304230423 0.00221221052890663\\
16650.9450945095 0.00221280992029526\\
16659.5859585959 0.00221340915638528\\
16668.2268226823 0.00221400823729735\\
16676.8676867687 0.00221460716315195\\
16685.5085508551 0.00221520593406942\\
16694.1494149415 0.00221580455016995\\
16702.7902790279 0.00221640301157355\\
16711.4311431143 0.00221700131840011\\
16720.0720072007 0.00221759947076932\\
16728.7128712871 0.00221819746880076\\
16737.3537353735 0.00221879531261383\\
16745.9945994599 0.00221939300232778\\
16754.6354635464 0.00221999053806172\\
16763.2763276328 0.00222058791993459\\
16771.9171917192 0.00222118514806517\\
16780.5580558056 0.00222178222257212\\
16789.198919892 0.00222237914357392\\
16797.8397839784 0.0022229759111889\\
16806.4806480648 0.00222357252553525\\
16815.1215121512 0.002224168986731\\
16823.7623762376 0.00222476529489403\\
16832.403240324 0.00222536145014207\\
16841.0441044104 0.00222595745259269\\
16849.6849684969 0.00222655330236333\\
16858.3258325833 0.00222714899957126\\
16866.9666966697 0.00222774454433362\\
16875.6075607561 0.00222833993676736\\
16884.2484248425 0.00222893517698934\\
16892.8892889289 0.00222953026511621\\
16901.5301530153 0.00223012520126453\\
16910.1710171017 0.00223071998555065\\
16918.8118811881 0.00223131461809082\\
16927.4527452745 0.00223190909900112\\
16936.0936093609 0.00223250342839749\\
16944.7344734473 0.00223309760639572\\
16953.3753375338 0.00223369163311144\\
16962.0162016202 0.00223428550866016\\
16970.6570657066 0.00223487923315721\\
16979.297929793 0.0022354728067178\\
16987.9387938794 0.00223606622945699\\
16996.5796579658 0.00223665950148968\\
17005.2205220522 0.00223725262293063\\
17013.8613861386 0.00223784559389446\\
17022.502250225 0.00223843841449565\\
17031.1431143114 0.00223903108484852\\
17039.7839783978 0.00223962360506725\\
17048.4248424842 0.00224021597526587\\
17057.0657065707 0.00224080819555829\\
17065.7065706571 0.00224140026605824\\
17074.3474347435 0.00224199218687934\\
17082.9882988299 0.00224258395813505\\
17091.6291629163 0.00224317557993867\\
17100.2700270027 0.0022437670524034\\
17108.9108910891 0.00224435837564226\\
17117.5517551755 0.00224494954976814\\
17126.1926192619 0.00224554057489379\\
17134.8334833483 0.00224613145113182\\
17143.4743474347 0.00224672217859468\\
17152.1152115212 0.0022473127573947\\
17160.7560756076 0.00224790318764407\\
17169.396939694 0.00224849346945482\\
17178.0378037804 0.00224908360293884\\
17186.6786678668 0.00224967358820791\\
17195.3195319532 0.00225026342537364\\
17203.9603960396 0.0022508531145475\\
17212.601260126 0.00225144265584084\\
17221.2421242124 0.00225203204936485\\
17229.8829882988 0.0022526212952306\\
17238.5238523852 0.002253210393549\\
17247.1647164716 0.00225379934443084\\
17255.8055805581 0.00225438814798675\\
17264.4464446445 0.00225497680432726\\
17273.0873087309 0.00225556531356272\\
17281.7281728173 0.00225615367580336\\
17290.3690369037 0.00225674189115927\\
17299.0099009901 0.00225732995974041\\
17307.6507650765 0.0022579178816566\\
17316.2916291629 0.00225850565701751\\
17324.9324932493 0.00225909328593269\\
17333.5733573357 0.00225968076851154\\
17342.2142214221 0.00226026810486334\\
17350.8550855086 0.00226085529509723\\
17359.495949595 0.00226144233932219\\
17368.1368136814 0.00226202923764711\\
17376.7776777678 0.00226261599018069\\
17385.4185418542 0.00226320259703155\\
17394.0594059406 0.00226378905830813\\
17402.700270027 0.00226437537411876\\
17411.3411341134 0.00226496154457164\\
17419.9819981998 0.00226554756977482\\
17428.6228622862 0.00226613344983621\\
17437.2637263726 0.00226671918486362\\
17445.904590459 0.00226730477496469\\
17454.5454545455 0.00226789022024695\\
17463.1863186319 0.00226847552081779\\
17471.8271827183 0.00226906067678446\\
17480.4680468047 0.00226964568825409\\
17489.1089108911 0.00227023055533367\\
17497.7497749775 0.00227081527813006\\
17506.3906390639 0.00227139985674999\\
17515.0315031503 0.00227198429130006\\
17523.6723672367 0.00227256858188673\\
17532.3132313231 0.00227315272861634\\
17540.9540954095 0.00227373673159509\\
17549.5949594959 0.00227432059092905\\
17558.2358235824 0.00227490430672416\\
17566.8766876688 0.00227548787908625\\
17575.5175517552 0.00227607130812099\\
17584.1584158416 0.00227665459393394\\
17592.799279928 0.00227723773663051\\
17601.4401440144 0.00227782073631601\\
17610.0810081008 0.00227840359309559\\
17618.7218721872 0.0022789863070743\\
17627.3627362736 0.00227956887835704\\
17636.00360036 0.00228015130704859\\
17644.6444644464 0.0022807335932536\\
17653.2853285329 0.0022813157370766\\
17661.9261926193 0.00228189773862198\\
17670.5670567057 0.00228247959799401\\
17679.2079207921 0.00228306131529683\\
17687.8487848785 0.00228364289063446\\
17696.4896489649 0.00228422432411077\\
17705.1305130513 0.00228480561582954\\
17713.7713771377 0.00228538676589439\\
17722.4122412241 0.00228596777440883\\
17731.0531053105 0.00228654864147624\\
17739.6939693969 0.00228712936719988\\
17748.3348334833 0.00228770995168289\\
17756.9756975698 0.00228829039502825\\
17765.6165616562 0.00228887069733886\\
17774.2574257426 0.00228945085871747\\
17782.898289829 0.00229003087926671\\
17791.5391539154 0.00229061075908908\\
17800.1800180018 0.00229119049828696\\
17808.8208820882 0.00229177009696262\\
17817.4617461746 0.00229234955521818\\
17826.102610261 0.00229292887315566\\
17834.7434743474 0.00229350805087694\\
17843.3843384338 0.00229408708848378\\
17852.0252025203 0.00229466598607783\\
17860.6660666067 0.0022952447437606\\
17869.3069306931 0.00229582336163349\\
17877.9477947795 0.00229640183979776\\
17886.5886588659 0.00229698017835457\\
17895.2295229523 0.00229755837740495\\
17903.8703870387 0.0022981364370498\\
17912.5112511251 0.0022987143573899\\
17921.1521152115 0.00229929213852592\\
17929.7929792979 0.00229986978055841\\
17938.4338433843 0.00230044728358777\\
17947.0747074707 0.00230102464771432\\
17955.7155715572 0.00230160187303823\\
17964.3564356436 0.00230217895965956\\
17972.99729973 0.00230275590767826\\
17981.6381638164 0.00230333271719413\\
17990.2790279028 0.00230390938830688\\
17998.9198919892 0.00230448592111609\\
18007.5607560756 0.00230506231572121\\
18016.201620162 0.0023056385722216\\
18024.8424842484 0.00230621469071647\\
18033.4833483348 0.00230679067130493\\
18042.1242124212 0.00230736651408596\\
18050.7650765077 0.00230794221915843\\
18059.4059405941 0.00230851778662108\\
18068.0468046805 0.00230909321657255\\
18076.6876687669 0.00230966850911135\\
18085.3285328533 0.00231024366433588\\
18093.9693969397 0.00231081868234441\\
18102.6102610261 0.0023113935632351\\
18111.2511251125 0.002311968307106\\
18119.8919891989 0.00231254291405504\\
18128.5328532853 0.00231311738418002\\
18137.1737173717 0.00231369171757864\\
18145.8145814581 0.00231426591434848\\
18154.4554455446 0.00231483997458699\\
18163.096309631 0.00231541389839153\\
18171.7371737174 0.00231598768585933\\
18180.3780378038 0.00231656133708749\\
18189.0189018902 0.00231713485217302\\
18197.6597659766 0.00231770823121281\\
18206.300630063 0.00231828147430362\\
18214.9414941494 0.00231885458154211\\
18223.5823582358 0.00231942755302482\\
18232.2232223222 0.00232000038884817\\
18240.8640864086 0.00232057308910848\\
18249.5049504951 0.00232114565390195\\
18258.1458145815 0.00232171808332466\\
18266.7866786679 0.00232229037747258\\
18275.4275427543 0.00232286253644157\\
18284.0684068407 0.00232343456032737\\
18292.7092709271 0.00232400644922562\\
18301.3501350135 0.00232457820323183\\
18309.9909990999 0.00232514982244142\\
18318.6318631863 0.00232572130694966\\
18327.2727272727 0.00232629265685176\\
18335.9135913591 0.00232686387224276\\
18344.5544554455 0.00232743495321765\\
18353.195319532 0.00232800589987125\\
18361.8361836184 0.0023285767122983\\
18370.4770477048 0.00232914739059343\\
18379.1179117912 0.00232971793485115\\
18387.7587758776 0.00233028834516586\\
18396.399639964 0.00233085862163185\\
18405.0405040504 0.0023314287643433\\
18413.6813681368 0.00233199877339427\\
18422.3222322232 0.00233256864887874\\
18430.9630963096 0.00233313839089055\\
18439.603960396 0.00233370799952343\\
18448.2448244824 0.00233427747487102\\
18456.8856885689 0.00233484681702683\\
18465.5265526553 0.00233541602608428\\
18474.1674167417 0.00233598510213668\\
18482.8082808281 0.0023365540452772\\
18491.4491449145 0.00233712285559894\\
18500.0900090009 0.00233769153319487\\
18508.7308730873 0.00233826007815786\\
18517.3717371737 0.00233882849058067\\
18526.0126012601 0.00233939677055595\\
18534.6534653465 0.00233996491817623\\
18543.2943294329 0.00234053293353396\\
18551.9351935194 0.00234110081672147\\
18560.5760576058 0.00234166856783096\\
18569.2169216922 0.00234223618695457\\
18577.8577857786 0.00234280367418428\\
18586.498649865 0.00234337102961201\\
18595.1395139514 0.00234393825332954\\
18603.7803780378 0.00234450534542855\\
18612.4212421242 0.00234507230600064\\
18621.0621062106 0.00234563913513727\\
18629.702970297 0.00234620583292981\\
18638.3438343834 0.00234677239946952\\
18646.9846984698 0.00234733883484755\\
18655.6255625563 0.00234790513915497\\
18664.2664266427 0.0023484713124827\\
18672.9072907291 0.0023490373549216\\
18681.5481548155 0.0023496032665624\\
18690.1890189019 0.00235016904749573\\
18698.8298829883 0.00235073469781211\\
18707.4707470747 0.00235130021760197\\
18716.1116111611 0.00235186560695561\\
18724.7524752475 0.00235243086596327\\
18733.3933393339 0.00235299599471504\\
18742.0342034203 0.00235356099330093\\
18750.6750675068 0.00235412586181083\\
18759.3159315932 0.00235469060033456\\
18767.9567956796 0.00235525520896181\\
18776.597659766 0.00235581968778216\\
18785.2385238524 0.0023563840368851\\
18793.8793879388 0.00235694825636002\\
18802.5202520252 0.00235751234629621\\
18811.1611161116 0.00235807630678285\\
18819.801980198 0.00235864013790901\\
18828.4428442844 0.00235920383976367\\
18837.0837083708 0.0023597674124357\\
18845.7245724572 0.00236033085601388\\
18854.3654365437 0.00236089417058689\\
18863.0063006301 0.00236145735624328\\
18871.6471647165 0.00236202041307153\\
18880.2880288029 0.00236258334116\\
18888.9288928893 0.00236314614059697\\
18897.5697569757 0.00236370881147059\\
18906.2106210621 0.00236427135386893\\
18914.8514851485 0.00236483376787997\\
18923.4923492349 0.00236539605359155\\
18932.1332133213 0.00236595821109144\\
18940.7740774077 0.00236652024046731\\
18949.4149414941 0.00236708214180673\\
18958.0558055806 0.00236764391519715\\
18966.696669667 0.00236820556072594\\
18975.3375337534 0.00236876707848037\\
18983.9783978398 0.00236932846854761\\
18992.6192619262 0.00236988973101471\\
19001.2601260126 0.00237045086596865\\
19009.900990099 0.0023710118734963\\
19018.5418541854 0.00237157275368442\\
19027.1827182718 0.0023721335066197\\
19035.8235823582 0.0023726941323887\\
19044.4644464446 0.00237325463107791\\
19053.1053105311 0.00237381500277369\\
19061.7461746175 0.00237437524756233\\
19070.3870387039 0.00237493536553002\\
19079.0279027903 0.00237549535676283\\
19087.6687668767 0.00237605522134676\\
19096.3096309631 0.00237661495936769\\
19104.9504950495 0.00237717457091143\\
19113.5913591359 0.00237773405606365\\
19122.2322232223 0.00237829341490998\\
19130.8730873087 0.0023788526475359\\
19139.5139513951 0.00237941175402684\\
19148.1548154815 0.00237997073446808\\
19156.795679568 0.00238052958894486\\
19165.4365436544 0.00238108831754229\\
19174.0774077408 0.00238164692034539\\
19182.7182718272 0.0023822053974391\\
19191.3591359136 0.00238276374890823\\
19200 0.00238332197483753\\
19208.6408640864 0.00238388007531164\\
19217.2817281728 0.0023844380504151\\
19225.9225922592 0.00238499590023237\\
19234.5634563456 0.00238555362484781\\
19243.204320432 0.00238611122434566\\
19251.8451845185 0.00238666869881011\\
19260.4860486049 0.00238722604832522\\
19269.1269126913 0.00238778327297497\\
19277.7677767777 0.00238834037284325\\
19286.4086408641 0.00238889734801384\\
19295.0495049505 0.00238945419857045\\
19303.6903690369 0.00239001092459668\\
19312.3312331233 0.00239056752617603\\
19320.9720972097 0.00239112400339193\\
19329.6129612961 0.0023916803563277\\
19338.2538253825 0.00239223658506657\\
19346.8946894689 0.00239279268969167\\
19355.5355535554 0.00239334867028605\\
19364.1764176418 0.00239390452693267\\
19372.8172817282 0.00239446025971438\\
19381.4581458146 0.00239501586871395\\
19390.099009901 0.00239557135401406\\
19398.7398739874 0.00239612671569728\\
19407.3807380738 0.00239668195384612\\
19416.0216021602 0.00239723706854298\\
19424.6624662466 0.00239779205987015\\
19433.303330333 0.00239834692790986\\
19441.9441944194 0.00239890167274423\\
19450.5850585059 0.0023994562944553\\
19459.2259225923 0.00240001079312501\\
19467.8667866787 0.00240056516883521\\
19476.5076507651 0.00240111942166767\\
19485.1485148515 0.00240167355170406\\
19493.7893789379 0.00240222755902596\\
19502.4302430243 0.00240278144371486\\
19511.0711071107 0.00240333520585215\\
19519.7119711971 0.00240388884551916\\
19528.3528352835 0.0024044423627971\\
19536.9936993699 0.0024049957577671\\
19545.6345634563 0.0024055490305102\\
19554.2754275428 0.00240610218110736\\
19562.9162916292 0.00240665520963944\\
19571.5571557156 0.0024072081161872\\
19580.198019802 0.00240776090083135\\
19588.8388838884 0.00240831356365246\\
19597.4797479748 0.00240886610473105\\
19606.1206120612 0.00240941852414754\\
19614.7614761476 0.00240997082198226\\
19623.402340234 0.00241052299831544\\
19632.0432043204 0.00241107505322724\\
19640.6840684068 0.00241162698679773\\
19649.3249324932 0.00241217879910687\\
19657.9657965797 0.00241273049023457\\
19666.6066606661 0.00241328206026062\\
19675.2475247525 0.00241383350926474\\
19683.8883888389 0.00241438483732654\\
19692.5292529253 0.00241493604452558\\
19701.1701170117 0.00241548713094131\\
19709.8109810981 0.00241603809665308\\
19718.4518451845 0.00241658894174018\\
19727.0927092709 0.0024171396662818\\
19735.7335733573 0.00241769027035704\\
19744.3744374437 0.00241824075404492\\
19753.0153015302 0.00241879111742438\\
19761.6561656166 0.00241934136057425\\
19770.297029703 0.00241989148357331\\
19778.9378937894 0.00242044148650022\\
19787.5787578758 0.00242099136943356\\
19796.2196219622 0.00242154113245186\\
19804.8604860486 0.00242209077563351\\
19813.501350135 0.00242264029905685\\
19822.1422142214 0.00242318970280013\\
19830.7830783078 0.0024237389869415\\
19839.4239423942 0.00242428815155905\\
19848.0648064806 0.00242483719673076\\
19856.7056705671 0.00242538612253455\\
19865.3465346535 0.00242593492904822\\
19873.9873987399 0.00242648361634952\\
19882.6282628263 0.00242703218451611\\
19891.2691269127 0.00242758063362554\\
19899.9099909991 0.00242812896375531\\
19908.5508550855 0.00242867717498281\\
19917.1917191719 0.00242922526738536\\
19925.8325832583 0.00242977324104019\\
19934.4734473447 0.00243032109602445\\
19943.1143114311 0.00243086883241521\\
19951.7551755176 0.00243141645028944\\
19960.396039604 0.00243196394972405\\
19969.0369036904 0.00243251133079586\\
19977.6777677768 0.00243305859358159\\
19986.3186318632 0.00243360573815789\\
19994.9594959496 0.00243415276460134\\
20003.600360036 0.00243469967298841\\
20012.2412241224 0.0024352464633955\\
20020.8820882088 0.00243579313589895\\
20029.5229522952 0.00243633969057498\\
20038.1638163816 0.00243688612749974\\
20046.804680468 0.00243743244674932\\
20055.4455445545 0.00243797864839971\\
20064.0864086409 0.0024385247325268\\
20072.7272727273 0.00243907069920644\\
20081.3681368137 0.00243961654851437\\
20090.0090009001 0.00244016228052624\\
20098.6498649865 0.00244070789531766\\
20107.2907290729 0.00244125339296411\\
20115.9315931593 0.00244179877354102\\
20124.5724572457 0.00244234403712373\\
20133.2133213321 0.0024428891837875\\
20141.8541854185 0.00244343421360751\\
20150.4950495049 0.00244397912665886\\
20159.1359135914 0.00244452392301656\\
20167.7767776778 0.00244506860275556\\
20176.4176417642 0.00244561316595071\\
20185.0585058506 0.00244615761267679\\
20193.699369937 0.0024467019430085\\
20202.3402340234 0.00244724615702045\\
20210.9810981098 0.00244779025478719\\
20219.6219621962 0.00244833423638317\\
20228.2628262826 0.00244887810188277\\
20236.903690369 0.00244942185136029\\
20245.5445544554 0.00244996548488996\\
20254.1854185419 0.00245050900254591\\
20262.8262826283 0.00245105240440221\\
20271.4671467147 0.00245159569053283\\
20280.1080108011 0.0024521388610117\\
20288.7488748875 0.00245268191591262\\
20297.3897389739 0.00245322485530935\\
20306.0306030603 0.00245376767927557\\
20314.6714671467 0.00245431038788485\\
20323.3123312331 0.00245485298121072\\
20331.9531953195 0.00245539545932661\\
20340.5940594059 0.00245593782230587\\
20349.2349234923 0.00245648007022178\\
20357.8757875788 0.00245702220314756\\
20366.5166516652 0.00245756422115631\\
20375.1575157516 0.00245810612432109\\
20383.798379838 0.00245864791271486\\
20392.4392439244 0.00245918958641053\\
20401.0801080108 0.00245973114548089\\
20409.7209720972 0.0024602725899987\\
20418.3618361836 0.00246081392003661\\
20427.00270027 0.0024613551356672\\
20435.6435643564 0.00246189623696299\\
20444.2844284428 0.0024624372239964\\
20452.9252925293 0.00246297809683979\\
20461.5661566157 0.00246351885556543\\
20470.2070207021 0.00246405950024554\\
20478.8478847885 0.00246460003095222\\
20487.4887488749 0.00246514044775754\\
20496.1296129613 0.00246568075073347\\
20504.7704770477 0.00246622093995191\\
20513.4113411341 0.00246676101548467\\
20522.0522052205 0.00246730097740351\\
20530.6930693069 0.00246784082578009\\
20539.3339333933 0.00246838056068602\\
20547.9747974797 0.00246892018219281\\
20556.6156615662 0.00246945969037191\\
20565.2565256526 0.0024699990852947\\
20573.897389739 0.00247053836703246\\
20582.5382538254 0.00247107753565641\\
20591.1791179118 0.00247161659123771\\
20599.8199819982 0.00247215553384743\\
20608.4608460846 0.00247269436355657\\
20617.101710171 0.00247323308043603\\
20625.7425742574 0.00247377168455669\\
20634.3834383438 0.00247431017598931\\
20643.0243024302 0.00247484855480459\\
20651.6651665167 0.00247538682107316\\
20660.3060306031 0.00247592497486557\\
20668.9468946895 0.0024764630162523\\
20677.5877587759 0.00247700094530376\\
20686.2286228623 0.00247753876209028\\
20694.8694869487 0.00247807646668212\\
20703.5103510351 0.00247861405914947\\
20712.1512151215 0.00247915153956243\\
20720.7920792079 0.00247968890799105\\
20729.4329432943 0.0024802261645053\\
20738.0738073807 0.00248076330917506\\
20746.7146714671 0.00248130034207016\\
20755.3555355536 0.00248183726326036\\
20763.99639964 0.00248237407281531\\
20772.6372637264 0.00248291077080464\\
20781.2781278128 0.00248344735729786\\
20789.9189918992 0.00248398383236444\\
20798.5598559856 0.00248452019607377\\
20807.200720072 0.00248505644849515\\
20815.8415841584 0.00248559258969785\\
20824.4824482448 0.00248612861975101\\
20833.1233123312 0.00248666453872375\\
20841.7641764176 0.0024872003466851\\
20850.4050405041 0.002487736043704\\
20859.0459045905 0.00248827162984935\\
20867.6867686769 0.00248880710518996\\
20876.3276327633 0.00248934246979457\\
20884.9684968497 0.00248987772373186\\
20893.6093609361 0.00249041286707042\\
20902.2502250225 0.00249094789987879\\
20910.8910891089 0.00249148282222542\\
20919.5319531953 0.00249201763417871\\
20928.1728172817 0.00249255233580696\\
20936.8136813681 0.00249308692717844\\
20945.4545454545 0.00249362140836132\\
20954.095409541 0.00249415577942369\\
20962.7362736274 0.00249469004043361\\
20971.3771377138 0.00249522419145904\\
20980.0180018002 0.00249575823256788\\
20988.6588658866 0.00249629216382794\\
20997.299729973 0.002496825985307\\
21005.9405940594 0.00249735969707273\\
21014.5814581458 0.00249789329919277\\
21023.2223222322 0.00249842679173464\\
21031.8631863186 0.00249896017476584\\
21040.504050405 0.00249949344835378\\
21049.1449144914 0.0025000266125658\\
21057.7857785779 0.00250055966746916\\
21066.4266426643 0.00250109261313108\\
21075.0675067507 0.00250162544961868\\
21083.7083708371 0.00250215817699904\\
21092.3492349235 0.00250269079533914\\
21100.9900990099 0.00250322330470593\\
21109.6309630963 0.00250375570516625\\
21118.2718271827 0.0025042879967869\\
21126.9126912691 0.00250482017963461\\
21135.5535553555 0.00250535225377603\\
21144.1944194419 0.00250588421927774\\
21152.8352835284 0.00250641607620627\\
21161.4761476148 0.00250694782462806\\
21170.1170117012 0.00250747946460951\\
21178.7578757876 0.00250801099621693\\
21187.398739874 0.00250854241951656\\
21196.0396039604 0.00250907373457459\\
21204.6804680468 0.00250960494145713\\
21213.3213321332 0.00251013604023022\\
21221.9621962196 0.00251066703095986\\
21230.603060306 0.00251119791371194\\
21239.2439243924 0.00251172868855231\\
21247.8847884788 0.00251225935554676\\
21256.5256525653 0.00251278991476099\\
21265.1665166517 0.00251332036626065\\
21273.8073807381 0.00251385071011132\\
21282.4482448245 0.00251438094637851\\
21291.0891089109 0.00251491107512765\\
21299.7299729973 0.00251544109642415\\
21308.3708370837 0.00251597101033329\\
21317.0117011701 0.00251650081692034\\
21325.6525652565 0.00251703051625047\\
21334.2934293429 0.0025175601083888\\
21342.9342934293 0.00251808959340037\\
21351.5751575158 0.00251861897135018\\
21360.2160216022 0.00251914824230312\\
21368.8568856886 0.00251967740632407\\
21377.497749775 0.0025202064634778\\
21386.1386138614 0.00252073541382903\\
21394.7794779478 0.00252126425744243\\
21403.4203420342 0.00252179299438257\\
21412.0612061206 0.00252232162471399\\
21420.702070207 0.00252285014850114\\
21429.3429342934 0.00252337856580843\\
21437.9837983798 0.00252390687670018\\
21446.6246624662 0.00252443508124065\\
21455.2655265527 0.00252496317949405\\
21463.9063906391 0.00252549117152452\\
21472.5472547255 0.00252601905739612\\
21481.1881188119 0.00252654683717287\\
21489.8289828983 0.0025270745109187\\
21498.4698469847 0.0025276020786975\\
21507.1107110711 0.00252812954057309\\
21515.7515751575 0.0025286568966092\\
21524.3924392439 0.00252918414686953\\
21533.0333033303 0.00252971129141771\\
21541.6741674167 0.00253023833031728\\
21550.3150315031 0.00253076526363176\\
21558.9558955896 0.00253129209142456\\
21567.596759676 0.00253181881375906\\
21576.2376237624 0.00253234543069856\\
21584.8784878488 0.00253287194230631\\
21593.5193519352 0.00253339834864548\\
21602.1602160216 0.0025339246497792\\
21610.801080108 0.0025344508457705\\
21619.4419441944 0.00253497693668239\\
21628.0828082808 0.00253550292257778\\
21636.7236723672 0.00253602880351955\\
21645.3645364536 0.00253655457957049\\
21654.0054005401 0.00253708025079334\\
21662.6462646265 0.00253760581725078\\
21671.2871287129 0.00253813127900542\\
21679.9279927993 0.00253865663611982\\
21688.5688568857 0.00253918188865646\\
21697.2097209721 0.00253970703667776\\
21705.8505850585 0.0025402320802461\\
21714.4914491449 0.00254075701942378\\
21723.1323132313 0.00254128185427304\\
21731.7731773177 0.00254180658485605\\
21740.4140414041 0.00254233121123494\\
21749.0549054905 0.00254285573347175\\
21757.695769577 0.00254338015162849\\
21766.3366336634 0.00254390446576709\\
21774.9774977498 0.00254442867594942\\
21783.6183618362 0.00254495278223728\\
21792.2592259226 0.00254547678469243\\
21800.900090009 0.00254600068337656\\
21809.5409540954 0.00254652447835128\\
21818.1818181818 0.00254704816967817\\
21826.8226822682 0.00254757175741874\\
21835.4635463546 0.00254809524163441\\
21844.104410441 0.00254861862238659\\
21852.7452745275 0.00254914189973658\\
21861.3861386139 0.00254966507374566\\
21870.0270027003 0.00255018814447502\\
21878.6678667867 0.00255071111198581\\
21887.3087308731 0.0025512339763391\\
21895.9495949595 0.00255175673759592\\
21904.5904590459 0.00255227939581723\\
21913.2313231323 0.00255280195106392\\
21921.8721872187 0.00255332440339684\\
21930.5130513051 0.00255384675287677\\
21939.1539153915 0.00255436899956443\\
21947.7947794779 0.00255489114352047\\
21956.4356435644 0.00255541318480551\\
21965.0765076508 0.00255593512348008\\
21973.7173717372 0.00255645695960467\\
21982.3582358236 0.00255697869323969\\
21990.99909991 0.00255750032444552\\
21999.6399639964 0.00255802185328245\\
22008.2808280828 0.00255854327981073\\
22016.9216921692 0.00255906460409055\\
22025.5625562556 0.00255958582618204\\
22034.203420342 0.00256010694614526\\
22042.8442844284 0.00256062796404022\\
22051.4851485149 0.00256114887992687\\
22060.1260126013 0.00256166969386511\\
22068.7668766877 0.00256219040591477\\
22077.4077407741 0.00256271101613562\\
22086.0486048605 0.00256323152458739\\
22094.6894689469 0.00256375193132972\\
22103.3303330333 0.00256427223642222\\
22111.9711971197 0.00256479243992443\\
22120.6120612061 0.00256531254189583\\
22129.2529252925 0.00256583254239585\\
22137.8937893789 0.00256635244148386\\
22146.5346534653 0.00256687223921916\\
22155.1755175518 0.00256739193566102\\
22163.8163816382 0.00256791153086862\\
22172.4572457246 0.00256843102490109\\
22181.098109811 0.00256895041781753\\
22189.7389738974 0.00256946970967694\\
22198.3798379838 0.0025699889005383\\
22207.0207020702 0.00257050799046052\\
22215.6615661566 0.00257102697950243\\
22224.302430243 0.00257154586772284\\
22232.9432943294 0.00257206465518048\\
22241.5841584158 0.00257258334193402\\
22250.2250225023 0.00257310192804209\\
22258.8658865887 0.00257362041356326\\
22267.5067506751 0.00257413879855603\\
22276.1476147615 0.00257465708307885\\
22284.7884788479 0.00257517526719012\\
22293.4293429343 0.00257569335094817\\
22302.0702070207 0.00257621133441129\\
22310.7110711071 0.00257672921763771\\
22319.3519351935 0.00257724700068558\\
22327.9927992799 0.00257776468361303\\
22336.6336633663 0.00257828226647811\\
22345.2745274527 0.00257879974933881\\
22353.9153915392 0.0025793171322531\\
22362.5562556256 0.00257983441527884\\
22371.197119712 0.00258035159847388\\
22379.8379837984 0.002580868681896\\
22388.4788478848 0.0025813856656029\\
22397.1197119712 0.00258190254965227\\
22405.7605760576 0.0025824193341017\\
22414.401440144 0.00258293601900876\\
22423.0423042304 0.00258345260443094\\
22431.6831683168 0.00258396909042568\\
22440.3240324032 0.00258448547705038\\
22448.9648964896 0.00258500176436236\\
22457.6057605761 0.00258551795241891\\
22466.2466246625 0.00258603404127725\\
22474.8874887489 0.00258655003099455\\
22483.5283528353 0.00258706592162791\\
22492.1692169217 0.00258758171323441\\
22500.8100810081 0.00258809740587103\\
22509.4509450945 0.00258861299959474\\
22518.0918091809 0.00258912849446242\\
22526.7326732673 0.00258964389053091\\
22535.3735373537 0.002590159187857\\
22544.0144014401 0.00259067438649743\\
22552.6552655266 0.00259118948650886\\
22561.296129613 0.00259170448794793\\
22569.9369936994 0.00259221939087119\\
22578.5778577858 0.00259273419533516\\
22587.2187218722 0.00259324890139631\\
22595.8595859586 0.00259376350911104\\
22604.500450045 0.0025942780185357\\
22613.1413141314 0.00259479242972659\\
22621.7821782178 0.00259530674273995\\
22630.4230423042 0.00259582095763199\\
22639.0639063906 0.00259633507445882\\
22647.704770477 0.00259684909327655\\
22656.3456345635 0.00259736301414119\\
22664.9864986499 0.00259787683710873\\
22673.6273627363 0.00259839056223509\\
22682.2682268227 0.00259890418957615\\
22690.9090909091 0.00259941771918771\\
22699.5499549955 0.00259993115112554\\
22708.1908190819 0.00260044448544537\\
22716.8316831683 0.00260095772220283\\
22725.4725472547 0.00260147086145355\\
22734.1134113411 0.00260198390325306\\
22742.7542754275 0.00260249684765688\\
22751.395139514 0.00260300969472046\\
22760.0360036004 0.00260352244449917\\
22768.6768676868 0.00260403509704838\\
22777.3177317732 0.00260454765242336\\
22785.9585958596 0.00260506011067936\\
22794.599459946 0.00260557247187156\\
22803.2403240324 0.00260608473605509\\
22811.8811881188 0.00260659690328504\\
22820.5220522052 0.00260710897361644\\
22829.1629162916 0.00260762094710425\\
22837.803780378 0.00260813282380341\\
22846.4446444644 0.00260864460376879\\
22855.0855085509 0.0026091562870552\\
22863.7263726373 0.00260966787371743\\
22872.3672367237 0.00261017936381018\\
22881.0081008101 0.00261069075738813\\
22889.6489648965 0.00261120205450588\\
22898.2898289829 0.002611713255218\\
22906.9306930693 0.00261222435957901\\
22915.5715571557 0.00261273536764335\\
22924.2124212421 0.00261324627946545\\
22932.8532853285 0.00261375709509965\\
22941.4941494149 0.00261426781460027\\
22950.1350135013 0.00261477843802155\\
22958.7758775878 0.00261528896541772\\
22967.4167416742 0.00261579939684291\\
22976.0576057606 0.00261630973235123\\
22984.698469847 0.00261681997199673\\
22993.3393339334 0.00261733011583342\\
23001.9801980198 0.00261784016391524\\
23010.6210621062 0.0026183501162961\\
23019.2619261926 0.00261885997302984\\
23027.902790279 0.00261936973417026\\
23036.5436543654 0.00261987939977111\\
23045.1845184518 0.00262038896988608\\
23053.8253825383 0.00262089844456884\\
23062.4662466247 0.00262140782387296\\
23071.1071107111 0.00262191710785201\\
23079.7479747975 0.00262242629655947\\
23088.3888388839 0.0026229353900488\\
23097.0297029703 0.00262344438837339\\
23105.6705670567 0.00262395329158659\\
23114.3114311431 0.0026244620997417\\
23122.9522952295 0.00262497081289196\\
23131.5931593159 0.00262547943109058\\
23140.2340234023 0.00262598795439069\\
23148.8748874887 0.00262649638284541\\
23157.5157515752 0.00262700471650778\\
23166.1566156616 0.00262751295543079\\
23174.797479748 0.00262802109966741\\
23183.4383438344 0.00262852914927052\\
23192.0792079208 0.00262903710429299\\
23200.7200720072 0.00262954496478762\\
23209.3609360936 0.00263005273080716\\
23218.00180018 0.00263056040240431\\
23226.6426642664 0.00263106797963173\\
23235.2835283528 0.00263157546254203\\
23243.9243924392 0.00263208285118777\\
23252.5652565257 0.00263259014562145\\
23261.2061206121 0.00263309734589553\\
23269.8469846985 0.00263360445206243\\
23278.4878487849 0.00263411146417451\\
23287.1287128713 0.00263461838228409\\
23295.7695769577 0.00263512520644342\\
23304.4104410441 0.00263563193670474\\
23313.0513051305 0.0026361385731202\\
23321.6921692169 0.00263664511574193\\
23330.3330333033 0.002637151564622\\
23338.9738973897 0.00263765791981243\\
23347.6147614761 0.00263816418136521\\
23356.2556255626 0.00263867034933227\\
23364.896489649 0.00263917642376547\\
23373.5373537354 0.00263968240471666\\
23382.1782178218 0.00264018829223761\\
23390.8190819082 0.00264069408638007\\
23399.4599459946 0.00264119978719573\\
23408.100810081 0.00264170539473622\\
23416.7416741674 0.00264221090905315\\
23425.3825382538 0.00264271633019805\\
23434.0234023402 0.00264322165822243\\
23442.6642664266 0.00264372689317774\\
23451.3051305131 0.00264423203511538\\
23459.9459945995 0.0026447370840867\\
23468.5868586859 0.00264524204014303\\
23477.2277227723 0.00264574690333562\\
23485.8685868587 0.00264625167371568\\
23494.5094509451 0.00264675635133439\\
23503.1503150315 0.00264726093624286\\
23511.7911791179 0.00264776542849217\\
23520.4320432043 0.00264826982813335\\
23529.0729072907 0.00264877413521738\\
23537.7137713771 0.00264927834979519\\
23546.3546354635 0.00264978247191767\\
23554.99549955 0.00265028650163566\\
23563.6363636364 0.00265079043899995\\
23572.2772277228 0.00265129428406129\\
23580.9180918092 0.00265179803687039\\
23589.5589558956 0.00265230169747789\\
23598.199819982 0.0026528052659344\\
23606.8406840684 0.0026533087422905\\
23615.4815481548 0.00265381212659668\\
23624.1224122412 0.00265431541890343\\
23632.7632763276 0.00265481861926116\\
23641.404140414 0.00265532172772026\\
23650.0450045005 0.00265582474433104\\
23658.6858685869 0.00265632766914381\\
23667.3267326733 0.0026568305022088\\
23675.9675967597 0.00265733324357621\\
23684.6084608461 0.00265783589329617\\
23693.2493249325 0.0026583384514188\\
23701.8901890189 0.00265884091799415\\
23710.5310531053 0.00265934329307223\\
23719.1719171917 0.002659845576703\\
23727.8127812781 0.0026603477689364\\
23736.4536453645 0.00266084986982229\\
23745.0945094509 0.00266135187941049\\
23753.7353735374 0.00266185379775081\\
23762.3762376238 0.00266235562489297\\
23771.0171017102 0.00266285736088667\\
23779.6579657966 0.00266335900578156\\
23788.298829883 0.00266386055962724\\
23796.9396939694 0.00266436202247326\\
23805.5805580558 0.00266486339436915\\
23814.2214221422 0.00266536467536438\\
23822.8622862286 0.00266586586550836\\
23831.503150315 0.00266636696485047\\
23840.1440144014 0.00266686797344006\\
23848.7848784878 0.0026673688913264\\
23857.4257425743 0.00266786971855875\\
23866.0666066607 0.0026683704551863\\
23874.7074707471 0.00266887110125822\\
23883.3483348335 0.00266937165682361\\
23891.9891989199 0.00266987212193154\\
23900.6300630063 0.00267037249663103\\
23909.2709270927 0.00267087278097107\\
23917.9117911791 0.00267137297500058\\
23926.5526552655 0.00267187307876846\\
23935.1935193519 0.00267237309232356\\
23943.8343834383 0.00267287301571467\\
23952.4752475248 0.00267337284899056\\
23961.1161116112 0.00267387259219993\\
23969.7569756976 0.00267437224539146\\
23978.397839784 0.00267487180861378\\
23987.0387038704 0.00267537128191547\\
23995.6795679568 0.00267587066534507\\
24004.3204320432 0.00267636995895107\\
24012.9612961296 0.00267686916278193\\
24021.602160216 0.00267736827688605\\
24030.2430243024 0.0026778673013118\\
24038.8838883888 0.0026783662361075\\
24047.5247524752 0.00267886508132143\\
24056.1656165617 0.00267936383700182\\
24064.8064806481 0.00267986250319687\\
24073.4473447345 0.00268036107995472\\
24082.0882088209 0.00268085956732348\\
24090.7290729073 0.00268135796535121\\
24099.3699369937 0.00268185627408593\\
24108.0108010801 0.00268235449357561\\
24116.6516651665 0.0026828526238682\\
24125.2925292529 0.00268335066501157\\
24133.9333933393 0.00268384861705357\\
24142.5742574257 0.00268434648004202\\
24151.2151215122 0.00268484425402467\\
24159.8559855986 0.00268534193904924\\
24168.496849685 0.00268583953516341\\
24177.1377137714 0.00268633704241481\\
24185.7785778578 0.00268683446085103\\
24194.4194419442 0.00268733179051963\\
24203.0603060306 0.0026878290314681\\
24211.701170117 0.00268832618374392\\
24220.3420342034 0.0026888232473945\\
24228.9828982898 0.00268932022246723\\
24237.6237623762 0.00268981710900943\\
24246.2646264626 0.00269031390706842\\
24254.9054905491 0.00269081061669143\\
24263.5463546355 0.00269130723792568\\
24272.1872187219 0.00269180377081835\\
24280.8280828083 0.00269230021541655\\
24289.4689468947 0.00269279657176737\\
24298.1098109811 0.00269329283991786\\
24306.7506750675 0.00269378901991502\\
24315.3915391539 0.00269428511180581\\
24324.0324032403 0.00269478111563714\\
24332.6732673267 0.00269527703145589\\
24341.3141314131 0.0026957728593089\\
24349.9549954995 0.00269626859924296\\
24358.595859586 0.00269676425130483\\
24367.2367236724 0.00269725981554121\\
24375.8775877588 0.00269775529199877\\
24384.5184518452 0.00269825068072415\\
24393.1593159316 0.00269874598176392\\
24401.800180018 0.00269924119516463\\
24410.4410441044 0.0026997363209728\\
24419.0819081908 0.00270023135923487\\
24427.7227722772 0.00270072630999728\\
24436.3636363636 0.0027012211733064\\
24445.00450045 0.00270171594920857\\
24453.6453645365 0.0027022106377501\\
24462.2862286229 0.00270270523897724\\
24470.9270927093 0.00270319975293621\\
24479.5679567957 0.00270369417967319\\
24488.2088208821 0.00270418851923431\\
24496.8496849685 0.00270468277166567\\
24505.4905490549 0.00270517693701332\\
24514.1314131413 0.00270567101532327\\
24522.7722772277 0.00270616500664151\\
24531.4131413141 0.00270665891101397\\
24540.0540054005 0.00270715272848653\\
24548.6948694869 0.00270764645910504\\
24557.3357335734 0.00270814010291534\\
24565.9765976598 0.00270863365996317\\
24574.6174617462 0.00270912713029428\\
24583.2583258326 0.00270962051395436\\
24591.899189919 0.00271011381098906\\
24600.5400540054 0.00271060702144399\\
24609.1809180918 0.00271110014536471\\
24617.8217821782 0.00271159318279678\\
24626.4626462646 0.00271208613378566\\
24635.103510351 0.00271257899837683\\
24643.7443744374 0.00271307177661568\\
24652.3852385239 0.00271356446854759\\
24661.0261026103 0.0027140570742179\\
24669.6669666967 0.00271454959367189\\
24678.3078307831 0.00271504202695482\\
24686.9486948695 0.0027155343741119\\
24695.5895589559 0.00271602663518831\\
24704.2304230423 0.00271651881022918\\
24712.8712871287 0.00271701089927961\\
24721.5121512151 0.00271750290238465\\
24730.1530153015 0.00271799481958931\\
24738.7938793879 0.00271848665093858\\
24747.4347434743 0.00271897839647739\\
24756.0756075608 0.00271947005625065\\
24764.7164716472 0.0027199616303032\\
24773.3573357336 0.00272045311867988\\
24781.99819982 0.00272094452142546\\
24790.6390639064 0.00272143583858469\\
24799.2799279928 0.00272192707020226\\
24807.9207920792 0.00272241821632285\\
24816.5616561656 0.00272290927699107\\
24825.202520252 0.00272340025225153\\
24833.8433843384 0.00272389114214875\\
24842.4842484248 0.00272438194672726\\
24851.1251125113 0.00272487266603153\\
24859.7659765977 0.00272536330010598\\
24868.4068406841 0.00272585384899501\\
24877.0477047705 0.00272634431274298\\
24885.6885688569 0.00272683469139419\\
24894.3294329433 0.00272732498499294\\
24902.9702970297 0.00272781519358346\\
24911.6111611161 0.00272830531720994\\
24920.2520252025 0.00272879535591656\\
24928.8928892889 0.00272928530974744\\
24937.5337533753 0.00272977517874667\\
24946.1746174617 0.00273026496295829\\
24954.8154815482 0.00273075466242631\\
24963.4563456346 0.00273124427719471\\
24972.097209721 0.00273173380730742\\
24980.7380738074 0.00273222325280834\\
24989.3789378938 0.00273271261374132\\
24998.0198019802 0.00273320189015019\\
25006.6606660666 0.00273369108207873\\
25015.301530153 0.00273418018957068\\
25023.9423942394 0.00273466921266975\\
25032.5832583258 0.00273515815141962\\
25041.2241224122 0.00273564700586391\\
25049.8649864987 0.00273613577604621\\
25058.5058505851 0.00273662446201009\\
25067.1467146715 0.00273711306379906\\
25075.7875787579 0.0027376015814566\\
25084.4284428443 0.00273809001502616\\
25093.0693069307 0.00273857836455115\\
25101.7101710171 0.00273906663007493\\
25110.3510351035 0.00273955481164084\\
25118.9918991899 0.00274004290929218\\
25127.6327632763 0.00274053092307219\\
25136.2736273627 0.00274101885302411\\
25144.9144914491 0.00274150669919111\\
25153.5553555356 0.00274199446161634\\
25162.196219622 0.00274248214034291\\
25170.8370837084 0.00274296973541389\\
25179.4779477948 0.00274345724687232\\
25188.1188118812 0.0027439446747612\\
25196.7596759676 0.00274443201912349\\
25205.400540054 0.00274491928000212\\
25214.0414041404 0.00274540645743997\\
25222.6822682268 0.00274589355147989\\
25231.3231323132 0.0027463805621647\\
25239.9639963996 0.00274686748953718\\
25248.604860486 0.00274735433364007\\
25257.2457245725 0.00274784109451607\\
25265.8865886589 0.00274832777220786\\
25274.5274527453 0.00274881436675806\\
25283.1683168317 0.00274930087820928\\
25291.8091809181 0.00274978730660406\\
25300.4500450045 0.00275027365198494\\
25309.0909090909 0.0027507599143944\\
25317.7317731773 0.00275124609387489\\
25326.3726372637 0.00275173219046883\\
25335.0135013501 0.0027522182042186\\
25343.6543654365 0.00275270413516653\\
25352.295229523 0.00275318998335494\\
25360.9360936094 0.00275367574882608\\
25369.5769576958 0.00275416143162221\\
25378.2178217822 0.00275464703178551\\
25386.8586858686 0.00275513254935815\\
25395.499549955 0.00275561798438226\\
25404.1404140414 0.00275610333689992\\
25412.7812781278 0.0027565886069532\\
25421.4221422142 0.0027570737945841\\
25430.0630063006 0.00275755889983462\\
25438.703870387 0.0027580439227467\\
25447.3447344734 0.00275852886336225\\
25455.9855985599 0.00275901372172316\\
25464.6264626463 0.00275949849787125\\
25473.2673267327 0.00275998319184835\\
25481.9081908191 0.00276046780369621\\
25490.5490549055 0.00276095233345658\\
25499.1899189919 0.00276143678117116\\
25507.8307830783 0.0027619211468816\\
25516.4716471647 0.00276240543062954\\
25525.1125112511 0.00276288963245657\\
25533.7533753375 0.00276337375240425\\
25542.3942394239 0.0027638577905141\\
25551.0351035104 0.00276434174682762\\
25559.6759675968 0.00276482562138625\\
25568.3168316832 0.00276530941423142\\
25576.9576957696 0.00276579312540451\\
25585.598559856 0.00276627675494687\\
25594.2394239424 0.0027667603028998\\
25602.8802880288 0.0027672437693046\\
25611.5211521152 0.0027677271542025\\
25620.1620162016 0.00276821045763471\\
25628.802880288 0.00276869367964242\\
25637.4437443744 0.00276917682026675\\
25646.0846084608 0.00276965987954882\\
25654.7254725473 0.00277014285752969\\
25663.3663366337 0.0027706257542504\\
25672.0072007201 0.00277110856975195\\
25680.6480648065 0.00277159130407532\\
25689.2889288929 0.00277207395726143\\
25697.9297929793 0.00277255652935118\\
25706.5706570657 0.00277303902038544\\
25715.2115211521 0.00277352143040503\\
25723.8523852385 0.00277400375945075\\
25732.4932493249 0.00277448600756337\\
25741.1341134113 0.0027749681747836\\
25749.7749774977 0.00277545026115215\\
25758.4158415842 0.00277593226670967\\
25767.0567056706 0.00277641419149679\\
25775.697569757 0.00277689603555409\\
25784.3384338434 0.00277737779892214\\
25792.9792979298 0.00277785948164146\\
25801.6201620162 0.00277834108375253\\
25810.2610261026 0.00277882260529581\\
25818.901890189 0.00277930404631173\\
25827.5427542754 0.00277978540684066\\
25836.1836183618 0.00278026668692297\\
25844.8244824482 0.00278074788659897\\
25853.4653465347 0.00278122900590895\\
25862.1062106211 0.00278171004489317\\
25870.7470747075 0.00278219100359184\\
25879.3879387939 0.00278267188204514\\
25888.0288028803 0.00278315268029324\\
25896.6696669667 0.00278363339837624\\
25905.3105310531 0.00278411403633423\\
25913.9513951395 0.00278459459420728\\
25922.5922592259 0.00278507507203538\\
25931.2331233123 0.00278555546985853\\
25939.8739873987 0.00278603578771669\\
25948.5148514851 0.00278651602564976\\
25957.1557155716 0.00278699618369764\\
25965.796579658 0.00278747626190017\\
25974.4374437444 0.00278795626029718\\
25983.0783078308 0.00278843617892846\\
25991.7191719172 0.00278891601783374\\
26000.3600360036 0.00278939577705276\\
26009.00090009 0.00278987545662521\\
26017.6417641764 0.00279035505659073\\
26026.2826282628 0.00279083457698894\\
26034.9234923492 0.00279131401785944\\
26043.5643564356 0.00279179337924179\\
26052.2052205221 0.00279227266117549\\
26060.8460846085 0.00279275186370005\\
26069.4869486949 0.00279323098685492\\
26078.1278127813 0.00279371003067953\\
26086.7686768677 0.00279418899521326\\
26095.4095409541 0.00279466788049548\\
26104.0504050405 0.00279514668656551\\
26112.6912691269 0.00279562541346265\\
26121.3321332133 0.00279610406122616\\
26129.9729972997 0.00279658262989527\\
26138.6138613861 0.00279706111950918\\
26147.2547254725 0.00279753953010704\\
26155.895589559 0.002798017861728\\
26164.5364536454 0.00279849611441115\\
26173.1773177318 0.00279897428819556\\
26181.8181818182 0.00279945238312027\\
26190.4590459046 0.00279993039922428\\
26199.099909991 0.00280040833654656\\
26207.7407740774 0.00280088619512605\\
26216.3816381638 0.00280136397500166\\
26225.0225022502 0.00280184167621226\\
26233.6633663366 0.00280231929879669\\
26242.304230423 0.00280279684279377\\
26250.9450945095 0.00280327430824228\\
26259.5859585959 0.00280375169518095\\
26268.2268226823 0.00280422900364852\\
26276.8676867687 0.00280470623368366\\
26285.5085508551 0.00280518338532502\\
26294.1494149415 0.00280566045861123\\
26302.7902790279 0.00280613745358086\\
26311.4311431143 0.00280661437027249\\
26320.0720072007 0.00280709120872463\\
26328.7128712871 0.00280756796897578\\
26337.3537353735 0.0028080446510644\\
26345.9945994599 0.00280852125502892\\
26354.6354635464 0.00280899778090773\\
26363.2763276328 0.00280947422873922\\
26371.9171917192 0.0028099505985617\\
26380.5580558056 0.0028104268904135\\
26389.198919892 0.00281090310433287\\
26397.8397839784 0.00281137924035807\\
26406.4806480648 0.0028118552985273\\
26415.1215121512 0.00281233127887874\\
26423.7623762376 0.00281280718145055\\
26432.403240324 0.00281328300628084\\
26441.0441044104 0.00281375875340769\\
26449.6849684969 0.00281423442286916\\
26458.3258325833 0.00281471001470327\\
26466.9666966697 0.00281518552894803\\
26475.6075607561 0.00281566096564138\\
26484.2484248425 0.00281613632482126\\
26492.8892889289 0.00281661160652558\\
26501.5301530153 0.00281708681079219\\
26510.1710171017 0.00281756193765894\\
26518.8118811881 0.00281803698716364\\
26527.4527452745 0.00281851195934406\\
26536.0936093609 0.00281898685423795\\
26544.7344734473 0.00281946167188302\\
26553.3753375338 0.00281993641231697\\
26562.0162016202 0.00282041107557743\\
26570.6570657066 0.00282088566170205\\
26579.297929793 0.0028213601707284\\
26587.9387938794 0.00282183460269405\\
26596.5796579658 0.00282230895763654\\
26605.2205220522 0.00282278323559337\\
26613.8613861386 0.002823257436602\\
26622.502250225 0.00282373156069988\\
26631.1431143114 0.00282420560792441\\
26639.7839783978 0.00282467957831299\\
26648.4248424842 0.00282515347190295\\
26657.0657065707 0.00282562728873162\\
26665.7065706571 0.00282610102883628\\
26674.3474347435 0.0028265746922542\\
26682.9882988299 0.00282704827902261\\
26691.6291629163 0.00282752178917869\\
26700.2700270027 0.00282799522275963\\
26708.9108910891 0.00282846857980256\\
26717.5517551755 0.00282894186034459\\
26726.1926192619 0.0028294150644228\\
26734.8334833483 0.00282988819207424\\
26743.4743474347 0.00283036124333592\\
26752.1152115212 0.00283083421824483\\
26760.7560756076 0.00283130711683794\\
26769.396939694 0.00283177993915217\\
26778.0378037804 0.00283225268522443\\
26786.6786678668 0.00283272535509158\\
26795.3195319532 0.00283319794879046\\
26803.9603960396 0.00283367046635788\\
26812.601260126 0.00283414290783063\\
26821.2421242124 0.00283461527324545\\
26829.8829882988 0.00283508756263907\\
26838.5238523852 0.00283555977604818\\
26847.1647164716 0.00283603191350944\\
26855.8055805581 0.00283650397505948\\
26864.4464446445 0.00283697596073492\\
26873.0873087309 0.00283744787057231\\
26881.7281728173 0.00283791970460821\\
26890.3690369037 0.00283839146287914\\
26899.0099009901 0.00283886314542158\\
26907.6507650765 0.00283933475227198\\
26916.2916291629 0.00283980628346677\\
26924.9324932493 0.00284027773904236\\
26933.5733573357 0.00284074911903511\\
26942.2142214221 0.00284122042348136\\
26950.8550855086 0.00284169165241743\\
26959.495949595 0.00284216280587959\\
26968.1368136814 0.0028426338839041\\
26976.7776777678 0.00284310488652718\\
26985.4185418542 0.00284357581378503\\
26994.0594059406 0.00284404666571381\\
27002.700270027 0.00284451744234966\\
27011.3411341134 0.00284498814372869\\
27019.9819981998 0.00284545876988697\\
27028.6228622862 0.00284592932086057\\
27037.2637263726 0.00284639979668549\\
27045.904590459 0.00284687019739774\\
27054.5454545455 0.00284734052303327\\
27063.1863186319 0.00284781077362803\\
27071.8271827183 0.00284828094921792\\
27080.4680468047 0.00284875104983882\\
27089.1089108911 0.00284922107552657\\
27097.7497749775 0.002849691026317\\
27106.3906390639 0.00285016090224591\\
27115.0315031503 0.00285063070334905\\
27123.6723672367 0.00285110042966216\\
27132.3132313231 0.00285157008122095\\
27140.9540954095 0.0028520396580611\\
27149.5949594959 0.00285250916021825\\
27158.2358235824 0.00285297858772804\\
27166.8766876688 0.00285344794062605\\
27175.5175517552 0.00285391721894785\\
27184.1584158416 0.00285438642272897\\
27192.799279928 0.00285485555200493\\
27201.4401440144 0.00285532460681121\\
27210.0810081008 0.00285579358718325\\
27218.7218721872 0.00285626249315649\\
27227.3627362736 0.00285673132476632\\
27236.00360036 0.00285720008204811\\
27244.6444644464 0.0028576687650372\\
27253.2853285329 0.0028581373737689\\
27261.9261926193 0.0028586059082785\\
27270.5670567057 0.00285907436860124\\
27279.2079207921 0.00285954275477237\\
27287.8487848785 0.00286001106682708\\
27296.4896489649 0.00286047930480055\\
27305.1305130513 0.00286094746872792\\
27313.7713771377 0.0028614155586443\\
27322.4122412241 0.00286188357458479\\
27331.0531053105 0.00286235151658445\\
27339.6939693969 0.00286281938467831\\
27348.3348334833 0.00286328717890138\\
27356.9756975698 0.00286375489928864\\
27365.6165616562 0.00286422254587504\\
27374.2574257426 0.0028646901186955\\
27382.898289829 0.00286515761778492\\
27391.5391539154 0.00286562504317817\\
27400.1800180018 0.00286609239491008\\
27408.8208820882 0.00286655967301548\\
27417.4617461746 0.00286702687752915\\
27426.102610261 0.00286749400848585\\
27434.7434743474 0.00286796106592031\\
27443.3843384338 0.00286842804986723\\
27452.0252025203 0.0028688949603613\\
27460.6660666067 0.00286936179743715\\
27469.3069306931 0.00286982856112942\\
27477.9477947795 0.0028702952514727\\
27486.5886588659 0.00287076186850155\\
27495.2295229523 0.00287122841225052\\
27503.8703870387 0.00287169488275412\\
27512.5112511251 0.00287216128004684\\
27521.1521152115 0.00287262760416314\\
27529.7929792979 0.00287309385513745\\
27538.4338433843 0.00287356003300418\\
27547.0747074707 0.0028740261377977\\
27555.7155715572 0.00287449216955236\\
27564.3564356436 0.0028749581283025\\
27572.99729973 0.00287542401408241\\
27581.6381638164 0.00287588982692636\\
27590.2790279028 0.0028763555668686\\
27598.9198919892 0.00287682123394334\\
27607.5607560756 0.00287728682818477\\
27616.201620162 0.00287775234962706\\
27624.8424842484 0.00287821779830435\\
27633.4833483348 0.00287868317425074\\
27642.1242124212 0.00287914847750032\\
27650.7650765077 0.00287961370808715\\
27659.4059405941 0.00288007886604526\\
27668.0468046805 0.00288054395140865\\
27676.6876687669 0.0028810089642113\\
27685.3285328533 0.00288147390448716\\
27693.9693969397 0.00288193877227016\\
27702.6102610261 0.00288240356759419\\
27711.2511251125 0.00288286829049313\\
27719.8919891989 0.00288333294100082\\
27728.5328532853 0.00288379751915108\\
27737.1737173717 0.00288426202497771\\
27745.8145814581 0.00288472645851447\\
27754.4554455446 0.0028851908197951\\
27763.096309631 0.00288565510885331\\
27771.7371737174 0.0028861193257228\\
27780.3780378038 0.00288658347043722\\
27789.0189018902 0.00288704754303022\\
27797.6597659766 0.00288751154353539\\
27806.300630063 0.00288797547198633\\
27814.9414941494 0.00288843932841659\\
27823.5823582358 0.0028889031128597\\
27832.2232223222 0.00288936682534917\\
27840.8640864086 0.00288983046591847\\
27849.5049504951 0.00289029403460106\\
27858.1458145815 0.00289075753143037\\
27866.7866786679 0.0028912209564398\\
27875.4275427543 0.00289168430966272\\
27884.0684068407 0.00289214759113249\\
27892.7092709271 0.00289261080088243\\
27901.3501350135 0.00289307393894583\\
27909.9909990999 0.00289353700535596\\
27918.6318631863 0.00289400000014609\\
27927.2727272727 0.00289446292334942\\
27935.9135913591 0.00289492577499915\\
27944.5544554455 0.00289538855512845\\
27953.195319532 0.00289585126377047\\
27961.8361836184 0.00289631390095832\\
27970.4770477048 0.00289677646672509\\
27979.1179117912 0.00289723896110386\\
27987.7587758776 0.00289770138412766\\
27996.399639964 0.00289816373582952\\
28005.0405040504 0.00289862601624242\\
28013.6813681368 0.00289908822539932\\
28022.3222322232 0.00289955036333318\\
28030.9630963096 0.00290001243007689\\
28039.603960396 0.00290047442566336\\
28048.2448244824 0.00290093635012545\\
28056.8856885689 0.00290139820349599\\
28065.5265526553 0.0029018599858078\\
28074.1674167417 0.00290232169709366\\
28082.8082808281 0.00290278333738634\\
28091.4491449145 0.00290324490671858\\
28100.0900090009 0.0029037064051231\\
28108.7308730873 0.00290416783263256\\
28117.3717371737 0.00290462918927965\\
28126.0126012601 0.002905090475097\\
28134.6534653465 0.00290555169011722\\
28143.2943294329 0.00290601283437289\\
28151.9351935194 0.00290647390789658\\
28160.5760576058 0.00290693491072083\\
28169.2169216922 0.00290739584287815\\
28177.8577857786 0.00290785670440103\\
28186.498649865 0.00290831749532192\\
28195.1395139514 0.00290877821567327\\
28203.7803780378 0.0029092388654875\\
28212.4212421242 0.00290969944479698\\
28221.0621062106 0.00291015995363408\\
28229.702970297 0.00291062039203114\\
28238.3438343834 0.00291108076002048\\
28246.9846984698 0.00291154105763438\\
28255.6255625563 0.00291200128490511\\
28264.2664266427 0.00291246144186491\\
28272.9072907291 0.00291292152854599\\
28281.5481548155 0.00291338154498054\\
28290.1890189019 0.00291384149120074\\
28298.8298829883 0.00291430136723871\\
28307.4707470747 0.00291476117312659\\
28316.1116111611 0.00291522090889645\\
28324.7524752475 0.00291568057458037\\
28333.3933393339 0.0029161401702104\\
28342.0342034203 0.00291659969581855\\
28350.6750675068 0.00291705915143681\\
28359.3159315932 0.00291751853709717\\
28367.9567956796 0.00291797785283155\\
28376.597659766 0.0029184370986719\\
28385.2385238524 0.00291889627465009\\
28393.8793879388 0.00291935538079801\\
28402.5202520252 0.00291981441714751\\
28411.1611161116 0.00292027338373041\\
28419.801980198 0.0029207322805785\\
28428.4428442844 0.00292119110772358\\
28437.0837083708 0.00292164986519739\\
28445.7245724572 0.00292210855303165\\
28454.3654365437 0.00292256717125808\\
28463.0063006301 0.00292302571990835\\
28471.6471647165 0.00292348419901412\\
28480.2880288029 0.00292394260860702\\
28488.9288928893 0.00292440094871865\\
28497.5697569757 0.00292485921938062\\
28506.2106210621 0.00292531742062447\\
28514.8514851485 0.00292577555248173\\
28523.4923492349 0.00292623361498394\\
28532.1332133213 0.00292669160816256\\
28540.7740774077 0.00292714953204907\\
28549.4149414941 0.00292760738667491\\
28558.0558055806 0.00292806517207149\\
28566.696669667 0.00292852288827021\\
28575.3375337534 0.00292898053530244\\
28583.9783978398 0.00292943811319953\\
28592.6192619262 0.0029298956219928\\
28601.2601260126 0.00293035306171354\\
28609.900990099 0.00293081043239303\\
28618.5418541854 0.00293126773406252\\
28627.1827182718 0.00293172496675325\\
28635.8235823582 0.00293218213049641\\
28644.4644464446 0.0029326392253232\\
28653.1053105311 0.00293309625126475\\
28661.7461746175 0.00293355320835221\\
28670.3870387039 0.0029340100966167\\
28679.0279027903 0.00293446691608929\\
28687.6687668767 0.00293492366680105\\
28696.3096309631 0.00293538034878302\\
28704.9504950495 0.00293583696206622\\
28713.5913591359 0.00293629350668164\\
28722.2322232223 0.00293674998266025\\
28730.8730873087 0.002937206390033\\
28739.5139513951 0.00293766272883082\\
28748.1548154815 0.0029381189990846\\
28756.795679568 0.00293857520082522\\
28765.4365436544 0.00293903133408354\\
28774.0774077408 0.00293948739889038\\
28782.7182718272 0.00293994339527656\\
28791.3591359136 0.00294039932327286\\
28800 0.00294085518291004\\
28808.6408640864 0.00294131097421884\\
28817.2817281728 0.00294176669722998\\
28825.9225922592 0.00294222235197415\\
28834.5634563456 0.00294267793848201\\
28843.204320432 0.00294313345678423\\
28851.8451845185 0.00294358890691141\\
28860.4860486049 0.00294404428889416\\
28869.1269126913 0.00294449960276307\\
28877.7677767777 0.00294495484854867\\
28886.4086408641 0.00294541002628152\\
28895.0495049505 0.00294586513599211\\
28903.6903690369 0.00294632017771093\\
28912.3312331233 0.00294677515146845\\
28920.9720972097 0.00294723005729512\\
28929.6129612961 0.00294768489522133\\
28938.2538253825 0.0029481396652775\\
28946.8946894689 0.002948594367494\\
28955.5355535554 0.00294904900190117\\
28964.1764176418 0.00294950356852935\\
28972.8172817282 0.00294995806740884\\
28981.4581458146 0.00295041249856992\\
28990.099009901 0.00295086686204286\\
28998.7398739874 0.00295132115785789\\
29007.3807380738 0.00295177538604522\\
29016.0216021602 0.00295222954663505\\
29024.6624662466 0.00295268363965755\\
29033.303330333 0.00295313766514287\\
29041.9441944194 0.00295359162312113\\
29050.5850585059 0.00295404551362243\\
29059.2259225923 0.00295449933667686\\
29067.8667866787 0.00295495309231448\\
29076.5076507651 0.00295540678056531\\
29085.1485148515 0.00295586040145938\\
29093.7893789379 0.00295631395502668\\
29102.4302430243 0.00295676744129717\\
29111.0711071107 0.0029572208603008\\
29119.7119711971 0.0029576742120675\\
29128.3528352835 0.00295812749662716\\
29136.9936993699 0.00295858071400968\\
29145.6345634563 0.0029590338642449\\
29154.2754275428 0.00295948694736266\\
29162.9162916292 0.00295993996339279\\
29171.5571557156 0.00296039291236506\\
29180.198019802 0.00296084579430925\\
29188.8388838884 0.00296129860925511\\
29197.4797479748 0.00296175135723237\\
29206.1206120612 0.00296220403827072\\
29214.7614761476 0.00296265665239985\\
29223.402340234 0.00296310919964942\\
29232.0432043204 0.00296356168004907\\
29240.6840684068 0.00296401409362841\\
29249.3249324932 0.00296446644041704\\
29257.9657965797 0.00296491872044453\\
29266.6066606661 0.00296537093374043\\
29275.2475247525 0.00296582308033427\\
29283.8883888389 0.00296627516025556\\
29292.5292529253 0.00296672717353378\\
29301.1701170117 0.0029671791201984\\
29309.8109810981 0.00296763100027886\\
29318.4518451845 0.00296808281380458\\
29327.0927092709 0.00296853456080496\\
29335.7335733573 0.00296898624130937\\
29344.3744374437 0.00296943785534718\\
29353.0153015302 0.0029698894029477\\
29361.6561656166 0.00297034088414027\\
29370.297029703 0.00297079229895416\\
29378.9378937894 0.00297124364741864\\
29387.5787578758 0.00297169492956298\\
29396.2196219622 0.00297214614541638\\
29404.8604860486 0.00297259729500805\\
29413.501350135 0.00297304837836719\\
29422.1422142214 0.00297349939552295\\
29430.7830783078 0.00297395034650446\\
29439.4239423942 0.00297440123134086\\
29448.0648064806 0.00297485205006123\\
29456.7056705671 0.00297530280269466\\
29465.3465346535 0.0029757534892702\\
29473.9873987399 0.00297620410981688\\
29482.6282628263 0.00297665466436372\\
29491.2691269127 0.0029771051529397\\
29499.9099909991 0.0029775555755738\\
29508.5508550855 0.00297800593229497\\
29517.1917191719 0.00297845622313213\\
29525.8325832583 0.0029789064481142\\
29534.4734473447 0.00297935660727005\\
29543.1143114311 0.00297980670062854\\
29551.7551755176 0.00298025672821854\\
29560.396039604 0.00298070669006885\\
29569.0369036904 0.00298115658620827\\
29577.6777677768 0.0029816064166656\\
29586.3186318632 0.00298205618146958\\
29594.9594959496 0.00298250588064895\\
29603.600360036 0.00298295551423244\\
29612.2412241224 0.00298340508224873\\
29620.8820882088 0.00298385458472651\\
29629.5229522952 0.00298430402169443\\
29638.1638163816 0.00298475339318111\\
29646.804680468 0.00298520269921519\\
29655.4455445545 0.00298565193982523\\
29664.0864086409 0.00298610111503983\\
29672.7272727273 0.00298655022488752\\
29681.3681368137 0.00298699926939684\\
29690.0090009001 0.00298744824859629\\
29698.6498649865 0.00298789716251437\\
29707.2907290729 0.00298834601117954\\
29715.9315931593 0.00298879479462025\\
29724.5724572457 0.00298924351286492\\
29733.2133213321 0.00298969216594197\\
29741.8541854185 0.00299014075387976\\
29750.4950495049 0.00299058927670667\\
29759.1359135914 0.00299103773445105\\
29767.7767776778 0.00299148612714121\\
29776.4176417642 0.00299193445480546\\
29785.0585058506 0.00299238271747207\\
29793.699369937 0.00299283091516932\\
29802.3402340234 0.00299327904792544\\
29810.9810981098 0.00299372711576865\\
29819.6219621962 0.00299417511872715\\
29828.2628262826 0.00299462305682913\\
29836.903690369 0.00299507093010273\\
29845.5445544554 0.00299551873857611\\
29854.1854185419 0.00299596648227737\\
29862.8262826283 0.00299641416123462\\
29871.4671467147 0.00299686177547594\\
29880.1080108011 0.00299730932502938\\
29888.7488748875 0.00299775680992298\\
29897.3897389739 0.00299820423018475\\
29906.0306030603 0.0029986515858427\\
29914.6714671467 0.0029990988769248\\
29923.3123312331 0.00299954610345901\\
29931.9531953195 0.00299999326547326\\
29940.5940594059 0.00300044036299547\\
29949.2349234923 0.00300088739605354\\
29957.8757875788 0.00300133436467533\\
29966.5166516652 0.00300178126888872\\
29975.1575157516 0.00300222810872153\\
29983.798379838 0.00300267488420157\\
29992.4392439244 0.00300312159535666\\
30001.0801080108 0.00300356824221455\\
30009.7209720972 0.00300401482480301\\
30018.3618361836 0.00300446134314977\\
30027.00270027 0.00300490779728255\\
30035.6435643564 0.00300535418722904\\
30044.2844284428 0.00300580051301693\\
30052.9252925293 0.00300624677467386\\
30061.5661566157 0.00300669297222747\\
30070.2070207021 0.00300713910570538\\
30078.8478847885 0.00300758517513519\\
30087.4887488749 0.00300803118054447\\
30096.1296129613 0.00300847712196077\\
30104.7704770477 0.00300892299941165\\
30113.4113411341 0.00300936881292461\\
30122.0522052205 0.00300981456252715\\
30130.6930693069 0.00301026024824675\\
30139.3339333933 0.00301070587011087\\
30147.9747974797 0.00301115142814694\\
30156.6156615662 0.0030115969223824\\
30165.2565256526 0.00301204235284462\\
30173.897389739 0.00301248771956101\\
30182.5382538254 0.00301293302255891\\
30191.1791179118 0.00301337826186566\\
30199.8199819982 0.0030138234375086\\
30208.4608460846 0.00301426854951501\\
30217.101710171 0.00301471359791218\\
30225.7425742574 0.00301515858272738\\
30234.3834383438 0.00301560350398785\\
30243.0243024302 0.0030160483617208\\
30251.6651665167 0.00301649315595346\\
30260.3060306031 0.00301693788671299\\
30268.9468946895 0.00301738255402657\\
30277.5877587759 0.00301782715792134\\
30286.2286228623 0.00301827169842443\\
30294.8694869487 0.00301871617556294\\
30303.5103510351 0.00301916058936397\\
30312.1512151215 0.00301960493985458\\
30320.7920792079 0.00302004922706182\\
30329.4329432943 0.00302049345101273\\
30338.0738073807 0.00302093761173431\\
30346.7146714671 0.00302138170925356\\
30355.3555355536 0.00302182574359744\\
30363.99639964 0.00302226971479292\\
30372.6372637264 0.00302271362286692\\
30381.2781278128 0.00302315746784636\\
30389.9189918992 0.00302360124975814\\
30398.5598559856 0.00302404496862914\\
30407.200720072 0.00302448862448621\\
30415.8415841584 0.00302493221735618\\
30424.4824482448 0.00302537574726589\\
30433.1233123312 0.00302581921424213\\
30441.7641764176 0.00302626261831169\\
30450.4050405041 0.00302670595950132\\
30459.0459045905 0.00302714923783776\\
30467.6867686769 0.00302759245334776\\
30476.3276327633 0.003028035606058\\
30484.9684968497 0.00302847869599518\\
30493.6093609361 0.00302892172318596\\
30502.2502250225 0.003029364687657\\
30510.8910891089 0.00302980758943492\\
30519.5319531953 0.00303025042854634\\
30528.1728172817 0.00303069320501785\\
30536.8136813681 0.00303113591887602\\
30545.4545454545 0.00303157857014741\\
30554.095409541 0.00303202115885855\\
30562.7362736274 0.00303246368503597\\
30571.3771377138 0.00303290614870615\\
30580.0180018002 0.00303334854989559\\
30588.6588658866 0.00303379088863073\\
30597.299729973 0.00303423316493803\\
30605.9405940594 0.00303467537884391\\
30614.5814581458 0.00303511753037477\\
30623.2223222322 0.00303555961955701\\
30631.8631863186 0.00303600164641698\\
30640.504050405 0.00303644361098104\\
30649.1449144914 0.00303688551327553\\
30657.7857785779 0.00303732735332674\\
30666.4266426643 0.00303776913116098\\
30675.0675067507 0.00303821084680453\\
30683.7083708371 0.00303865250028363\\
30692.3492349235 0.00303909409162453\\
30700.9900990099 0.00303953562085345\\
30709.6309630963 0.00303997708799658\\
30718.2718271827 0.00304041849308012\\
30726.9126912691 0.00304085983613023\\
30735.5535553555 0.00304130111717305\\
30744.1944194419 0.00304174233623472\\
30752.8352835284 0.00304218349334133\\
30761.4761476148 0.00304262458851899\\
30770.1170117012 0.00304306562179377\\
30778.7578757876 0.00304350659319171\\
30787.398739874 0.00304394750273887\\
30796.0396039604 0.00304438835046125\\
30804.6804680468 0.00304482913638486\\
30813.3213321332 0.00304526986053567\\
30821.9621962196 0.00304571052293966\\
30830.603060306 0.00304615112362276\\
30839.2439243924 0.00304659166261091\\
30847.8847884788 0.00304703213993002\\
30856.5256525653 0.00304747255560596\\
30865.1665166517 0.00304791290966463\\
30873.8073807381 0.00304835320213187\\
30882.4482448245 0.00304879343303352\\
30891.0891089109 0.00304923360239539\\
30899.7299729973 0.0030496737102433\\
30908.3708370837 0.00305011375660301\\
30917.0117011701 0.0030505537415003\\
30925.6525652565 0.00305099366496092\\
30934.2934293429 0.00305143352701059\\
30942.9342934293 0.00305187332767501\\
30951.5751575158 0.0030523130669799\\
30960.2160216022 0.00305275274495091\\
30968.8568856886 0.00305319236161371\\
30977.497749775 0.00305363191699393\\
30986.1386138614 0.00305407141111721\\
30994.7794779478 0.00305451084400913\\
31003.4203420342 0.00305495021569529\\
31012.0612061206 0.00305538952620125\\
31020.702070207 0.00305582877555257\\
31029.3429342934 0.00305626796377477\\
31037.9837983798 0.00305670709089337\\
31046.6246624662 0.00305714615693387\\
31055.2655265527 0.00305758516192174\\
31063.9063906391 0.00305802410588245\\
31072.5472547255 0.00305846298884144\\
31081.1881188119 0.00305890181082414\\
31089.8289828983 0.00305934057185595\\
31098.4698469847 0.00305977927196227\\
31107.1107110711 0.00306021791116846\\
31115.7515751575 0.00306065648949989\\
31124.3924392439 0.00306109500698189\\
31133.0333033303 0.00306153346363978\\
31141.6741674167 0.00306197185949886\\
31150.3150315031 0.00306241019458442\\
31158.9558955896 0.00306284846892173\\
31167.596759676 0.00306328668253603\\
31176.2376237624 0.00306372483545257\\
31184.8784878488 0.00306416292769655\\
31193.5193519352 0.00306460095929316\\
31202.1602160216 0.0030650389302676\\
31210.801080108 0.00306547684064503\\
31219.4419441944 0.00306591469045058\\
31228.0828082808 0.0030663524797094\\
31236.7236723672 0.00306679020844658\\
31245.3645364536 0.00306722787668722\\
31254.0054005401 0.0030676654844564\\
31262.6462646265 0.00306810303177918\\
31271.2871287129 0.00306854051868059\\
31279.9279927993 0.00306897794518567\\
31288.5688568857 0.00306941531131941\\
31297.2097209721 0.00306985261710681\\
31305.8505850585 0.00307028986257285\\
31314.4914491449 0.00307072704774246\\
31323.1323132313 0.0030711641726406\\
31331.7731773177 0.00307160123729219\\
31340.4140414041 0.00307203824172212\\
31349.0549054905 0.00307247518595528\\
31357.695769577 0.00307291207001655\\
31366.3366336634 0.00307334889393077\\
31374.9774977498 0.00307378565772278\\
31383.6183618362 0.0030742223614174\\
31392.2592259226 0.00307465900503943\\
31400.900090009 0.00307509558861365\\
31409.5409540954 0.00307553211216483\\
31418.1818181818 0.00307596857571772\\
31426.8226822682 0.00307640497929704\\
31435.4635463546 0.00307684132292753\\
31444.104410441 0.00307727760663387\\
31452.7452745275 0.00307771383044075\\
31461.3861386139 0.00307814999437283\\
31470.0270027003 0.00307858609845477\\
31478.6678667867 0.00307902214271118\\
31487.3087308731 0.00307945812716669\\
31495.9495949595 0.00307989405184589\\
31504.5904590459 0.00308032991677337\\
31513.2313231323 0.00308076572197368\\
31521.8721872187 0.00308120146747137\\
31530.5130513051 0.00308163715329098\\
31539.1539153915 0.00308207277945702\\
31547.7947794779 0.00308250834599398\\
31556.4356435644 0.00308294385292634\\
31565.0765076508 0.00308337930027857\\
31573.7173717372 0.0030838146880751\\
31582.3582358236 0.00308425001634038\\
31590.99909991 0.00308468528509882\\
31599.6399639964 0.0030851204943748\\
31608.2808280828 0.00308555564419271\\
31616.9216921692 0.00308599073457692\\
31625.5625562556 0.00308642576555176\\
31634.203420342 0.00308686073714157\\
31642.8442844284 0.00308729564937066\\
31651.4851485149 0.00308773050226332\\
31660.1260126013 0.00308816529584384\\
31668.7668766877 0.00308860003013648\\
31677.4077407741 0.00308903470516549\\
31686.0486048605 0.0030894693209551\\
31694.6894689469 0.00308990387752951\\
31703.3303330333 0.00309033837491293\\
31711.9711971197 0.00309077281312954\\
31720.6120612061 0.0030912071922035\\
31729.2529252925 0.00309164151215896\\
31737.8937893789 0.00309207577302005\\
31746.5346534653 0.00309250997481088\\
31755.1755175518 0.00309294411755556\\
31763.8163816382 0.00309337820127817\\
31772.4572457246 0.00309381222600277\\
31781.098109811 0.0030942461917534\\
31789.7389738974 0.00309468009855412\\
31798.3798379838 0.00309511394642892\\
31807.0207020702 0.00309554773540182\\
31815.6615661566 0.00309598146549679\\
31824.302430243 0.00309641513673781\\
31832.9432943294 0.00309684874914882\\
31841.5841584158 0.00309728230275376\\
31850.2250225023 0.00309771579757655\\
31858.8658865887 0.00309814923364109\\
31867.5067506751 0.00309858261097127\\
31876.1476147615 0.00309901592959097\\
31884.7884788479 0.00309944918952402\\
31893.4293429343 0.00309988239079428\\
31902.0702070207 0.00310031553342557\\
31910.7110711071 0.00310074861744168\\
31919.3519351935 0.00310118164286642\\
31927.9927992799 0.00310161460972355\\
31936.6336633663 0.00310204751803683\\
31945.2745274527 0.00310248036783001\\
31953.9153915392 0.00310291315912681\\
31962.5562556256 0.00310334589195093\\
31971.197119712 0.00310377856632608\\
31979.8379837984 0.00310421118227593\\
31988.4788478848 0.00310464373982414\\
31997.1197119712 0.00310507623899436\\
32005.7605760576 0.00310550867981021\\
32014.401440144 0.00310594106229532\\
32023.0423042304 0.00310637338647328\\
32031.6831683168 0.00310680565236766\\
32040.3240324032 0.00310723786000205\\
32048.9648964896 0.00310767000939998\\
32057.6057605761 0.00310810210058499\\
32066.2466246625 0.0031085341335806\\
32074.8874887489 0.00310896610841031\\
32083.5283528353 0.00310939802509762\\
32092.1692169217 0.00310982988366598\\
32100.8100810081 0.00311026168413886\\
32109.4509450945 0.00311069342653969\\
32118.0918091809 0.00311112511089191\\
32126.7326732673 0.00311155673721891\\
32135.3735373537 0.00311198830554408\\
32144.0144014401 0.00311241981589082\\
32152.6552655266 0.00311285126828247\\
32161.296129613 0.00311328266274238\\
32169.9369936994 0.00311371399929388\\
32178.5778577858 0.00311414527796028\\
32187.2187218722 0.00311457649876489\\
32195.8595859586 0.00311500766173099\\
32204.500450045 0.00311543876688183\\
32213.1413141314 0.00311586981424068\\
32221.7821782178 0.00311630080383077\\
32230.4230423042 0.00311673173567532\\
32239.0639063906 0.00311716260979753\\
32247.704770477 0.0031175934262206\\
32256.3456345635 0.00311802418496769\\
32264.9864986499 0.00311845488606197\\
32273.6273627363 0.00311888552952657\\
32282.2682268227 0.00311931611538463\\
32290.9090909091 0.00311974664365925\\
32299.5499549955 0.00312017711437354\\
32308.1908190819 0.00312060752755057\\
32316.8316831683 0.00312103788321341\\
32325.4725472547 0.0031214681813851\\
32334.1134113411 0.00312189842208869\\
32342.7542754275 0.0031223286053472\\
32351.395139514 0.00312275873118361\\
32360.0360036004 0.00312318879962093\\
32368.6768676868 0.00312361881068213\\
32377.3177317732 0.00312404876439017\\
32385.9585958596 0.00312447866076798\\
32394.599459946 0.00312490849983849\\
32403.2403240324 0.00312533828162463\\
32411.8811881188 0.00312576800614927\\
32420.5220522052 0.00312619767343531\\
32429.1629162916 0.00312662728350561\\
32437.803780378 0.00312705683638303\\
32446.4446444644 0.00312748633209039\\
32455.0855085509 0.00312791577065051\\
32463.7263726373 0.00312834515208622\\
32472.3672367237 0.00312877447642028\\
32481.0081008101 0.00312920374367549\\
32489.6489648965 0.00312963295387459\\
32498.2898289829 0.00313006210704034\\
32506.9306930693 0.00313049120319546\\
32515.5715571557 0.00313092024236266\\
32524.2124212421 0.00313134922456466\\
32532.8532853285 0.00313177814982412\\
32541.4941494149 0.00313220701816373\\
32550.1350135013 0.00313263582960613\\
32558.7758775878 0.00313306458417397\\
32567.4167416742 0.00313349328188986\\
32576.0576057606 0.00313392192277643\\
32584.698469847 0.00313435050685625\\
32593.3393339334 0.00313477903415192\\
32601.9801980198 0.003135207504686\\
32610.6210621062 0.00313563591848103\\
32619.2619261926 0.00313606427555955\\
32627.902790279 0.00313649257594408\\
32636.5436543654 0.00313692081965713\\
32645.1845184518 0.00313734900672118\\
32653.8253825383 0.00313777713715872\\
32662.4662466247 0.00313820521099219\\
32671.1071107111 0.00313863322824404\\
32679.7479747975 0.00313906118893672\\
32688.3888388839 0.00313948909309262\\
32697.0297029703 0.00313991694073416\\
32705.6705670567 0.00314034473188371\\
32714.3114311431 0.00314077246656366\\
32722.9522952295 0.00314120014479635\\
32731.5931593159 0.00314162776660412\\
32740.2340234023 0.00314205533200932\\
32748.8748874887 0.00314248284103424\\
32757.5157515752 0.00314291029370118\\
32766.1566156616 0.00314333769003243\\
32774.797479748 0.00314376503005026\\
32783.4383438344 0.00314419231377692\\
32792.0792079208 0.00314461954123465\\
32800.7200720072 0.00314504671244566\\
32809.3609360936 0.00314547382743219\\
32818.00180018 0.0031459008862164\\
32826.6426642664 0.0031463278888205\\
32835.2835283528 0.00314675483526664\\
32843.9243924392 0.00314718172557697\\
32852.5652565257 0.00314760855977364\\
32861.2061206121 0.00314803533787876\\
32869.8469846985 0.00314846205991444\\
32878.4878487849 0.00314888872590278\\
32887.1287128713 0.00314931533586585\\
32895.7695769577 0.00314974188982572\\
32904.4104410441 0.00315016838780443\\
32913.0513051305 0.00315059482982403\\
32921.6921692169 0.00315102121590652\\
32930.3330333033 0.00315144754607393\\
32938.9738973897 0.00315187382034824\\
32947.6147614762 0.00315230003875143\\
32956.2556255626 0.00315272620130546\\
32964.896489649 0.00315315230803228\\
32973.5373537354 0.00315357835895382\\
32982.1782178218 0.00315400435409201\\
32990.8190819082 0.00315443029346875\\
32999.4599459946 0.00315485617710593\\
33008.100810081 0.00315528200502543\\
33016.7416741674 0.00315570777724911\\
33025.3825382538 0.00315613349379882\\
33034.0234023402 0.00315655915469639\\
33042.6642664266 0.00315698475996365\\
33051.3051305131 0.00315741030962239\\
33059.9459945995 0.00315783580369442\\
33068.5868586859 0.0031582612422015\\
33077.2277227723 0.0031586866251654\\
33085.8685868587 0.00315911195260787\\
33094.5094509451 0.00315953722455064\\
33103.1503150315 0.00315996244101544\\
33111.7911791179 0.00316038760202396\\
33120.4320432043 0.00316081270759791\\
33129.0729072907 0.00316123775775895\\
33137.7137713771 0.00316166275252876\\
33146.3546354635 0.00316208769192898\\
33154.99549955 0.00316251257598125\\
33163.6363636364 0.00316293740470719\\
33172.2772277228 0.00316336217812841\\
33180.9180918092 0.00316378689626649\\
33189.5589558956 0.00316421155914303\\
33198.199819982 0.00316463616677958\\
33206.8406840684 0.00316506071919769\\
33215.4815481548 0.00316548521641891\\
33224.1224122412 0.00316590965846476\\
33232.7632763276 0.00316633404535673\\
33241.404140414 0.00316675837711634\\
33250.0450045004 0.00316718265376507\\
33258.6858685869 0.00316760687532436\\
33267.3267326733 0.0031680310418157\\
33275.9675967597 0.0031684551532605\\
33284.6084608461 0.0031688792096802\\
33293.2493249325 0.00316930321109621\\
33301.8901890189 0.00316972715752993\\
33310.5310531053 0.00317015104900274\\
33319.1719171917 0.003170574885536\\
33327.8127812781 0.00317099866715109\\
33336.4536453645 0.00317142239386933\\
33345.0945094509 0.00317184606571206\\
33353.7353735374 0.0031722696827006\\
33362.3762376238 0.00317269324485624\\
33371.0171017102 0.00317311675220026\\
33379.6579657966 0.00317354020475396\\
33388.298829883 0.00317396360253857\\
33396.9396939694 0.00317438694557536\\
33405.5805580558 0.00317481023388554\\
33414.2214221422 0.00317523346749035\\
33422.8622862286 0.00317565664641098\\
33431.503150315 0.00317607977066863\\
33440.1440144014 0.00317650284028447\\
33448.7848784879 0.00317692585527966\\
33457.4257425743 0.00317734881567537\\
33466.0666066607 0.00317777172149271\\
33474.7074707471 0.00317819457275282\\
33483.3483348335 0.00317861736947681\\
33491.9891989199 0.00317904011168577\\
33500.6300630063 0.00317946279940078\\
33509.2709270927 0.00317988543264291\\
33517.9117911791 0.00318030801143322\\
33526.5526552655 0.00318073053579274\\
33535.1935193519 0.00318115300574251\\
33543.8343834383 0.00318157542130353\\
33552.4752475248 0.00318199778249682\\
33561.1161116112 0.00318242008934335\\
33569.7569756976 0.0031828423418641\\
33578.397839784 0.00318326454008003\\
33587.0387038704 0.00318368668401209\\
33595.6795679568 0.0031841087736812\\
33604.3204320432 0.0031845308091083\\
33612.9612961296 0.00318495279031428\\
33621.602160216 0.00318537471732004\\
33630.2430243024 0.00318579659014646\\
33638.8838883888 0.0031862184088144\\
33647.5247524752 0.00318664017334471\\
33656.1656165617 0.00318706188375825\\
33664.8064806481 0.00318748354007582\\
33673.4473447345 0.00318790514231825\\
33682.0882088209 0.00318832669050633\\
33690.7290729073 0.00318874818466085\\
33699.3699369937 0.00318916962480258\\
33708.0108010801 0.00318959101095229\\
33716.6516651665 0.00319001234313071\\
33725.2925292529 0.00319043362135858\\
33733.9333933393 0.00319085484565661\\
33742.5742574257 0.00319127601604553\\
33751.2151215121 0.00319169713254601\\
33759.8559855986 0.00319211819517874\\
33768.496849685 0.00319253920396438\\
33777.1377137714 0.00319296015892359\\
33785.7785778578 0.00319338106007701\\
33794.4194419442 0.00319380190744526\\
33803.0603060306 0.00319422270104896\\
33811.701170117 0.0031946434409087\\
33820.3420342034 0.00319506412704509\\
33828.9828982898 0.00319548475947868\\
33837.6237623762 0.00319590533823004\\
33846.2646264626 0.00319632586331972\\
33854.9054905491 0.00319674633476826\\
33863.5463546355 0.00319716675259617\\
33872.1872187219 0.00319758711682396\\
33880.8280828083 0.00319800742747214\\
33889.4689468947 0.00319842768456118\\
33898.1098109811 0.00319884788811154\\
33906.7506750675 0.0031992680381437\\
33915.3915391539 0.00319968813467809\\
33924.0324032403 0.00320010817773514\\
33932.6732673267 0.00320052816733526\\
33941.3141314131 0.00320094810349888\\
33949.9549954996 0.00320136798624637\\
33958.595859586 0.00320178781559811\\
33967.2367236724 0.00320220759157447\\
33975.8775877588 0.0032026273141958\\
33984.5184518452 0.00320304698348244\\
33993.1593159316 0.00320346659945473\\
34001.800180018 0.00320388616213296\\
34010.4410441044 0.00320430567153745\\
34019.0819081908 0.00320472512768848\\
34027.7227722772 0.00320514453060632\\
34036.3636363636 0.00320556388031125\\
34045.00450045 0.0032059831768235\\
34053.6453645365 0.00320640242016332\\
34062.2862286229 0.00320682161035092\\
34070.9270927093 0.00320724074740652\\
34079.5679567957 0.00320765983135033\\
34088.2088208821 0.00320807886220251\\
34096.8496849685 0.00320849783998325\\
34105.4905490549 0.00320891676471271\\
34114.1314131413 0.00320933563641103\\
34122.7722772277 0.00320975445509834\\
34131.4131413141 0.00321017322079478\\
34140.0540054005 0.00321059193352044\\
34148.6948694869 0.00321101059329542\\
34157.3357335734 0.00321142920013982\\
34165.9765976598 0.00321184775407369\\
34174.6174617462 0.00321226625511709\\
34183.2583258326 0.00321268470329008\\
34191.899189919 0.00321310309861268\\
34200.5400540054 0.00321352144110491\\
34209.1809180918 0.00321393973078679\\
34217.8217821782 0.0032143579676783\\
34226.4626462646 0.00321477615179943\\
34235.103510351 0.00321519428317015\\
34243.7443744374 0.00321561236181041\\
34252.3852385238 0.00321603038774016\\
34261.0261026103 0.00321644836097933\\
34269.6669666967 0.00321686628154785\\
34278.3078307831 0.00321728414946561\\
34286.9486948695 0.00321770196475251\\
34295.5895589559 0.00321811972742844\\
34304.2304230423 0.00321853743751326\\
34312.8712871287 0.00321895509502683\\
34321.5121512151 0.00321937269998899\\
34330.1530153015 0.00321979025241958\\
34338.7938793879 0.00322020775233841\\
34347.4347434743 0.00322062519976529\\
34356.0756075608 0.00322104259472002\\
34364.7164716472 0.00322145993722237\\
34373.3573357336 0.00322187722729212\\
34381.99819982 0.00322229446494903\\
34390.6390639064 0.00322271165021284\\
34399.2799279928 0.00322312878310327\\
34407.9207920792 0.00322354586364006\\
34416.5616561656 0.0032239628918429\\
34425.202520252 0.0032243798677315\\
34433.8433843384 0.00322479679132553\\
34442.4842484248 0.00322521366264467\\
34451.1251125113 0.00322563048170857\\
34459.7659765977 0.00322604724853689\\
34468.4068406841 0.00322646396314925\\
34477.0477047705 0.00322688062556527\\
34485.6885688569 0.00322729723580457\\
34494.3294329433 0.00322771379388674\\
34502.9702970297 0.00322813029983136\\
34511.6111611161 0.00322854675365801\\
34520.2520252025 0.00322896315538625\\
34528.8928892889 0.00322937950503562\\
34537.5337533753 0.00322979580262566\\
34546.1746174617 0.0032302120481759\\
34554.8154815482 0.00323062824170584\\
34563.4563456346 0.00323104438323498\\
};
\addplot [
color=red,
solid,
forget plot
]
table[row sep=crcr]{
34563.4563456346 0.00323104438323498\\
34572.097209721 0.00323146047278282\\
34580.7380738074 0.00323187651036882\\
34589.3789378938 0.00323229249601244\\
34598.0198019802 0.00323270842973314\\
34606.6606660666 0.00323312431155036\\
34615.301530153 0.00323354014148351\\
34623.9423942394 0.00323395591955202\\
34632.5832583258 0.00323437164577529\\
34641.2241224122 0.00323478732017271\\
34649.8649864987 0.00323520294276364\\
34658.5058505851 0.00323561851356747\\
34667.1467146715 0.00323603403260354\\
34675.7875787579 0.00323644949989119\\
34684.4284428443 0.00323686491544975\\
34693.0693069307 0.00323728027929855\\
34701.7101710171 0.00323769559145688\\
34710.3510351035 0.00323811085194404\\
34718.9918991899 0.00323852606077931\\
34727.6327632763 0.00323894121798195\\
34736.2736273627 0.00323935632357124\\
34744.9144914491 0.0032397713775664\\
34753.5553555356 0.00324018637998669\\
34762.196219622 0.0032406013308513\\
34770.8370837084 0.00324101623017947\\
34779.4779477948 0.00324143107799037\\
34788.1188118812 0.00324184587430321\\
34796.7596759676 0.00324226061913715\\
34805.400540054 0.00324267531251135\\
34814.0414041404 0.00324308995444497\\
34822.6822682268 0.00324350454495714\\
34831.3231323132 0.00324391908406699\\
34839.9639963996 0.00324433357179363\\
34848.604860486 0.00324474800815617\\
34857.2457245725 0.0032451623931737\\
34865.8865886589 0.00324557672686529\\
34874.5274527453 0.00324599100925001\\
34883.1683168317 0.00324640524034692\\
34891.8091809181 0.00324681942017506\\
34900.4500450045 0.00324723354875346\\
34909.0909090909 0.00324764762610115\\
34917.7317731773 0.00324806165223713\\
34926.3726372637 0.00324847562718039\\
34935.0135013501 0.00324888955094992\\
34943.6543654365 0.0032493034235647\\
34952.295229523 0.00324971724504369\\
34960.9360936094 0.00325013101540583\\
34969.5769576958 0.00325054473467006\\
34978.2178217822 0.00325095840285531\\
34986.8586858686 0.0032513720199805\\
34995.499549955 0.00325178558606452\\
35004.1404140414 0.00325219910112627\\
35012.7812781278 0.00325261256518463\\
35021.4221422142 0.00325302597825846\\
35030.0630063006 0.00325343934036662\\
35038.703870387 0.00325385265152795\\
35047.3447344734 0.00325426591176129\\
35055.9855985599 0.00325467912108547\\
35064.6264626463 0.00325509227951928\\
35073.2673267327 0.00325550538708153\\
35081.9081908191 0.003255918443791\\
35090.5490549055 0.00325633144966647\\
35099.1899189919 0.0032567444047267\\
35107.8307830783 0.00325715730899045\\
35116.4716471647 0.00325757016247645\\
35125.1125112511 0.00325798296520344\\
35133.7533753375 0.00325839571719013\\
35142.3942394239 0.00325880841845522\\
35151.0351035104 0.00325922106901741\\
35159.6759675968 0.00325963366889539\\
35168.3168316832 0.00326004621810782\\
35176.9576957696 0.00326045871667337\\
35185.598559856 0.00326087116461069\\
35194.2394239424 0.0032612835619384\\
35202.8802880288 0.00326169590867514\\
35211.5211521152 0.00326210820483952\\
35220.1620162016 0.00326252045045014\\
35228.802880288 0.0032629326455256\\
35237.4437443744 0.00326334479008448\\
35246.0846084608 0.00326375688414533\\
35254.7254725473 0.00326416892772673\\
35263.3663366337 0.00326458092084722\\
35272.0072007201 0.00326499286352532\\
35280.6480648065 0.00326540475577957\\
35289.2889288929 0.00326581659762848\\
35297.9297929793 0.00326622838909054\\
35306.5706570657 0.00326664013018425\\
35315.2115211521 0.00326705182092809\\
35323.8523852385 0.00326746346134051\\
35332.4932493249 0.00326787505143998\\
35341.1341134113 0.00326828659124494\\
35349.7749774977 0.00326869808077382\\
35358.4158415842 0.00326910952004505\\
35367.0567056706 0.00326952090907703\\
35375.697569757 0.00326993224788817\\
35384.3384338434 0.00327034353649685\\
35392.9792979298 0.00327075477492144\\
35401.6201620162 0.00327116596318032\\
35410.2610261026 0.00327157710129184\\
35418.901890189 0.00327198818927433\\
35427.5427542754 0.00327239922714614\\
35436.1836183618 0.00327281021492558\\
35444.8244824482 0.00327322115263097\\
35453.4653465347 0.0032736320402806\\
35462.1062106211 0.00327404287789275\\
35470.7470747075 0.00327445366548571\\
35479.3879387939 0.00327486440307774\\
35488.0288028803 0.00327527509068709\\
35496.6696669667 0.00327568572833201\\
35505.3105310531 0.00327609631603072\\
35513.9513951395 0.00327650685380145\\
35522.5922592259 0.00327691734166241\\
35531.2331233123 0.0032773277796318\\
35539.8739873987 0.0032777381677278\\
35548.5148514851 0.00327814850596859\\
35557.1557155716 0.00327855879437233\\
35565.796579658 0.00327896903295718\\
35574.4374437444 0.00327937922174129\\
35583.0783078308 0.00327978936074278\\
35591.7191719172 0.00328019944997978\\
35600.3600360036 0.00328060948947039\\
35609.00090009 0.00328101947923271\\
35617.6417641764 0.00328142941928484\\
35626.2826282628 0.00328183930964485\\
35634.9234923492 0.00328224915033081\\
35643.5643564356 0.00328265894136077\\
35652.2052205221 0.00328306868275278\\
35660.8460846085 0.00328347837452487\\
35669.4869486949 0.00328388801669506\\
35678.1278127813 0.00328429760928137\\
35686.7686768677 0.00328470715230179\\
35695.4095409541 0.00328511664577432\\
35704.0504050405 0.00328552608971693\\
35712.6912691269 0.00328593548414759\\
35721.3321332133 0.00328634482908427\\
35729.9729972997 0.0032867541245449\\
35738.6138613861 0.00328716337054743\\
35747.2547254726 0.00328757256710977\\
35755.895589559 0.00328798171424984\\
35764.5364536454 0.00328839081198555\\
35773.1773177318 0.00328879986033478\\
35781.8181818182 0.00328920885931542\\
35790.4590459046 0.00328961780894534\\
35799.099909991 0.0032900267092424\\
35807.7407740774 0.00329043556022445\\
35816.3816381638 0.00329084436190933\\
35825.0225022502 0.00329125311431485\\
35833.6633663366 0.00329166181745885\\
35842.304230423 0.00329207047135913\\
35850.9450945095 0.00329247907603348\\
35859.5859585959 0.00329288763149969\\
35868.2268226823 0.00329329613777553\\
35876.8676867687 0.00329370459487876\\
35885.5085508551 0.00329411300282715\\
35894.1494149415 0.00329452136163842\\
35902.7902790279 0.00329492967133031\\
35911.4311431143 0.00329533793192055\\
35920.0720072007 0.00329574614342683\\
35928.7128712871 0.00329615430586687\\
35937.3537353735 0.00329656241925835\\
35945.9945994599 0.00329697048361895\\
35954.6354635464 0.00329737849896633\\
35963.2763276328 0.00329778646531816\\
35971.9171917192 0.00329819438269207\\
35980.5580558056 0.00329860225110571\\
35989.198919892 0.0032990100705767\\
35997.8397839784 0.00329941784112265\\
36006.4806480648 0.00329982556276117\\
36015.1215121512 0.00330023323550986\\
36023.7623762376 0.00330064085938628\\
36032.403240324 0.00330104843440803\\
36041.0441044104 0.00330145596059265\\
36049.6849684968 0.0033018634379577\\
36058.3258325833 0.00330227086652073\\
36066.9666966697 0.00330267824629925\\
36075.6075607561 0.0033030855773108\\
36084.2484248425 0.00330349285957288\\
36092.8892889289 0.00330390009310298\\
36101.5301530153 0.00330430727791861\\
36110.1710171017 0.00330471441403722\\
36118.8118811881 0.00330512150147631\\
36127.4527452745 0.00330552854025331\\
36136.0936093609 0.00330593553038568\\
36144.7344734473 0.00330634247189085\\
36153.3753375338 0.00330674936478625\\
36162.0162016202 0.00330715620908929\\
36170.6570657066 0.00330756300481738\\
36179.297929793 0.00330796975198792\\
36187.9387938794 0.00330837645061828\\
36196.5796579658 0.00330878310072585\\
36205.2205220522 0.00330918970232798\\
36213.8613861386 0.00330959625544203\\
36222.502250225 0.00331000276008534\\
36231.1431143114 0.00331040921627524\\
36239.7839783978 0.00331081562402906\\
36248.4248424843 0.0033112219833641\\
36257.0657065707 0.00331162829429767\\
36265.7065706571 0.00331203455684707\\
36274.3474347435 0.00331244077102956\\
36282.9882988299 0.00331284693686242\\
36291.6291629163 0.00331325305436292\\
36300.2700270027 0.0033136591235483\\
36308.9108910891 0.0033140651444358\\
36317.5517551755 0.00331447111704265\\
36326.1926192619 0.00331487704138607\\
36334.8334833483 0.00331528291748327\\
36343.4743474347 0.00331568874535145\\
36352.1152115212 0.00331609452500779\\
36360.7560756076 0.00331650025646948\\
36369.396939694 0.00331690593975369\\
36378.0378037804 0.00331731157487756\\
36386.6786678668 0.00331771716185826\\
36395.3195319532 0.00331812270071291\\
36403.9603960396 0.00331852819145865\\
36412.601260126 0.00331893363411259\\
36421.2421242124 0.00331933902869183\\
36429.8829882988 0.00331974437521349\\
36438.5238523852 0.00332014967369463\\
36447.1647164716 0.00332055492415235\\
36455.8055805581 0.0033209601266037\\
36464.4464446445 0.00332136528106575\\
36473.0873087309 0.00332177038755554\\
36481.7281728173 0.0033221754460901\\
36490.3690369037 0.00332258045668646\\
36499.0099009901 0.00332298541936165\\
36507.6507650765 0.00332339033413266\\
36516.2916291629 0.00332379520101648\\
36524.9324932493 0.00332420002003012\\
36533.5733573357 0.00332460479119054\\
36542.2142214221 0.00332500951451471\\
36550.8550855085 0.00332541419001958\\
36559.495949595 0.00332581881772211\\
36568.1368136814 0.00332622339763922\\
36576.7776777678 0.00332662792978785\\
36585.4185418542 0.0033270324141849\\
36594.0594059406 0.00332743685084729\\
36602.700270027 0.00332784123979191\\
36611.3411341134 0.00332824558103565\\
36619.9819981998 0.00332864987459538\\
36628.6228622862 0.00332905412048797\\
36637.2637263726 0.00332945831873027\\
36645.904590459 0.00332986246933914\\
36654.5454545455 0.00333026657233141\\
36663.1863186319 0.0033306706277239\\
36671.8271827183 0.00333107463553343\\
36680.4680468047 0.00333147859577681\\
36689.1089108911 0.00333188250847083\\
36697.7497749775 0.00333228637363229\\
36706.3906390639 0.00333269019127795\\
36715.0315031503 0.00333309396142459\\
36723.6723672367 0.00333349768408897\\
36732.3132313231 0.00333390135928782\\
36740.9540954095 0.00333430498703789\\
36749.594959496 0.0033347085673559\\
36758.2358235824 0.00333511210025858\\
36766.8766876688 0.00333551558576263\\
36775.5175517552 0.00333591902388475\\
36784.1584158416 0.00333632241464163\\
36792.799279928 0.00333672575804995\\
36801.4401440144 0.00333712905412637\\
36810.0810081008 0.00333753230288756\\
36818.7218721872 0.00333793550435016\\
36827.3627362736 0.00333833865853081\\
36836.00360036 0.00333874176544615\\
36844.6444644464 0.00333914482511279\\
36853.2853285329 0.00333954783754735\\
36861.9261926193 0.00333995080276642\\
36870.5670567057 0.0033403537207866\\
36879.2079207921 0.00334075659162446\\
36887.8487848785 0.00334115941529658\\
36896.4896489649 0.00334156219181953\\
36905.1305130513 0.00334196492120984\\
36913.7713771377 0.00334236760348407\\
36922.4122412241 0.00334277023865874\\
36931.0531053105 0.00334317282675039\\
36939.6939693969 0.00334357536777552\\
36948.3348334833 0.00334397786175064\\
36956.9756975698 0.00334438030869224\\
36965.6165616562 0.0033447827086168\\
36974.2574257426 0.00334518506154081\\
36982.898289829 0.00334558736748072\\
36991.5391539154 0.003345989626453\\
37000.1800180018 0.00334639183847408\\
37008.8208820882 0.0033467940035604\\
37017.4617461746 0.0033471961217284\\
37026.102610261 0.00334759819299449\\
37034.7434743474 0.00334800021737507\\
37043.3843384338 0.00334840219488654\\
37052.0252025202 0.0033488041255453\\
37060.6660666067 0.00334920600936772\\
37069.3069306931 0.00334960784637017\\
37077.9477947795 0.00335000963656901\\
37086.5886588659 0.0033504113799806\\
37095.2295229523 0.00335081307662126\\
37103.8703870387 0.00335121472650734\\
37112.5112511251 0.00335161632965516\\
37121.1521152115 0.00335201788608102\\
37129.7929792979 0.00335241939580124\\
37138.4338433843 0.00335282085883209\\
37147.0747074707 0.00335322227518988\\
37155.7155715572 0.00335362364489087\\
37164.3564356436 0.00335402496795133\\
37172.99729973 0.00335442624438751\\
37181.6381638164 0.00335482747421566\\
37190.2790279028 0.00335522865745202\\
37198.9198919892 0.00335562979411281\\
37207.5607560756 0.00335603088421425\\
37216.201620162 0.00335643192777255\\
37224.8424842484 0.00335683292480391\\
37233.4833483348 0.00335723387532452\\
37242.1242124212 0.00335763477935056\\
37250.7650765077 0.0033580356368982\\
37259.4059405941 0.00335843644798359\\
37268.0468046805 0.00335883721262291\\
37276.6876687669 0.00335923793083227\\
37285.3285328533 0.00335963860262783\\
37293.9693969397 0.00336003922802571\\
37302.6102610261 0.00336043980704201\\
37311.2511251125 0.00336084033969284\\
37319.8919891989 0.0033612408259943\\
37328.5328532853 0.00336164126596249\\
37337.1737173717 0.00336204165961346\\
37345.8145814581 0.0033624420069633\\
37354.4554455446 0.00336284230802806\\
37363.096309631 0.00336324256282379\\
37371.7371737174 0.00336364277136654\\
37380.3780378038 0.00336404293367232\\
37389.0189018902 0.00336444304975718\\
37397.6597659766 0.0033648431196371\\
37406.300630063 0.00336524314332811\\
37414.9414941494 0.0033656431208462\\
37423.5823582358 0.00336604305220734\\
37432.2232223222 0.00336644293742752\\
37440.8640864086 0.0033668427765227\\
37449.5049504951 0.00336724256950884\\
37458.1458145815 0.00336764231640189\\
37466.7866786679 0.00336804201721778\\
37475.4275427543 0.00336844167197245\\
37484.0684068407 0.00336884128068181\\
37492.7092709271 0.00336924084336179\\
37501.3501350135 0.00336964036002827\\
37509.9909990999 0.00337003983069715\\
37518.6318631863 0.00337043925538432\\
37527.2727272727 0.00337083863410566\\
37535.9135913591 0.00337123796687702\\
37544.5544554455 0.00337163725371426\\
37553.195319532 0.00337203649463323\\
37561.8361836184 0.00337243568964978\\
37570.4770477048 0.00337283483877972\\
37579.1179117912 0.00337323394203888\\
37587.7587758776 0.00337363299944306\\
37596.399639964 0.00337403201100808\\
37605.0405040504 0.00337443097674973\\
37613.6813681368 0.00337482989668378\\
37622.3222322232 0.00337522877082601\\
37630.9630963096 0.00337562759919219\\
37639.603960396 0.00337602638179808\\
37648.2448244824 0.00337642511865942\\
37656.8856885689 0.00337682380979195\\
37665.5265526553 0.00337722245521139\\
37674.1674167417 0.00337762105493348\\
37682.8082808281 0.00337801960897393\\
37691.4491449145 0.00337841811734842\\
37700.0900090009 0.00337881658007266\\
37708.7308730873 0.00337921499716234\\
37717.3717371737 0.00337961336863313\\
37726.0126012601 0.00338001169450068\\
37734.6534653465 0.00338040997478068\\
37743.2943294329 0.00338080820948875\\
37751.9351935194 0.00338120639864054\\
37760.5760576058 0.00338160454225168\\
37769.2169216922 0.0033820026403378\\
37777.8577857786 0.0033824006929145\\
37786.498649865 0.00338279869999739\\
37795.1395139514 0.00338319666160207\\
37803.7803780378 0.00338359457774412\\
37812.4212421242 0.00338399244843911\\
37821.0621062106 0.00338439027370263\\
37829.702970297 0.00338478805355021\\
37838.3438343834 0.00338518578799743\\
37846.9846984698 0.00338558347705981\\
37855.6255625563 0.00338598112075289\\
37864.2664266427 0.0033863787190922\\
37872.9072907291 0.00338677627209325\\
37881.5481548155 0.00338717377977154\\
37890.1890189019 0.00338757124214258\\
37898.8298829883 0.00338796865922185\\
37907.4707470747 0.00338836603102483\\
37916.1116111611 0.003388763357567\\
37924.7524752475 0.00338916063886381\\
37933.3933393339 0.00338955787493072\\
37942.0342034203 0.00338995506578317\\
37950.6750675068 0.0033903522114366\\
37959.3159315932 0.00339074931190644\\
37967.9567956796 0.0033911463672081\\
37976.597659766 0.00339154337735699\\
37985.2385238524 0.00339194034236852\\
37993.8793879388 0.00339233726225807\\
38002.5202520252 0.00339273413704104\\
38011.1611161116 0.00339313096673278\\
38019.801980198 0.00339352775134868\\
38028.4428442844 0.00339392449090408\\
38037.0837083708 0.00339432118541434\\
38045.7245724572 0.00339471783489479\\
38054.3654365437 0.00339511443936077\\
38063.0063006301 0.00339551099882759\\
38071.6471647165 0.00339590751331058\\
38080.2880288029 0.00339630398282503\\
38088.9288928893 0.00339670040738624\\
38097.5697569757 0.0033970967870095\\
38106.2106210621 0.00339749312171009\\
38114.8514851485 0.00339788941150327\\
38123.4923492349 0.00339828565640432\\
38132.1332133213 0.00339868185642848\\
38140.7740774077 0.00339907801159099\\
38149.4149414941 0.0033994741219071\\
38158.0558055806 0.00339987018739203\\
38166.696669667 0.00340026620806099\\
38175.3375337534 0.0034006621839292\\
38183.9783978398 0.00340105811501186\\
38192.6192619262 0.00340145400132415\\
38201.2601260126 0.00340184984288128\\
38209.900990099 0.0034022456396984\\
38218.5418541854 0.00340264139179069\\
38227.1827182718 0.00340303709917331\\
38235.8235823582 0.0034034327618614\\
38244.4644464446 0.0034038283798701\\
38253.1053105311 0.00340422395321455\\
38261.7461746175 0.00340461948190988\\
38270.3870387039 0.00340501496597119\\
38279.0279027903 0.0034054104054136\\
38287.6687668767 0.0034058058002522\\
38296.3096309631 0.00340620115050209\\
38304.9504950495 0.00340659645617834\\
38313.5913591359 0.00340699171729602\\
38322.2322232223 0.00340738693387021\\
38330.8730873087 0.00340778210591596\\
38339.5139513951 0.00340817723344831\\
38348.1548154815 0.0034085723164823\\
38356.795679568 0.00340896735503297\\
38365.4365436544 0.00340936234911534\\
38374.0774077408 0.00340975729874442\\
38382.7182718272 0.00341015220393521\\
38391.3591359136 0.00341054706470271\\
38400 0.00341094188106192\\
38408.6408640864 0.00341133665302781\\
38417.2817281728 0.00341173138061534\\
38425.9225922592 0.0034121260638395\\
38434.5634563456 0.00341252070271522\\
38443.204320432 0.00341291529725746\\
38451.8451845185 0.00341330984748115\\
38460.4860486049 0.00341370435340122\\
38469.1269126913 0.0034140988150326\\
38477.7677767777 0.00341449323239019\\
38486.4086408641 0.00341488760548891\\
38495.0495049505 0.00341528193434364\\
38503.6903690369 0.00341567621896928\\
38512.3312331233 0.0034160704593807\\
38520.9720972097 0.00341646465559278\\
38529.6129612961 0.00341685880762037\\
38538.2538253825 0.00341725291547834\\
38546.8946894689 0.00341764697918152\\
38555.5355535554 0.00341804099874476\\
38564.1764176418 0.00341843497418288\\
38572.8172817282 0.00341882890551071\\
38581.4581458146 0.00341922279274305\\
38590.099009901 0.00341961663589472\\
38598.7398739874 0.00342001043498051\\
38607.3807380738 0.0034204041900152\\
38616.0216021602 0.00342079790101358\\
38624.6624662466 0.00342119156799042\\
38633.303330333 0.00342158519096047\\
38641.9441944194 0.00342197876993851\\
38650.5850585059 0.00342237230493926\\
38659.2259225923 0.00342276579597748\\
38667.8667866787 0.00342315924306788\\
38676.5076507651 0.0034235526462252\\
38685.1485148515 0.00342394600546415\\
38693.7893789379 0.00342433932079942\\
38702.4302430243 0.00342473259224573\\
38711.0711071107 0.00342512581981775\\
38719.7119711971 0.00342551900353017\\
38728.3528352835 0.00342591214339767\\
38736.9936993699 0.00342630523943489\\
38745.6345634563 0.00342669829165652\\
38754.2754275428 0.00342709130007718\\
38762.9162916292 0.00342748426471152\\
38771.5571557156 0.00342787718557417\\
38780.198019802 0.00342827006267976\\
38788.8388838884 0.0034286628960429\\
38797.4797479748 0.0034290556856782\\
38806.1206120612 0.00342944843160026\\
38814.7614761476 0.00342984113382367\\
38823.402340234 0.00343023379236301\\
38832.0432043204 0.00343062640723286\\
38840.6840684068 0.00343101897844778\\
38849.3249324932 0.00343141150602234\\
38857.9657965797 0.00343180398997108\\
38866.6066606661 0.00343219643030854\\
38875.2475247525 0.00343258882704927\\
38883.8883888389 0.00343298118020778\\
38892.5292529253 0.0034333734897986\\
38901.1701170117 0.00343376575583624\\
38909.8109810981 0.00343415797833519\\
38918.4518451845 0.00343455015730996\\
38927.0927092709 0.00343494229277502\\
38935.7335733573 0.00343533438474487\\
38944.3744374437 0.00343572643323395\\
38953.0153015302 0.00343611843825675\\
38961.6561656166 0.00343651039982771\\
38970.297029703 0.00343690231796128\\
38978.9378937894 0.0034372941926719\\
38987.5787578758 0.00343768602397399\\
38996.2196219622 0.00343807781188198\\
39004.8604860486 0.00343846955641029\\
39013.501350135 0.00343886125757331\\
39022.1422142214 0.00343925291538546\\
39030.7830783078 0.00343964452986111\\
39039.4239423942 0.00344003610101465\\
39048.0648064806 0.00344042762886045\\
39056.7056705671 0.00344081911341289\\
39065.3465346535 0.00344121055468631\\
39073.9873987399 0.00344160195269508\\
39082.6282628263 0.00344199330745352\\
39091.2691269127 0.00344238461897598\\
39099.9099909991 0.00344277588727679\\
39108.5508550855 0.00344316711237025\\
39117.1917191719 0.00344355829427069\\
39125.8325832583 0.00344394943299241\\
39134.4734473447 0.00344434052854969\\
39143.1143114311 0.00344473158095684\\
39151.7551755176 0.00344512259022812\\
39160.396039604 0.00344551355637781\\
39169.0369036904 0.00344590447942017\\
39177.6777677768 0.00344629535936946\\
39186.3186318632 0.00344668619623992\\
39194.9594959496 0.00344707699004581\\
39203.600360036 0.00344746774080133\\
39212.2412241224 0.00344785844852074\\
39220.8820882088 0.00344824911321822\\
39229.5229522952 0.00344863973490801\\
39238.1638163816 0.0034490303136043\\
39246.804680468 0.00344942084932127\\
39255.4455445545 0.00344981134207312\\
39264.0864086409 0.00345020179187403\\
39272.7272727273 0.00345059219873816\\
39281.3681368137 0.00345098256267967\\
39290.0090009001 0.00345137288371273\\
39298.6498649865 0.00345176316185147\\
39307.2907290729 0.00345215339711003\\
39315.9315931593 0.00345254358950255\\
39324.5724572457 0.00345293373904314\\
39333.2133213321 0.00345332384574593\\
39341.8541854185 0.00345371390962502\\
39350.4950495049 0.00345410393069451\\
39359.1359135914 0.00345449390896849\\
39367.7767776778 0.00345488384446105\\
39376.4176417642 0.00345527373718626\\
39385.0585058506 0.00345566358715819\\
39393.699369937 0.00345605339439091\\
39402.3402340234 0.00345644315889846\\
39410.9810981098 0.00345683288069489\\
39419.6219621962 0.00345722255979425\\
39428.2628262826 0.00345761219621055\\
39436.903690369 0.00345800178995782\\
39445.5445544554 0.00345839134105009\\
39454.1854185419 0.00345878084950135\\
39462.8262826283 0.0034591703153256\\
39471.4671467147 0.00345955973853684\\
39480.1080108011 0.00345994911914905\\
39488.7488748875 0.00346033845717621\\
39497.3897389739 0.00346072775263229\\
39506.0306030603 0.00346111700553124\\
39514.6714671467 0.00346150621588703\\
39523.3123312331 0.0034618953837136\\
39531.9531953195 0.00346228450902488\\
39540.5940594059 0.0034626735918348\\
39549.2349234923 0.0034630626321573\\
39557.8757875788 0.00346345163000629\\
39566.5166516652 0.00346384058539567\\
39575.1575157516 0.00346422949833934\\
39583.798379838 0.00346461836885121\\
39592.4392439244 0.00346500719694514\\
39601.0801080108 0.00346539598263503\\
39609.7209720972 0.00346578472593474\\
39618.3618361836 0.00346617342685813\\
39627.00270027 0.00346656208541907\\
39635.6435643564 0.00346695070163139\\
39644.2844284428 0.00346733927550893\\
39652.9252925293 0.00346772780706554\\
39661.5661566157 0.00346811629631503\\
39670.2070207021 0.00346850474327122\\
39678.8478847885 0.00346889314794792\\
39687.4887488749 0.00346928151035893\\
39696.1296129613 0.00346966983051806\\
39704.7704770477 0.00347005810843908\\
39713.4113411341 0.00347044634413578\\
39722.0522052205 0.00347083453762192\\
39730.6930693069 0.00347122268891128\\
39739.3339333933 0.00347161079801761\\
39747.9747974798 0.00347199886495466\\
39756.6156615662 0.00347238688973617\\
39765.2565256526 0.00347277487237589\\
39773.897389739 0.00347316281288752\\
39782.5382538254 0.00347355071128481\\
39791.1791179118 0.00347393856758145\\
39799.8199819982 0.00347432638179115\\
39808.4608460846 0.00347471415392762\\
39817.101710171 0.00347510188400454\\
39825.7425742574 0.0034754895720356\\
39834.3834383438 0.00347587721803446\\
39843.0243024302 0.0034762648220148\\
39851.6651665167 0.00347665238399029\\
39860.3060306031 0.00347703990397456\\
39868.9468946895 0.00347742738198127\\
39877.5877587759 0.00347781481802406\\
39886.2286228623 0.00347820221211656\\
39894.8694869487 0.00347858956427239\\
39903.5103510351 0.00347897687450517\\
39912.1512151215 0.00347936414282851\\
39920.7920792079 0.003479751369256\\
39929.4329432943 0.00348013855380125\\
39938.0738073807 0.00348052569647784\\
39946.7146714671 0.00348091279729934\\
39955.3555355536 0.00348129985627934\\
39963.99639964 0.0034816868734314\\
39972.6372637264 0.00348207384876907\\
39981.2781278128 0.00348246078230591\\
39989.9189918992 0.00348284767405545\\
39998.5598559856 0.00348323452403124\\
40007.200720072 0.0034836213322468\\
40015.8415841584 0.00348400809871565\\
40024.4824482448 0.00348439482345131\\
40033.1233123312 0.00348478150646729\\
40041.7641764176 0.00348516814777707\\
40050.405040504 0.00348555474739416\\
40059.0459045905 0.00348594130533204\\
40067.6867686769 0.00348632782160418\\
40076.3276327633 0.00348671429622406\\
40084.9684968497 0.00348710072920514\\
40093.6093609361 0.00348748712056088\\
40102.2502250225 0.00348787347030471\\
40110.8910891089 0.0034882597784501\\
40119.5319531953 0.00348864604501046\\
40128.1728172817 0.00348903226999922\\
40136.8136813681 0.00348941845342981\\
40145.4545454545 0.00348980459531564\\
40154.095409541 0.0034901906956701\\
40162.7362736274 0.00349057675450661\\
40171.3771377138 0.00349096277183854\\
40180.0180018002 0.00349134874767929\\
40188.6588658866 0.00349173468204223\\
40197.299729973 0.00349212057494073\\
40205.9405940594 0.00349250642638815\\
40214.5814581458 0.00349289223639784\\
40223.2223222322 0.00349327800498315\\
40231.8631863186 0.00349366373215743\\
40240.504050405 0.003494049417934\\
40249.1449144915 0.0034944350623262\\
40257.7857785779 0.00349482066534733\\
40266.4266426643 0.00349520622701072\\
40275.0675067507 0.00349559174732967\\
40283.7083708371 0.00349597722631746\\
40292.3492349235 0.00349636266398741\\
40300.9900990099 0.00349674806035278\\
40309.6309630963 0.00349713341542686\\
40318.2718271827 0.00349751872922291\\
40326.9126912691 0.00349790400175419\\
40335.5535553555 0.00349828923303396\\
40344.1944194419 0.00349867442307548\\
40352.8352835284 0.00349905957189196\\
40361.4761476148 0.00349944467949666\\
40370.1170117012 0.0034998297459028\\
40378.7578757876 0.0035002147711236\\
40387.398739874 0.00350059975517226\\
40396.0396039604 0.003500984698062\\
40404.6804680468 0.00350136959980602\\
40413.3213321332 0.0035017544604175\\
40421.9621962196 0.00350213927990962\\
40430.603060306 0.00350252405829557\\
40439.2439243924 0.00350290879558852\\
40447.8847884788 0.00350329349180162\\
40456.5256525653 0.00350367814694804\\
40465.1665166517 0.00350406276104092\\
40473.8073807381 0.0035044473340934\\
40482.4482448245 0.00350483186611863\\
40491.0891089109 0.00350521635712972\\
40499.7299729973 0.0035056008071398\\
40508.3708370837 0.00350598521616198\\
40517.0117011701 0.00350636958420938\\
40525.6525652565 0.00350675391129508\\
40534.2934293429 0.00350713819743218\\
40542.9342934293 0.00350752244263377\\
40551.5751575157 0.00350790664691293\\
40560.2160216022 0.00350829081028273\\
40568.8568856886 0.00350867493275623\\
40577.497749775 0.0035090590143465\\
40586.1386138614 0.00350944305506658\\
40594.7794779478 0.00350982705492951\\
40603.4203420342 0.00351021101394834\\
40612.0612061206 0.0035105949321361\\
40620.702070207 0.0035109788095058\\
40629.3429342934 0.00351136264607047\\
40637.9837983798 0.00351174644184312\\
40646.6246624662 0.00351213019683674\\
40655.2655265527 0.00351251391106433\\
40663.9063906391 0.00351289758453889\\
40672.5472547255 0.00351328121727339\\
40681.1881188119 0.00351366480928082\\
40689.8289828983 0.00351404836057412\\
40698.4698469847 0.00351443187116628\\
40707.1107110711 0.00351481534107024\\
40715.7515751575 0.00351519877029895\\
40724.3924392439 0.00351558215886536\\
40733.0333033303 0.00351596550678239\\
40741.6741674167 0.00351634881406297\\
40750.3150315032 0.00351673208072002\\
40758.9558955896 0.00351711530676646\\
40767.596759676 0.00351749849221519\\
40776.2376237624 0.00351788163707911\\
40784.8784878488 0.00351826474137112\\
40793.5193519352 0.0035186478051041\\
40802.1602160216 0.00351903082829093\\
40810.801080108 0.00351941381094448\\
40819.4419441944 0.00351979675307762\\
40828.0828082808 0.0035201796547032\\
40836.7236723672 0.00352056251583409\\
40845.3645364536 0.00352094533648311\\
40854.0054005401 0.00352132811666312\\
40862.6462646265 0.00352171085638694\\
40871.2871287129 0.0035220935556674\\
40879.9279927993 0.00352247621451732\\
40888.5688568857 0.0035228588329495\\
40897.2097209721 0.00352324141097676\\
40905.8505850585 0.00352362394861188\\
40914.4914491449 0.00352400644586766\\
40923.1323132313 0.00352438890275689\\
40931.7731773177 0.00352477131929235\\
40940.4140414041 0.00352515369548679\\
40949.0549054905 0.00352553603135299\\
40957.695769577 0.00352591832690371\\
40966.3366336634 0.00352630058215169\\
40974.9774977498 0.00352668279710968\\
40983.6183618362 0.00352706497179041\\
40992.2592259226 0.00352744710620662\\
41000.900090009 0.00352782920037103\\
41009.5409540954 0.00352821125429636\\
41018.1818181818 0.00352859326799531\\
41026.8226822682 0.00352897524148059\\
41035.4635463546 0.00352935717476491\\
41044.104410441 0.00352973906786093\\
41052.7452745274 0.00353012092078136\\
41061.3861386139 0.00353050273353887\\
41070.0270027003 0.00353088450614613\\
41078.6678667867 0.0035312662386158\\
41087.3087308731 0.00353164793096053\\
41095.9495949595 0.00353202958319299\\
41104.5904590459 0.00353241119532581\\
41113.2313231323 0.00353279276737163\\
41121.8721872187 0.00353317429934307\\
41130.5130513051 0.00353355579125278\\
41139.1539153915 0.00353393724311335\\
41147.7947794779 0.0035343186549374\\
41156.4356435644 0.00353470002673754\\
41165.0765076508 0.00353508135852636\\
41173.7173717372 0.00353546265031645\\
41182.3582358236 0.0035358439021204\\
41190.99909991 0.00353622511395079\\
41199.6399639964 0.00353660628582018\\
41208.2808280828 0.00353698741774114\\
41216.9216921692 0.00353736850972622\\
41225.5625562556 0.00353774956178799\\
41234.203420342 0.00353813057393898\\
41242.8442844284 0.00353851154619172\\
41251.4851485149 0.00353889247855876\\
41260.1260126013 0.00353927337105262\\
41268.7668766877 0.00353965422368581\\
41277.4077407741 0.00354003503647085\\
41286.0486048605 0.00354041580942024\\
41294.6894689469 0.00354079654254648\\
41303.3303330333 0.00354117723586206\\
41311.9711971197 0.00354155788937947\\
41320.6120612061 0.00354193850311119\\
41329.2529252925 0.00354231907706969\\
41337.8937893789 0.00354269961126743\\
41346.5346534653 0.00354308010571688\\
41355.1755175518 0.00354346056043049\\
41363.8163816382 0.0035438409754207\\
41372.4572457246 0.00354422135069996\\
41381.098109811 0.00354460168628069\\
41389.7389738974 0.00354498198217533\\
41398.3798379838 0.0035453622383963\\
41407.0207020702 0.00354574245495601\\
41415.6615661566 0.00354612263186686\\
41424.302430243 0.00354650276914126\\
41432.9432943294 0.00354688286679161\\
41441.5841584158 0.00354726292483028\\
41450.2250225023 0.00354764294326967\\
41458.8658865887 0.00354802292212215\\
41467.5067506751 0.00354840286140009\\
41476.1476147615 0.00354878276111584\\
41484.7884788479 0.00354916262128177\\
41493.4293429343 0.00354954244191023\\
41502.0702070207 0.00354992222301355\\
41510.7110711071 0.00355030196460408\\
41519.3519351935 0.00355068166669415\\
41527.9927992799 0.00355106132929607\\
41536.6336633663 0.00355144095242217\\
41545.2745274527 0.00355182053608476\\
41553.9153915392 0.00355220008029614\\
41562.5562556256 0.00355257958506861\\
41571.197119712 0.00355295905041446\\
41579.8379837984 0.00355333847634598\\
41588.4788478848 0.00355371786287545\\
41597.1197119712 0.00355409721001514\\
41605.7605760576 0.00355447651777731\\
41614.401440144 0.00355485578617423\\
41623.0423042304 0.00355523501521816\\
41631.6831683168 0.00355561420492132\\
41640.3240324032 0.00355599335529598\\
41648.9648964896 0.00355637246635436\\
41657.6057605761 0.00355675153810869\\
41666.2466246625 0.00355713057057119\\
41674.8874887489 0.00355750956375408\\
41683.5283528353 0.00355788851766957\\
41692.1692169217 0.00355826743232985\\
41700.8100810081 0.00355864630774714\\
41709.4509450945 0.00355902514393361\\
41718.0918091809 0.00355940394090146\\
41726.7326732673 0.00355978269866285\\
41735.3735373537 0.00356016141722997\\
41744.0144014401 0.00356054009661497\\
41752.6552655266 0.00356091873683001\\
41761.296129613 0.00356129733788725\\
41769.9369936994 0.00356167589979884\\
41778.5778577858 0.00356205442257691\\
41787.2187218722 0.00356243290623359\\
41795.8595859586 0.00356281135078102\\
41804.500450045 0.00356318975623131\\
41813.1413141314 0.00356356812259659\\
41821.7821782178 0.00356394644988895\\
41830.4230423042 0.00356432473812051\\
41839.0639063906 0.00356470298730335\\
41847.704770477 0.00356508119744957\\
41856.3456345635 0.00356545936857124\\
41864.9864986499 0.00356583750068046\\
41873.6273627363 0.00356621559378929\\
41882.2682268227 0.00356659364790979\\
41890.9090909091 0.00356697166305402\\
41899.5499549955 0.00356734963923404\\
41908.1908190819 0.0035677275764619\\
41916.8316831683 0.00356810547474962\\
41925.4725472547 0.00356848333410925\\
41934.1134113411 0.00356886115455281\\
41942.7542754275 0.00356923893609233\\
41951.395139514 0.00356961667873982\\
41960.0360036004 0.00356999438250729\\
41968.6768676868 0.00357037204740674\\
41977.3177317732 0.00357074967345017\\
41985.9585958596 0.00357112726064958\\
41994.599459946 0.00357150480901693\\
42003.2403240324 0.00357188231856422\\
42011.8811881188 0.00357225978930341\\
42020.5220522052 0.00357263722124648\\
42029.1629162916 0.00357301461440537\\
42037.803780378 0.00357339196879205\\
42046.4446444644 0.00357376928441846\\
42055.0855085509 0.00357414656129655\\
42063.7263726373 0.00357452379943824\\
42072.3672367237 0.00357490099885547\\
42081.0081008101 0.00357527815956017\\
42089.6489648965 0.00357565528156424\\
42098.2898289829 0.0035760323648796\\
42106.9306930693 0.00357640940951815\\
42115.5715571557 0.00357678641549179\\
42124.2124212421 0.00357716338281242\\
42132.8532853285 0.00357754031149192\\
42141.4941494149 0.00357791720154217\\
42150.1350135013 0.00357829405297505\\
42158.7758775878 0.00357867086580242\\
42167.4167416742 0.00357904764003615\\
42176.0576057606 0.00357942437568808\\
42184.698469847 0.00357980107277008\\
42193.3393339334 0.00358017773129399\\
42201.9801980198 0.00358055435127164\\
42210.6210621062 0.00358093093271486\\
42219.2619261926 0.00358130747563548\\
42227.902790279 0.00358168398004533\\
42236.5436543654 0.0035820604459562\\
42245.1845184518 0.00358243687337992\\
42253.8253825383 0.00358281326232829\\
42262.4662466247 0.00358318961281309\\
42271.1071107111 0.00358356592484612\\
42279.7479747975 0.00358394219843916\\
42288.3888388839 0.00358431843360399\\
42297.0297029703 0.00358469463035238\\
42305.6705670567 0.00358507078869609\\
42314.3114311431 0.00358544690864689\\
42322.9522952295 0.00358582299021653\\
42331.5931593159 0.00358619903341676\\
42340.2340234023 0.00358657503825931\\
42348.8748874887 0.00358695100475593\\
42357.5157515752 0.00358732693291833\\
42366.1566156616 0.00358770282275826\\
42374.797479748 0.00358807867428741\\
42383.4383438344 0.00358845448751752\\
42392.0792079208 0.00358883026246027\\
42400.7200720072 0.00358920599912737\\
42409.3609360936 0.00358958169753052\\
42418.00180018 0.00358995735768139\\
42426.6426642664 0.00359033297959168\\
42435.2835283528 0.00359070856327306\\
42443.9243924392 0.0035910841087372\\
42452.5652565257 0.00359145961599576\\
42461.2061206121 0.0035918350850604\\
42469.8469846985 0.00359221051594277\\
42478.4878487849 0.00359258590865453\\
42487.1287128713 0.0035929612632073\\
42495.7695769577 0.00359333657961272\\
42504.4104410441 0.00359371185788243\\
42513.0513051305 0.00359408709802804\\
42521.6921692169 0.00359446230006117\\
42530.3330333033 0.00359483746399344\\
42538.9738973897 0.00359521258983644\\
42547.6147614762 0.00359558767760177\\
42556.2556255626 0.00359596272730103\\
42564.896489649 0.00359633773894581\\
42573.5373537354 0.00359671271254769\\
42582.1782178218 0.00359708764811824\\
42590.8190819082 0.00359746254566902\\
42599.4599459946 0.00359783740521162\\
42608.100810081 0.00359821222675758\\
42616.7416741674 0.00359858701031845\\
42625.3825382538 0.00359896175590578\\
42634.0234023402 0.00359933646353112\\
42642.6642664266 0.00359971113320599\\
42651.3051305131 0.00360008576494192\\
42659.9459945995 0.00360046035875044\\
42668.5868586859 0.00360083491464307\\
42677.2277227723 0.00360120943263131\\
42685.8685868587 0.00360158391272667\\
42694.5094509451 0.00360195835494065\\
42703.1503150315 0.00360233275928474\\
42711.7911791179 0.00360270712577043\\
42720.4320432043 0.00360308145440921\\
42729.0729072907 0.00360345574521255\\
42737.7137713771 0.00360382999819191\\
42746.3546354635 0.00360420421335877\\
42754.99549955 0.00360457839072458\\
42763.6363636364 0.0036049525303008\\
42772.2772277228 0.00360532663209887\\
42780.9180918092 0.00360570069613024\\
42789.5589558956 0.00360607472240633\\
42798.199819982 0.00360644871093859\\
42806.8406840684 0.00360682266173843\\
42815.4815481548 0.00360719657481727\\
42824.1224122412 0.00360757045018653\\
42832.7632763276 0.00360794428785761\\
42841.404140414 0.00360831808784192\\
42850.0450045004 0.00360869185015084\\
42858.6858685869 0.00360906557479577\\
42867.3267326733 0.00360943926178809\\
42875.9675967597 0.00360981291113918\\
42884.6084608461 0.00361018652286042\\
42893.2493249325 0.00361056009696316\\
42901.8901890189 0.00361093363345878\\
42910.5310531053 0.00361130713235862\\
42919.1719171917 0.00361168059367404\\
42927.8127812781 0.00361205401741637\\
42936.4536453645 0.00361242740359696\\
42945.0945094509 0.00361280075222715\\
42953.7353735374 0.00361317406331825\\
42962.3762376238 0.00361354733688158\\
42971.0171017102 0.00361392057292847\\
42979.6579657966 0.00361429377147022\\
42988.298829883 0.00361466693251814\\
42996.9396939694 0.00361504005608352\\
43005.5805580558 0.00361541314217766\\
43014.2214221422 0.00361578619081185\\
43022.8622862286 0.00361615920199736\\
43031.503150315 0.00361653217574547\\
43040.1440144014 0.00361690511206745\\
43048.7848784879 0.00361727801097457\\
43057.4257425743 0.00361765087247808\\
43066.0666066607 0.00361802369658924\\
43074.7074707471 0.0036183964833193\\
43083.3483348335 0.00361876923267949\\
43091.9891989199 0.00361914194468105\\
43100.6300630063 0.00361951461933522\\
43109.2709270927 0.00361988725665321\\
43117.9117911791 0.00362025985664626\\
43126.5526552655 0.00362063241932556\\
43135.1935193519 0.00362100494470233\\
43143.8343834383 0.00362137743278778\\
43152.4752475248 0.00362174988359309\\
43161.1161116112 0.00362212229712946\\
43169.7569756976 0.00362249467340807\\
43178.397839784 0.00362286701244012\\
43187.0387038704 0.00362323931423676\\
43195.6795679568 0.00362361157880916\\
43204.3204320432 0.0036239838061685\\
43212.9612961296 0.00362435599632593\\
43221.602160216 0.0036247281492926\\
43230.2430243024 0.00362510026507966\\
43238.8838883888 0.00362547234369825\\
43247.5247524752 0.0036258443851595\\
43256.1656165617 0.00362621638947454\\
43264.8064806481 0.00362658835665451\\
43273.4473447345 0.00362696028671051\\
43282.0882088209 0.00362733217965366\\
43290.7290729073 0.00362770403549507\\
43299.3699369937 0.00362807585424583\\
43308.0108010801 0.00362844763591706\\
43316.6516651665 0.00362881938051983\\
43325.2925292529 0.00362919108806523\\
43333.9333933393 0.00362956275856435\\
43342.5742574257 0.00362993439202826\\
43351.2151215121 0.00363030598846802\\
43359.8559855986 0.0036306775478947\\
43368.496849685 0.00363104907031936\\
43377.1377137714 0.00363142055575304\\
43385.7785778578 0.00363179200420681\\
43394.4194419442 0.00363216341569169\\
43403.0603060306 0.00363253479021873\\
43411.701170117 0.00363290612779896\\
43420.3420342034 0.00363327742844339\\
43428.9828982898 0.00363364869216305\\
43437.6237623762 0.00363401991896896\\
43446.2646264626 0.00363439110887212\\
43454.9054905491 0.00363476226188353\\
43463.5463546355 0.00363513337801419\\
43472.1872187219 0.0036355044572751\\
43480.8280828083 0.00363587549967724\\
43489.4689468947 0.00363624650523159\\
43498.1098109811 0.00363661747394914\\
43506.7506750675 0.00363698840584083\\
43515.3915391539 0.00363735930091766\\
43524.0324032403 0.00363773015919056\\
43532.6732673267 0.0036381009806705\\
43541.3141314131 0.00363847176536843\\
43549.9549954996 0.00363884251329529\\
43558.595859586 0.00363921322446201\\
43567.2367236724 0.00363958389887953\\
43575.8775877588 0.00363995453655877\\
43584.5184518452 0.00364032513751066\\
43593.1593159316 0.00364069570174611\\
43601.800180018 0.00364106622927604\\
43610.4410441044 0.00364143672011135\\
43619.0819081908 0.00364180717426293\\
43627.7227722772 0.00364217759174169\\
43636.3636363636 0.0036425479725585\\
43645.00450045 0.00364291831672427\\
43653.6453645365 0.00364328862424985\\
43662.2862286229 0.00364365889514614\\
43670.9270927093 0.00364402912942398\\
43679.5679567957 0.00364439932709426\\
43688.2088208821 0.00364476948816781\\
43696.8496849685 0.00364513961265551\\
43705.4905490549 0.00364550970056818\\
43714.1314131413 0.00364587975191667\\
43722.7722772277 0.00364624976671182\\
43731.4131413141 0.00364661974496445\\
43740.0540054005 0.0036469896866854\\
43748.6948694869 0.00364735959188548\\
43757.3357335734 0.0036477294605755\\
43765.9765976598 0.00364809929276627\\
43774.6174617462 0.0036484690884686\\
43783.2583258326 0.00364883884769328\\
43791.899189919 0.00364920857045111\\
43800.5400540054 0.00364957825675288\\
43809.1809180918 0.00364994790660936\\
43817.8217821782 0.00365031752003133\\
43826.4626462646 0.00365068709702957\\
43835.103510351 0.00365105663761483\\
43843.7443744374 0.00365142614179789\\
43852.3852385238 0.0036517956095895\\
43861.0261026103 0.0036521650410004\\
43869.6669666967 0.00365253443604134\\
43878.3078307831 0.00365290379472306\\
43886.9486948695 0.0036532731170563\\
43895.5895589559 0.00365364240305179\\
43904.2304230423 0.00365401165272024\\
43912.8712871287 0.00365438086607237\\
43921.5121512151 0.00365475004311891\\
43930.1530153015 0.00365511918387055\\
43938.7938793879 0.00365548828833801\\
43947.4347434743 0.00365585735653197\\
43956.0756075608 0.00365622638846313\\
43964.7164716472 0.00365659538414218\\
43973.3573357336 0.0036569643435798\\
43981.99819982 0.00365733326678666\\
43990.6390639064 0.00365770215377344\\
43999.2799279928 0.0036580710045508\\
44007.9207920792 0.0036584398191294\\
44016.5616561656 0.0036588085975199\\
44025.202520252 0.00365917733973295\\
44033.8433843384 0.00365954604577919\\
44042.4842484248 0.00365991471566927\\
44051.1251125113 0.00366028334941381\\
44059.7659765977 0.00366065194702345\\
44068.4068406841 0.00366102050850881\\
44077.0477047705 0.00366138903388051\\
44085.6885688569 0.00366175752314917\\
44094.3294329433 0.00366212597632538\\
44102.9702970297 0.00366249439341977\\
44111.6111611161 0.00366286277444292\\
44120.2520252025 0.00366323111940542\\
44128.8928892889 0.00366359942831787\\
44137.5337533753 0.00366396770119085\\
44146.1746174617 0.00366433593803493\\
44154.8154815482 0.00366470413886069\\
44163.4563456346 0.00366507230367869\\
44172.097209721 0.0036654404324995\\
44180.7380738074 0.00366580852533367\\
44189.3789378938 0.00366617658219176\\
44198.0198019802 0.0036665446030843\\
44206.6606660666 0.00366691258802185\\
44215.301530153 0.00366728053701494\\
44223.9423942394 0.0036676484500741\\
44232.5832583258 0.00366801632720985\\
44241.2241224122 0.00366838416843272\\
44249.8649864987 0.00366875197375323\\
44258.5058505851 0.00366911974318187\\
44267.1467146715 0.00366948747672916\\
44275.7875787579 0.0036698551744056\\
44284.4284428443 0.00367022283622168\\
44293.0693069307 0.00367059046218789\\
44301.7101710171 0.00367095805231473\\
44310.3510351035 0.00367132560661266\\
44318.9918991899 0.00367169312509216\\
44327.6327632763 0.0036720606077637\\
44336.2736273627 0.00367242805463775\\
44344.9144914491 0.00367279546572477\\
44353.5553555356 0.0036731628410352\\
44362.196219622 0.0036735301805795\\
44370.8370837084 0.00367389748436812\\
44379.4779477948 0.00367426475241149\\
44388.1188118812 0.00367463198472004\\
44396.7596759676 0.00367499918130421\\
44405.400540054 0.00367536634217442\\
44414.0414041404 0.00367573346734109\\
44422.6822682268 0.00367610055681462\\
44431.3231323132 0.00367646761060544\\
44439.9639963996 0.00367683462872394\\
44448.604860486 0.00367720161118052\\
44457.2457245725 0.00367756855798557\\
44465.8865886589 0.00367793546914949\\
44474.5274527453 0.00367830234468266\\
44483.1683168317 0.00367866918459545\\
44491.8091809181 0.00367903598889823\\
44500.4500450045 0.00367940275760139\\
44509.0909090909 0.00367976949071527\\
44517.7317731773 0.00368013618825025\\
44526.3726372637 0.00368050285021666\\
44535.0135013501 0.00368086947662486\\
44543.6543654365 0.0036812360674852\\
44552.295229523 0.003681602622808\\
44560.9360936094 0.00368196914260361\\
44569.5769576958 0.00368233562688235\\
44578.2178217822 0.00368270207565454\\
44586.8586858686 0.0036830684889305\\
44595.499549955 0.00368343486672056\\
44604.1404140414 0.003683801209035\\
44612.7812781278 0.00368416751588414\\
44621.4221422142 0.00368453378727828\\
44630.0630063006 0.0036849000232277\\
44638.703870387 0.00368526622374271\\
44647.3447344734 0.00368563238883357\\
44655.9855985599 0.00368599851851056\\
44664.6264626463 0.00368636461278397\\
44673.2673267327 0.00368673067166406\\
44681.9081908191 0.00368709669516109\\
44690.5490549055 0.00368746268328532\\
44699.1899189919 0.003687828636047\\
44707.8307830783 0.00368819455345639\\
44716.4716471647 0.00368856043552372\\
44725.1125112511 0.00368892628225924\\
44733.7533753375 0.00368929209367317\\
44742.3942394239 0.00368965786977576\\
44751.0351035104 0.00369002361057721\\
44759.6759675968 0.00369038931608775\\
44768.3168316832 0.00369075498631759\\
44776.9576957696 0.00369112062127695\\
44785.598559856 0.00369148622097602\\
44794.2394239424 0.00369185178542501\\
44802.8802880288 0.0036922173146341\\
44811.5211521152 0.0036925828086135\\
44820.1620162016 0.00369294826737338\\
44828.802880288 0.00369331369092392\\
44837.4437443744 0.00369367907927529\\
44846.0846084608 0.00369404443243768\\
44854.7254725473 0.00369440975042123\\
44863.3663366337 0.00369477503323611\\
44872.0072007201 0.00369514028089249\\
44880.6480648065 0.00369550549340049\\
44889.2889288929 0.00369587067077028\\
44897.9297929793 0.00369623581301199\\
44906.5706570657 0.00369660092013576\\
44915.2115211521 0.00369696599215171\\
44923.8523852385 0.00369733102906998\\
44932.4932493249 0.00369769603090068\\
44941.1341134113 0.00369806099765394\\
44949.7749774977 0.00369842592933985\\
44958.4158415842 0.00369879082596853\\
44967.0567056706 0.00369915568755008\\
44975.697569757 0.00369952051409459\\
44984.3384338434 0.00369988530561217\\
44992.9792979298 0.00370025006211288\\
45001.6201620162 0.00370061478360682\\
45010.2610261026 0.00370097947010407\\
45018.901890189 0.00370134412161469\\
45027.5427542754 0.00370170873814875\\
45036.1836183618 0.00370207331971632\\
45044.8244824482 0.00370243786632745\\
45053.4653465347 0.0037028023779922\\
45062.1062106211 0.00370316685472062\\
45070.7470747075 0.00370353129652274\\
45079.3879387939 0.00370389570340861\\
45088.0288028803 0.00370426007538826\\
45096.6696669667 0.00370462441247172\\
45105.3105310531 0.00370498871466901\\
45113.9513951395 0.00370535298199016\\
45122.5922592259 0.00370571721444517\\
45131.2331233123 0.00370608141204406\\
45139.8739873987 0.00370644557479683\\
45148.5148514851 0.00370680970271349\\
45157.1557155716 0.00370717379580402\\
45165.796579658 0.00370753785407843\\
45174.4374437444 0.00370790187754669\\
45183.0783078308 0.00370826586621878\\
45191.7191719172 0.00370862982010469\\
45200.3600360036 0.00370899373921438\\
45209.00090009 0.00370935762355782\\
45217.6417641764 0.00370972147314497\\
45226.2826282628 0.00371008528798579\\
45234.9234923492 0.00371044906809024\\
45243.5643564356 0.00371081281346825\\
45252.2052205221 0.00371117652412978\\
45260.8460846085 0.00371154020008477\\
45269.4869486949 0.00371190384134313\\
45278.1278127813 0.00371226744791481\\
45286.7686768677 0.00371263101980973\\
45295.4095409541 0.00371299455703781\\
45304.0504050405 0.00371335805960895\\
45312.6912691269 0.00371372152753308\\
45321.3321332133 0.0037140849608201\\
45329.9729972997 0.00371444835947991\\
45338.6138613861 0.00371481172352239\\
45347.2547254726 0.00371517505295746\\
45355.895589559 0.00371553834779498\\
45364.5364536454 0.00371590160804485\\
45373.1773177318 0.00371626483371693\\
45381.8181818182 0.00371662802482111\\
45390.4590459046 0.00371699118136725\\
45399.099909991 0.0037173543033652\\
45407.7407740774 0.00371771739082484\\
45416.3816381638 0.00371808044375601\\
45425.0225022502 0.00371844346216857\\
45433.6633663366 0.00371880644607235\\
45442.304230423 0.00371916939547719\\
45450.9450945095 0.00371953231039294\\
45459.5859585959 0.00371989519082941\\
45468.2268226823 0.00372025803679645\\
45476.8676867687 0.00372062084830386\\
45485.5085508551 0.00372098362536146\\
45494.1494149415 0.00372134636797907\\
45502.7902790279 0.00372170907616648\\
45511.4311431143 0.00372207174993351\\
45520.0720072007 0.00372243438928995\\
45528.7128712871 0.0037227969942456\\
45537.3537353735 0.00372315956481023\\
45545.9945994599 0.00372352210099364\\
45554.6354635464 0.00372388460280561\\
45563.2763276328 0.0037242470702559\\
45571.9171917192 0.00372460950335429\\
45580.5580558056 0.00372497190211054\\
45589.198919892 0.00372533426653441\\
45597.8397839784 0.00372569659663567\\
45606.4806480648 0.00372605889242405\\
45615.1215121512 0.00372642115390931\\
45623.7623762376 0.00372678338110119\\
45632.403240324 0.00372714557400942\\
45641.0441044104 0.00372750773264374\\
45649.6849684968 0.00372786985701388\\
45658.3258325833 0.00372823194712956\\
45666.9666966697 0.0037285940030005\\
45675.6075607561 0.0037289560246364\\
45684.2484248425 0.003729318012047\\
45692.8892889289 0.00372967996524197\\
45701.5301530153 0.00373004188423104\\
45710.1710171017 0.00373040376902389\\
45718.8118811881 0.00373076561963022\\
45727.4527452745 0.0037311274360597\\
45736.0936093609 0.00373148921832204\\
45744.7344734473 0.00373185096642689\\
45753.3753375338 0.00373221268038394\\
45762.0162016202 0.00373257436020285\\
45770.6570657066 0.00373293600589329\\
45779.297929793 0.00373329761746492\\
45787.9387938794 0.00373365919492739\\
45796.5796579658 0.00373402073829035\\
45805.2205220522 0.00373438224756344\\
45813.8613861386 0.00373474372275632\\
45822.502250225 0.00373510516387861\\
45831.1431143114 0.00373546657093995\\
45839.7839783978 0.00373582794394996\\
45848.4248424843 0.00373618928291827\\
45857.0657065707 0.0037365505878545\\
45865.7065706571 0.00373691185876826\\
45874.3474347435 0.00373727309566916\\
45882.9882988299 0.0037376342985668\\
45891.6291629163 0.00373799546747079\\
45900.2700270027 0.00373835660239071\\
45908.9108910891 0.00373871770333617\\
45917.5517551755 0.00373907877031675\\
45926.1926192619 0.00373943980334203\\
45934.8334833483 0.00373980080242159\\
45943.4743474347 0.00374016176756501\\
45952.1152115212 0.00374052269878184\\
45960.7560756076 0.00374088359608166\\
45969.396939694 0.00374124445947402\\
45978.0378037804 0.00374160528896849\\
45986.6786678668 0.00374196608457461\\
45995.3195319532 0.00374232684630192\\
46003.9603960396 0.00374268757415998\\
46012.601260126 0.00374304826815831\\
46021.2421242124 0.00374340892830645\\
46029.8829882988 0.00374376955461393\\
46038.5238523852 0.00374413014709028\\
46047.1647164716 0.003744490705745\\
46055.8055805581 0.00374485123058763\\
46064.4464446445 0.00374521172162766\\
46073.0873087309 0.00374557217887461\\
46081.7281728173 0.00374593260233797\\
46090.3690369037 0.00374629299202724\\
46099.0099009901 0.00374665334795192\\
46107.6507650765 0.0037470136701215\\
46116.2916291629 0.00374737395854545\\
46124.9324932493 0.00374773421323326\\
46133.5733573357 0.0037480944341944\\
46142.2142214221 0.00374845462143835\\
46150.8550855085 0.00374881477497456\\
46159.495949595 0.00374917489481251\\
46168.1368136814 0.00374953498096164\\
46176.7776777678 0.00374989503343142\\
46185.4185418542 0.00375025505223128\\
46194.0594059406 0.00375061503737068\\
46202.700270027 0.00375097498885906\\
46211.3411341134 0.00375133490670584\\
46219.9819981998 0.00375169479092046\\
46228.6228622862 0.00375205464151234\\
46237.2637263726 0.00375241445849092\\
46245.904590459 0.0037527742418656\\
46254.5454545455 0.0037531339916458\\
46263.1863186319 0.00375349370784092\\
46271.8271827183 0.00375385339046038\\
46280.4680468047 0.00375421303951357\\
46289.1089108911 0.00375457265500988\\
46297.7497749775 0.00375493223695872\\
46306.3906390639 0.00375529178536946\\
46315.0315031503 0.00375565130025149\\
46323.6723672367 0.00375601078161418\\
46332.3132313231 0.00375637022946692\\
46340.9540954095 0.00375672964381906\\
46349.594959496 0.00375708902467998\\
46358.2358235824 0.00375744837205904\\
46366.8766876688 0.00375780768596559\\
46375.5175517552 0.00375816696640897\\
46384.1584158416 0.00375852621339856\\
46392.799279928 0.00375888542694367\\
46401.4401440144 0.00375924460705366\\
46410.0810081008 0.00375960375373785\\
46418.7218721872 0.00375996286700559\\
46427.3627362736 0.00376032194686619\\
46436.00360036 0.00376068099332897\\
46444.6444644464 0.00376104000640325\\
46453.2853285329 0.00376139898609835\\
46461.9261926193 0.00376175793242358\\
46470.5670567057 0.00376211684538823\\
46479.2079207921 0.00376247572500161\\
46487.8487848785 0.00376283457127302\\
46496.4896489649 0.00376319338421174\\
46505.1305130513 0.00376355216382706\\
46513.7713771377 0.00376391091012826\\
46522.4122412241 0.00376426962312463\\
46531.0531053105 0.00376462830282544\\
46539.6939693969 0.00376498694923995\\
46548.3348334833 0.00376534556237743\\
46556.9756975698 0.00376570414224715\\
46565.6165616562 0.00376606268885836\\
46574.2574257426 0.00376642120222031\\
46582.898289829 0.00376677968234225\\
46591.5391539154 0.00376713812923342\\
46600.1800180018 0.00376749654290307\\
46608.8208820882 0.00376785492336043\\
46617.4617461746 0.00376821327061472\\
46626.102610261 0.00376857158467519\\
46634.7434743474 0.00376892986555105\\
46643.3843384338 0.00376928811325151\\
46652.0252025202 0.0037696463277858\\
46660.6660666067 0.00377000450916312\\
46669.3069306931 0.00377036265739267\\
46677.9477947795 0.00377072077248367\\
46686.5886588659 0.0037710788544453\\
46695.2295229523 0.00377143690328676\\
46703.8703870387 0.00377179491901724\\
46712.5112511251 0.00377215290164592\\
46721.1521152115 0.00377251085118197\\
46729.7929792979 0.00377286876763459\\
46738.4338433843 0.00377322665101294\\
46747.0747074707 0.00377358450132618\\
46755.7155715572 0.00377394231858348\\
46764.3564356436 0.00377430010279399\\
46772.99729973 0.00377465785396688\\
46781.6381638164 0.00377501557211129\\
46790.2790279028 0.00377537325723638\\
46798.9198919892 0.00377573090935127\\
46807.5607560756 0.00377608852846511\\
46816.201620162 0.00377644611458703\\
46824.8424842484 0.00377680366772616\\
46833.4833483348 0.00377716118789163\\
46842.1242124212 0.00377751867509255\\
46850.7650765077 0.00377787612933805\\
46859.4059405941 0.00377823355063723\\
46868.0468046805 0.00377859093899921\\
46876.6876687669 0.00377894829443309\\
46885.3285328533 0.00377930561694797\\
46893.9693969397 0.00377966290655293\\
46902.6102610261 0.00378002016325709\\
46911.2511251125 0.00378037738706951\\
46919.8919891989 0.00378073457799929\\
46928.5328532853 0.00378109173605551\\
46937.1737173717 0.00378144886124724\\
46945.8145814581 0.00378180595358354\\
46954.4554455446 0.00378216301307349\\
46963.096309631 0.00378252003972615\\
46971.7371737174 0.00378287703355058\\
46980.3780378038 0.00378323399455583\\
46989.0189018902 0.00378359092275094\\
46997.6597659766 0.00378394781814497\\
47006.300630063 0.00378430468074696\\
47014.9414941494 0.00378466151056595\\
47023.5823582358 0.00378501830761096\\
47032.2232223222 0.00378537507189103\\
47040.8640864086 0.00378573180341518\\
47049.5049504951 0.00378608850219243\\
47058.1458145815 0.0037864451682318\\
47066.7866786679 0.0037868018015423\\
47075.4275427543 0.00378715840213294\\
47084.0684068407 0.00378751497001272\\
47092.7092709271 0.00378787150519065\\
47101.3501350135 0.00378822800767571\\
47109.9909990999 0.00378858447747691\\
47118.6318631863 0.00378894091460323\\
47127.2727272727 0.00378929731906365\\
47135.9135913591 0.00378965369086715\\
47144.5544554455 0.0037900100300227\\
47153.195319532 0.00379036633653929\\
47161.8361836184 0.00379072261042587\\
47170.4770477048 0.00379107885169141\\
47179.1179117912 0.00379143506034486\\
47187.7587758776 0.00379179123639519\\
47196.399639964 0.00379214737985134\\
47205.0405040504 0.00379250349072225\\
47213.6813681368 0.00379285956901688\\
47222.3222322232 0.00379321561474416\\
47230.9630963096 0.00379357162791303\\
47239.603960396 0.00379392760853241\\
47248.2448244824 0.00379428355661123\\
47256.8856885689 0.00379463947215841\\
47265.5265526553 0.00379499535518287\\
47274.1674167417 0.00379535120569353\\
47282.8082808281 0.00379570702369929\\
47291.4491449145 0.00379606280920906\\
47300.0900090009 0.00379641856223175\\
47308.7308730873 0.00379677428277624\\
47317.3717371737 0.00379712997085144\\
47326.0126012601 0.00379748562646623\\
47334.6534653465 0.00379784124962949\\
47343.2943294329 0.00379819684035012\\
47351.9351935194 0.00379855239863698\\
47360.5760576058 0.00379890792449896\\
47369.2169216922 0.00379926341794491\\
47377.8577857786 0.00379961887898371\\
47386.498649865 0.00379997430762422\\
47395.1395139514 0.00380032970387528\\
47403.7803780378 0.00380068506774577\\
47412.4212421242 0.00380104039924452\\
47421.0621062106 0.00380139569838038\\
47429.702970297 0.00380175096516219\\
47438.3438343834 0.00380210619959879\\
47446.9846984698 0.00380246140169901\\
47455.6255625563 0.00380281657147168\\
47464.2664266427 0.00380317170892563\\
47472.9072907291 0.00380352681406967\\
47481.5481548155 0.00380388188691262\\
47490.1890189019 0.00380423692746331\\
47498.8298829883 0.00380459193573052\\
47507.4707470747 0.00380494691172308\\
47516.1116111611 0.00380530185544978\\
47524.7524752475 0.00380565676691941\\
47533.3933393339 0.00380601164614078\\
47542.0342034203 0.00380636649312267\\
47550.6750675068 0.00380672130787386\\
47559.3159315932 0.00380707609040313\\
47567.9567956796 0.00380743084071927\\
47576.597659766 0.00380778555883104\\
47585.2385238524 0.00380814024474722\\
47593.8793879388 0.00380849489847657\\
47602.5202520252 0.00380884952002784\\
47611.1611161116 0.00380920410940981\\
47619.801980198 0.00380955866663121\\
47628.4428442844 0.00380991319170081\\
47637.0837083708 0.00381026768462733\\
47645.7245724572 0.00381062214541954\\
47654.3654365437 0.00381097657408616\\
47663.0063006301 0.00381133097063592\\
47671.6471647165 0.00381168533507756\\
47680.2880288029 0.0038120396674198\\
47688.9288928893 0.00381239396767136\\
47697.5697569757 0.00381274823584096\\
47706.2106210621 0.00381310247193732\\
47714.8514851485 0.00381345667596914\\
47723.4923492349 0.00381381084794513\\
47732.1332133213 0.00381416498787398\\
47740.7740774077 0.00381451909576441\\
47749.4149414941 0.0038148731716251\\
47758.0558055806 0.00381522721546473\\
47766.696669667 0.00381558122729201\\
47775.3375337534 0.0038159352071156\\
47783.9783978398 0.0038162891549442\\
47792.6192619262 0.00381664307078646\\
47801.2601260126 0.00381699695465107\\
47809.900990099 0.00381735080654669\\
47818.5418541854 0.00381770462648197\\
47827.1827182718 0.00381805841446559\\
47835.8235823582 0.00381841217050619\\
47844.4644464446 0.00381876589461242\\
47853.1053105311 0.00381911958679294\\
47861.7461746175 0.00381947324705638\\
47870.3870387039 0.00381982687541138\\
47879.0279027903 0.00382018047186659\\
47887.6687668767 0.00382053403643062\\
47896.3096309631 0.00382088756911211\\
47904.9504950495 0.00382124106991968\\
47913.5913591359 0.00382159453886195\\
47922.2322232223 0.00382194797594753\\
47930.8730873087 0.00382230138118505\\
47939.5139513951 0.0038226547545831\\
47948.1548154815 0.00382300809615028\\
47956.795679568 0.00382336140589521\\
47965.4365436544 0.00382371468382647\\
47974.0774077408 0.00382406792995267\\
47982.7182718272 0.00382442114428237\\
47991.3591359136 0.00382477432682418\\
48000 0.00382512747758667\\
48008.6408640864 0.00382548059657843\\
48017.2817281728 0.00382583368380801\\
48025.9225922592 0.00382618673928399\\
48034.5634563456 0.00382653976301494\\
48043.204320432 0.00382689275500942\\
48051.8451845185 0.00382724571527599\\
48060.4860486049 0.0038275986438232\\
48069.1269126913 0.00382795154065959\\
48077.7677767777 0.00382830440579373\\
48086.4086408641 0.00382865723923414\\
48095.0495049505 0.00382901004098936\\
48103.6903690369 0.00382936281106794\\
48112.3312331233 0.0038297155494784\\
48120.9720972097 0.00383006825622927\\
48129.6129612961 0.00383042093132908\\
48138.2538253825 0.00383077357478633\\
48146.8946894689 0.00383112618660956\\
48155.5355535554 0.00383147876680726\\
48164.1764176418 0.00383183131538795\\
48172.8172817282 0.00383218383236013\\
48181.4581458146 0.0038325363177323\\
48190.099009901 0.00383288877151296\\
48198.7398739874 0.0038332411937106\\
48207.3807380738 0.00383359358433371\\
48216.0216021602 0.00383394594339077\\
48224.6624662466 0.00383429827089027\\
48233.303330333 0.00383465056684068\\
48241.9441944194 0.00383500283125048\\
48250.5850585059 0.00383535506412813\\
48259.2259225923 0.00383570726548211\\
48267.8667866787 0.00383605943532087\\
48276.5076507651 0.00383641157365288\\
48285.1485148515 0.00383676368048658\\
48293.7893789379 0.00383711575583043\\
48302.4302430243 0.00383746779969288\\
48311.0711071107 0.00383781981208237\\
48319.7119711971 0.00383817179300734\\
48328.3528352835 0.00383852374247622\\
48336.9936993699 0.00383887566049745\\
48345.6345634563 0.00383922754707946\\
48354.2754275428 0.00383957940223066\\
48362.9162916292 0.00383993122595949\\
48371.5571557156 0.00384028301827435\\
48380.198019802 0.00384063477918366\\
48388.8388838884 0.00384098650869584\\
48397.4797479748 0.00384133820681928\\
48406.1206120612 0.00384168987356239\\
48414.7614761476 0.00384204150893356\\
48423.402340234 0.0038423931129412\\
48432.0432043204 0.00384274468559368\\
48440.6840684068 0.00384309622689941\\
48449.3249324932 0.00384344773686676\\
48457.9657965797 0.00384379921550411\\
48466.6066606661 0.00384415066281983\\
48475.2475247525 0.00384450207882231\\
48483.8883888389 0.0038448534635199\\
48492.5292529253 0.00384520481692098\\
48501.1701170117 0.0038455561390339\\
48509.8109810981 0.00384590742986701\\
48518.4518451845 0.00384625868942869\\
48527.0927092709 0.00384660991772726\\
48535.7335733573 0.00384696111477109\\
48544.3744374437 0.00384731228056851\\
48553.0153015302 0.00384766341512786\\
48561.6561656166 0.00384801451845747\\
48570.297029703 0.00384836559056568\\
48578.9378937894 0.00384871663146082\\
48587.5787578758 0.0038490676411512\\
48596.2196219622 0.00384941861964515\\
48604.8604860486 0.00384976956695099\\
48613.501350135 0.00385012048307702\\
48622.1422142214 0.00385047136803156\\
48630.7830783078 0.00385082222182291\\
48639.4239423942 0.00385117304445937\\
48648.0648064806 0.00385152383594924\\
48656.7056705671 0.00385187459630082\\
48665.3465346535 0.0038522253255224\\
48673.9873987399 0.00385257602362225\\
48682.6282628263 0.00385292669060868\\
48691.2691269127 0.00385327732648994\\
48699.9099909991 0.00385362793127433\\
48708.5508550855 0.00385397850497012\\
48717.1917191719 0.00385432904758556\\
48725.8325832583 0.00385467955912893\\
48734.4734473447 0.00385503003960849\\
48743.1143114311 0.0038553804890325\\
48751.7551755176 0.0038557309074092\\
48760.396039604 0.00385608129474686\\
48769.0369036904 0.00385643165105371\\
48777.6777677768 0.003856781976338\\
48786.3186318632 0.00385713227060797\\
48794.9594959496 0.00385748253387185\\
48803.600360036 0.00385783276613788\\
48812.2412241224 0.00385818296741429\\
48820.8820882088 0.00385853313770929\\
48829.5229522952 0.00385888327703112\\
48838.1638163816 0.00385923338538798\\
48846.804680468 0.00385958346278809\\
48855.4455445545 0.00385993350923967\\
48864.0864086409 0.00386028352475091\\
48872.7272727273 0.00386063350933003\\
48881.3681368137 0.00386098346298521\\
48890.0090009001 0.00386133338572466\\
48898.6498649865 0.00386168327755657\\
48907.2907290729 0.00386203313848913\\
48915.9315931593 0.00386238296853052\\
48924.5724572457 0.00386273276768892\\
48933.2133213321 0.00386308253597251\\
48941.8541854185 0.00386343227338947\\
48950.4950495049 0.00386378197994796\\
48959.1359135914 0.00386413165565615\\
48967.7767776778 0.00386448130052221\\
48976.4176417642 0.00386483091455429\\
48985.0585058506 0.00386518049776055\\
48993.699369937 0.00386553005014915\\
49002.3402340234 0.00386587957172822\\
49010.9810981098 0.00386622906250593\\
49019.6219621962 0.0038665785224904\\
49028.2628262826 0.00386692795168978\\
49036.903690369 0.00386727735011219\\
49045.5445544554 0.00386762671776578\\
49054.1854185419 0.00386797605465867\\
49062.8262826283 0.00386832536079899\\
49071.4671467147 0.00386867463619484\\
49080.1080108011 0.00386902388085435\\
49088.7488748875 0.00386937309478564\\
49097.3897389739 0.00386972227799681\\
49106.0306030603 0.00387007143049596\\
49114.6714671467 0.00387042055229121\\
49123.3123312331 0.00387076964339064\\
49131.9531953195 0.00387111870380236\\
49140.5940594059 0.00387146773353445\\
49149.2349234923 0.00387181673259501\\
49157.8757875788 0.00387216570099212\\
49166.5166516652 0.00387251463873385\\
49175.1575157516 0.00387286354582829\\
49183.798379838 0.00387321242228351\\
49192.4392439244 0.00387356126810758\\
49201.0801080108 0.00387391008330857\\
49209.7209720972 0.00387425886789454\\
49218.3618361836 0.00387460762187355\\
49227.00270027 0.00387495634525365\\
49235.6435643564 0.00387530503804291\\
49244.2844284428 0.00387565370024936\\
49252.9252925293 0.00387600233188106\\
49261.5661566157 0.00387635093294604\\
49270.2070207021 0.00387669950345235\\
49278.8478847885 0.00387704804340801\\
49287.4887488749 0.00387739655282107\\
49296.1296129613 0.00387774503169955\\
49304.7704770477 0.00387809348005146\\
49313.4113411341 0.00387844189788485\\
49322.0522052205 0.00387879028520771\\
49330.6930693069 0.00387913864202807\\
49339.3339333933 0.00387948696835393\\
49347.9747974798 0.00387983526419331\\
49356.6156615662 0.0038801835295542\\
49365.2565256526 0.00388053176444461\\
49373.897389739 0.00388087996887253\\
49382.5382538254 0.00388122814284596\\
49391.1791179118 0.00388157628637288\\
49399.8199819982 0.00388192439946128\\
49408.4608460846 0.00388227248211915\\
49417.101710171 0.00388262053435445\\
49425.7425742574 0.00388296855617518\\
49434.3834383438 0.00388331654758929\\
49443.0243024302 0.00388366450860476\\
49451.6651665167 0.00388401243922955\\
49460.3060306031 0.00388436033947163\\
49468.9468946895 0.00388470820933895\\
49477.5877587759 0.00388505604883947\\
49486.2286228623 0.00388540385798113\\
49494.8694869487 0.00388575163677189\\
49503.5103510351 0.00388609938521969\\
49512.1512151215 0.00388644710333246\\
49520.7920792079 0.00388679479111816\\
49529.4329432943 0.00388714244858471\\
49538.0738073807 0.00388749007574004\\
49546.7146714671 0.00388783767259207\\
49555.3555355536 0.00388818523914874\\
49563.99639964 0.00388853277541796\\
49572.6372637264 0.00388888028140765\\
49581.2781278128 0.00388922775712572\\
49589.9189918992 0.00388957520258009\\
49598.5598559856 0.00388992261777864\\
49607.200720072 0.0038902700027293\\
49615.8415841584 0.00389061735743996\\
49624.4824482448 0.00389096468191851\\
49633.1233123312 0.00389131197617285\\
49641.7641764176 0.00389165924021087\\
49650.405040504 0.00389200647404044\\
49659.0459045905 0.00389235367766946\\
49667.6867686769 0.00389270085110581\\
49676.3276327633 0.00389304799435735\\
49684.9684968497 0.00389339510743196\\
49693.6093609361 0.00389374219033751\\
49702.2502250225 0.00389408924308186\\
49710.8910891089 0.00389443626567288\\
49719.5319531953 0.00389478325811842\\
49728.1728172817 0.00389513022042633\\
49736.8136813681 0.00389547715260448\\
49745.4545454545 0.0038958240546607\\
49754.095409541 0.00389617092660285\\
49762.7362736274 0.00389651776843875\\
49771.3771377138 0.00389686458017626\\
49780.0180018002 0.0038972113618232\\
49788.6588658866 0.00389755811338741\\
49797.299729973 0.00389790483487671\\
49805.9405940594 0.00389825152629892\\
49814.5814581458 0.00389859818766188\\
49823.2223222322 0.00389894481897339\\
49831.8631863186 0.00389929142024127\\
49840.504050405 0.00389963799147332\\
49849.1449144915 0.00389998453267737\\
49857.7857785779 0.00390033104386121\\
49866.4266426643 0.00390067752503263\\
49875.0675067507 0.00390102397619945\\
49883.7083708371 0.00390137039736944\\
49892.3492349235 0.00390171678855041\\
49900.9900990099 0.00390206314975014\\
49909.6309630963 0.00390240948097641\\
49918.2718271827 0.003902755782237\\
49926.9126912691 0.0039031020535397\\
49935.5535553555 0.00390344829489226\\
49944.1944194419 0.00390379450630247\\
49952.8352835284 0.00390414068777809\\
49961.4761476148 0.00390448683932688\\
49970.1170117012 0.00390483296095661\\
49978.7578757876 0.00390517905267503\\
49987.398739874 0.00390552511448988\\
49996.0396039604 0.00390587114640894\\
50004.6804680468 0.00390621714843993\\
50013.3213321332 0.00390656312059061\\
50021.9621962196 0.0039069090628687\\
50030.603060306 0.00390725497528196\\
50039.2439243924 0.00390760085783812\\
50047.8847884788 0.00390794671054489\\
50056.5256525653 0.00390829253341001\\
50065.1665166517 0.00390863832644121\\
50073.8073807381 0.0039089840896462\\
50082.4482448245 0.0039093298230327\\
50091.0891089109 0.00390967552660842\\
50099.7299729973 0.00391002120038108\\
50108.3708370837 0.00391036684435837\\
50117.0117011701 0.00391071245854801\\
50125.6525652565 0.00391105804295769\\
50134.2934293429 0.00391140359759511\\
50142.9342934293 0.00391174912246796\\
50151.5751575157 0.00391209461758393\\
50160.2160216022 0.00391244008295072\\
50168.8568856886 0.00391278551857599\\
50177.497749775 0.00391313092446744\\
50186.1386138614 0.00391347630063274\\
50194.7794779478 0.00391382164707956\\
50203.4203420342 0.00391416696381557\\
50212.0612061206 0.00391451225084845\\
50220.702070207 0.00391485750818585\\
50229.3429342934 0.00391520273583543\\
50237.9837983798 0.00391554793380486\\
50246.6246624662 0.00391589310210178\\
50255.2655265527 0.00391623824073384\\
50263.9063906391 0.0039165833497087\\
50272.5472547255 0.003916928429034\\
50281.1881188119 0.00391727347871737\\
50289.8289828983 0.00391761849876645\\
50298.4698469847 0.00391796348918889\\
50307.1107110711 0.0039183084499923\\
50315.7515751575 0.00391865338118432\\
50324.3924392439 0.00391899828277257\\
50333.0333033303 0.00391934315476468\\
50341.6741674167 0.00391968799716825\\
50350.3150315032 0.0039200328099909\\
50358.9558955896 0.00392037759324026\\
50367.596759676 0.00392072234692391\\
50376.2376237624 0.00392106707104947\\
50384.8784878488 0.00392141176562454\\
50393.5193519352 0.00392175643065672\\
50402.1602160216 0.0039221010661536\\
50410.801080108 0.00392244567212277\\
50419.4419441944 0.00392279024857182\\
50428.0828082808 0.00392313479550834\\
50436.7236723672 0.00392347931293991\\
50445.3645364536 0.00392382380087411\\
50454.0054005401 0.00392416825931851\\
50462.6462646265 0.00392451268828068\\
50471.2871287129 0.0039248570877682\\
50479.9279927993 0.00392520145778862\\
50488.5688568857 0.00392554579834952\\
50497.2097209721 0.00392589010945845\\
50505.8505850585 0.00392623439112296\\
50514.4914491449 0.00392657864335061\\
50523.1323132313 0.00392692286614896\\
50531.7731773177 0.00392726705952553\\
50540.4140414041 0.00392761122348789\\
50549.0549054905 0.00392795535804356\\
50557.695769577 0.00392829946320009\\
50566.3366336634 0.003928643538965\\
50574.9774977498 0.00392898758534584\\
50583.6183618362 0.00392933160235011\\
50592.2592259226 0.00392967558998536\\
50600.900090009 0.00393001954825909\\
50609.5409540954 0.00393036347717883\\
50618.1818181818 0.00393070737675209\\
50626.8226822682 0.00393105124698638\\
50635.4635463546 0.00393139508788921\\
50644.104410441 0.00393173889946809\\
50652.7452745274 0.00393208268173051\\
50661.3861386139 0.00393242643468397\\
50670.0270027003 0.00393277015833597\\
50678.6678667867 0.00393311385269399\\
50687.3087308731 0.00393345751776554\\
50695.9495949595 0.00393380115355809\\
50704.5904590459 0.00393414476007912\\
50713.2313231323 0.00393448833733612\\
50721.8721872187 0.00393483188533655\\
50730.5130513051 0.0039351754040879\\
50739.1539153915 0.00393551889359763\\
50747.7947794779 0.00393586235387321\\
50756.4356435644 0.0039362057849221\\
50765.0765076508 0.00393654918675176\\
50773.7173717372 0.00393689255936965\\
50782.3582358236 0.00393723590278321\\
50790.99909991 0.00393757921699991\\
50799.6399639964 0.00393792250202719\\
50808.2808280828 0.00393826575787249\\
50816.9216921692 0.00393860898454325\\
50825.5625562556 0.00393895218204692\\
50834.203420342 0.00393929535039092\\
50842.8442844284 0.00393963848958268\\
50851.4851485149 0.00393998159962965\\
50860.1260126013 0.00394032468053923\\
50868.7668766877 0.00394066773231885\\
50877.4077407741 0.00394101075497594\\
50886.0486048605 0.0039413537485179\\
50894.6894689469 0.00394169671295214\\
50903.3303330333 0.00394203964828609\\
50911.9711971197 0.00394238255452714\\
50920.6120612061 0.0039427254316827\\
50929.2529252925 0.00394306827976017\\
50937.8937893789 0.00394341109876694\\
50946.5346534653 0.0039437538887104\\
50955.1755175518 0.00394409664959796\\
50963.8163816382 0.00394443938143699\\
50972.4572457246 0.00394478208423488\\
50981.098109811 0.00394512475799901\\
50989.7389738974 0.00394546740273676\\
50998.3798379838 0.0039458100184555\\
51007.0207020702 0.00394615260516261\\
51015.6615661566 0.00394649516286545\\
51024.302430243 0.00394683769157139\\
51032.9432943294 0.00394718019128779\\
51041.5841584158 0.00394752266202201\\
51050.2250225023 0.00394786510378141\\
51058.8658865887 0.00394820751657334\\
51067.5067506751 0.00394854990040515\\
51076.1476147615 0.00394889225528419\\
51084.7884788479 0.00394923458121779\\
51093.4293429343 0.00394957687821331\\
51102.0702070207 0.00394991914627808\\
51110.7110711071 0.00395026138541942\\
51119.3519351935 0.00395060359564469\\
51127.9927992799 0.00395094577696119\\
51136.6336633663 0.00395128792937626\\
51145.2745274527 0.00395163005289723\\
51153.9153915392 0.0039519721475314\\
51162.5562556256 0.00395231421328609\\
51171.197119712 0.00395265625016862\\
51179.8379837984 0.0039529982581863\\
51188.4788478848 0.00395334023734643\\
51197.1197119712 0.00395368218765632\\
51205.7605760576 0.00395402410912327\\
51214.401440144 0.00395436600175457\\
51223.0423042304 0.00395470786555751\\
51231.6831683168 0.0039550497005394\\
51240.3240324032 0.00395539150670752\\
51248.9648964896 0.00395573328406915\\
51257.6057605761 0.00395607503263157\\
51266.2466246625 0.00395641675240207\\
51274.8874887489 0.00395675844338791\\
51283.5283528353 0.00395710010559638\\
51292.1692169217 0.00395744173903474\\
51300.8100810081 0.00395778334371027\\
51309.4509450945 0.00395812491963021\\
51318.0918091809 0.00395846646680183\\
51326.7326732673 0.0039588079852324\\
51335.3735373537 0.00395914947492916\\
51344.0144014401 0.00395949093589937\\
51352.6552655266 0.00395983236815027\\
51361.296129613 0.00396017377168911\\
51369.9369936994 0.00396051514652313\\
51378.5778577858 0.00396085649265958\\
51387.2187218722 0.00396119781010568\\
51395.8595859586 0.00396153909886867\\
51404.500450045 0.00396188035895578\\
51413.1413141314 0.00396222159037424\\
51421.7821782178 0.00396256279313128\\
51430.4230423042 0.0039629039672341\\
51439.0639063906 0.00396324511268994\\
51447.704770477 0.003963586229506\\
51456.3456345635 0.00396392731768949\\
51464.9864986499 0.00396426837724764\\
51473.6273627363 0.00396460940818763\\
51482.2682268227 0.00396495041051668\\
51490.9090909091 0.00396529138424198\\
51499.5499549955 0.00396563232937074\\
51508.1908190819 0.00396597324591014\\
51516.8316831683 0.00396631413386738\\
51525.4725472547 0.00396665499324963\\
51534.1134113411 0.0039669958240641\\
51542.7542754275 0.00396733662631796\\
51551.395139514 0.00396767740001838\\
51560.0360036004 0.00396801814517255\\
51568.6768676868 0.00396835886178764\\
51577.3177317732 0.00396869954987081\\
51585.9585958596 0.00396904020942924\\
51594.599459946 0.00396938084047008\\
51603.2403240324 0.0039697214430005\\
51611.8811881188 0.00397006201702765\\
51620.5220522052 0.0039704025625587\\
51629.1629162916 0.00397074307960078\\
51637.803780378 0.00397108356816106\\
51646.4446444644 0.00397142402824667\\
51655.0855085509 0.00397176445986476\\
51663.7263726373 0.00397210486302247\\
51672.3672367237 0.00397244523772693\\
51681.0081008101 0.00397278558398528\\
51689.6489648965 0.00397312590180465\\
51698.2898289829 0.00397346619119217\\
51706.9306930693 0.00397380645215496\\
51715.5715571557 0.00397414668470015\\
51724.2124212421 0.00397448688883485\\
51732.8532853285 0.00397482706456617\\
51741.4941494149 0.00397516721190124\\
51750.1350135013 0.00397550733084716\\
51758.7758775878 0.00397584742141104\\
51767.4167416742 0.00397618748359998\\
51776.0576057606 0.00397652751742109\\
51784.698469847 0.00397686752288145\\
51793.3393339334 0.00397720749998818\\
51801.9801980198 0.00397754744874835\\
51810.6210621062 0.00397788736916906\\
51819.2619261926 0.0039782272612574\\
51827.902790279 0.00397856712502045\\
51836.5436543654 0.00397890696046528\\
51845.1845184518 0.00397924676759898\\
51853.8253825383 0.00397958654642862\\
51862.4662466247 0.00397992629696127\\
51871.1071107111 0.003980266019204\\
51879.7479747975 0.00398060571316388\\
51888.3888388839 0.00398094537884797\\
51897.0297029703 0.00398128501626332\\
51905.6705670567 0.003981624625417\\
51914.3114311431 0.00398196420631606\\
51922.9522952295 0.00398230375896756\\
51931.5931593159 0.00398264328337852\\
51940.2340234023 0.00398298277955602\\
51948.8748874887 0.00398332224750707\\
51957.5157515752 0.00398366168723874\\
51966.1566156616 0.00398400109875804\\
51974.797479748 0.00398434048207202\\
51983.4383438344 0.0039846798371877\\
51992.0792079208 0.00398501916411212\\
52000.7200720072 0.0039853584628523\\
52009.3609360936 0.00398569773341525\\
52018.00180018 0.003986036975808\\
52026.6426642664 0.00398637619003756\\
52035.2835283528 0.00398671537611095\\
52043.9243924392 0.00398705453403518\\
52052.5652565257 0.00398739366381725\\
52061.2061206121 0.00398773276546417\\
52069.8469846985 0.00398807183898294\\
52078.4878487849 0.00398841088438056\\
52087.1287128713 0.00398874990166402\\
52095.7695769577 0.00398908889084032\\
52104.4104410441 0.00398942785191644\\
52113.0513051305 0.00398976678489938\\
52121.6921692169 0.00399010568979611\\
52130.3330333033 0.00399044456661362\\
52138.9738973897 0.00399078341535888\\
52147.6147614762 0.00399112223603888\\
52156.2556255626 0.00399146102866058\\
52164.896489649 0.00399179979323096\\
52173.5373537354 0.00399213852975697\\
52182.1782178218 0.00399247723824559\\
52190.8190819082 0.00399281591870377\\
52199.4599459946 0.00399315457113847\\
52208.100810081 0.00399349319555665\\
52216.7416741674 0.00399383179196526\\
52225.3825382538 0.00399417036037125\\
52234.0234023402 0.00399450890078156\\
52242.6642664266 0.00399484741320315\\
52251.3051305131 0.00399518589764294\\
52259.9459945995 0.00399552435410788\\
52268.5868586859 0.0039958627826049\\
52277.2277227723 0.00399620118314094\\
52285.8685868587 0.00399653955572292\\
52294.5094509451 0.00399687790035778\\
52303.1503150315 0.00399721621705242\\
52311.7911791179 0.00399755450581378\\
52320.4320432043 0.00399789276664878\\
52329.0729072907 0.00399823099956432\\
52337.7137713771 0.00399856920456732\\
52346.3546354635 0.00399890738166469\\
52354.99549955 0.00399924553086333\\
52363.6363636364 0.00399958365217016\\
52372.2772277228 0.00399992174559206\\
52380.9180918092 0.00400025981113594\\
52389.5589558956 0.00400059784880869\\
52398.199819982 0.00400093585861721\\
52406.8406840684 0.00400127384056838\\
52415.4815481548 0.0040016117946691\\
52424.1224122412 0.00400194972092623\\
52432.7632763276 0.00400228761934667\\
52441.404140414 0.0040026254899373\\
52450.0450045004 0.00400296333270498\\
52458.6858685869 0.0040033011476566\\
52467.3267326733 0.00400363893479901\\
52475.9675967597 0.00400397669413909\\
52484.6084608461 0.0040043144256837\\
52493.2493249325 0.00400465212943971\\
52501.8901890189 0.00400498980541396\\
52510.5310531053 0.00400532745361332\\
52519.1719171917 0.00400566507404464\\
52527.8127812781 0.00400600266671476\\
52536.4536453645 0.00400634023163054\\
52545.0945094509 0.00400667776879882\\
52553.7353735374 0.00400701527822644\\
52562.3762376238 0.00400735275992023\\
52571.0171017102 0.00400769021388705\\
52579.6579657966 0.0040080276401337\\
52588.298829883 0.00400836503866704\\
52596.9396939694 0.00400870240949388\\
52605.5805580558 0.00400903975262105\\
52614.2214221422 0.00400937706805537\\
52622.8622862286 0.00400971435580366\\
52631.503150315 0.00401005161587274\\
52640.1440144014 0.00401038884826941\\
52648.7848784879 0.0040107260530005\\
52657.4257425743 0.0040110632300728\\
52666.0666066607 0.00401140037949312\\
52674.7074707471 0.00401173750126827\\
52683.3483348335 0.00401207459540504\\
52691.9891989199 0.00401241166191024\\
52700.6300630063 0.00401274870079064\\
52709.2709270927 0.00401308571205305\\
52717.9117911791 0.00401342269570426\\
52726.5526552655 0.00401375965175104\\
52735.1935193519 0.00401409658020019\\
52743.8343834383 0.00401443348105847\\
52752.4752475248 0.00401477035433268\\
52761.1161116112 0.00401510720002957\\
52769.7569756976 0.00401544401815593\\
52778.397839784 0.00401578080871852\\
52787.0387038704 0.00401611757172411\\
52795.6795679568 0.00401645430717947\\
52804.3204320432 0.00401679101509134\\
52812.9612961296 0.0040171276954665\\
52821.602160216 0.00401746434831169\\
52830.2430243024 0.00401780097363367\\
52838.8838883888 0.00401813757143918\\
52847.5247524752 0.00401847414173498\\
52856.1656165617 0.0040188106845278\\
52864.8064806481 0.0040191471998244\\
52873.4473447345 0.0040194836876315\\
52882.0882088209 0.00401982014795584\\
52890.7290729073 0.00402015658080416\\
52899.3699369937 0.00402049298618318\\
52908.0108010801 0.00402082936409964\\
52916.6516651665 0.00402116571456025\\
52925.2925292529 0.00402150203757174\\
52933.9333933393 0.00402183833314083\\
52942.5742574257 0.00402217460127424\\
52951.2151215121 0.00402251084197868\\
52959.8559855986 0.00402284705526085\\
52968.496849685 0.00402318324112747\\
52977.1377137714 0.00402351939958525\\
52985.7785778578 0.00402385553064088\\
52994.4194419442 0.00402419163430107\\
53003.0603060306 0.00402452771057251\\
53011.701170117 0.0040248637594619\\
53020.3420342034 0.00402519978097592\\
53028.9828982898 0.00402553577512128\\
53037.6237623762 0.00402587174190465\\
53046.2646264626 0.00402620768133272\\
53054.9054905491 0.00402654359341217\\
53063.5463546355 0.00402687947814968\\
53072.1872187219 0.00402721533555192\\
53080.8280828083 0.00402755116562557\\
53089.4689468947 0.0040278869683773\\
53098.1098109811 0.00402822274381377\\
53106.7506750675 0.00402855849194164\\
53115.3915391539 0.00402889421276759\\
53124.0324032403 0.00402922990629826\\
53132.6732673267 0.00402956557254033\\
53141.3141314131 0.00402990121150043\\
53149.9549954996 0.00403023682318522\\
53158.595859586 0.00403057240760135\\
53167.2367236724 0.00403090796475547\\
53175.8775877588 0.00403124349465421\\
53184.5184518452 0.00403157899730423\\
53193.1593159316 0.00403191447271214\\
53201.800180018 0.0040322499208846\\
53210.4410441044 0.00403258534182824\\
53219.0819081908 0.00403292073554967\\
53227.7227722772 0.00403325610205554\\
53236.3636363636 0.00403359144135246\\
53245.00450045 0.00403392675344705\\
53253.6453645365 0.00403426203834594\\
53262.2862286229 0.00403459729605573\\
53270.9270927093 0.00403493252658305\\
53279.5679567957 0.00403526772993451\\
53288.2088208821 0.0040356029061167\\
53296.8496849685 0.00403593805513625\\
53305.4905490549 0.00403627317699974\\
53314.1314131413 0.00403660827171379\\
53322.7722772277 0.00403694333928498\\
53331.4131413141 0.00403727837971991\\
53340.0540054005 0.00403761339302518\\
53348.6948694869 0.00403794837920738\\
53357.3357335734 0.00403828333827308\\
53365.9765976598 0.00403861827022888\\
53374.6174617462 0.00403895317508135\\
53383.2583258326 0.00403928805283708\\
53391.899189919 0.00403962290350264\\
53400.5400540054 0.0040399577270846\\
53409.1809180918 0.00404029252358953\\
53417.8217821782 0.00404062729302401\\
53426.4626462646 0.0040409620353946\\
53435.103510351 0.00404129675070786\\
53443.7443744374 0.00404163143897035\\
53452.3852385238 0.00404196610018862\\
53461.0261026103 0.00404230073436924\\
53469.6669666967 0.00404263534151876\\
53478.3078307831 0.00404296992164372\\
53486.9486948695 0.00404330447475067\\
53495.5895589559 0.00404363900084616\\
53504.2304230423 0.00404397349993672\\
53512.8712871287 0.0040443079720289\\
53521.5121512151 0.00404464241712924\\
53530.1530153015 0.00404497683524426\\
53538.7938793879 0.0040453112263805\\
53547.4347434743 0.00404564559054448\\
53556.0756075608 0.00404597992774273\\
53564.7164716472 0.00404631423798178\\
53573.3573357336 0.00404664852126815\\
53581.99819982 0.00404698277760834\\
53590.6390639064 0.00404731700700889\\
53599.2799279928 0.0040476512094763\\
53607.9207920792 0.00404798538501707\\
53616.5616561656 0.00404831953363773\\
53625.202520252 0.00404865365534476\\
53633.8433843384 0.00404898775014469\\
53642.4842484248 0.00404932181804399\\
53651.1251125113 0.00404965585904918\\
53659.7659765977 0.00404998987316675\\
53668.4068406841 0.00405032386040318\\
53677.0477047705 0.00405065782076497\\
53685.6885688569 0.0040509917542586\\
53694.3294329433 0.00405132566089056\\
53702.9702970297 0.00405165954066732\\
53711.6111611161 0.00405199339359537\\
53720.2520252025 0.00405232721968119\\
53728.8928892889 0.00405266101893124\\
53737.5337533753 0.004052994791352\\
53746.1746174617 0.00405332853694993\\
53754.8154815482 0.0040536622557315\\
53763.4563456346 0.00405399594770318\\
53772.097209721 0.00405432961287142\\
53780.7380738074 0.00405466325124269\\
53789.3789378938 0.00405499686282343\\
53798.0198019802 0.0040553304476201\\
53806.6606660666 0.00405566400563915\\
53815.301530153 0.00405599753688702\\
53823.9423942394 0.00405633104137017\\
53832.5832583258 0.00405666451909504\\
53841.2241224122 0.00405699797006806\\
53849.8649864987 0.00405733139429566\\
53858.5058505851 0.0040576647917843\\
53867.1467146715 0.00405799816254039\\
53875.7875787579 0.00405833150657037\\
53884.4284428443 0.00405866482388067\\
53893.0693069307 0.0040589981144777\\
53901.7101710171 0.0040593313783679\\
53910.3510351035 0.00405966461555767\\
53918.9918991899 0.00405999782605343\\
53927.6327632763 0.00406033100986161\\
53936.2736273627 0.00406066416698861\\
53944.9144914491 0.00406099729744084\\
53953.5553555356 0.0040613304012247\\
53962.196219622 0.0040616634783466\\
53970.8370837084 0.00406199652881295\\
53979.4779477948 0.00406232955263013\\
53988.1188118812 0.00406266254980455\\
53996.7596759676 0.0040629955203426\\
54005.400540054 0.00406332846425067\\
54014.0414041404 0.00406366138153514\\
54022.6822682268 0.00406399427220241\\
54031.3231323132 0.00406432713625886\\
54039.9639963996 0.00406465997371086\\
54048.604860486 0.0040649927845648\\
54057.2457245725 0.00406532556882705\\
54065.8865886589 0.00406565832650399\\
54074.5274527453 0.00406599105760198\\
54083.1683168317 0.00406632376212739\\
54091.8091809181 0.00406665644008659\\
54100.4500450045 0.00406698909148595\\
54109.0909090909 0.00406732171633181\\
54117.7317731773 0.00406765431463055\\
54126.3726372637 0.0040679868863885\\
54135.0135013501 0.00406831943161204\\
54143.6543654365 0.0040686519503075\\
54152.295229523 0.00406898444248124\\
54160.9360936094 0.0040693169081396\\
54169.5769576958 0.00406964934728892\\
54178.2178217822 0.00406998175993554\\
54186.8586858686 0.0040703141460858\\
54195.499549955 0.00407064650574604\\
54204.1404140414 0.00407097883892259\\
54212.7812781278 0.00407131114562177\\
54221.4221422142 0.00407164342584992\\
54230.0630063006 0.00407197567961336\\
54238.703870387 0.00407230790691841\\
54247.3447344734 0.0040726401077714\\
54255.9855985599 0.00407297228217863\\
54264.6264626463 0.00407330443014643\\
54273.2673267327 0.00407363655168111\\
54281.9081908191 0.00407396864678897\\
54290.5490549055 0.00407430071547633\\
54299.1899189919 0.00407463275774948\\
54307.8307830783 0.00407496477361474\\
54316.4716471647 0.00407529676307839\\
54325.1125112511 0.00407562872614675\\
54333.7533753375 0.0040759606628261\\
54342.3942394239 0.00407629257312273\\
54351.0351035104 0.00407662445704294\\
54359.6759675968 0.00407695631459301\\
54368.3168316832 0.00407728814577923\\
54376.9576957696 0.00407761995060788\\
54385.598559856 0.00407795172908524\\
54394.2394239424 0.00407828348121758\\
54402.8802880288 0.00407861520701118\\
54411.5211521152 0.00407894690647232\\
54420.1620162016 0.00407927857960727\\
54428.802880288 0.00407961022642228\\
54437.4437443744 0.00407994184692363\\
54446.0846084608 0.00408027344111758\\
54454.7254725473 0.00408060500901038\\
54463.3663366337 0.0040809365506083\\
54472.0072007201 0.0040812680659176\\
54480.6480648065 0.00408159955494451\\
54489.2889288929 0.0040819310176953\\
54497.9297929793 0.0040822624541762\\
54506.5706570657 0.00408259386439348\\
54515.2115211521 0.00408292524835336\\
54523.8523852385 0.00408325660606208\\
54532.4932493249 0.0040835879375259\\
54541.1341134113 0.00408391924275103\\
54549.7749774977 0.00408425052174372\\
54558.4158415842 0.00408458177451019\\
54567.0567056706 0.00408491300105667\\
54575.697569757 0.00408524420138939\\
54584.3384338434 0.00408557537551457\\
54592.9792979298 0.00408590652343843\\
54601.6201620162 0.00408623764516718\\
54610.2610261026 0.00408656874070705\\
54618.901890189 0.00408689981006425\\
54627.5427542754 0.00408723085324498\\
54636.1836183618 0.00408756187025545\\
54644.8244824482 0.00408789286110188\\
54653.4653465347 0.00408822382579046\\
54662.1062106211 0.00408855476432739\\
54670.7470747075 0.00408888567671887\\
54679.3879387939 0.0040892165629711\\
54688.0288028803 0.00408954742309028\\
54696.6696669667 0.00408987825708258\\
54705.3105310531 0.00409020906495421\\
54713.9513951395 0.00409053984671134\\
54722.5922592259 0.00409087060236017\\
54731.2331233123 0.00409120133190686\\
54739.8739873987 0.00409153203535761\\
54748.5148514851 0.00409186271271858\\
54757.1557155716 0.00409219336399595\\
54765.796579658 0.0040925239891959\\
54774.4374437444 0.00409285458832458\\
54783.0783078308 0.00409318516138817\\
54791.7191719172 0.00409351570839283\\
54800.3600360036 0.00409384622934473\\
54809.00090009 0.00409417672425002\\
54817.6417641764 0.00409450719311485\\
54826.2826282628 0.00409483763594539\\
54834.9234923492 0.00409516805274779\\
54843.5643564356 0.00409549844352819\\
54852.2052205221 0.00409582880829274\\
54860.8460846085 0.00409615914704759\\
54869.4869486949 0.00409648945979888\\
54878.1278127813 0.00409681974655274\\
54886.7686768677 0.00409715000731532\\
54895.4095409541 0.00409748024209276\\
54904.0504050405 0.00409781045089117\\
54912.6912691269 0.00409814063371671\\
54921.3321332133 0.00409847079057548\\
54929.9729972997 0.00409880092147362\\
54938.6138613861 0.00409913102641725\\
54947.2547254726 0.00409946110541249\\
54955.895589559 0.00409979115846547\\
54964.5364536454 0.00410012118558228\\
54973.1773177318 0.00410045118676906\\
54981.8181818182 0.0041007811620319\\
54990.4590459046 0.00410111111137692\\
54999.099909991 0.00410144103481023\\
55007.7407740774 0.00410177093233792\\
55016.3816381638 0.0041021008039661\\
55025.0225022502 0.00410243064970087\\
55033.6633663366 0.00410276046954832\\
55042.304230423 0.00410309026351456\\
55050.9450945095 0.00410342003160566\\
55059.5859585959 0.00410374977382772\\
55068.2268226823 0.00410407949018683\\
55076.8676867687 0.00410440918068908\\
55085.5085508551 0.00410473884534053\\
55094.1494149415 0.00410506848414728\\
55102.7902790279 0.0041053980971154\\
55111.4311431143 0.00410572768425097\\
55120.0720072007 0.00410605724556006\\
55128.7128712871 0.00410638678104873\\
55137.3537353735 0.00410671629072307\\
55145.9945994599 0.00410704577458912\\
55154.6354635464 0.00410737523265296\\
55163.2763276328 0.00410770466492065\\
55171.9171917192 0.00410803407139825\\
55180.5580558056 0.00410836345209181\\
55189.198919892 0.00410869280700738\\
55197.8397839784 0.00410902213615102\\
55206.4806480648 0.00410935143952877\\
55215.1215121512 0.00410968071714669\\
55223.7623762376 0.00411000996901081\\
55232.403240324 0.00411033919512718\\
55241.0441044104 0.00411066839550184\\
55249.6849684968 0.00411099757014083\\
55258.3258325833 0.00411132671905017\\
55266.9666966697 0.00411165584223591\\
55275.6075607561 0.00411198493970407\\
55284.2484248425 0.00411231401146068\\
55292.8892889289 0.00411264305751177\\
55301.5301530153 0.00411297207786335\\
55310.1710171017 0.00411330107252146\\
55318.8118811881 0.0041136300414921\\
55327.4527452745 0.00411395898478129\\
55336.0936093609 0.00411428790239506\\
55344.7344734473 0.0041146167943394\\
55353.3753375338 0.00411494566062033\\
55362.0162016202 0.00411527450124385\\
55370.6570657066 0.00411560331621597\\
55379.297929793 0.0041159321055427\\
55387.9387938794 0.00411626086923002\\
55396.5796579658 0.00411658960728394\\
55405.2205220522 0.00411691831971045\\
55413.8613861386 0.00411724700651555\\
55422.502250225 0.00411757566770522\\
55431.1431143114 0.00411790430328545\\
55439.7839783978 0.00411823291326224\\
55448.4248424843 0.00411856149764155\\
55457.0657065707 0.00411889005642938\\
55465.7065706571 0.00411921858963169\\
55474.3474347435 0.00411954709725448\\
55482.9882988299 0.00411987557930371\\
55491.6291629163 0.00412020403578535\\
55500.2700270027 0.00412053246670537\\
55508.9108910891 0.00412086087206974\\
55517.5517551755 0.00412118925188442\\
55526.1926192619 0.00412151760615538\\
55534.8334833483 0.00412184593488858\\
55543.4743474347 0.00412217423808997\\
55552.1152115212 0.00412250251576551\\
55560.7560756076 0.00412283076792115\\
55569.396939694 0.00412315899456284\\
55578.0378037804 0.00412348719569654\\
55586.6786678668 0.00412381537132818\\
55595.3195319532 0.00412414352146372\\
55603.9603960396 0.00412447164610908\\
55612.601260126 0.00412479974527022\\
55621.2421242124 0.00412512781895307\\
55629.8829882988 0.00412545586716356\\
55638.5238523852 0.00412578388990763\\
55647.1647164716 0.0041261118871912\\
55655.8055805581 0.0041264398590202\\
55664.4464446445 0.00412676780540056\\
55673.0873087309 0.00412709572633821\\
55681.7281728173 0.00412742362183905\\
55690.3690369037 0.00412775149190902\\
55699.0099009901 0.00412807933655403\\
55707.6507650765 0.00412840715577998\\
55716.2916291629 0.0041287349495928\\
55724.9324932493 0.00412906271799839\\
55733.5733573357 0.00412939046100266\\
55742.2142214221 0.00412971817861151\\
55750.8550855085 0.00413004587083085\\
55759.495949595 0.00413037353766657\\
55768.1368136814 0.00413070117912458\\
55776.7776777678 0.00413102879521077\\
55785.4185418542 0.00413135638593103\\
55794.0594059406 0.00413168395129125\\
55802.700270027 0.00413201149129733\\
55811.3411341134 0.00413233900595515\\
55819.9819981998 0.00413266649527059\\
55828.6228622862 0.00413299395924953\\
55837.2637263726 0.00413332139789786\\
55845.904590459 0.00413364881122145\\
55854.5454545455 0.00413397619922618\\
55863.1863186319 0.00413430356191792\\
55871.8271827183 0.00413463089930253\\
55880.4680468047 0.0041349582113859\\
55889.1089108911 0.00413528549817388\\
55897.7497749775 0.00413561275967234\\
55906.3906390639 0.00413593999588715\\
55915.0315031503 0.00413626720682415\\
55923.6723672367 0.0041365943924892\\
55932.3132313231 0.00413692155288817\\
55940.9540954095 0.0041372486880269\\
55949.594959496 0.00413757579791124\\
55958.2358235824 0.00413790288254705\\
55966.8766876688 0.00413822994194016\\
55975.5175517552 0.00413855697609642\\
55984.1584158416 0.00413888398502167\\
55992.799279928 0.00413921096872176\\
56001.4401440144 0.00413953792720251\\
56010.0810081008 0.00413986486046976\\
56018.7218721872 0.00414019176852935\\
56027.3627362736 0.0041405186513871\\
56036.00360036 0.00414084550904884\\
56044.6444644464 0.0041411723415204\\
56053.2853285329 0.0041414991488076\\
56061.9261926193 0.00414182593091625\\
56070.5670567057 0.00414215268785219\\
56079.2079207921 0.00414247941962122\\
56087.8487848785 0.00414280612622916\\
56096.4896489649 0.00414313280768183\\
56105.1305130513 0.00414345946398503\\
56113.7713771377 0.00414378609514456\\
56122.4122412241 0.00414411270116624\\
56131.0531053105 0.00414443928205587\\
56139.6939693969 0.00414476583781924\\
56148.3348334833 0.00414509236846216\\
56156.9756975698 0.00414541887399043\\
56165.6165616562 0.00414574535440983\\
56174.2574257426 0.00414607180972616\\
56182.898289829 0.00414639823994521\\
56191.5391539154 0.00414672464507276\\
56200.1800180018 0.00414705102511461\\
56208.8208820882 0.00414737738007652\\
56217.4617461746 0.0041477037099643\\
56226.102610261 0.00414803001478371\\
56234.7434743474 0.00414835629454052\\
56243.3843384338 0.00414868254924052\\
56252.0252025202 0.00414900877888948\\
56260.6660666067 0.00414933498349316\\
56269.3069306931 0.00414966116305734\\
56277.9477947795 0.00414998731758777\\
56286.5886588659 0.00415031344709022\\
56295.2295229523 0.00415063955157045\\
56303.8703870387 0.00415096563103423\\
56312.5112511251 0.0041512916854873\\
56321.1521152115 0.00415161771493542\\
56329.7929792979 0.00415194371938435\\
56338.4338433843 0.00415226969883982\\
56347.0747074707 0.0041525956533076\\
56355.7155715572 0.00415292158279342\\
56364.3564356436 0.00415324748730304\\
56372.99729973 0.00415357336684218\\
56381.6381638164 0.00415389922141659\\
56390.2790279028 0.004154225051032\\
56398.9198919892 0.00415455085569415\\
56407.5607560756 0.00415487663540878\\
56416.201620162 0.0041552023901816\\
56424.8424842484 0.00415552812001836\\
56433.4833483348 0.00415585382492477\\
56442.1242124212 0.00415617950490656\\
56450.7650765077 0.00415650515996944\\
56459.4059405941 0.00415683079011915\\
56468.0468046805 0.00415715639536139\\
56476.6876687669 0.00415748197570188\\
56485.3285328533 0.00415780753114633\\
56493.9693969397 0.00415813306170046\\
56502.6102610261 0.00415845856736996\\
56511.2511251125 0.00415878404816056\\
56519.8919891989 0.00415910950407794\\
56528.5328532853 0.00415943493512781\\
56537.1737173717 0.00415976034131588\\
56545.8145814581 0.00416008572264783\\
56554.4554455446 0.00416041107912937\\
56563.096309631 0.00416073641076618\\
56571.7371737174 0.00416106171756396\\
56580.3780378038 0.0041613869995284\\
56589.0189018902 0.00416171225666518\\
56597.6597659766 0.00416203748897998\\
56606.300630063 0.00416236269647849\\
56614.9414941494 0.00416268787916639\\
56623.5823582358 0.00416301303704936\\
56632.2232223222 0.00416333817013306\\
56640.8640864086 0.00416366327842318\\
56649.5049504951 0.00416398836192539\\
56658.1458145815 0.00416431342064535\\
56666.7866786679 0.00416463845458873\\
56675.4275427543 0.0041649634637612\\
56684.0684068407 0.00416528844816841\\
56692.7092709271 0.00416561340781604\\
56701.3501350135 0.00416593834270973\\
56709.9909990999 0.00416626325285514\\
56718.6318631863 0.00416658813825794\\
56727.2727272727 0.00416691299892376\\
56735.9135913591 0.00416723783485826\\
56744.5544554455 0.00416756264606708\\
56753.195319532 0.00416788743255588\\
56761.8361836184 0.00416821219433028\\
56770.4770477048 0.00416853693139595\\
56779.1179117912 0.00416886164375851\\
56787.7587758776 0.0041691863314236\\
56796.399639964 0.00416951099439685\\
56805.0405040504 0.00416983563268391\\
56813.6813681368 0.00417016024629039\\
56822.3222322232 0.00417048483522193\\
56830.9630963096 0.00417080939948415\\
56839.603960396 0.00417113393908268\\
56848.2448244824 0.00417145845402313\\
56856.8856885689 0.00417178294431113\\
56865.5265526553 0.0041721074099523\\
56874.1674167417 0.00417243185095225\\
56882.8082808281 0.00417275626731659\\
56891.4491449145 0.00417308065905093\\
56900.0900090009 0.00417340502616089\\
56908.7308730873 0.00417372936865207\\
56917.3717371737 0.00417405368653008\\
56926.0126012601 0.00417437797980051\\
56934.6534653465 0.00417470224846897\\
56943.2943294329 0.00417502649254106\\
56951.9351935194 0.00417535071202237\\
56960.5760576058 0.00417567490691849\\
56969.2169216922 0.00417599907723503\\
56977.8577857786 0.00417632322297756\\
56986.498649865 0.00417664734415168\\
56995.1395139514 0.00417697144076297\\
57003.7803780378 0.00417729551281702\\
57012.4212421242 0.0041776195603194\\
57021.0621062106 0.0041779435832757\\
57029.702970297 0.00417826758169149\\
57038.3438343834 0.00417859155557235\\
57046.9846984698 0.00417891550492385\\
57055.6255625563 0.00417923942975157\\
57064.2664266427 0.00417956333006106\\
57072.9072907291 0.0041798872058579\\
57081.5481548155 0.00418021105714765\\
57090.1890189019 0.00418053488393588\\
57098.8298829883 0.00418085868622814\\
57107.4707470747 0.00418118246402999\\
57116.1116111611 0.00418150621734699\\
57124.7524752475 0.00418182994618469\\
57133.3933393339 0.00418215365054865\\
57142.0342034203 0.00418247733044441\\
57150.6750675068 0.00418280098587752\\
57159.3159315932 0.00418312461685353\\
57167.9567956796 0.00418344822337798\\
57176.597659766 0.00418377180545641\\
57185.2385238524 0.00418409536309436\\
57193.8793879388 0.00418441889629738\\
57202.5202520252 0.00418474240507098\\
57211.1611161116 0.00418506588942071\\
57219.801980198 0.0041853893493521\\
57228.4428442844 0.00418571278487068\\
57237.0837083708 0.00418603619598197\\
57245.7245724572 0.0041863595826915\\
57254.3654365437 0.00418668294500479\\
57263.0063006301 0.00418700628292736\\
57271.6471647165 0.00418732959646473\\
57280.2880288029 0.00418765288562241\\
57288.9288928893 0.00418797615040593\\
57297.5697569757 0.00418829939082078\\
57306.2106210621 0.00418862260687249\\
57314.8514851485 0.00418894579856656\\
57323.4923492349 0.0041892689659085\\
57332.1332133213 0.00418959210890381\\
57340.7740774077 0.00418991522755798\\
57349.4149414941 0.00419023832187653\\
57358.0558055806 0.00419056139186495\\
57366.696669667 0.00419088443752874\\
57375.3375337534 0.00419120745887338\\
57383.9783978398 0.00419153045590438\\
57392.6192619262 0.00419185342862721\\
57401.2601260126 0.00419217637704737\\
57409.900990099 0.00419249930117034\\
57418.5418541854 0.00419282220100161\\
57427.1827182718 0.00419314507654666\\
57435.8235823582 0.00419346792781096\\
57444.4644464446 0.00419379075479999\\
57453.1053105311 0.00419411355751923\\
57461.7461746175 0.00419443633597415\\
57470.3870387039 0.00419475909017022\\
57479.0279027903 0.00419508182011291\\
57487.6687668767 0.00419540452580769\\
57496.3096309631 0.00419572720726002\\
57504.9504950495 0.00419604986447537\\
57513.5913591359 0.00419637249745919\\
57522.2322232223 0.00419669510621694\\
57530.8730873087 0.00419701769075408\\
57539.5139513951 0.00419734025107607\\
57548.1548154815 0.00419766278718835\\
57556.795679568 0.00419798529909638\\
57565.4365436544 0.00419830778680561\\
57574.0774077408 0.00419863025032148\\
57582.7182718272 0.00419895268964943\\
57591.3591359136 0.00419927510479491\\
57600 0.00419959749576336\\
57608.6408640864 0.00419991986256021\\
57617.2817281728 0.00420024220519091\\
57625.9225922592 0.00420056452366088\\
57634.5634563456 0.00420088681797556\\
57643.204320432 0.00420120908814038\\
57651.8451845185 0.00420153133416077\\
57660.4860486049 0.00420185355604215\\
57669.1269126913 0.00420217575378994\\
57677.7677767777 0.00420249792740957\\
57686.4086408641 0.00420282007690646\\
57695.0495049505 0.00420314220228602\\
57703.6903690369 0.00420346430355368\\
57712.3312331233 0.00420378638071484\\
57720.9720972097 0.00420410843377491\\
57729.6129612961 0.00420443046273931\\
57738.2538253825 0.00420475246761345\\
57746.8946894689 0.00420507444840272\\
57755.5355535554 0.00420539640511254\\
57764.1764176418 0.0042057183377483\\
57772.8172817282 0.0042060402463154\\
57781.4581458146 0.00420636213081925\\
57790.099009901 0.00420668399126523\\
57798.7398739874 0.00420700582765874\\
57807.3807380738 0.00420732764000518\\
57816.0216021602 0.00420764942830992\\
57824.6624662466 0.00420797119257837\\
57833.303330333 0.0042082929328159\\
57841.9441944194 0.00420861464902789\\
57850.5850585059 0.00420893634121974\\
57859.2259225923 0.00420925800939681\\
57867.8667866787 0.00420957965356449\\
57876.5076507651 0.00420990127372815\\
57885.1485148515 0.00421022286989317\\
57893.7893789379 0.00421054444206491\\
57902.4302430243 0.00421086599024875\\
57911.0711071107 0.00421118751445005\\
57919.7119711971 0.00421150901467417\\
57928.3528352835 0.00421183049092649\\
57936.9936993699 0.00421215194321237\\
57945.6345634563 0.00421247337153715\\
57954.2754275428 0.00421279477590621\\
57962.9162916292 0.00421311615632489\\
57971.5571557156 0.00421343751279855\\
57980.198019802 0.00421375884533253\\
57988.8388838884 0.0042140801539322\\
57997.4797479748 0.0042144014386029\\
58006.1206120612 0.00421472269934996\\
58014.7614761476 0.00421504393617874\\
58023.402340234 0.00421536514909458\\
58032.0432043204 0.00421568633810282\\
58040.6840684068 0.00421600750320879\\
58049.3249324932 0.00421632864441783\\
58057.9657965797 0.00421664976173527\\
58066.6066606661 0.00421697085516645\\
58075.2475247525 0.00421729192471669\\
58083.8883888389 0.00421761297039132\\
58092.5292529253 0.00421793399219567\\
58101.1701170117 0.00421825499013506\\
58109.8109810981 0.00421857596421481\\
58118.4518451845 0.00421889691444025\\
58127.0927092709 0.00421921784081668\\
58135.7335733573 0.00421953874334943\\
58144.3744374437 0.00421985962204381\\
58153.0153015302 0.00422018047690513\\
58161.6561656166 0.0042205013079387\\
58170.297029703 0.00422082211514982\\
58178.9378937894 0.00422114289854381\\
58187.5787578758 0.00422146365812597\\
58196.2196219622 0.00422178439390161\\
58204.8604860486 0.00422210510587601\\
58213.501350135 0.00422242579405447\\
58222.1422142214 0.00422274645844231\\
58230.7830783078 0.00422306709904479\\
58239.4239423942 0.00422338771586723\\
58248.0648064806 0.00422370830891491\\
58256.7056705671 0.00422402887819312\\
58265.3465346535 0.00422434942370714\\
58273.9873987399 0.00422466994546226\\
58282.6282628263 0.00422499044346376\\
58291.2691269127 0.00422531091771692\\
58299.9099909991 0.00422563136822703\\
58308.5508550855 0.00422595179499934\\
58317.1917191719 0.00422627219803915\\
58325.8325832583 0.00422659257735172\\
58334.4734473447 0.00422691293294232\\
58343.1143114311 0.00422723326481623\\
58351.7551755176 0.0042275535729787\\
58360.396039604 0.00422787385743501\\
58369.0369036904 0.00422819411819041\\
58377.6777677768 0.00422851435525017\\
58386.3186318632 0.00422883456861954\\
58394.9594959496 0.00422915475830379\\
58403.600360036 0.00422947492430817\\
58412.2412241224 0.00422979506663792\\
58420.8820882088 0.00423011518529831\\
58429.5229522952 0.00423043528029457\\
58438.1638163816 0.00423075535163197\\
58446.804680468 0.00423107539931573\\
58455.4455445545 0.00423139542335112\\
58464.0864086409 0.00423171542374336\\
58472.7272727273 0.0042320354004977\\
58481.3681368137 0.00423235535361937\\
58490.0090009001 0.00423267528311362\\
58498.6498649865 0.00423299518898566\\
58507.2907290729 0.00423331507124075\\
58515.9315931593 0.0042336349298841\\
58524.5724572457 0.00423395476492094\\
58533.2133213321 0.0042342745763565\\
58541.8541854185 0.004234594364196\\
58550.4950495049 0.00423491412844467\\
58559.1359135914 0.00423523386910772\\
58567.7767776778 0.00423555358619038\\
58576.4176417642 0.00423587327969786\\
58585.0585058506 0.00423619294963537\\
58593.699369937 0.00423651259600813\\
58602.3402340234 0.00423683221882135\\
58610.9810981098 0.00423715181808023\\
58619.6219621962 0.00423747139378998\\
58628.2628262826 0.00423779094595581\\
58636.903690369 0.00423811047458292\\
58645.5445544554 0.00423842997967651\\
58654.1854185419 0.00423874946124178\\
58662.8262826283 0.00423906891928393\\
58671.4671467147 0.00423938835380815\\
58680.1080108011 0.00423970776481964\\
58688.7488748875 0.00424002715232358\\
58697.3897389739 0.00424034651632517\\
58706.0306030603 0.0042406658568296\\
58714.6714671467 0.00424098517384204\\
58723.3123312331 0.00424130446736769\\
58731.9531953195 0.00424162373741172\\
58740.5940594059 0.00424194298397931\\
58749.2349234923 0.00424226220707565\\
58757.8757875788 0.00424258140670591\\
58766.5166516652 0.00424290058287526\\
58775.1575157516 0.00424321973558888\\
58783.798379838 0.00424353886485193\\
58792.4392439244 0.00424385797066959\\
58801.0801080108 0.00424417705304702\\
58809.7209720972 0.00424449611198939\\
58818.3618361836 0.00424481514750185\\
58827.00270027 0.00424513415958957\\
58835.6435643564 0.00424545314825772\\
58844.2844284428 0.00424577211351143\\
58852.9252925293 0.00424609105535588\\
58861.5661566157 0.00424640997379622\\
58870.2070207021 0.00424672886883759\\
58878.8478847885 0.00424704774048514\\
58887.4887488749 0.00424736658874403\\
58896.1296129613 0.00424768541361939\\
58904.7704770477 0.00424800421511638\\
58913.4113411341 0.00424832299324014\\
58922.0522052205 0.0042486417479958\\
58930.6930693069 0.0042489604793885\\
58939.3339333933 0.00424927918742338\\
58947.9747974798 0.00424959787210558\\
58956.6156615662 0.00424991653344022\\
58965.2565256526 0.00425023517143244\\
58973.897389739 0.00425055378608737\\
58982.5382538254 0.00425087237741013\\
58991.1791179118 0.00425119094540585\\
58999.8199819982 0.00425150949007965\\
59008.4608460846 0.00425182801143665\\
59017.101710171 0.00425214650948198\\
59025.7425742574 0.00425246498422074\\
59034.3834383438 0.00425278343565806\\
59043.0243024302 0.00425310186379906\\
59051.6651665167 0.00425342026864883\\
59060.3060306031 0.00425373865021249\\
59068.9468946895 0.00425405700849516\\
59077.5877587759 0.00425437534350193\\
59086.2286228623 0.00425469365523791\\
59094.8694869487 0.0042550119437082\\
59103.5103510351 0.00425533020891791\\
59112.1512151215 0.00425564845087213\\
59120.7920792079 0.00425596666957596\\
59129.4329432943 0.0042562848650345\\
59138.0738073807 0.00425660303725284\\
59146.7146714671 0.00425692118623607\\
59155.3555355536 0.00425723931198927\\
59163.99639964 0.00425755741451755\\
59172.6372637264 0.00425787549382597\\
59181.2781278128 0.00425819354991964\\
59189.9189918992 0.00425851158280362\\
59198.5598559856 0.00425882959248301\\
59207.200720072 0.00425914757896287\\
59215.8415841584 0.00425946554224829\\
59224.4824482448 0.00425978348234434\\
59233.1233123312 0.00426010139925609\\
59241.7641764176 0.00426041929298861\\
59250.405040504 0.00426073716354698\\
59259.0459045905 0.00426105501093626\\
59267.6867686769 0.00426137283516152\\
59276.3276327633 0.00426169063622781\\
59284.9684968497 0.00426200841414021\\
59293.6093609361 0.00426232616890377\\
59302.2502250225 0.00426264390052354\\
59310.8910891089 0.0042629616090046\\
59319.5319531953 0.00426327929435198\\
59328.1728172817 0.00426359695657075\\
59336.8136813681 0.00426391459566595\\
59345.4545454545 0.00426423221164264\\
59354.095409541 0.00426454980450585\\
59362.7362736274 0.00426486737426064\\
59371.3771377138 0.00426518492091205\\
59380.0180018002 0.00426550244446512\\
59388.6588658866 0.00426581994492489\\
59397.299729973 0.0042661374222964\\
59405.9405940594 0.00426645487658468\\
59414.5814581458 0.00426677230779478\\
59423.2223222322 0.00426708971593171\\
59431.8631863186 0.00426740710100051\\
59440.504050405 0.00426772446300622\\
59449.1449144915 0.00426804180195385\\
59457.7857785779 0.00426835911784844\\
59466.4266426643 0.004268676410695\\
59475.0675067507 0.00426899368049855\\
59483.7083708371 0.00426931092726413\\
59492.3492349235 0.00426962815099674\\
59500.9900990099 0.0042699453517014\\
59509.6309630963 0.00427026252938313\\
59518.2718271827 0.00427057968404693\\
59526.9126912691 0.00427089681569783\\
59535.5535553555 0.00427121392434082\\
59544.1944194419 0.00427153100998091\\
59552.8352835284 0.00427184807262311\\
59561.4761476148 0.00427216511227243\\
59570.1170117012 0.00427248212893386\\
59578.7578757876 0.0042727991226124\\
59587.398739874 0.00427311609331305\\
59596.0396039604 0.00427343304104082\\
59604.6804680468 0.00427374996580068\\
59613.3213321332 0.00427406686759764\\
59621.9621962196 0.00427438374643668\\
59630.603060306 0.0042747006023228\\
59639.2439243924 0.00427501743526097\\
59647.8847884788 0.0042753342452562\\
59656.5256525653 0.00427565103231345\\
59665.1665166517 0.00427596779643771\\
59673.8073807381 0.00427628453763397\\
59682.4482448245 0.00427660125590719\\
59691.0891089109 0.00427691795126236\\
59699.7299729973 0.00427723462370444\\
59708.3708370837 0.00427755127323842\\
59717.0117011701 0.00427786789986926\\
59725.6525652565 0.00427818450360194\\
59734.2934293429 0.00427850108444141\\
59742.9342934293 0.00427881764239265\\
59751.5751575157 0.00427913417746061\\
59760.2160216022 0.00427945068965026\\
59768.8568856886 0.00427976717896657\\
59777.497749775 0.00428008364541448\\
59786.1386138614 0.00428040008899896\\
59794.7794779478 0.00428071650972495\\
59803.4203420342 0.00428103290759742\\
59812.0612061206 0.00428134928262131\\
59820.702070207 0.00428166563480158\\
59829.3429342934 0.00428198196414316\\
59837.9837983798 0.00428229827065101\\
59846.6246624662 0.00428261455433007\\
59855.2655265527 0.00428293081518528\\
59863.9063906391 0.00428324705322158\\
59872.5472547255 0.00428356326844391\\
59881.1881188119 0.00428387946085721\\
59889.8289828983 0.00428419563046642\\
59898.4698469847 0.00428451177727646\\
59907.1107110711 0.00428482790129226\\
59915.7515751575 0.00428514400251877\\
59924.3924392439 0.00428546008096089\\
59933.0333033303 0.00428577613662357\\
59941.6741674167 0.00428609216951173\\
59950.3150315032 0.00428640817963028\\
59958.9558955896 0.00428672416698415\\
59967.596759676 0.00428704013157826\\
59976.2376237624 0.00428735607341753\\
59984.8784878488 0.00428767199250686\\
59993.5193519352 0.00428798788885118\\
60002.1602160216 0.0042883037624554\\
60010.801080108 0.00428861961332442\\
60019.4419441944 0.00428893544146315\\
60028.0828082808 0.00428925124687651\\
60036.7236723672 0.0042895670295694\\
60045.3645364536 0.00428988278954671\\
60054.0054005401 0.00429019852681336\\
60062.6462646265 0.00429051424137423\\
60071.2871287129 0.00429082993323424\\
60079.9279927993 0.00429114560239827\\
60088.5688568857 0.00429146124887121\\
60097.2097209721 0.00429177687265797\\
60105.8505850585 0.00429209247376343\\
60114.4914491449 0.00429240805219249\\
60123.1323132313 0.00429272360795002\\
60131.7731773177 0.00429303914104091\\
60140.4140414041 0.00429335465147006\\
60149.0549054905 0.00429367013924233\\
60157.695769577 0.00429398560436261\\
60166.3366336634 0.00429430104683579\\
60174.9774977498 0.00429461646666673\\
60183.6183618362 0.0042949318638603\\
60192.2592259226 0.0042952472384214\\
60200.900090009 0.00429556259035487\\
60209.5409540954 0.00429587791966561\\
60218.1818181818 0.00429619322635846\\
60226.8226822682 0.00429650851043831\\
60235.4635463546 0.00429682377191001\\
60244.104410441 0.00429713901077842\\
60252.7452745274 0.00429745422704842\\
60261.3861386139 0.00429776942072485\\
60270.0270027003 0.00429808459181258\\
60278.6678667867 0.00429839974031645\\
60287.3087308731 0.00429871486624134\\
60295.9495949595 0.00429902996959208\\
60304.5904590459 0.00429934505037352\\
60313.2313231323 0.00429966010859053\\
60321.8721872187 0.00429997514424793\\
60330.5130513051 0.00430029015735059\\
60339.1539153915 0.00430060514790334\\
60347.7947794779 0.00430092011591102\\
60356.4356435644 0.00430123506137848\\
60365.0765076508 0.00430154998431055\\
60373.7173717372 0.00430186488471208\\
60382.3582358236 0.00430217976258789\\
60390.99909991 0.00430249461794281\\
60399.6399639964 0.00430280945078169\\
60408.2808280828 0.00430312426110935\\
60416.9216921692 0.00430343904893061\\
60425.5625562556 0.00430375381425031\\
60434.203420342 0.00430406855707327\\
60442.8442844284 0.00430438327740431\\
60451.4851485149 0.00430469797524825\\
60460.1260126013 0.00430501265060992\\
60468.7668766877 0.00430532730349413\\
60477.4077407741 0.00430564193390569\\
60486.0486048605 0.00430595654184943\\
60494.6894689469 0.00430627112733014\\
60503.3303330333 0.00430658569035266\\
60511.9711971197 0.00430690023092177\\
60520.6120612061 0.0043072147490423\\
60529.2529252925 0.00430752924471904\\
60537.8937893789 0.00430784371795681\\
60546.5346534653 0.0043081581687604\\
60555.1755175518 0.00430847259713461\\
60563.8163816382 0.00430878700308424\\
60572.4572457246 0.00430910138661409\\
60581.098109811 0.00430941574772896\\
60589.7389738974 0.00430973008643364\\
60598.3798379838 0.00431004440273292\\
60607.0207020702 0.00431035869663159\\
60615.6615661566 0.00431067296813444\\
60624.302430243 0.00431098721724626\\
60632.9432943294 0.00431130144397183\\
60641.5841584158 0.00431161564831594\\
60650.2250225023 0.00431192983028336\\
60658.8658865887 0.00431224398987888\\
60667.5067506751 0.00431255812710728\\
60676.1476147615 0.00431287224197333\\
60684.7884788479 0.00431318633448181\\
60693.4293429343 0.00431350040463749\\
60702.0702070207 0.00431381445244514\\
60710.7110711071 0.00431412847790953\\
60719.3519351935 0.00431444248103544\\
60727.9927992799 0.00431475646182762\\
60736.6336633663 0.00431507042029084\\
60745.2745274527 0.00431538435642987\\
60753.9153915392 0.00431569827024947\\
60762.5562556256 0.00431601216175439\\
60771.197119712 0.0043163260309494\\
60779.8379837984 0.00431663987783925\\
60788.4788478848 0.0043169537024287\\
60797.1197119712 0.00431726750472249\\
60805.7605760576 0.00431758128472539\\
60814.401440144 0.00431789504244213\\
60823.0423042304 0.00431820877787747\\
60831.6831683168 0.00431852249103616\\
60840.3240324032 0.00431883618192293\\
60848.9648964896 0.00431914985054253\\
60857.6057605761 0.00431946349689971\\
60866.2466246625 0.0043197771209992\\
60874.8874887489 0.00432009072284574\\
60883.5283528353 0.00432040430244406\\
60892.1692169217 0.00432071785979891\\
60900.8100810081 0.00432103139491501\\
60909.4509450945 0.00432134490779709\\
60918.0918091809 0.00432165839844989\\
60926.7326732673 0.00432197186687812\\
60935.3735373537 0.00432228531308653\\
60944.0144014401 0.00432259873707983\\
60952.6552655266 0.00432291213886274\\
60961.296129613 0.00432322551843999\\
60969.9369936994 0.00432353887581629\\
60978.5778577858 0.00432385221099637\\
60987.2187218722 0.00432416552398493\\
60995.8595859586 0.0043244788147867\\
61004.500450045 0.00432479208340638\\
61013.1413141314 0.00432510532984869\\
61021.7821782178 0.00432541855411834\\
61030.4230423042 0.00432573175622003\\
61039.0639063906 0.00432604493615848\\
61047.704770477 0.00432635809393837\\
61056.3456345635 0.00432667122956443\\
61064.9864986499 0.00432698434304134\\
61073.6273627363 0.00432729743437381\\
61082.2682268227 0.00432761050356653\\
61090.9090909091 0.00432792355062421\\
61099.5499549955 0.00432823657555153\\
61108.1908190819 0.00432854957835319\\
61116.8316831683 0.00432886255903388\\
61125.4725472547 0.00432917551759829\\
61134.1134113411 0.00432948845405111\\
61142.7542754275 0.00432980136839702\\
61151.395139514 0.00433011426064071\\
61160.0360036004 0.00433042713078686\\
61168.6768676868 0.00433073997884014\\
61177.3177317732 0.00433105280480525\\
61185.9585958596 0.00433136560868686\\
61194.599459946 0.00433167839048965\\
61203.2403240324 0.00433199115021828\\
61211.8811881188 0.00433230388787744\\
61220.5220522052 0.00433261660347179\\
61229.1629162916 0.004332929297006\\
61237.803780378 0.00433324196848474\\
61246.4446444644 0.00433355461791269\\
61255.0855085509 0.00433386724529449\\
61263.7263726373 0.00433417985063482\\
61272.3672367237 0.00433449243393833\\
61281.0081008101 0.0043348049952097\\
61289.6489648965 0.00433511753445356\\
61298.2898289829 0.00433543005167459\\
61306.9306930693 0.00433574254687743\\
61315.5715571557 0.00433605502006675\\
61324.2124212421 0.00433636747124718\\
61332.8532853285 0.00433667990042338\\
61341.4941494149 0.0043369923076\\
61350.1350135013 0.00433730469278169\\
61358.7758775878 0.00433761705597309\\
61367.4167416742 0.00433792939717884\\
61376.0576057606 0.00433824171640358\\
61384.698469847 0.00433855401365196\\
61393.3393339334 0.00433886628892862\\
61401.9801980198 0.00433917854223818\\
61410.6210621062 0.00433949077358529\\
61419.2619261926 0.00433980298297458\\
61427.902790279 0.00434011517041069\\
61436.5436543654 0.00434042733589823\\
61445.1845184518 0.00434073947944184\\
61453.8253825383 0.00434105160104615\\
61462.4662466247 0.00434136370071579\\
61471.1071107111 0.00434167577845537\\
61479.7479747975 0.00434198783426952\\
61488.3888388839 0.00434229986816286\\
61497.0297029703 0.00434261188014\\
61505.6705670567 0.00434292387020557\\
61514.3114311431 0.00434323583836419\\
61522.9522952295 0.00434354778462046\\
61531.5931593159 0.00434385970897899\\
61540.2340234023 0.00434417161144441\\
61548.8748874887 0.00434448349202131\\
61557.5157515752 0.00434479535071431\\
61566.1566156616 0.004345107187528\\
61574.797479748 0.00434541900246701\\
61583.4383438344 0.00434573079553592\\
61592.0792079208 0.00434604256673934\\
61600.7200720072 0.00434635431608188\\
61609.3609360936 0.00434666604356811\\
61618.00180018 0.00434697774920266\\
61626.6426642664 0.0043472894329901\\
61635.2835283528 0.00434760109493503\\
61643.9243924392 0.00434791273504204\\
61652.5652565257 0.00434822435331572\\
61661.2061206121 0.00434853594976067\\
61669.8469846985 0.00434884752438146\\
61678.4878487849 0.00434915907718269\\
61687.1287128713 0.00434947060816893\\
61695.7695769577 0.00434978211734477\\
61704.4104410441 0.00435009360471478\\
61713.0513051305 0.00435040507028355\\
61721.6921692169 0.00435071651405566\\
61730.3330333033 0.00435102793603568\\
61738.9738973897 0.00435133933622817\\
61747.6147614762 0.00435165071463773\\
61756.2556255626 0.0043519620712689\\
61764.896489649 0.00435227340612627\\
61773.5373537354 0.00435258471921441\\
61782.1782178218 0.00435289601053787\\
61790.8190819082 0.00435320728010122\\
61799.4599459946 0.00435351852790903\\
61808.100810081 0.00435382975396585\\
61816.7416741674 0.00435414095827625\\
61825.3825382538 0.00435445214084478\\
61834.0234023402 0.004354763301676\\
61842.6642664266 0.00435507444077447\\
61851.3051305131 0.00435538555814473\\
61859.9459945995 0.00435569665379134\\
61868.5868586859 0.00435600772771885\\
61877.2277227723 0.0043563187799318\\
61885.8685868587 0.00435662981043475\\
61894.5094509451 0.00435694081923223\\
61903.1503150315 0.0043572518063288\\
61911.7911791179 0.00435756277172899\\
61920.4320432043 0.00435787371543735\\
61929.0729072907 0.00435818463745841\\
61937.7137713771 0.00435849553779671\\
61946.3546354635 0.00435880641645678\\
61954.99549955 0.00435911727344317\\
61963.6363636364 0.0043594281087604\\
61972.2772277228 0.004359738922413\\
61980.9180918092 0.0043600497144055\\
61989.5589558956 0.00436036048474244\\
61998.199819982 0.00436067123342833\\
62006.8406840684 0.0043609819604677\\
62015.4815481548 0.00436129266586508\\
62024.1224122412 0.00436160334962498\\
62032.7632763276 0.00436191401175193\\
62041.404140414 0.00436222465225045\\
62050.0450045004 0.00436253527112504\\
62058.6858685869 0.00436284586838023\\
62067.3267326733 0.00436315644402053\\
62075.9675967597 0.00436346699805044\\
62084.6084608461 0.0043637775304745\\
62093.2493249325 0.00436408804129719\\
62101.8901890189 0.00436439853052303\\
62110.5310531053 0.00436470899815652\\
62119.1719171917 0.00436501944420217\\
62127.8127812781 0.00436532986866448\\
62136.4536453645 0.00436564027154795\\
62145.0945094509 0.00436595065285709\\
62153.7353735374 0.00436626101259638\\
62162.3762376238 0.00436657135077032\\
62171.0171017102 0.00436688166738342\\
62179.6579657966 0.00436719196244016\\
62188.298829883 0.00436750223594504\\
62196.9396939694 0.00436781248790254\\
62205.5805580558 0.00436812271831715\\
62214.2214221422 0.00436843292719336\\
62222.8622862286 0.00436874311453566\\
62231.503150315 0.00436905328034853\\
62240.1440144014 0.00436936342463645\\
62248.7848784879 0.0043696735474039\\
62257.4257425743 0.00436998364865537\\
62266.0666066607 0.00437029372839532\\
62274.7074707471 0.00437060378662824\\
62283.3483348335 0.0043709138233586\\
62291.9891989199 0.00437122383859088\\
62300.6300630063 0.00437153383232954\\
62309.2709270927 0.00437184380457905\\
62317.9117911791 0.00437215375534388\\
62326.5526552655 0.00437246368462851\\
62335.1935193519 0.00437277359243739\\
62343.8343834383 0.00437308347877499\\
62352.4752475248 0.00437339334364577\\
62361.1161116112 0.00437370318705419\\
62369.7569756976 0.00437401300900471\\
62378.397839784 0.00437432280950179\\
62387.0387038704 0.00437463258854988\\
62395.6795679568 0.00437494234615344\\
62404.3204320432 0.00437525208231693\\
62412.9612961296 0.00437556179704478\\
62421.602160216 0.00437587149034146\\
62430.2430243024 0.00437618116221141\\
62438.8838883888 0.00437649081265907\\
62447.5247524752 0.0043768004416889\\
62456.1656165617 0.00437711004930534\\
62464.8064806481 0.00437741963551282\\
62473.4473447345 0.00437772920031579\\
62482.0882088209 0.00437803874371869\\
62490.7290729073 0.00437834826572596\\
62499.3699369937 0.00437865776634203\\
62508.0108010801 0.00437896724557133\\
62516.6516651665 0.0043792767034183\\
62525.2925292529 0.00437958613988737\\
62533.9333933393 0.00437989555498298\\
62542.5742574257 0.00438020494870954\\
62551.2151215121 0.00438051432107148\\
62559.8559855986 0.00438082367207324\\
62568.496849685 0.00438113300171923\\
62577.1377137714 0.00438144231001388\\
62585.7785778578 0.0043817515969616\\
62594.4194419442 0.00438206086256682\\
62603.0603060306 0.00438237010683396\\
62611.701170117 0.00438267932976742\\
62620.3420342034 0.00438298853137163\\
62628.9828982898 0.004383297711651\\
62637.6237623762 0.00438360687060993\\
62646.2646264626 0.00438391600825285\\
62654.9054905491 0.00438422512458415\\
62663.5463546355 0.00438453421960825\\
62672.1872187219 0.00438484329332955\\
62680.8280828083 0.00438515234575245\\
62689.4689468947 0.00438546137688136\\
62698.1098109811 0.00438577038672068\\
62706.7506750675 0.00438607937527481\\
62715.3915391539 0.00438638834254813\\
62724.0324032403 0.00438669728854507\\
62732.6732673267 0.00438700621326999\\
62741.3141314131 0.00438731511672731\\
62749.9549954996 0.00438762399892141\\
62758.595859586 0.00438793285985668\\
62767.2367236724 0.00438824169953751\\
62775.8775877588 0.00438855051796828\\
62784.5184518452 0.0043888593151534\\
62793.1593159316 0.00438916809109722\\
62801.800180018 0.00438947684580415\\
62810.4410441044 0.00438978557927857\\
62819.0819081908 0.00439009429152484\\
62827.7227722772 0.00439040298254735\\
62836.3636363636 0.00439071165235048\\
62845.00450045 0.0043910203009386\\
62853.6453645365 0.00439132892831609\\
62862.2862286229 0.00439163753448732\\
62870.9270927093 0.00439194611945665\\
62879.5679567957 0.00439225468322846\\
62888.2088208821 0.00439256322580712\\
62896.8496849685 0.00439287174719699\\
62905.4905490549 0.00439318024740244\\
62914.1314131413 0.00439348872642783\\
62922.7722772277 0.00439379718427751\\
62931.4131413141 0.00439410562095586\\
62940.0540054005 0.00439441403646723\\
62948.6948694869 0.00439472243081598\\
62957.3357335734 0.00439503080400646\\
62965.9765976598 0.00439533915604303\\
62974.6174617462 0.00439564748693004\\
62983.2583258326 0.00439595579667185\\
62991.899189919 0.00439626408527279\\
63000.5400540054 0.00439657235273723\\
63009.1809180918 0.0043968805990695\\
63017.8217821782 0.00439718882427395\\
63026.4626462646 0.00439749702835493\\
63035.103510351 0.00439780521131678\\
63043.7443744374 0.00439811337316384\\
63052.3852385238 0.00439842151390044\\
63061.0261026103 0.00439872963353094\\
63069.6669666967 0.00439903773205965\\
63078.3078307831 0.00439934580949092\\
63086.9486948695 0.00439965386582909\\
63095.5895589559 0.00439996190107848\\
63104.2304230423 0.00440026991524342\\
63112.8712871287 0.00440057790832824\\
63121.5121512151 0.00440088588033728\\
63130.1530153015 0.00440119383127485\\
63138.7938793879 0.00440150176114528\\
63147.4347434743 0.00440180966995289\\
63156.0756075608 0.00440211755770202\\
63164.7164716472 0.00440242542439696\\
63173.3573357336 0.00440273327004205\\
63181.99819982 0.0044030410946416\\
63190.6390639064 0.00440334889819993\\
63199.2799279928 0.00440365668072135\\
63207.9207920792 0.00440396444221017\\
63216.5616561656 0.00440427218267071\\
63225.202520252 0.00440457990210727\\
63233.8433843384 0.00440488760052417\\
63242.4842484248 0.00440519527792571\\
63251.1251125113 0.00440550293431619\\
63259.7659765977 0.00440581056969992\\
63268.4068406841 0.0044061181840812\\
63277.0477047705 0.00440642577746434\\
63285.6885688569 0.00440673334985362\\
63294.3294329433 0.00440704090125336\\
63302.9702970297 0.00440734843166784\\
63311.6111611161 0.00440765594110137\\
63320.2520252025 0.00440796342955823\\
63328.8928892889 0.00440827089704272\\
63337.5337533753 0.00440857834355913\\
63346.1746174617 0.00440888576911175\\
63354.8154815482 0.00440919317370486\\
63363.4563456346 0.00440950055734275\\
63372.097209721 0.00440980792002971\\
63380.7380738074 0.00441011526177002\\
63389.3789378938 0.00441042258256796\\
63398.0198019802 0.00441072988242781\\
63406.6606660666 0.00441103716135385\\
63415.301530153 0.00441134441935037\\
63423.9423942394 0.00441165165642163\\
63432.5832583258 0.0044119588725719\\
63441.2241224122 0.00441226606780547\\
63449.8649864987 0.00441257324212661\\
63458.5058505851 0.00441288039553958\\
63467.1467146715 0.00441318752804865\\
63475.7875787579 0.0044134946396581\\
63484.4284428443 0.00441380173037218\\
63493.0693069307 0.00441410880019517\\
63501.7101710171 0.00441441584913131\\
63510.3510351035 0.00441472287718489\\
63518.9918991899 0.00441502988436015\\
63527.6327632763 0.00441533687066135\\
63536.2736273627 0.00441564383609276\\
63544.9144914491 0.00441595078065863\\
63553.5553555356 0.00441625770436321\\
63562.196219622 0.00441656460721075\\
63570.8370837084 0.00441687148920551\\
63579.4779477948 0.00441717835035173\\
63588.1188118812 0.00441748519065368\\
63596.7596759676 0.00441779201011558\\
63605.400540054 0.00441809880874169\\
63614.0414041404 0.00441840558653626\\
63622.6822682268 0.00441871234350351\\
63631.3231323132 0.00441901907964771\\
63639.9639963996 0.00441932579497308\\
63648.604860486 0.00441963248948387\\
63657.2457245725 0.0044199391631843\\
63665.8865886589 0.00442024581607863\\
63674.5274527453 0.00442055244817108\\
63683.1683168317 0.00442085905946588\\
63691.8091809181 0.00442116564996727\\
63700.4500450045 0.00442147221967947\\
63709.0909090909 0.00442177876860671\\
63717.7317731773 0.00442208529675323\\
63726.3726372637 0.00442239180412324\\
63735.0135013501 0.00442269829072098\\
63743.6543654365 0.00442300475655065\\
63752.295229523 0.0044233112016165\\
63760.9360936094 0.00442361762592272\\
63769.5769576958 0.00442392402947355\\
63778.2178217822 0.0044242304122732\\
63786.8586858686 0.00442453677432588\\
63795.499549955 0.00442484311563581\\
63804.1404140414 0.00442514943620721\\
63812.7812781278 0.00442545573604428\\
63821.4221422142 0.00442576201515123\\
63830.0630063006 0.00442606827353227\\
63838.703870387 0.00442637451119161\\
63847.3447344734 0.00442668072813346\\
63855.9855985599 0.00442698692436201\\
63864.6264626463 0.00442729309988148\\
63873.2673267327 0.00442759925469605\\
63881.9081908191 0.00442790538880994\\
63890.5490549055 0.00442821150222734\\
63899.1899189919 0.00442851759495245\\
63907.8307830783 0.00442882366698946\\
63916.4716471647 0.00442912971834257\\
63925.1125112511 0.00442943574901597\\
63933.7533753375 0.00442974175901385\\
63942.3942394239 0.0044300477483404\\
63951.0351035104 0.00443035371699982\\
63959.6759675968 0.00443065966499628\\
63968.3168316832 0.00443096559233397\\
63976.9576957696 0.00443127149901708\\
63985.598559856 0.00443157738504979\\
63994.2394239424 0.00443188325043629\\
64002.8802880288 0.00443218909518074\\
64011.5211521152 0.00443249491928734\\
64020.1620162016 0.00443280072276025\\
64028.802880288 0.00443310650560366\\
64037.4437443744 0.00443341226782174\\
64046.0846084608 0.00443371800941865\\
64054.7254725473 0.00443402373039858\\
64063.3663366337 0.0044343294307657\\
64072.0072007201 0.00443463511052416\\
64080.6480648065 0.00443494076967814\\
64089.2889288929 0.00443524640823181\\
64097.9297929793 0.00443555202618933\\
64106.5706570657 0.00443585762355486\\
64115.2115211521 0.00443616320033257\\
64123.8523852385 0.00443646875652661\\
64132.4932493249 0.00443677429214115\\
64141.1341134113 0.00443707980718033\\
64149.7749774977 0.00443738530164833\\
64158.4158415842 0.00443769077554929\\
64167.0567056706 0.00443799622888736\\
64175.697569757 0.00443830166166671\\
64184.3384338434 0.00443860707389147\\
64192.9792979298 0.0044389124655658\\
64201.6201620162 0.00443921783669384\\
64210.2610261026 0.00443952318727975\\
64218.901890189 0.00443982851732766\\
64227.5427542754 0.00444013382684173\\
64236.1836183618 0.00444043911582608\\
64244.8244824482 0.00444074438428488\\
64253.4653465347 0.00444104963222224\\
64262.1062106211 0.00444135485964232\\
64270.7470747075 0.00444166006654924\\
64279.3879387939 0.00444196525294715\\
64288.0288028803 0.00444227041884018\\
64296.6696669667 0.00444257556423245\\
64305.3105310531 0.00444288068912811\\
64313.9513951395 0.00444318579353127\\
64322.5922592259 0.00444349087744608\\
64331.2331233123 0.00444379594087665\\
64339.8739873987 0.00444410098382711\\
64348.5148514851 0.00444440600630158\\
64357.1557155716 0.0044447110083042\\
64365.796579658 0.00444501598983907\\
64374.4374437444 0.00444532095091033\\
64383.0783078308 0.00444562589152208\\
64391.7191719172 0.00444593081167845\\
64400.3600360036 0.00444623571138355\\
64409.00090009 0.0044465405906415\\
64417.6417641764 0.0044468454494564\\
64426.2826282628 0.00444715028783238\\
64434.9234923492 0.00444745510577353\\
64443.5643564356 0.00444775990328398\\
64452.2052205221 0.00444806468036783\\
64460.8460846085 0.00444836943702918\\
64469.4869486949 0.00444867417327214\\
64478.1278127813 0.00444897888910081\\
64486.7686768677 0.00444928358451929\\
64495.4095409541 0.00444958825953169\\
64504.0504050405 0.00444989291414211\\
64512.6912691269 0.00445019754835463\\
64521.3321332133 0.00445050216217337\\
64529.9729972997 0.00445080675560241\\
64538.6138613861 0.00445111132864585\\
64547.2547254726 0.00445141588130778\\
64555.895589559 0.0044517204135923\\
64564.5364536454 0.00445202492550348\\
64573.1773177318 0.00445232941704543\\
64581.8181818182 0.00445263388822222\\
64590.4590459046 0.00445293833903794\\
64599.099909991 0.00445324276949668\\
64607.7407740774 0.00445354717960253\\
64616.3816381638 0.00445385156935955\\
64625.0225022502 0.00445415593877184\\
64633.6633663366 0.00445446028784346\\
64642.304230423 0.00445476461657851\\
64650.9450945095 0.00445506892498105\\
64659.5859585959 0.00445537321305515\\
64668.2268226823 0.00445567748080491\\
64676.8676867687 0.00445598172823437\\
64685.5085508551 0.00445628595534762\\
64694.1494149415 0.00445659016214873\\
64702.7902790279 0.00445689434864175\\
64711.4311431143 0.00445719851483077\\
64720.0720072007 0.00445750266071984\\
64728.7128712871 0.00445780678631303\\
64737.3537353735 0.0044581108916144\\
64745.9945994599 0.00445841497662802\\
64754.6354635464 0.00445871904135793\\
64763.2763276328 0.0044590230858082\\
64771.9171917192 0.0044593271099829\\
64780.5580558056 0.00445963111388606\\
64789.198919892 0.00445993509752175\\
64797.8397839784 0.00446023906089403\\
64806.4806480648 0.00446054300400693\\
64815.1215121512 0.00446084692686452\\
64823.7623762376 0.00446115082947084\\
64832.403240324 0.00446145471182994\\
64841.0441044104 0.00446175857394587\\
64849.6849684968 0.00446206241582266\\
64858.3258325833 0.00446236623746437\\
64866.9666966697 0.00446267003887504\\
64875.6075607561 0.00446297382005871\\
64884.2484248425 0.00446327758101941\\
64892.8892889289 0.00446358132176118\\
64901.5301530153 0.00446388504228807\\
64910.1710171017 0.00446418874260411\\
64918.8118811881 0.00446449242271333\\
64927.4527452745 0.00446479608261976\\
64936.0936093609 0.00446509972232744\\
64944.7344734473 0.0044654033418404\\
64953.3753375338 0.00446570694116266\\
64962.0162016202 0.00446601052029826\\
64970.6570657066 0.00446631407925121\\
64979.297929793 0.00446661761802555\\
64987.9387938794 0.00446692113662529\\
64996.5796579658 0.00446722463505447\\
65005.2205220522 0.0044675281133171\\
65013.8613861386 0.00446783157141719\\
65022.502250225 0.00446813500935877\\
65031.1431143114 0.00446843842714586\\
65039.7839783978 0.00446874182478247\\
65048.4248424843 0.00446904520227261\\
65057.0657065707 0.0044693485596203\\
65065.7065706571 0.00446965189682955\\
65074.3474347435 0.00446995521390437\\
65082.9882988299 0.00447025851084877\\
65091.6291629163 0.00447056178766675\\
65100.2700270027 0.00447086504436232\\
65108.9108910891 0.00447116828093949\\
65117.5517551755 0.00447147149740226\\
65126.1926192619 0.00447177469375463\\
65134.8334833483 0.0044720778700006\\
65143.4743474347 0.00447238102614417\\
65152.1152115212 0.00447268416218935\\
65160.7560756076 0.00447298727814012\\
65169.396939694 0.00447329037400049\\
65178.0378037804 0.00447359344977444\\
65186.6786678668 0.00447389650546597\\
65195.3195319532 0.00447419954107908\\
65203.9603960396 0.00447450255661774\\
65212.601260126 0.00447480555208596\\
65221.2421242124 0.00447510852748772\\
65229.8829882988 0.004475411482827\\
65238.5238523852 0.00447571441810779\\
65247.1647164716 0.00447601733333407\\
65255.8055805581 0.00447632022850983\\
65264.4464446445 0.00447662310363905\\
65273.0873087309 0.0044769259587257\\
65281.7281728173 0.00447722879377376\\
65290.3690369037 0.00447753160878722\\
65299.0099009901 0.00447783440377005\\
65307.6507650765 0.00447813717872622\\
65316.2916291629 0.0044784399336597\\
65324.9324932493 0.00447874266857447\\
65333.5733573357 0.0044790453834745\\
65342.2142214221 0.00447934807836375\\
65350.8550855085 0.0044796507532462\\
65359.495949595 0.00447995340812581\\
65368.1368136814 0.00448025604300655\\
65376.7776777678 0.00448055865789237\\
65385.4185418542 0.00448086125278725\\
65394.0594059406 0.00448116382769514\\
65402.700270027 0.00448146638262001\\
65411.3411341134 0.00448176891756581\\
65419.9819981998 0.0044820714325365\\
65428.6228622862 0.00448237392753604\\
65437.2637263726 0.00448267640256838\\
65445.904590459 0.00448297885763747\\
65454.5454545455 0.00448328129274727\\
65463.1863186319 0.00448358370790173\\
65471.8271827183 0.00448388610310479\\
65480.4680468047 0.00448418847836042\\
65489.1089108911 0.00448449083367254\\
65497.7497749775 0.00448479316904511\\
65506.3906390639 0.00448509548448207\\
65515.0315031503 0.00448539777998737\\
65523.6723672367 0.00448570005556495\\
65532.3132313231 0.00448600231121875\\
65540.9540954095 0.0044863045469527\\
65549.594959496 0.00448660676277074\\
65558.2358235824 0.00448690895867682\\
65566.8766876688 0.00448721113467487\\
65575.5175517552 0.00448751329076881\\
65584.1584158416 0.00448781542696259\\
65592.799279928 0.00448811754326014\\
65601.4401440144 0.00448841963966538\\
65610.0810081008 0.00448872171618224\\
65618.7218721872 0.00448902377281465\\
65627.3627362736 0.00448932580956654\\
65636.00360036 0.00448962782644184\\
65644.6444644465 0.00448992982344446\\
65653.2853285329 0.00449023180057833\\
65661.9261926193 0.00449053375784736\\
65670.5670567057 0.00449083569525549\\
65679.2079207921 0.00449113761280663\\
65687.8487848785 0.00449143951050469\\
65696.4896489649 0.00449174138835359\\
65705.1305130513 0.00449204324635724\\
65713.7713771377 0.00449234508451956\\
65722.4122412241 0.00449264690284447\\
65731.0531053105 0.00449294870133587\\
65739.6939693969 0.00449325047999766\\
65748.3348334833 0.00449355223883377\\
65756.9756975698 0.00449385397784809\\
65765.6165616562 0.00449415569704453\\
65774.2574257426 0.004494457396427\\
65782.898289829 0.0044947590759994\\
65791.5391539154 0.00449506073576563\\
65800.1800180018 0.00449536237572959\\
65808.8208820882 0.00449566399589518\\
65817.4617461746 0.0044959655962663\\
65826.102610261 0.00449626717684685\\
65834.7434743474 0.00449656873764071\\
65843.3843384338 0.00449687027865179\\
65852.0252025202 0.00449717179988398\\
65860.6660666067 0.00449747330134117\\
65869.3069306931 0.00449777478302724\\
65877.9477947795 0.00449807624494609\\
65886.5886588659 0.00449837768710161\\
65895.2295229523 0.00449867910949769\\
65903.8703870387 0.00449898051213819\\
65912.5112511251 0.00449928189502702\\
65921.1521152115 0.00449958325816806\\
65929.7929792979 0.00449988460156518\\
65938.4338433843 0.00450018592522226\\
65947.0747074707 0.00450048722914319\\
65955.7155715572 0.00450078851333184\\
65964.3564356436 0.00450108977779209\\
65972.99729973 0.00450139102252782\\
65981.6381638164 0.00450169224754288\\
65990.2790279028 0.00450199345284117\\
65998.9198919892 0.00450229463842655\\
66007.5607560756 0.00450259580430289\\
66016.201620162 0.00450289695047406\\
66024.8424842484 0.00450319807694392\\
66033.4833483348 0.00450349918371635\\
66042.1242124212 0.00450380027079521\\
66050.7650765077 0.00450410133818435\\
66059.4059405941 0.00450440238588765\\
66068.0468046805 0.00450470341390896\\
66076.6876687669 0.00450500442225215\\
66085.3285328533 0.00450530541092107\\
66093.9693969397 0.00450560637991958\\
66102.6102610261 0.00450590732925154\\
66111.2511251125 0.0045062082589208\\
66119.8919891989 0.00450650916893121\\
66128.5328532853 0.00450681005928663\\
66137.1737173717 0.00450711092999091\\
66145.8145814582 0.00450741178104789\\
66154.4554455446 0.00450771261246143\\
66163.096309631 0.00450801342423538\\
66171.7371737174 0.00450831421637357\\
66180.3780378038 0.00450861498887985\\
66189.0189018902 0.00450891574175808\\
66197.6597659766 0.00450921647501208\\
66206.300630063 0.00450951718864571\\
66214.9414941494 0.00450981788266279\\
66223.5823582358 0.00451011855706717\\
66232.2232223222 0.00451041921186269\\
66240.8640864086 0.00451071984705318\\
66249.504950495 0.00451102046264248\\
66258.1458145815 0.00451132105863442\\
66266.7866786679 0.00451162163503283\\
66275.4275427543 0.00451192219184154\\
66284.0684068407 0.00451222272906439\\
66292.7092709271 0.00451252324670519\\
66301.3501350135 0.00451282374476779\\
66309.9909990999 0.004513124223256\\
66318.6318631863 0.00451342468217365\\
66327.2727272727 0.00451372512152456\\
66335.9135913591 0.00451402554131256\\
66344.5544554455 0.00451432594154146\\
66353.195319532 0.00451462632221509\\
66361.8361836184 0.00451492668333726\\
66370.4770477048 0.00451522702491179\\
66379.1179117912 0.0045155273469425\\
66387.7587758776 0.00451582764943321\\
66396.399639964 0.00451612793238772\\
66405.0405040504 0.00451642819580984\\
66413.6813681368 0.0045167284397034\\
66422.3222322232 0.00451702866407219\\
66430.9630963096 0.00451732886892003\\
66439.603960396 0.00451762905425073\\
66448.2448244824 0.00451792922006809\\
66456.8856885689 0.00451822936637591\\
66465.5265526553 0.004518529493178\\
66474.1674167417 0.00451882960047817\\
66482.8082808281 0.00451912968828021\\
66491.4491449145 0.00451942975658792\\
66500.0900090009 0.0045197298054051\\
66508.7308730873 0.00452002983473555\\
66517.3717371737 0.00452032984458306\\
66526.0126012601 0.00452062983495144\\
66534.6534653465 0.00452092980584447\\
66543.2943294329 0.00452122975726594\\
66551.9351935194 0.00452152968921966\\
66560.5760576058 0.0045218296017094\\
66569.2169216922 0.00452212949473896\\
66577.8577857786 0.00452242936831212\\
66586.498649865 0.00452272922243267\\
66595.1395139514 0.0045230290571044\\
66603.7803780378 0.00452332887233109\\
66612.4212421242 0.00452362866811651\\
66621.0621062106 0.00452392844446447\\
66629.702970297 0.00452422820137873\\
66638.3438343834 0.00452452793886307\\
66646.9846984699 0.00452482765692127\\
66655.6255625563 0.0045251273555571\\
66664.2664266427 0.00452542703477436\\
66672.9072907291 0.00452572669457679\\
66681.5481548155 0.00452602633496819\\
66690.1890189019 0.00452632595595232\\
66698.8298829883 0.00452662555753295\\
66707.4707470747 0.00452692513971385\\
66716.1116111611 0.00452722470249879\\
66724.7524752475 0.00452752424589154\\
66733.3933393339 0.00452782376989586\\
66742.0342034203 0.00452812327451551\\
66750.6750675067 0.00452842275975426\\
66759.3159315932 0.00452872222561588\\
66767.9567956796 0.00452902167210411\\
66776.597659766 0.00452932109922273\\
66785.2385238524 0.00452962050697548\\
66793.8793879388 0.00452991989536614\\
66802.5202520252 0.00453021926439844\\
66811.1611161116 0.00453051861407615\\
66819.801980198 0.00453081794440302\\
66828.4428442844 0.0045311172553828\\
66837.0837083708 0.00453141654701925\\
66845.7245724572 0.00453171581931611\\
66854.3654365437 0.00453201507227713\\
66863.0063006301 0.00453231430590606\\
66871.6471647165 0.00453261352020665\\
66880.2880288029 0.00453291271518264\\
66888.9288928893 0.00453321189083777\\
66897.5697569757 0.00453351104717578\\
66906.2106210621 0.00453381018420043\\
66914.8514851485 0.00453410930191544\\
66923.4923492349 0.00453440840032456\\
66932.1332133213 0.00453470747943152\\
66940.7740774077 0.00453500653924006\\
66949.4149414941 0.00453530557975392\\
66958.0558055806 0.00453560460097683\\
66966.696669667 0.00453590360291253\\
66975.3375337534 0.00453620258556474\\
66983.9783978398 0.00453650154893719\\
66992.6192619262 0.00453680049303362\\
67001.2601260126 0.00453709941785775\\
67009.900990099 0.00453739832341331\\
67018.5418541854 0.00453769720970402\\
67027.1827182718 0.00453799607673361\\
67035.8235823582 0.00453829492450581\\
67044.4644464446 0.00453859375302433\\
67053.1053105311 0.0045388925622929\\
67061.7461746175 0.00453919135231523\\
67070.3870387039 0.00453949012309505\\
67079.0279027903 0.00453978887463607\\
67087.6687668767 0.004540087606942\\
67096.3096309631 0.00454038632001657\\
67104.9504950495 0.00454068501386348\\
67113.5913591359 0.00454098368848645\\
67122.2322232223 0.00454128234388919\\
67130.8730873087 0.00454158098007541\\
67139.5139513951 0.00454187959704882\\
67148.1548154816 0.00454217819481312\\
67156.795679568 0.00454247677337203\\
67165.4365436544 0.00454277533272924\\
67174.0774077408 0.00454307387288847\\
67182.7182718272 0.00454337239385341\\
67191.3591359136 0.00454367089562777\\
67200 0.00454396937821524\\
67208.6408640864 0.00454426784161953\\
67217.2817281728 0.00454456628584434\\
67225.9225922592 0.00454486471089336\\
67234.5634563456 0.00454516311677028\\
67243.204320432 0.00454546150347881\\
67251.8451845184 0.00454575987102264\\
67260.4860486049 0.00454605821940545\\
67269.1269126913 0.00454635654863094\\
67277.7677767777 0.0045466548587028\\
67286.4086408641 0.00454695314962471\\
67295.0495049505 0.00454725142140038\\
67303.6903690369 0.00454754967403347\\
67312.3312331233 0.00454784790752768\\
67320.9720972097 0.00454814612188669\\
67329.6129612961 0.00454844431711418\\
67338.2538253825 0.00454874249321384\\
67346.8946894689 0.00454904065018934\\
67355.5355535554 0.00454933878804437\\
67364.1764176418 0.0045496369067826\\
67372.8172817282 0.00454993500640771\\
67381.4581458146 0.00455023308692337\\
67390.099009901 0.00455053114833326\\
67398.7398739874 0.00455082919064106\\
67407.3807380738 0.00455112721385043\\
67416.0216021602 0.00455142521796505\\
67424.6624662466 0.00455172320298858\\
67433.303330333 0.0045520211689247\\
67441.9441944194 0.00455231911577707\\
67450.5850585059 0.00455261704354936\\
67459.2259225923 0.00455291495224523\\
67467.8667866787 0.00455321284186835\\
67476.5076507651 0.00455351071242238\\
67485.1485148515 0.00455380856391098\\
67493.7893789379 0.00455410639633781\\
67502.4302430243 0.00455440420970654\\
67511.0711071107 0.00455470200402081\\
67519.7119711971 0.00455499977928429\\
67528.3528352835 0.00455529753550063\\
67536.9936993699 0.00455559527267349\\
67545.6345634563 0.00455589299080652\\
67554.2754275428 0.00455619068990337\\
67562.9162916292 0.0045564883699677\\
67571.5571557156 0.00455678603100314\\
67580.198019802 0.00455708367301336\\
67588.8388838884 0.004557381296002\\
67597.4797479748 0.00455767889997271\\
67606.1206120612 0.00455797648492912\\
67614.7614761476 0.00455827405087489\\
67623.402340234 0.00455857159781366\\
67632.0432043204 0.00455886912574907\\
67640.6840684068 0.00455916663468476\\
67649.3249324933 0.00455946412462437\\
67657.9657965797 0.00455976159557154\\
67666.6066606661 0.0045600590475299\\
67675.2475247525 0.0045603564805031\\
67683.8883888389 0.00456065389449476\\
67692.5292529253 0.00456095128950852\\
67701.1701170117 0.00456124866554801\\
67709.8109810981 0.00456154602261686\\
67718.4518451845 0.00456184336071871\\
67727.0927092709 0.00456214067985719\\
67735.7335733573 0.00456243798003591\\
67744.3744374437 0.00456273526125851\\
67753.0153015301 0.00456303252352861\\
67761.6561656166 0.00456332976684984\\
67770.297029703 0.00456362699122581\\
67778.9378937894 0.00456392419666017\\
67787.5787578758 0.00456422138315651\\
67796.2196219622 0.00456451855071846\\
67804.8604860486 0.00456481569934965\\
67813.501350135 0.00456511282905369\\
67822.1422142214 0.00456540993983419\\
67830.7830783078 0.00456570703169477\\
67839.4239423942 0.00456600410463905\\
67848.0648064806 0.00456630115867063\\
67856.7056705671 0.00456659819379313\\
67865.3465346535 0.00456689521001016\\
67873.9873987399 0.00456719220732533\\
67882.6282628263 0.00456748918574225\\
67891.2691269127 0.00456778614526452\\
67899.9099909991 0.00456808308589575\\
67908.5508550855 0.00456838000763955\\
67917.1917191719 0.00456867691049951\\
67925.8325832583 0.00456897379447924\\
67934.4734473447 0.00456927065958235\\
67943.1143114311 0.00456956750581243\\
67951.7551755176 0.00456986433317308\\
67960.396039604 0.0045701611416679\\
67969.0369036904 0.00457045793130049\\
67977.6777677768 0.00457075470207443\\
67986.3186318632 0.00457105145399334\\
67994.9594959496 0.00457134818706079\\
68003.600360036 0.00457164490128039\\
68012.2412241224 0.00457194159665572\\
68020.8820882088 0.00457223827319037\\
68029.5229522952 0.00457253493088793\\
68038.1638163816 0.004572831569752\\
68046.804680468 0.00457312818978614\\
68055.4455445545 0.00457342479099396\\
68064.0864086409 0.00457372137337904\\
68072.7272727273 0.00457401793694495\\
68081.3681368137 0.00457431448169529\\
68090.0090009001 0.00457461100763362\\
68098.6498649865 0.00457490751476354\\
68107.2907290729 0.00457520400308862\\
68115.9315931593 0.00457550047261243\\
68124.5724572457 0.00457579692333856\\
68133.2133213321 0.00457609335527057\\
68141.8541854185 0.00457638976841205\\
68150.495049505 0.00457668616276656\\
68159.1359135914 0.00457698253833768\\
68167.7767776778 0.00457727889512898\\
68176.4176417642 0.00457757523314403\\
68185.0585058506 0.00457787155238639\\
68193.699369937 0.00457816785285963\\
68202.3402340234 0.00457846413456733\\
68210.9810981098 0.00457876039751303\\
68219.6219621962 0.00457905664170032\\
68228.2628262826 0.00457935286713274\\
68236.903690369 0.00457964907381387\\
68245.5445544554 0.00457994526174726\\
68254.1854185418 0.00458024143093647\\
68262.8262826283 0.00458053758138506\\
68271.4671467147 0.00458083371309659\\
68280.1080108011 0.00458112982607462\\
68288.7488748875 0.00458142592032269\\
68297.3897389739 0.00458172199584436\\
68306.0306030603 0.00458201805264319\\
68314.6714671467 0.00458231409072272\\
68323.3123312331 0.00458261011008651\\
68331.9531953195 0.0045829061107381\\
68340.5940594059 0.00458320209268105\\
68349.2349234923 0.00458349805591889\\
68357.8757875788 0.00458379400045519\\
68366.5166516652 0.00458408992629347\\
68375.1575157516 0.00458438583343729\\
68383.798379838 0.00458468172189019\\
68392.4392439244 0.00458497759165571\\
68401.0801080108 0.00458527344273738\\
68409.7209720972 0.00458556927513875\\
68418.3618361836 0.00458586508886336\\
68427.00270027 0.00458616088391474\\
68435.6435643564 0.00458645666029643\\
68444.2844284428 0.00458675241801196\\
68452.9252925293 0.00458704815706487\\
68461.5661566157 0.00458734387745869\\
68470.2070207021 0.00458763957919695\\
68478.8478847885 0.00458793526228318\\
68487.4887488749 0.00458823092672091\\
68496.1296129613 0.00458852657251367\\
68504.7704770477 0.00458882219966498\\
68513.4113411341 0.00458911780817838\\
68522.0522052205 0.00458941339805738\\
68530.6930693069 0.0045897089693055\\
68539.3339333933 0.00459000452192628\\
68547.9747974798 0.00459030005592324\\
68556.6156615662 0.00459059557129988\\
68565.2565256526 0.00459089106805974\\
68573.897389739 0.00459118654620632\\
68582.5382538254 0.00459148200574316\\
68591.1791179118 0.00459177744667375\\
68599.8199819982 0.00459207286900163\\
68608.4608460846 0.00459236827273029\\
68617.101710171 0.00459266365786326\\
68625.7425742574 0.00459295902440404\\
68634.3834383438 0.00459325437235615\\
68643.0243024302 0.00459354970172309\\
68651.6651665167 0.00459384501250837\\
68660.3060306031 0.00459414030471551\\
68668.9468946895 0.00459443557834799\\
68677.5877587759 0.00459473083340934\\
68686.2286228623 0.00459502606990305\\
68694.8694869487 0.00459532128783263\\
68703.5103510351 0.00459561648720157\\
68712.1512151215 0.00459591166801338\\
68720.7920792079 0.00459620683027156\\
68729.4329432943 0.0045965019739796\\
68738.0738073807 0.00459679709914101\\
68746.7146714671 0.00459709220575927\\
68755.3555355535 0.00459738729383789\\
68763.99639964 0.00459768236338035\\
68772.6372637264 0.00459797741439015\\
68781.2781278128 0.00459827244687079\\
68789.9189918992 0.00459856746082574\\
68798.5598559856 0.00459886245625851\\
68807.200720072 0.00459915743317257\\
68815.8415841584 0.00459945239157142\\
68824.4824482448 0.00459974733145854\\
68833.1233123312 0.00460004225283742\\
68841.7641764176 0.00460033715571154\\
68850.405040504 0.00460063204008438\\
68859.0459045905 0.00460092690595943\\
68867.6867686769 0.00460122175334016\\
68876.3276327633 0.00460151658223006\\
68884.9684968497 0.0046018113926326\\
68893.6093609361 0.00460210618455127\\
68902.2502250225 0.00460240095798953\\
68910.8910891089 0.00460269571295087\\
68919.5319531953 0.00460299044943875\\
68928.1728172817 0.00460328516745665\\
68936.8136813681 0.00460357986700804\\
68945.4545454545 0.0046038745480964\\
68954.095409541 0.00460416921072518\\
68962.7362736274 0.00460446385489787\\
68971.3771377138 0.00460475848061793\\
68980.0180018002 0.00460505308788882\\
68988.6588658866 0.00460534767671401\\
68997.299729973 0.00460564224709697\\
69005.9405940594 0.00460593679904115\\
69014.5814581458 0.00460623133255002\\
69023.2223222322 0.00460652584762705\\
69031.8631863186 0.00460682034427568\\
69040.504050405 0.00460711482249939\\
69049.1449144915 0.00460740928230162\\
69057.7857785779 0.00460770372368583\\
69066.4266426643 0.00460799814665549\\
69075.0675067507 0.00460829255121404\\
69083.7083708371 0.00460858693736494\\
69092.3492349235 0.00460888130511163\\
69100.9900990099 0.00460917565445758\\
69109.6309630963 0.00460946998540623\\
69118.2718271827 0.00460976429796103\\
69126.9126912691 0.00461005859212543\\
};
\addplot [
color=red,
solid,
forget plot
]
table[row sep=crcr]{
69126.9126912691 0.00461005859212543\\
69135.5535553555 0.00461035286790288\\
69144.1944194419 0.00461064712529682\\
69152.8352835284 0.00461094136431069\\
69161.4761476148 0.00461123558494794\\
69170.1170117012 0.00461152978721202\\
69178.7578757876 0.00461182397110636\\
69187.398739874 0.00461211813663441\\
69196.0396039604 0.0046124122837996\\
69204.6804680468 0.00461270641260538\\
69213.3213321332 0.00461300052305517\\
69221.9621962196 0.00461329461515243\\
69230.603060306 0.00461358868890057\\
69239.2439243924 0.00461388274430304\\
69247.8847884788 0.00461417678136327\\
69256.5256525653 0.00461447080008469\\
69265.1665166517 0.00461476480047073\\
69273.8073807381 0.00461505878252482\\
69282.4482448245 0.0046153527462504\\
69291.0891089109 0.00461564669165088\\
69299.7299729973 0.0046159406187297\\
69308.3708370837 0.00461623452749027\\
69317.0117011701 0.00461652841793603\\
69325.6525652565 0.0046168222900704\\
69334.2934293429 0.00461711614389681\\
69342.9342934293 0.00461740997941866\\
69351.5751575157 0.00461770379663938\\
69360.2160216022 0.0046179975955624\\
69368.8568856886 0.00461829137619112\\
69377.497749775 0.00461858513852897\\
69386.1386138614 0.00461887888257937\\
69394.7794779478 0.00461917260834572\\
69403.4203420342 0.00461946631583144\\
69412.0612061206 0.00461976000503994\\
69420.702070207 0.00462005367597464\\
69429.3429342934 0.00462034732863895\\
69437.9837983798 0.00462064096303627\\
69446.6246624662 0.00462093457917002\\
69455.2655265527 0.00462122817704359\\
69463.9063906391 0.00462152175666041\\
69472.5472547255 0.00462181531802387\\
69481.1881188119 0.00462210886113738\\
69489.8289828983 0.00462240238600434\\
69498.4698469847 0.00462269589262815\\
69507.1107110711 0.00462298938101221\\
69515.7515751575 0.00462328285115993\\
69524.3924392439 0.00462357630307471\\
69533.0333033303 0.00462386973675993\\
69541.6741674167 0.00462416315221901\\
69550.3150315032 0.00462445654945532\\
69558.9558955896 0.00462474992847228\\
69567.596759676 0.00462504328927327\\
69576.2376237624 0.00462533663186168\\
69584.8784878488 0.00462562995624091\\
69593.5193519352 0.00462592326241435\\
69602.1602160216 0.00462621655038538\\
69610.801080108 0.00462650982015741\\
69619.4419441944 0.0046268030717338\\
69628.0828082808 0.00462709630511795\\
69636.7236723672 0.00462738952031325\\
69645.3645364536 0.00462768271732308\\
69654.0054005401 0.00462797589615081\\
69662.6462646265 0.00462826905679985\\
69671.2871287129 0.00462856219927355\\
69679.9279927993 0.00462885532357532\\
69688.5688568857 0.00462914842970851\\
69697.2097209721 0.00462944151767652\\
69705.8505850585 0.00462973458748272\\
69714.4914491449 0.00463002763913049\\
69723.1323132313 0.00463032067262319\\
69731.7731773177 0.00463061368796421\\
69740.4140414041 0.00463090668515691\\
69749.0549054905 0.00463119966420468\\
69757.695769577 0.00463149262511087\\
69766.3366336634 0.00463178556787886\\
69774.9774977498 0.00463207849251202\\
69783.6183618362 0.00463237139901371\\
69792.2592259226 0.0046326642873873\\
69800.900090009 0.00463295715763616\\
69809.5409540954 0.00463325000976366\\
69818.1818181818 0.00463354284377314\\
69826.8226822682 0.00463383565966798\\
69835.4635463546 0.00463412845745154\\
69844.104410441 0.00463442123712718\\
69852.7452745274 0.00463471399869826\\
69861.3861386139 0.00463500674216813\\
69870.0270027003 0.00463529946754016\\
69878.6678667867 0.00463559217481769\\
69887.3087308731 0.00463588486400409\\
69895.9495949595 0.0046361775351027\\
69904.5904590459 0.00463647018811689\\
69913.2313231323 0.00463676282305\\
69921.8721872187 0.00463705543990538\\
69930.5130513051 0.00463734803868639\\
69939.1539153915 0.00463764061939637\\
69947.7947794779 0.00463793318203867\\
69956.4356435644 0.00463822572661664\\
69965.0765076508 0.00463851825313362\\
69973.7173717372 0.00463881076159296\\
69982.3582358236 0.00463910325199799\\
69990.99909991 0.00463939572435208\\
69999.6399639964 0.00463968817865854\\
70008.2808280828 0.00463998061492074\\
70016.9216921692 0.004640273033142\\
70025.5625562556 0.00464056543332566\\
70034.203420342 0.00464085781547507\\
70042.8442844284 0.00464115017959355\\
70051.4851485149 0.00464144252568445\\
70060.1260126013 0.0046417348537511\\
70068.7668766877 0.00464202716379682\\
70077.4077407741 0.00464231945582496\\
70086.0486048605 0.00464261172983885\\
70094.6894689469 0.00464290398584181\\
70103.3303330333 0.00464319622383718\\
70111.9711971197 0.00464348844382827\\
70120.6120612061 0.00464378064581843\\
70129.2529252925 0.00464407282981098\\
70137.8937893789 0.00464436499580924\\
70146.5346534654 0.00464465714381653\\
70155.1755175518 0.00464494927383618\\
70163.8163816382 0.00464524138587151\\
70172.4572457246 0.00464553347992584\\
70181.098109811 0.0046458255560025\\
70189.7389738974 0.0046461176141048\\
70198.3798379838 0.00464640965423605\\
70207.0207020702 0.00464670167639958\\
70215.6615661566 0.00464699368059871\\
70224.302430243 0.00464728566683674\\
70232.9432943294 0.00464757763511699\\
70241.5841584158 0.00464786958544278\\
70250.2250225022 0.00464816151781741\\
70258.8658865887 0.0046484534322442\\
70267.5067506751 0.00464874532872645\\
70276.1476147615 0.00464903720726748\\
70284.7884788479 0.00464932906787059\\
70293.4293429343 0.0046496209105391\\
70302.0702070207 0.0046499127352763\\
70310.7110711071 0.0046502045420855\\
70319.3519351935 0.00465049633097\\
70327.9927992799 0.00465078810193311\\
70336.6336633663 0.00465107985497813\\
70345.2745274527 0.00465137159010836\\
70353.9153915392 0.0046516633073271\\
70362.5562556256 0.00465195500663764\\
70371.197119712 0.0046522466880433\\
70379.8379837984 0.00465253835154735\\
70388.4788478848 0.0046528299971531\\
70397.1197119712 0.00465312162486385\\
70405.7605760576 0.00465341323468288\\
70414.401440144 0.00465370482661349\\
70423.0423042304 0.00465399640065897\\
70431.6831683168 0.00465428795682261\\
70440.3240324032 0.00465457949510771\\
70448.9648964896 0.00465487101551754\\
70457.6057605761 0.0046551625180554\\
70466.2466246625 0.00465545400272457\\
70474.8874887489 0.00465574546952834\\
70483.5283528353 0.00465603691847\\
70492.1692169217 0.00465632834955282\\
70500.8100810081 0.00465661976278009\\
70509.4509450945 0.0046569111581551\\
70518.0918091809 0.00465720253568111\\
70526.7326732673 0.00465749389536142\\
70535.3735373537 0.0046577852371993\\
70544.0144014401 0.00465807656119802\\
70552.6552655266 0.00465836786736087\\
70561.296129613 0.00465865915569112\\
70569.9369936994 0.00465895042619204\\
70578.5778577858 0.00465924167886691\\
70587.2187218722 0.004659532913719\\
70595.8595859586 0.00465982413075158\\
70604.500450045 0.00466011532996792\\
70613.1413141314 0.0046604065113713\\
70621.7821782178 0.00466069767496497\\
70630.4230423042 0.00466098882075221\\
70639.0639063906 0.00466127994873629\\
70647.7047704771 0.00466157105892047\\
70656.3456345635 0.00466186215130801\\
70664.9864986499 0.00466215322590218\\
70673.6273627363 0.00466244428270624\\
70682.2682268227 0.00466273532172345\\
70690.9090909091 0.00466302634295707\\
70699.5499549955 0.00466331734641037\\
70708.1908190819 0.00466360833208659\\
70716.8316831683 0.004663899299989\\
70725.4725472547 0.00466419025012086\\
70734.1134113411 0.00466448118248542\\
70742.7542754275 0.00466477209708593\\
70751.3951395139 0.00466506299392565\\
70760.0360036004 0.00466535387300783\\
70768.6768676868 0.00466564473433573\\
70777.3177317732 0.00466593557791259\\
70785.9585958596 0.00466622640374165\\
70794.599459946 0.00466651721182619\\
70803.2403240324 0.00466680800216942\\
70811.8811881188 0.00466709877477462\\
70820.5220522052 0.00466738952964501\\
70829.1629162916 0.00466768026678385\\
70837.803780378 0.00466797098619439\\
70846.4446444644 0.00466826168787985\\
70855.0855085509 0.00466855237184348\\
70863.7263726373 0.00466884303808853\\
70872.3672367237 0.00466913368661824\\
70881.0081008101 0.00466942431743584\\
70889.6489648965 0.00466971493054457\\
70898.2898289829 0.00467000552594766\\
70906.9306930693 0.00467029610364836\\
70915.5715571557 0.0046705866636499\\
70924.2124212421 0.0046708772059555\\
70932.8532853285 0.00467116773056842\\
70941.4941494149 0.00467145823749186\\
70950.1350135013 0.00467174872672908\\
70958.7758775878 0.00467203919828329\\
70967.4167416742 0.00467232965215773\\
70976.0576057606 0.00467262008835562\\
70984.698469847 0.00467291050688019\\
70993.3393339334 0.00467320090773467\\
71001.9801980198 0.00467349129092228\\
71010.6210621062 0.00467378165644625\\
71019.2619261926 0.00467407200430979\\
71027.902790279 0.00467436233451614\\
71036.5436543654 0.0046746526470685\\
71045.1845184518 0.00467494294197011\\
71053.8253825383 0.00467523321922418\\
71062.4662466247 0.00467552347883392\\
71071.1071107111 0.00467581372080257\\
71079.7479747975 0.00467610394513332\\
71088.3888388839 0.0046763941518294\\
71097.0297029703 0.00467668434089402\\
71105.6705670567 0.0046769745123304\\
71114.3114311431 0.00467726466614174\\
71122.9522952295 0.00467755480233126\\
71131.5931593159 0.00467784492090217\\
71140.2340234023 0.00467813502185768\\
71148.8748874888 0.00467842510520099\\
71157.5157515752 0.00467871517093532\\
71166.1566156616 0.00467900521906386\\
71174.797479748 0.00467929524958983\\
71183.4383438344 0.00467958526251643\\
71192.0792079208 0.00467987525784686\\
71200.7200720072 0.00468016523558433\\
71209.3609360936 0.00468045519573203\\
71218.00180018 0.00468074513829317\\
71226.6426642664 0.00468103506327095\\
71235.2835283528 0.00468132497066856\\
71243.9243924393 0.00468161486048921\\
71252.5652565256 0.00468190473273609\\
71261.2061206121 0.00468219458741239\\
71269.8469846985 0.00468248442452131\\
71278.4878487849 0.00468277424406604\\
71287.1287128713 0.00468306404604979\\
71295.7695769577 0.00468335383047573\\
71304.4104410441 0.00468364359734706\\
71313.0513051305 0.00468393334666697\\
71321.6921692169 0.00468422307843864\\
71330.3330333033 0.00468451279266528\\
71338.9738973897 0.00468480248935006\\
71347.6147614761 0.00468509216849616\\
71356.2556255626 0.00468538183010678\\
71364.896489649 0.0046856714741851\\
71373.5373537354 0.0046859611007343\\
71382.1782178218 0.00468625070975757\\
71390.8190819082 0.00468654030125808\\
71399.4599459946 0.00468682987523901\\
71408.100810081 0.00468711943170355\\
71416.7416741674 0.00468740897065487\\
71425.3825382538 0.00468769849209616\\
71434.0234023402 0.00468798799603058\\
71442.6642664266 0.00468827748246131\\
71451.3051305131 0.00468856695139152\\
71459.9459945995 0.0046888564028244\\
71468.5868586859 0.00468914583676311\\
71477.2277227723 0.00468943525321083\\
71485.8685868587 0.00468972465217072\\
71494.5094509451 0.00469001403364595\\
71503.1503150315 0.0046903033976397\\
71511.7911791179 0.00469059274415512\\
71520.4320432043 0.0046908820731954\\
71529.0729072907 0.00469117138476369\\
71537.7137713771 0.00469146067886316\\
71546.3546354635 0.00469174995549697\\
71554.99549955 0.00469203921466829\\
71563.6363636364 0.00469232845638027\\
71572.2772277228 0.00469261768063609\\
71580.9180918092 0.00469290688743889\\
71589.5589558956 0.00469319607679184\\
71598.199819982 0.0046934852486981\\
71606.8406840684 0.00469377440316082\\
71615.4815481548 0.00469406354018316\\
71624.1224122412 0.00469435265976828\\
71632.7632763276 0.00469464176191932\\
71641.404140414 0.00469493084663945\\
71650.0450045005 0.00469521991393182\\
71658.6858685869 0.00469550896379957\\
71667.3267326733 0.00469579799624586\\
71675.9675967597 0.00469608701127384\\
71684.6084608461 0.00469637600888665\\
71693.2493249325 0.00469666498908745\\
71701.8901890189 0.00469695395187938\\
71710.5310531053 0.00469724289726558\\
71719.1719171917 0.00469753182524921\\
71727.8127812781 0.0046978207358334\\
71736.4536453645 0.00469810962902131\\
71745.094509451 0.00469839850481606\\
71753.7353735373 0.0046986873632208\\
71762.3762376238 0.00469897620423868\\
71771.0171017102 0.00469926502787283\\
71779.6579657966 0.00469955383412639\\
71788.298829883 0.0046998426230025\\
71796.9396939694 0.00470013139450428\\
71805.5805580558 0.00470042014863489\\
71814.2214221422 0.00470070888539745\\
71822.8622862286 0.0047009976047951\\
71831.503150315 0.00470128630683097\\
71840.1440144014 0.00470157499150819\\
71848.7848784878 0.0047018636588299\\
71857.4257425743 0.00470215230879921\\
71866.0666066607 0.00470244094141927\\
71874.7074707471 0.00470272955669319\\
71883.3483348335 0.00470301815462411\\
71891.9891989199 0.00470330673521516\\
71900.6300630063 0.00470359529846945\\
71909.2709270927 0.00470388384439011\\
71917.9117911791 0.00470417237298027\\
71926.5526552655 0.00470446088424304\\
71935.1935193519 0.00470474937818155\\
71943.8343834383 0.00470503785479892\\
71952.4752475248 0.00470532631409827\\
71961.1161116112 0.00470561475608272\\
71969.7569756976 0.00470590318075538\\
71978.397839784 0.00470619158811937\\
71987.0387038704 0.00470647997817781\\
71995.6795679568 0.00470676835093381\\
72004.3204320432 0.00470705670639048\\
72012.9612961296 0.00470734504455095\\
72021.602160216 0.00470763336541831\\
72030.2430243024 0.00470792166899569\\
72038.8838883888 0.00470820995528619\\
72047.5247524752 0.00470849822429292\\
72056.1656165617 0.00470878647601899\\
72064.8064806481 0.0047090747104675\\
72073.4473447345 0.00470936292764157\\
72082.0882088209 0.0047096511275443\\
72090.7290729073 0.00470993931017879\\
72099.3699369937 0.00471022747554815\\
72108.0108010801 0.00471051562365548\\
72116.6516651665 0.00471080375450388\\
72125.2925292529 0.00471109186809645\\
72133.9333933393 0.00471137996443629\\
72142.5742574257 0.00471166804352651\\
72151.2151215122 0.00471195610537019\\
72159.8559855986 0.00471224414997045\\
72168.496849685 0.00471253217733036\\
72177.1377137714 0.00471282018745303\\
72185.7785778578 0.00471310818034155\\
72194.4194419442 0.00471339615599902\\
72203.0603060306 0.00471368411442853\\
72211.701170117 0.00471397205563317\\
72220.3420342034 0.00471425997961603\\
72228.9828982898 0.0047145478863802\\
72237.6237623762 0.00471483577592876\\
72246.2646264627 0.00471512364826482\\
72254.905490549 0.00471541150339144\\
72263.5463546355 0.00471569934131173\\
72272.1872187219 0.00471598716202877\\
72280.8280828083 0.00471627496554563\\
72289.4689468947 0.0047165627518654\\
72298.1098109811 0.00471685052099117\\
72306.7506750675 0.00471713827292602\\
72315.3915391539 0.00471742600767302\\
72324.0324032403 0.00471771372523526\\
72332.6732673267 0.00471800142561582\\
72341.3141314131 0.00471828910881777\\
72349.9549954995 0.00471857677484418\\
72358.595859586 0.00471886442369815\\
72367.2367236724 0.00471915205538273\\
72375.8775877588 0.00471943966990101\\
72384.5184518452 0.00471972726725606\\
72393.1593159316 0.00472001484745094\\
72401.800180018 0.00472030241048874\\
72410.4410441044 0.00472058995637252\\
72419.0819081908 0.00472087748510535\\
72427.7227722772 0.00472116499669031\\
72436.3636363636 0.00472145249113045\\
72445.00450045 0.00472173996842884\\
72453.6453645365 0.00472202742858855\\
72462.2862286229 0.00472231487161265\\
72470.9270927093 0.0047226022975042\\
72479.5679567957 0.00472288970626626\\
72488.2088208821 0.0047231770979019\\
72496.8496849685 0.00472346447241417\\
72505.4905490549 0.00472375182980614\\
72514.1314131413 0.00472403917008086\\
72522.7722772277 0.0047243264932414\\
72531.4131413141 0.00472461379929081\\
72540.0540054005 0.00472490108823215\\
72548.6948694869 0.00472518836006847\\
72557.3357335734 0.00472547561480283\\
72565.9765976598 0.00472576285243829\\
72574.6174617462 0.00472605007297789\\
72583.2583258326 0.00472633727642469\\
72591.899189919 0.00472662446278174\\
72600.5400540054 0.00472691163205209\\
72609.1809180918 0.00472719878423879\\
72617.8217821782 0.00472748591934489\\
72626.4626462646 0.00472777303737343\\
72635.103510351 0.00472806013832747\\
72643.7443744374 0.00472834722221004\\
72652.3852385239 0.0047286342890242\\
72661.0261026103 0.00472892133877299\\
72669.6669666967 0.00472920837145944\\
72678.3078307831 0.00472949538708661\\
72686.9486948695 0.00472978238565754\\
72695.5895589559 0.00473006936717525\\
72704.2304230423 0.0047303563316428\\
72712.8712871287 0.00473064327906323\\
72721.5121512151 0.00473093020943956\\
72730.1530153015 0.00473121712277484\\
72738.7938793879 0.00473150401907211\\
72747.4347434744 0.00473179089833439\\
72756.0756075607 0.00473207776056472\\
72764.7164716472 0.00473236460576614\\
72773.3573357336 0.00473265143394168\\
72781.99819982 0.00473293824509436\\
72790.6390639064 0.00473322503922723\\
72799.2799279928 0.0047335118163433\\
72807.9207920792 0.00473379857644561\\
72816.5616561656 0.00473408531953719\\
72825.202520252 0.00473437204562105\\
72833.8433843384 0.00473465875470024\\
72842.4842484248 0.00473494544677777\\
72851.1251125112 0.00473523212185667\\
72859.7659765977 0.00473551877993996\\
72868.4068406841 0.00473580542103066\\
72877.0477047705 0.0047360920451318\\
72885.6885688569 0.00473637865224639\\
72894.3294329433 0.00473666524237746\\
72902.9702970297 0.00473695181552802\\
72911.6111611161 0.00473723837170109\\
72920.2520252025 0.00473752491089969\\
72928.8928892889 0.00473781143312684\\
72937.5337533753 0.00473809793838555\\
72946.1746174617 0.00473838442667883\\
72954.8154815482 0.0047386708980097\\
72963.4563456346 0.00473895735238117\\
72972.097209721 0.00473924378979626\\
72980.7380738074 0.00473953021025796\\
72989.3789378938 0.0047398166137693\\
72998.0198019802 0.00474010300033329\\
73006.6606660666 0.00474038936995292\\
73015.301530153 0.00474067572263121\\
73023.9423942394 0.00474096205837116\\
73032.5832583258 0.00474124837717579\\
73041.2241224122 0.00474153467904808\\
73049.8649864987 0.00474182096399106\\
73058.5058505851 0.00474210723200771\\
73067.1467146715 0.00474239348310105\\
73075.7875787579 0.00474267971727407\\
73084.4284428443 0.00474296593452978\\
73093.0693069307 0.00474325213487117\\
73101.7101710171 0.00474353831830123\\
73110.3510351035 0.00474382448482298\\
73118.9918991899 0.0047441106344394\\
73127.6327632763 0.0047443967671535\\
73136.2736273627 0.00474468288296825\\
73144.9144914491 0.00474496898188667\\
73153.5553555356 0.00474525506391174\\
73162.196219622 0.00474554112904646\\
73170.8370837084 0.00474582717729381\\
73179.4779477948 0.00474611320865679\\
73188.1188118812 0.00474639922313838\\
73196.7596759676 0.00474668522074158\\
73205.400540054 0.00474697120146937\\
73214.0414041404 0.00474725716532474\\
73222.6822682268 0.00474754311231067\\
73231.3231323132 0.00474782904243016\\
73239.9639963996 0.00474811495568618\\
73248.6048604861 0.00474840085208171\\
73257.2457245725 0.00474868673161975\\
73265.8865886589 0.00474897259430327\\
73274.5274527453 0.00474925844013526\\
73283.1683168317 0.00474954426911868\\
73291.8091809181 0.00474983008125653\\
73300.4500450045 0.00475011587655177\\
73309.0909090909 0.0047504016550074\\
73317.7317731773 0.00475068741662637\\
73326.3726372637 0.00475097316141168\\
73335.0135013501 0.00475125888936629\\
73343.6543654365 0.00475154460049317\\
73352.2952295229 0.0047518302947953\\
73360.9360936094 0.00475211597227566\\
73369.5769576958 0.00475240163293721\\
73378.2178217822 0.00475268727678292\\
73386.8586858686 0.00475297290381576\\
73395.499549955 0.0047532585140387\\
73404.1404140414 0.00475354410745471\\
73412.7812781278 0.00475382968406676\\
73421.4221422142 0.00475411524387781\\
73430.0630063006 0.00475440078689082\\
73438.703870387 0.00475468631310876\\
73447.3447344734 0.0047549718225346\\
73455.9855985599 0.00475525731517129\\
73464.6264626463 0.00475554279102179\\
73473.2673267327 0.00475582825008908\\
73481.9081908191 0.0047561136923761\\
73490.5490549055 0.00475639911788582\\
73499.1899189919 0.00475668452662119\\
73507.8307830783 0.00475696991858517\\
73516.4716471647 0.00475725529378072\\
73525.1125112511 0.00475754065221079\\
73533.7533753375 0.00475782599387834\\
73542.3942394239 0.00475811131878632\\
73551.0351035104 0.00475839662693768\\
73559.6759675968 0.00475868191833538\\
73568.3168316832 0.00475896719298237\\
73576.9576957696 0.00475925245088159\\
73585.598559856 0.004759537692036\\
73594.2394239424 0.00475982291644854\\
73602.8802880288 0.00476010812412217\\
73611.5211521152 0.00476039331505982\\
73620.1620162016 0.00476067848926445\\
73628.802880288 0.00476096364673899\\
73637.4437443744 0.00476124878748641\\
73646.0846084608 0.00476153391150962\\
73654.7254725473 0.00476181901881159\\
73663.3663366337 0.00476210410939525\\
73672.0072007201 0.00476238918326354\\
73680.6480648065 0.0047626742404194\\
73689.2889288929 0.00476295928086577\\
73697.9297929793 0.0047632443046056\\
73706.5706570657 0.0047635293116418\\
73715.2115211521 0.00476381430197733\\
73723.8523852385 0.00476409927561512\\
73732.4932493249 0.0047643842325581\\
73741.1341134113 0.0047646691728092\\
73749.7749774978 0.00476495409637136\\
73758.4158415842 0.00476523900324752\\
73767.0567056706 0.00476552389344059\\
73775.697569757 0.00476580876695352\\
73784.3384338434 0.00476609362378923\\
73792.9792979298 0.00476637846395066\\
73801.6201620162 0.00476666328744072\\
73810.2610261026 0.00476694809426234\\
73818.901890189 0.00476723288441846\\
73827.5427542754 0.00476751765791199\\
73836.1836183618 0.00476780241474587\\
73844.8244824482 0.00476808715492301\\
73853.4653465346 0.00476837187844634\\
73862.1062106211 0.00476865658531877\\
73870.7470747075 0.00476894127554324\\
73879.3879387939 0.00476922594912265\\
73888.0288028803 0.00476951060605994\\
73896.6696669667 0.00476979524635801\\
73905.3105310531 0.00477007987001979\\
73913.9513951395 0.00477036447704819\\
73922.5922592259 0.00477064906744613\\
73931.2331233123 0.00477093364121652\\
73939.8739873987 0.00477121819836227\\
73948.5148514851 0.00477150273888631\\
73957.1557155716 0.00477178726279154\\
73965.796579658 0.00477207177008087\\
73974.4374437444 0.00477235626075722\\
73983.0783078308 0.00477264073482349\\
73991.7191719172 0.0047729251922826\\
74000.3600360036 0.00477320963313745\\
74009.00090009 0.00477349405739094\\
74017.6417641764 0.00477377846504599\\
74026.2826282628 0.00477406285610551\\
74034.9234923492 0.00477434723057239\\
74043.5643564356 0.00477463158844954\\
74052.2052205221 0.00477491592973986\\
74060.8460846085 0.00477520025444626\\
74069.4869486949 0.00477548456257163\\
74078.1278127813 0.00477576885411888\\
74086.7686768677 0.00477605312909091\\
74095.4095409541 0.00477633738749061\\
74104.0504050405 0.00477662162932089\\
74112.6912691269 0.00477690585458464\\
74121.3321332133 0.00477719006328475\\
74129.9729972997 0.00477747425542413\\
74138.6138613861 0.00477775843100567\\
74147.2547254726 0.00477804259003225\\
74155.895589559 0.00477832673250678\\
74164.5364536454 0.00477861085843215\\
74173.1773177318 0.00477889496781124\\
74181.8181818182 0.00477917906064695\\
74190.4590459046 0.00477946313694217\\
74199.099909991 0.00477974719669978\\
74207.7407740774 0.00478003123992268\\
74216.3816381638 0.00478031526661375\\
74225.0225022502 0.00478059927677588\\
74233.6633663366 0.00478088327041195\\
74242.304230423 0.00478116724752484\\
74250.9450945095 0.00478145120811745\\
74259.5859585959 0.00478173515219265\\
74268.2268226823 0.00478201907975333\\
74276.8676867687 0.00478230299080236\\
74285.5085508551 0.00478258688534263\\
74294.1494149415 0.00478287076337702\\
74302.7902790279 0.0047831546249084\\
74311.4311431143 0.00478343846993966\\
74320.0720072007 0.00478372229847366\\
74328.7128712871 0.00478400611051329\\
74337.3537353735 0.00478428990606143\\
74345.9945994599 0.00478457368512094\\
74354.6354635464 0.0047848574476947\\
74363.2763276328 0.00478514119378558\\
74371.9171917192 0.00478542492339646\\
74380.5580558056 0.0047857086365302\\
74389.198919892 0.00478599233318968\\
74397.8397839784 0.00478627601337777\\
74406.4806480648 0.00478655967709732\\
74415.1215121512 0.00478684332435123\\
74423.7623762376 0.00478712695514234\\
74432.403240324 0.00478741056947352\\
74441.0441044104 0.00478769416734765\\
74449.6849684968 0.00478797774876758\\
74458.3258325833 0.00478826131373618\\
74466.9666966697 0.00478854486225631\\
74475.6075607561 0.00478882839433083\\
74484.2484248425 0.00478911190996261\\
74492.8892889289 0.00478939540915451\\
74501.5301530153 0.00478967889190938\\
74510.1710171017 0.00478996235823008\\
74518.8118811881 0.00479024580811947\\
74527.4527452745 0.00479052924158041\\
74536.0936093609 0.00479081265861576\\
74544.7344734473 0.00479109605922836\\
74553.3753375338 0.00479137944342108\\
74562.0162016202 0.00479166281119677\\
74570.6570657066 0.00479194616255828\\
74579.297929793 0.00479222949750846\\
74587.9387938794 0.00479251281605016\\
74596.5796579658 0.00479279611818624\\
74605.2205220522 0.00479307940391955\\
74613.8613861386 0.00479336267325292\\
74622.502250225 0.00479364592618922\\
74631.1431143114 0.00479392916273129\\
74639.7839783978 0.00479421238288197\\
74648.4248424843 0.00479449558664411\\
74657.0657065707 0.00479477877402055\\
74665.7065706571 0.00479506194501414\\
74674.3474347435 0.00479534509962772\\
74682.9882988299 0.00479562823786414\\
74691.6291629163 0.00479591135972623\\
74700.2700270027 0.00479619446521684\\
74708.9108910891 0.0047964775543388\\
74717.5517551755 0.00479676062709495\\
74726.1926192619 0.00479704368348814\\
74734.8334833483 0.00479732672352119\\
74743.4743474347 0.00479760974719696\\
74752.1152115212 0.00479789275451826\\
74760.7560756076 0.00479817574548794\\
74769.396939694 0.00479845872010883\\
74778.0378037804 0.00479874167838376\\
74786.6786678668 0.00479902462031556\\
74795.3195319532 0.00479930754590708\\
74803.9603960396 0.00479959045516113\\
74812.601260126 0.00479987334808055\\
74821.2421242124 0.00480015622466816\\
74829.8829882988 0.0048004390849268\\
74838.5238523852 0.0048007219288593\\
74847.1647164716 0.00480100475646847\\
74855.8055805581 0.00480128756775714\\
74864.4464446445 0.00480157036272815\\
74873.0873087309 0.00480185314138431\\
74881.7281728173 0.00480213590372845\\
74890.3690369037 0.00480241864976338\\
74899.0099009901 0.00480270137949194\\
74907.6507650765 0.00480298409291694\\
74916.2916291629 0.0048032667900412\\
74924.9324932493 0.00480354947086755\\
74933.5733573357 0.00480383213539879\\
74942.2142214221 0.00480411478363776\\
74950.8550855085 0.00480439741558725\\
74959.495949595 0.0048046800312501\\
74968.1368136814 0.00480496263062912\\
74976.7776777678 0.00480524521372711\\
74985.4185418542 0.0048055277805469\\
74994.0594059406 0.0048058103310913\\
75002.700270027 0.00480609286536312\\
75011.3411341134 0.00480637538336516\\
75019.9819981998 0.00480665788510025\\
75028.6228622862 0.00480694037057118\\
75037.2637263726 0.00480722283978078\\
75045.904590459 0.00480750529273184\\
75054.5454545455 0.00480778772942717\\
75063.1863186319 0.00480807014986959\\
75071.8271827183 0.00480835255406189\\
75080.4680468047 0.00480863494200688\\
75089.1089108911 0.00480891731370736\\
75097.7497749775 0.00480919966916615\\
75106.3906390639 0.00480948200838603\\
75115.0315031503 0.00480976433136981\\
75123.6723672367 0.0048100466381203\\
75132.3132313231 0.00481032892864029\\
75140.9540954095 0.00481061120293257\\
75149.594959496 0.00481089346099996\\
75158.2358235824 0.00481117570284524\\
75166.8766876688 0.00481145792847122\\
75175.5175517552 0.00481174013788068\\
75184.1584158416 0.00481202233107643\\
75192.799279928 0.00481230450806126\\
75201.4401440144 0.00481258666883796\\
75210.0810081008 0.00481286881340933\\
75218.7218721872 0.00481315094177815\\
75227.3627362736 0.00481343305394722\\
75236.00360036 0.00481371514991932\\
75244.6444644465 0.00481399722969725\\
75253.2853285329 0.0048142792932838\\
75261.9261926193 0.00481456134068175\\
75270.5670567057 0.0048148433718939\\
75279.2079207921 0.00481512538692301\\
75287.8487848785 0.0048154073857719\\
75296.4896489649 0.00481568936844332\\
75305.1305130513 0.00481597133494008\\
75313.7713771377 0.00481625328526495\\
75322.4122412241 0.00481653521942072\\
75331.0531053105 0.00481681713741017\\
75339.6939693969 0.00481709903923607\\
75348.3348334833 0.00481738092490122\\
75356.9756975698 0.00481766279440838\\
75365.6165616562 0.00481794464776033\\
75374.2574257426 0.00481822648495986\\
75382.898289829 0.00481850830600973\\
75391.5391539154 0.00481879011091273\\
75400.1800180018 0.00481907189967164\\
75408.8208820882 0.00481935367228921\\
75417.4617461746 0.00481963542876824\\
75426.102610261 0.00481991716911148\\
75434.7434743474 0.00482019889332172\\
75443.3843384338 0.00482048060140172\\
75452.0252025202 0.00482076229335426\\
75460.6660666067 0.0048210439691821\\
75469.3069306931 0.00482132562888801\\
75477.9477947795 0.00482160727247477\\
75486.5886588659 0.00482188889994513\\
75495.2295229523 0.00482217051130186\\
75503.8703870387 0.00482245210654774\\
75512.5112511251 0.00482273368568551\\
75521.1521152115 0.00482301524871796\\
75529.7929792979 0.00482329679564784\\
75538.4338433843 0.00482357832647791\\
75547.0747074707 0.00482385984121094\\
75555.7155715572 0.00482414133984968\\
75564.3564356436 0.0048244228223969\\
75572.99729973 0.00482470428885535\\
75581.6381638164 0.0048249857392278\\
75590.2790279028 0.004825267173517\\
75598.9198919892 0.00482554859172571\\
75607.5607560756 0.00482582999385668\\
75616.201620162 0.00482611137991268\\
75624.8424842484 0.00482639274989645\\
75633.4833483348 0.00482667410381074\\
75642.1242124212 0.00482695544165832\\
75650.7650765077 0.00482723676344193\\
75659.4059405941 0.00482751806916432\\
75668.0468046805 0.00482779935882825\\
75676.6876687669 0.00482808063243647\\
75685.3285328533 0.00482836188999172\\
75693.9693969397 0.00482864313149675\\
75702.6102610261 0.00482892435695431\\
75711.2511251125 0.00482920556636715\\
75719.8919891989 0.00482948675973801\\
75728.5328532853 0.00482976793706963\\
75737.1737173717 0.00483004909836477\\
75745.8145814582 0.00483033024362616\\
75754.4554455446 0.00483061137285656\\
75763.096309631 0.00483089248605869\\
75771.7371737174 0.0048311735832353\\
75780.3780378038 0.00483145466438913\\
75789.0189018902 0.00483173572952293\\
75797.6597659766 0.00483201677863943\\
75806.300630063 0.00483229781174136\\
75814.9414941494 0.00483257882883147\\
75823.5823582358 0.00483285982991249\\
75832.2232223222 0.00483314081498716\\
75840.8640864086 0.00483342178405821\\
75849.504950495 0.00483370273712838\\
75858.1458145815 0.0048339836742004\\
75866.7866786679 0.004834264595277\\
75875.4275427543 0.00483454550036091\\
75884.0684068407 0.00483482638945487\\
75892.7092709271 0.00483510726256161\\
75901.3501350135 0.00483538811968385\\
75909.9909990999 0.00483566896082433\\
75918.6318631863 0.00483594978598577\\
75927.2727272727 0.00483623059517089\\
75935.9135913591 0.00483651138838244\\
75944.5544554455 0.00483679216562312\\
75953.195319532 0.00483707292689567\\
75961.8361836184 0.00483735367220281\\
75970.4770477048 0.00483763440154727\\
75979.1179117912 0.00483791511493176\\
75987.7587758776 0.00483819581235901\\
75996.399639964 0.00483847649383173\\
76005.0405040504 0.00483875715935266\\
76013.6813681368 0.0048390378089245\\
76022.3222322232 0.00483931844254998\\
76030.9630963096 0.00483959906023182\\
76039.603960396 0.00483987966197273\\
76048.2448244824 0.00484016024777542\\
76056.8856885689 0.00484044081764263\\
76065.5265526553 0.00484072137157704\\
76074.1674167417 0.0048410019095814\\
76082.8082808281 0.00484128243165839\\
76091.4491449145 0.00484156293781075\\
76100.0900090009 0.00484184342804118\\
76108.7308730873 0.00484212390235239\\
76117.3717371737 0.00484240436074709\\
76126.0126012601 0.00484268480322799\\
76134.6534653465 0.0048429652297978\\
76143.2943294329 0.00484324564045923\\
76151.9351935194 0.00484352603521498\\
76160.5760576058 0.00484380641406776\\
76169.2169216922 0.00484408677702028\\
76177.8577857786 0.00484436712407524\\
76186.498649865 0.00484464745523534\\
76195.1395139514 0.0048449277705033\\
76203.7803780378 0.0048452080698818\\
76212.4212421242 0.00484548835337356\\
76221.0621062106 0.00484576862098127\\
76229.702970297 0.00484604887270764\\
76238.3438343834 0.00484632910855535\\
76246.9846984699 0.00484660932852712\\
76255.6255625563 0.00484688953262564\\
76264.2664266427 0.00484716972085361\\
76272.9072907291 0.00484744989321372\\
76281.5481548155 0.00484773004970867\\
76290.1890189019 0.00484801019034116\\
76298.8298829883 0.00484829031511387\\
76307.4707470747 0.0048485704240295\\
76316.1116111611 0.00484885051709075\\
76324.7524752475 0.0048491305943003\\
76333.3933393339 0.00484941065566085\\
76342.0342034203 0.00484969070117509\\
76350.6750675067 0.0048499707308457\\
76359.3159315932 0.00485025074467538\\
76367.9567956796 0.00485053074266681\\
76376.597659766 0.00485081072482269\\
76385.2385238524 0.00485109069114569\\
76393.8793879388 0.0048513706416385\\
76402.5202520252 0.00485165057630381\\
76411.1611161116 0.00485193049514431\\
76419.801980198 0.00485221039816266\\
76428.4428442844 0.00485249028536157\\
76437.0837083708 0.00485277015674371\\
76445.7245724572 0.00485305001231176\\
76454.3654365437 0.0048533298520684\\
76463.0063006301 0.00485360967601631\\
76471.6471647165 0.00485388948415817\\
76480.2880288029 0.00485416927649666\\
76488.9288928893 0.00485444905303446\\
76497.5697569757 0.00485472881377424\\
76506.2106210621 0.00485500855871867\\
76514.8514851485 0.00485528828787044\\
76523.4923492349 0.00485556800123222\\
76532.1332133213 0.00485584769880668\\
76540.7740774077 0.0048561273805965\\
76549.4149414941 0.00485640704660434\\
76558.0558055806 0.00485668669683287\\
76566.696669667 0.00485696633128478\\
76575.3375337534 0.00485724594996273\\
76583.9783978398 0.00485752555286938\\
76592.6192619262 0.00485780514000741\\
76601.2601260126 0.00485808471137948\\
76609.900990099 0.00485836426698826\\
76618.5418541854 0.00485864380683641\\
76627.1827182718 0.00485892333092661\\
76635.8235823582 0.00485920283926152\\
76644.4644464446 0.00485948233184379\\
76653.1053105311 0.0048597618086761\\
76661.7461746175 0.0048600412697611\\
76670.3870387039 0.00486032071510147\\
76679.0279027903 0.00486060014469984\\
76687.6687668767 0.0048608795585589\\
76696.3096309631 0.00486115895668129\\
76704.9504950495 0.00486143833906968\\
76713.5913591359 0.00486171770572672\\
76722.2322232223 0.00486199705665508\\
76730.8730873087 0.0048622763918574\\
76739.5139513951 0.00486255571133634\\
76748.1548154816 0.00486283501509456\\
76756.795679568 0.00486311430313471\\
76765.4365436544 0.00486339357545945\\
76774.0774077408 0.00486367283207142\\
76782.7182718272 0.00486395207297329\\
76791.3591359136 0.00486423129816769\\
76800 0.00486451050765728\\
76808.6408640864 0.00486478970144472\\
76817.2817281728 0.00486506887953264\\
76825.9225922592 0.00486534804192371\\
76834.5634563456 0.00486562718862055\\
76843.204320432 0.00486590631962583\\
76851.8451845184 0.00486618543494219\\
76860.4860486049 0.00486646453457227\\
76869.1269126913 0.00486674361851871\\
76877.7677767777 0.00486702268678417\\
76886.4086408641 0.00486730173937128\\
76895.0495049505 0.00486758077628268\\
76903.6903690369 0.00486785979752103\\
76912.3312331233 0.00486813880308895\\
76920.9720972097 0.00486841779298908\\
76929.6129612961 0.00486869676722408\\
76938.2538253825 0.00486897572579657\\
76946.8946894689 0.0048692546687092\\
76955.5355535554 0.00486953359596459\\
76964.1764176418 0.00486981250756539\\
76972.8172817282 0.00487009140351423\\
76981.4581458146 0.00487037028381375\\
76990.099009901 0.00487064914846659\\
76998.7398739874 0.00487092799747536\\
77007.3807380738 0.00487120683084271\\
77016.0216021602 0.00487148564857127\\
77024.6624662466 0.00487176445066367\\
77033.303330333 0.00487204323712254\\
77041.9441944194 0.00487232200795051\\
77050.5850585059 0.00487260076315021\\
77059.2259225923 0.00487287950272427\\
77067.8667866787 0.00487315822667531\\
77076.5076507651 0.00487343693500596\\
77085.1485148515 0.00487371562771884\\
77093.7893789379 0.00487399430481659\\
77102.4302430243 0.00487427296630182\\
77111.0711071107 0.00487455161217717\\
77119.7119711971 0.00487483024244525\\
77128.3528352835 0.00487510885710868\\
77136.9936993699 0.00487538745617009\\
77145.6345634563 0.00487566603963209\\
77154.2754275428 0.00487594460749732\\
77162.9162916292 0.00487622315976838\\
77171.5571557156 0.0048765016964479\\
77180.198019802 0.00487678021753849\\
77188.8388838884 0.00487705872304277\\
77197.4797479748 0.00487733721296335\\
77206.1206120612 0.00487761568730286\\
77214.7614761476 0.00487789414606391\\
77223.402340234 0.00487817258924911\\
77232.0432043204 0.00487845101686107\\
77240.6840684068 0.00487872942890242\\
77249.3249324933 0.00487900782537575\\
77257.9657965797 0.00487928620628368\\
77266.6066606661 0.00487956457162883\\
77275.2475247525 0.00487984292141379\\
77283.8883888389 0.00488012125564119\\
77292.5292529253 0.00488039957431362\\
77301.1701170117 0.0048806778774337\\
77309.8109810981 0.00488095616500404\\
77318.4518451845 0.00488123443702723\\
77327.0927092709 0.00488151269350589\\
77335.7335733573 0.00488179093444262\\
77344.3744374437 0.00488206915984002\\
77353.0153015301 0.0048823473697007\\
77361.6561656166 0.00488262556402726\\
77370.297029703 0.0048829037428223\\
77378.9378937894 0.00488318190608843\\
77387.5787578758 0.00488346005382823\\
77396.2196219622 0.00488373818604432\\
77404.8604860486 0.00488401630273929\\
77413.501350135 0.00488429440391574\\
77422.1422142214 0.00488457248957626\\
77430.7830783078 0.00488485055972346\\
77439.4239423942 0.00488512861435993\\
77448.0648064806 0.00488540665348826\\
77456.7056705671 0.00488568467711106\\
77465.3465346535 0.0048859626852309\\
77473.9873987399 0.00488624067785039\\
77482.6282628263 0.00488651865497212\\
77491.2691269127 0.00488679661659868\\
77499.9099909991 0.00488707456273267\\
77508.5508550855 0.00488735249337666\\
77517.1917191719 0.00488763040853326\\
77525.8325832583 0.00488790830820504\\
77534.4734473447 0.0048881861923946\\
77543.1143114311 0.00488846406110453\\
77551.7551755176 0.00488874191433741\\
77560.396039604 0.00488901975209583\\
77569.0369036904 0.00488929757438237\\
77577.6777677768 0.00488957538119962\\
77586.3186318632 0.00488985317255016\\
77594.9594959496 0.00489013094843657\\
77603.600360036 0.00489040870886144\\
77612.2412241224 0.00489068645382735\\
77620.8820882088 0.00489096418333687\\
77629.5229522952 0.0048912418973926\\
77638.1638163816 0.0048915195959971\\
77646.804680468 0.00489179727915296\\
77655.4455445545 0.00489207494686275\\
77664.0864086409 0.00489235259912906\\
77672.7272727273 0.00489263023595446\\
77681.3681368137 0.00489290785734152\\
77690.0090009001 0.00489318546329282\\
77698.6498649865 0.00489346305381094\\
77707.2907290729 0.00489374062889844\\
77715.9315931593 0.00489401818855791\\
77724.5724572457 0.0048942957327919\\
77733.2133213321 0.00489457326160301\\
77741.8541854185 0.00489485077499378\\
77750.495049505 0.00489512827296681\\
77759.1359135914 0.00489540575552464\\
77767.7767776778 0.00489568322266987\\
77776.4176417642 0.00489596067440504\\
77785.0585058506 0.00489623811073273\\
77793.699369937 0.00489651553165551\\
77802.3402340234 0.00489679293717594\\
77810.9810981098 0.00489707032729659\\
77819.6219621962 0.00489734770202001\\
77828.2628262826 0.00489762506134879\\
77836.903690369 0.00489790240528547\\
77845.5445544554 0.00489817973383262\\
77854.1854185418 0.0048984570469928\\
77862.8262826283 0.00489873434476857\\
77871.4671467147 0.0048990116271625\\
77880.1080108011 0.00489928889417713\\
77888.7488748875 0.00489956614581504\\
77897.3897389739 0.00489984338207878\\
77906.0306030603 0.0049001206029709\\
77914.6714671467 0.00490039780849397\\
77923.3123312331 0.00490067499865053\\
77931.9531953195 0.00490095217344315\\
77940.5940594059 0.00490122933287437\\
77949.2349234923 0.00490150647694676\\
77957.8757875788 0.00490178360566287\\
77966.5166516652 0.00490206071902523\\
77975.1575157516 0.00490233781703642\\
77983.798379838 0.00490261489969898\\
77992.4392439244 0.00490289196701546\\
78001.0801080108 0.00490316901898841\\
78009.7209720972 0.00490344605562038\\
78018.3618361836 0.00490372307691391\\
78027.00270027 0.00490400008287156\\
78035.6435643564 0.00490427707349587\\
78044.2844284428 0.00490455404878938\\
78052.9252925293 0.00490483100875465\\
78061.5661566157 0.00490510795339421\\
78070.2070207021 0.00490538488271061\\
78078.8478847885 0.0049056617967064\\
78087.4887488749 0.00490593869538412\\
78096.1296129613 0.0049062155787463\\
78104.7704770477 0.00490649244679549\\
78113.4113411341 0.00490676929953423\\
78122.0522052205 0.00490704613696506\\
78130.6930693069 0.00490732295909051\\
78139.3339333933 0.00490759976591314\\
78147.9747974798 0.00490787655743547\\
78156.6156615662 0.00490815333366003\\
78165.2565256526 0.00490843009458938\\
78173.897389739 0.00490870684022604\\
78182.5382538254 0.00490898357057254\\
78191.1791179118 0.00490926028563143\\
78199.8199819982 0.00490953698540523\\
78208.4608460846 0.00490981366989647\\
78217.101710171 0.0049100903391077\\
78225.7425742574 0.00491036699304144\\
78234.3834383438 0.00491064363170021\\
78243.0243024302 0.00491092025508656\\
78251.6651665167 0.00491119686320301\\
78260.3060306031 0.00491147345605209\\
78268.9468946895 0.00491175003363632\\
78277.5877587759 0.00491202659595824\\
78286.2286228623 0.00491230314302037\\
78294.8694869487 0.00491257967482523\\
78303.5103510351 0.00491285619137535\\
78312.1512151215 0.00491313269267326\\
78320.7920792079 0.00491340917872147\\
78329.4329432943 0.00491368564952252\\
78338.0738073807 0.00491396210507892\\
78346.7146714671 0.00491423854539319\\
78355.3555355535 0.00491451497046786\\
78363.99639964 0.00491479138030545\\
78372.6372637264 0.00491506777490847\\
78381.2781278128 0.00491534415427944\\
78389.9189918992 0.00491562051842089\\
78398.5598559856 0.00491589686733533\\
78407.200720072 0.00491617320102527\\
78415.8415841584 0.00491644951949323\\
78424.4824482448 0.00491672582274173\\
78433.1233123312 0.00491700211077329\\
78441.7641764176 0.00491727838359041\\
78450.405040504 0.0049175546411956\\
78459.0459045905 0.00491783088359139\\
78467.6867686769 0.00491810711078028\\
78476.3276327633 0.00491838332276479\\
78484.9684968497 0.00491865951954742\\
78493.6093609361 0.00491893570113068\\
78502.2502250225 0.00491921186751709\\
78510.8910891089 0.00491948801870915\\
78519.5319531953 0.00491976415470937\\
78528.1728172817 0.00492004027552026\\
78536.8136813681 0.00492031638114432\\
78545.4545454545 0.00492059247158405\\
78554.095409541 0.00492086854684197\\
78562.7362736274 0.00492114460692058\\
78571.3771377138 0.00492142065182238\\
78580.0180018002 0.00492169668154987\\
78588.6588658866 0.00492197269610555\\
78597.299729973 0.00492224869549194\\
78605.9405940594 0.00492252467971152\\
78614.5814581458 0.0049228006487668\\
78623.2223222322 0.00492307660266028\\
78631.8631863186 0.00492335254139445\\
78640.504050405 0.00492362846497182\\
78649.1449144915 0.00492390437339488\\
78657.7857785779 0.00492418026666613\\
78666.4266426643 0.00492445614478806\\
78675.0675067507 0.00492473200776318\\
78683.7083708371 0.00492500785559396\\
78692.3492349235 0.00492528368828292\\
78700.9900990099 0.00492555950583253\\
78709.6309630963 0.0049258353082453\\
78718.2718271827 0.00492611109552372\\
78726.9126912691 0.00492638686767027\\
78735.5535553555 0.00492666262468745\\
78744.1944194419 0.00492693836657774\\
78752.8352835284 0.00492721409334364\\
78761.4761476148 0.00492748980498764\\
78770.1170117012 0.00492776550151222\\
78778.7578757876 0.00492804118291987\\
78787.398739874 0.00492831684921307\\
78796.0396039604 0.00492859250039432\\
78804.6804680468 0.00492886813646609\\
78813.3213321332 0.00492914375743088\\
78821.9621962196 0.00492941936329116\\
78830.603060306 0.00492969495404941\\
78839.2439243924 0.00492997052970813\\
78847.8847884788 0.00493024609026979\\
78856.5256525653 0.00493052163573687\\
78865.1665166517 0.00493079716611185\\
78873.8073807381 0.00493107268139722\\
78882.4482448245 0.00493134818159545\\
78891.0891089109 0.00493162366670901\\
78899.7299729973 0.0049318991367404\\
78908.3708370837 0.00493217459169208\\
78917.0117011701 0.00493245003156653\\
78925.6525652565 0.00493272545636622\\
78934.2934293429 0.00493300086609364\\
78942.9342934293 0.00493327626075125\\
78951.5751575157 0.00493355164034152\\
78960.2160216022 0.00493382700486694\\
78968.8568856886 0.00493410235432998\\
78977.497749775 0.00493437768873309\\
78986.1386138614 0.00493465300807877\\
78994.7794779478 0.00493492831236947\\
79003.4203420342 0.00493520360160766\\
79012.0612061206 0.00493547887579582\\
79020.702070207 0.00493575413493641\\
79029.3429342934 0.00493602937903189\\
79037.9837983798 0.00493630460808475\\
79046.6246624662 0.00493657982209744\\
79055.2655265527 0.00493685502107242\\
79063.9063906391 0.00493713020501217\\
79072.5472547255 0.00493740537391914\\
79081.1881188119 0.00493768052779581\\
79089.8289828983 0.00493795566664462\\
79098.4698469847 0.00493823079046805\\
79107.1107110711 0.00493850589926856\\
79115.7515751575 0.0049387809930486\\
79124.3924392439 0.00493905607181064\\
79133.0333033303 0.00493933113555713\\
79141.6741674167 0.00493960618429054\\
79150.3150315032 0.00493988121801332\\
79158.9558955896 0.00494015623672793\\
79167.596759676 0.00494043124043683\\
79176.2376237624 0.00494070622914247\\
79184.8784878488 0.00494098120284731\\
79193.5193519352 0.0049412561615538\\
79202.1602160216 0.00494153110526439\\
79210.801080108 0.00494180603398155\\
79219.4419441944 0.00494208094770772\\
79228.0828082808 0.00494235584644535\\
79236.7236723672 0.0049426307301969\\
79245.3645364536 0.00494290559896481\\
79254.0054005401 0.00494318045275154\\
79262.6462646265 0.00494345529155954\\
79271.2871287129 0.00494373011539125\\
79279.9279927993 0.00494400492424912\\
79288.5688568857 0.00494427971813561\\
79297.2097209721 0.00494455449705314\\
79305.8505850585 0.00494482926100418\\
79314.4914491449 0.00494510400999117\\
79323.1323132313 0.00494537874401655\\
79331.7731773177 0.00494565346308277\\
79340.4140414041 0.00494592816719226\\
79349.0549054905 0.00494620285634748\\
79357.695769577 0.00494647753055086\\
79366.3366336634 0.00494675218980484\\
79374.9774977498 0.00494702683411187\\
79383.6183618362 0.00494730146347439\\
79392.2592259226 0.00494757607789483\\
79400.900090009 0.00494785067737564\\
79409.5409540954 0.00494812526191925\\
79418.1818181818 0.00494839983152809\\
79426.8226822682 0.00494867438620462\\
79435.4635463546 0.00494894892595125\\
79444.104410441 0.00494922345077043\\
79452.7452745274 0.0049494979606646\\
79461.3861386139 0.00494977245563618\\
79470.0270027003 0.0049500469356876\\
79478.6678667867 0.00495032140082131\\
79487.3087308731 0.00495059585103974\\
79495.9495949595 0.00495087028634531\\
79504.5904590459 0.00495114470674045\\
79513.2313231323 0.0049514191122276\\
79521.8721872187 0.00495169350280919\\
79530.5130513051 0.00495196787848764\\
79539.1539153915 0.00495224223926538\\
79547.7947794779 0.00495251658514485\\
79556.4356435644 0.00495279091612846\\
79565.0765076508 0.00495306523221864\\
79573.7173717372 0.00495333953341782\\
79582.3582358236 0.00495361381972842\\
79590.99909991 0.00495388809115287\\
79599.6399639964 0.00495416234769359\\
79608.2808280828 0.004954436589353\\
79616.9216921692 0.00495471081613352\\
79625.5625562556 0.00495498502803759\\
79634.203420342 0.00495525922506761\\
79642.8442844284 0.00495553340722601\\
79651.4851485149 0.00495580757451521\\
79660.1260126013 0.00495608172693762\\
79668.7668766877 0.00495635586449567\\
79677.4077407741 0.00495662998719176\\
79686.0486048605 0.00495690409502833\\
79694.6894689469 0.00495717818800778\\
79703.3303330333 0.00495745226613254\\
79711.9711971197 0.00495772632940501\\
79720.6120612061 0.0049580003778276\\
79729.2529252925 0.00495827441140275\\
79737.8937893789 0.00495854843013284\\
79746.5346534654 0.00495882243402031\\
79755.1755175518 0.00495909642306756\\
79763.8163816382 0.004959370397277\\
79772.4572457246 0.00495964435665104\\
79781.098109811 0.00495991830119209\\
79789.7389738974 0.00496019223090256\\
79798.3798379838 0.00496046614578486\\
79807.0207020702 0.0049607400458414\\
79815.6615661566 0.00496101393107458\\
79824.302430243 0.00496128780148682\\
79832.9432943294 0.00496156165708051\\
79841.5841584158 0.00496183549785805\\
79850.2250225022 0.00496210932382187\\
79858.8658865887 0.00496238313497435\\
79867.5067506751 0.00496265693131791\\
79876.1476147615 0.00496293071285495\\
79884.7884788479 0.00496320447958786\\
79893.4293429343 0.00496347823151904\\
79902.0702070207 0.00496375196865091\\
79910.7110711071 0.00496402569098586\\
79919.3519351935 0.00496429939852628\\
79927.9927992799 0.00496457309127459\\
79936.6336633663 0.00496484676923316\\
79945.2745274527 0.00496512043240441\\
79953.9153915392 0.00496539408079073\\
79962.5562556256 0.00496566771439451\\
79971.197119712 0.00496594133321816\\
79979.8379837984 0.00496621493726405\\
79988.4788478848 0.0049664885265346\\
79997.1197119712 0.00496676210103219\\
80005.7605760576 0.00496703566075922\\
80014.401440144 0.00496730920571808\\
80023.0423042304 0.00496758273591115\\
80031.6831683168 0.00496785625134083\\
80040.3240324032 0.00496812975200952\\
80048.9648964896 0.00496840323791959\\
80057.6057605761 0.00496867670907345\\
80066.2466246625 0.00496895016547347\\
80074.8874887489 0.00496922360712204\\
80083.5283528353 0.00496949703402156\\
80092.1692169217 0.00496977044617441\\
80100.8100810081 0.00497004384358297\\
80109.4509450945 0.00497031722624963\\
80118.0918091809 0.00497059059417678\\
80126.7326732673 0.00497086394736679\\
80135.3735373537 0.00497113728582205\\
80144.0144014401 0.00497141060954495\\
80152.6552655266 0.00497168391853786\\
80161.296129613 0.00497195721280317\\
80169.9369936994 0.00497223049234325\\
80178.5778577858 0.00497250375716049\\
80187.2187218722 0.00497277700725727\\
80195.8595859586 0.00497305024263596\\
80204.500450045 0.00497332346329895\\
80213.1413141314 0.00497359666924861\\
80221.7821782178 0.00497386986048731\\
80230.4230423042 0.00497414303701744\\
80239.0639063906 0.00497441619884136\\
80247.7047704771 0.00497468934596146\\
80256.3456345635 0.00497496247838011\\
80264.9864986499 0.00497523559609967\\
80273.6273627363 0.00497550869912254\\
80282.2682268227 0.00497578178745106\\
80290.9090909091 0.00497605486108763\\
80299.5499549955 0.00497632792003461\\
80308.1908190819 0.00497660096429436\\
80316.8316831683 0.00497687399386927\\
80325.4725472547 0.00497714700876169\\
80334.1134113411 0.004977420008974\\
80342.7542754275 0.00497769299450857\\
80351.3951395139 0.00497796596536775\\
80360.0360036004 0.00497823892155393\\
80368.6768676868 0.00497851186306946\\
80377.3177317732 0.00497878478991671\\
80385.9585958596 0.00497905770209804\\
80394.599459946 0.00497933059961582\\
80403.2403240324 0.00497960348247241\\
80411.8811881188 0.00497987635067018\\
80420.5220522052 0.00498014920421149\\
80429.1629162916 0.00498042204309869\\
80437.803780378 0.00498069486733415\\
80446.4446444644 0.00498096767692023\\
80455.0855085509 0.00498124047185928\\
80463.7263726373 0.00498151325215368\\
80472.3672367237 0.00498178601780577\\
80481.0081008101 0.00498205876881791\\
80489.6489648965 0.00498233150519247\\
80498.2898289829 0.00498260422693179\\
80506.9306930693 0.00498287693403823\\
80515.5715571557 0.00498314962651415\\
80524.2124212421 0.00498342230436191\\
80532.8532853285 0.00498369496758385\\
80541.4941494149 0.00498396761618233\\
80550.1350135013 0.0049842402501597\\
80558.7758775878 0.00498451286951831\\
80567.4167416742 0.00498478547426052\\
80576.0576057606 0.00498505806438868\\
80584.698469847 0.00498533063990513\\
80593.3393339334 0.00498560320081223\\
80601.9801980198 0.00498587574711232\\
80610.6210621062 0.00498614827880776\\
80619.2619261926 0.00498642079590088\\
80627.902790279 0.00498669329839404\\
80636.5436543654 0.00498696578628959\\
80645.1845184518 0.00498723825958986\\
80653.8253825383 0.00498751071829721\\
80662.4662466247 0.00498778316241398\\
80671.1071107111 0.00498805559194251\\
80679.7479747975 0.00498832800688515\\
80688.3888388839 0.00498860040724423\\
80697.0297029703 0.0049888727930221\\
80705.6705670567 0.00498914516422111\\
80714.3114311431 0.00498941752084359\\
80722.9522952295 0.00498968986289188\\
80731.5931593159 0.00498996219036832\\
80740.2340234023 0.00499023450327525\\
80748.8748874888 0.00499050680161501\\
80757.5157515752 0.00499077908538993\\
80766.1566156616 0.00499105135460236\\
80774.797479748 0.00499132360925463\\
80783.4383438344 0.00499159584934907\\
80792.0792079208 0.00499186807488802\\
80800.7200720072 0.00499214028587382\\
80809.3609360936 0.00499241248230879\\
80818.00180018 0.00499268466419528\\
80826.6426642664 0.0049929568315356\\
80835.2835283528 0.00499322898433211\\
80843.9243924393 0.00499350112258712\\
80852.5652565256 0.00499377324630297\\
80861.2061206121 0.00499404535548199\\
80869.8469846985 0.0049943174501265\\
80878.4878487849 0.00499458953023884\\
80887.1287128713 0.00499486159582134\\
80895.7695769577 0.00499513364687631\\
80904.4104410441 0.0049954056834061\\
80913.0513051305 0.00499567770541302\\
80921.6921692169 0.0049959497128994\\
80930.3330333033 0.00499622170586756\\
80938.9738973897 0.00499649368431984\\
80947.6147614761 0.00499676564825855\\
80956.2556255626 0.00499703759768602\\
80964.896489649 0.00499730953260457\\
80973.5373537354 0.00499758145301651\\
80982.1782178218 0.00499785335892419\\
80990.8190819082 0.00499812525032991\\
80999.4599459946 0.00499839712723599\\
81008.100810081 0.00499866898964475\\
81016.7416741674 0.00499894083755852\\
81025.3825382538 0.00499921267097961\\
81034.0234023402 0.00499948448991033\\
81042.6642664266 0.00499975629435302\\
81051.3051305131 0.00500002808430997\\
81059.9459945995 0.00500029985978351\\
81068.5868586859 0.00500057162077595\\
81077.2277227723 0.00500084336728962\\
81085.8685868587 0.00500111509932681\\
81094.5094509451 0.00500138681688985\\
81103.1503150315 0.00500165851998104\\
81111.7911791179 0.00500193020860271\\
81120.4320432043 0.00500220188275715\\
81129.0729072907 0.00500247354244669\\
81137.7137713771 0.00500274518767363\\
81146.3546354635 0.00500301681844028\\
81154.99549955 0.00500328843474896\\
81163.6363636364 0.00500356003660196\\
81172.2772277228 0.0050038316240016\\
81180.9180918092 0.00500410319695018\\
81189.5589558956 0.00500437475545001\\
81198.199819982 0.00500464629950341\\
81206.8406840684 0.00500491782911266\\
81215.4815481548 0.00500518934428008\\
81224.1224122412 0.00500546084500797\\
81232.7632763276 0.00500573233129863\\
81241.404140414 0.00500600380315437\\
81250.0450045005 0.0050062752605775\\
81258.6858685869 0.0050065467035703\\
81267.3267326733 0.00500681813213508\\
81275.9675967597 0.00500708954627415\\
81284.6084608461 0.0050073609459898\\
81293.2493249325 0.00500763233128434\\
81301.8901890189 0.00500790370216005\\
81310.5310531053 0.00500817505861924\\
81319.1719171917 0.00500844640066421\\
81327.8127812781 0.00500871772829725\\
81336.4536453645 0.00500898904152066\\
81345.094509451 0.00500926034033673\\
81353.7353735373 0.00500953162474776\\
81362.3762376238 0.00500980289475605\\
81371.0171017102 0.00501007415036388\\
81379.6579657966 0.00501034539157356\\
81388.298829883 0.00501061661838736\\
81396.9396939694 0.0050108878308076\\
81405.5805580558 0.00501115902883654\\
81414.2214221422 0.0050114302124765\\
81422.8622862286 0.00501170138172975\\
81431.503150315 0.00501197253659859\\
81440.1440144014 0.0050122436770853\\
81448.7848784878 0.00501251480319218\\
81457.4257425743 0.00501278591492151\\
81466.0666066607 0.00501305701227558\\
81474.7074707471 0.00501332809525667\\
81483.3483348335 0.00501359916386707\\
81491.9891989199 0.00501387021810907\\
81500.6300630063 0.00501414125798495\\
81509.2709270927 0.00501441228349699\\
81517.9117911791 0.00501468329464748\\
81526.5526552655 0.0050149542914387\\
81535.1935193519 0.00501522527387294\\
81543.8343834383 0.00501549624195247\\
81552.4752475248 0.00501576719567957\\
81561.1161116112 0.00501603813505653\\
81569.7569756976 0.00501630906008563\\
81578.397839784 0.00501657997076914\\
81587.0387038704 0.00501685086710935\\
81595.6795679568 0.00501712174910852\\
81604.3204320432 0.00501739261676895\\
81612.9612961296 0.0050176634700929\\
81621.602160216 0.00501793430908265\\
81630.2430243024 0.00501820513374048\\
81638.8838883888 0.00501847594406866\\
81647.5247524752 0.00501874674006947\\
81656.1656165617 0.00501901752174518\\
81664.8064806481 0.00501928828909807\\
81673.4473447345 0.00501955904213039\\
81682.0882088209 0.00501982978084444\\
81690.7290729073 0.00502010050524248\\
81699.3699369937 0.00502037121532678\\
81708.0108010801 0.0050206419110996\\
81716.6516651665 0.00502091259256323\\
81725.2925292529 0.00502118325971993\\
81733.9333933393 0.00502145391257196\\
81742.5742574257 0.0050217245511216\\
81751.2151215122 0.00502199517537111\\
81759.8559855986 0.00502226578532276\\
81768.496849685 0.00502253638097881\\
81777.1377137714 0.00502280696234154\\
81785.7785778578 0.0050230775294132\\
81794.4194419442 0.00502334808219606\\
81803.0603060306 0.00502361862069238\\
81811.701170117 0.00502388914490442\\
81820.3420342034 0.00502415965483446\\
81828.9828982898 0.00502443015048474\\
81837.6237623762 0.00502470063185754\\
81846.2646264627 0.00502497109895511\\
81854.905490549 0.0050252415517797\\
81863.5463546355 0.00502551199033359\\
81872.1872187219 0.00502578241461903\\
81880.8280828083 0.00502605282463828\\
81889.4689468947 0.00502632322039359\\
81898.1098109811 0.00502659360188723\\
81906.7506750675 0.00502686396912144\\
81915.3915391539 0.00502713432209849\\
81924.0324032403 0.00502740466082063\\
81932.6732673267 0.00502767498529011\\
81941.3141314131 0.00502794529550919\\
81949.9549954995 0.00502821559148013\\
81958.595859586 0.00502848587320517\\
81967.2367236724 0.00502875614068656\\
81975.8775877588 0.00502902639392657\\
81984.5184518452 0.00502929663292743\\
81993.1593159316 0.00502956685769141\\
82001.800180018 0.00502983706822075\\
82010.4410441044 0.00503010726451769\\
82019.0819081908 0.00503037744658449\\
82027.7227722772 0.0050306476144234\\
82036.3636363636 0.00503091776803667\\
82045.00450045 0.00503118790742653\\
82053.6453645365 0.00503145803259525\\
82062.2862286229 0.00503172814354505\\
82070.9270927093 0.00503199824027819\\
82079.5679567957 0.00503226832279692\\
82088.2088208821 0.00503253839110347\\
82096.8496849685 0.00503280844520009\\
82105.4905490549 0.00503307848508903\\
82114.1314131413 0.00503334851077252\\
82122.7722772277 0.00503361852225281\\
82131.4131413141 0.00503388851953213\\
82140.0540054005 0.00503415850261273\\
82148.6948694869 0.00503442847149685\\
82157.3357335734 0.00503469842618673\\
82165.9765976598 0.0050349683666846\\
82174.6174617462 0.00503523829299271\\
82183.2583258326 0.00503550820511329\\
82191.899189919 0.00503577810304857\\
82200.5400540054 0.0050360479868008\\
82209.1809180918 0.00503631785637221\\
82217.8217821782 0.00503658771176504\\
82226.4626462646 0.00503685755298152\\
82235.103510351 0.00503712738002388\\
82243.7443744374 0.00503739719289435\\
82252.3852385239 0.00503766699159518\\
82261.0261026103 0.00503793677612859\\
82269.6669666967 0.00503820654649682\\
82278.3078307831 0.00503847630270208\\
82286.9486948695 0.00503874604474663\\
82295.5895589559 0.00503901577263268\\
82304.2304230423 0.00503928548636246\\
82312.8712871287 0.00503955518593821\\
82321.5121512151 0.00503982487136215\\
82330.1530153015 0.0050400945426365\\
82338.7938793879 0.00504036419976351\\
82347.4347434744 0.00504063384274538\\
82356.0756075607 0.00504090347158436\\
82364.7164716472 0.00504117308628265\\
82373.3573357336 0.0050414426868425\\
82381.99819982 0.00504171227326611\\
82390.6390639064 0.00504198184555572\\
82399.2799279928 0.00504225140371355\\
82407.9207920792 0.00504252094774182\\
82416.5616561656 0.00504279047764275\\
82425.202520252 0.00504305999341856\\
82433.8433843384 0.00504332949507148\\
82442.4842484248 0.00504359898260372\\
82451.1251125112 0.00504386845601751\\
82459.7659765977 0.00504413791531505\\
82468.4068406841 0.00504440736049858\\
82477.0477047705 0.0050446767915703\\
82485.6885688569 0.00504494620853244\\
82494.3294329433 0.00504521561138721\\
82502.9702970297 0.00504548500013682\\
82511.6111611161 0.0050457543747835\\
82520.2520252025 0.00504602373532945\\
82528.8928892889 0.0050462930817769\\
82537.5337533753 0.00504656241412805\\
82546.1746174617 0.00504683173238512\\
82554.8154815482 0.00504710103655031\\
82563.4563456346 0.00504737032662585\\
82572.097209721 0.00504763960261394\\
82580.7380738074 0.0050479088645168\\
82589.3789378938 0.00504817811233663\\
82598.0198019802 0.00504844734607564\\
82606.6606660666 0.00504871656573605\\
82615.301530153 0.00504898577132005\\
82623.9423942394 0.00504925496282987\\
82632.5832583258 0.0050495241402677\\
82641.2241224122 0.00504979330363575\\
82649.8649864987 0.00505006245293623\\
82658.5058505851 0.00505033158817134\\
82667.1467146715 0.00505060070934329\\
82675.7875787579 0.00505086981645429\\
82684.4284428443 0.00505113890950653\\
82693.0693069307 0.00505140798850222\\
82701.7101710171 0.00505167705344356\\
82710.3510351035 0.00505194610433276\\
82718.9918991899 0.00505221514117202\\
82727.6327632763 0.00505248416396353\\
82736.2736273627 0.00505275317270949\\
82744.9144914491 0.00505302216741212\\
82753.5553555356 0.0050532911480736\\
82762.196219622 0.00505356011469613\\
82770.8370837084 0.00505382906728192\\
82779.4779477948 0.00505409800583316\\
82788.1188118812 0.00505436693035204\\
82796.7596759676 0.00505463584084077\\
82805.400540054 0.00505490473730153\\
82814.0414041404 0.00505517361973653\\
82822.6822682268 0.00505544248814796\\
82831.3231323132 0.005055711342538\\
82839.9639963996 0.00505598018290887\\
82848.6048604861 0.00505624900926274\\
82857.2457245725 0.00505651782160182\\
82865.8865886589 0.00505678661992829\\
82874.5274527453 0.00505705540424434\\
82883.1683168317 0.00505732417455216\\
82891.8091809181 0.00505759293085395\\
82900.4500450045 0.00505786167315189\\
82909.0909090909 0.00505813040144818\\
82917.7317731773 0.00505839911574499\\
82926.3726372637 0.00505866781604453\\
82935.0135013501 0.00505893650234897\\
82943.6543654365 0.0050592051746605\\
82952.2952295229 0.00505947383298131\\
82960.9360936094 0.00505974247731358\\
82969.5769576958 0.00506001110765951\\
82978.2178217822 0.00506027972402126\\
82986.8586858686 0.00506054832640103\\
82995.499549955 0.005060816914801\\
83004.1404140414 0.00506108548922336\\
83012.7812781278 0.00506135404967027\\
83021.4221422142 0.00506162259614394\\
83030.0630063006 0.00506189112864653\\
83038.703870387 0.00506215964718022\\
83047.3447344734 0.00506242815174721\\
83055.9855985599 0.00506269664234965\\
83064.6264626463 0.00506296511898975\\
83073.2673267327 0.00506323358166966\\
83081.9081908191 0.00506350203039158\\
83090.5490549055 0.00506377046515767\\
83099.1899189919 0.00506403888597012\\
83107.8307830783 0.00506430729283109\\
83116.4716471647 0.00506457568574277\\
83125.1125112511 0.00506484406470733\\
83133.7533753375 0.00506511242972694\\
83142.3942394239 0.00506538078080377\\
83151.0351035104 0.00506564911794001\\
83159.6759675968 0.00506591744113782\\
83168.3168316832 0.00506618575039937\\
83176.9576957696 0.00506645404572684\\
83185.598559856 0.00506672232712239\\
83194.2394239424 0.0050669905945882\\
83202.8802880288 0.00506725884812643\\
83211.5211521152 0.00506752708773926\\
83220.1620162016 0.00506779531342885\\
83228.802880288 0.00506806352519737\\
83237.4437443744 0.00506833172304699\\
83246.0846084608 0.00506859990697987\\
83254.7254725473 0.00506886807699819\\
83263.3663366337 0.0050691362331041\\
83272.0072007201 0.00506940437529977\\
83280.6480648065 0.00506967250358737\\
83289.2889288929 0.00506994061796906\\
83297.9297929793 0.005070208718447\\
83306.5706570657 0.00507047680502336\\
83315.2115211521 0.0050707448777003\\
83323.8523852385 0.00507101293647998\\
83332.4932493249 0.00507128098136456\\
83341.1341134113 0.0050715490123562\\
83349.7749774978 0.00507181702945706\\
83358.4158415842 0.00507208503266931\\
83367.0567056706 0.0050723530219951\\
83375.697569757 0.00507262099743658\\
83384.3384338434 0.00507288895899592\\
83392.9792979298 0.00507315690667528\\
83401.6201620162 0.00507342484047681\\
83410.2610261026 0.00507369276040266\\
83418.901890189 0.005073960666455\\
83427.5427542754 0.00507422855863597\\
83436.1836183618 0.00507449643694773\\
83444.8244824482 0.00507476430139244\\
83453.4653465346 0.00507503215197225\\
83462.1062106211 0.00507529998868931\\
83470.7470747075 0.00507556781154577\\
83479.3879387939 0.00507583562054379\\
83488.0288028803 0.00507610341568552\\
83496.6696669667 0.00507637119697311\\
83505.3105310531 0.0050766389644087\\
83513.9513951395 0.00507690671799444\\
83522.5922592259 0.0050771744577325\\
83531.2331233123 0.005077442183625\\
83539.8739873987 0.00507770989567411\\
83548.5148514851 0.00507797759388197\\
83557.1557155716 0.00507824527825072\\
83565.796579658 0.00507851294878252\\
83574.4374437444 0.0050787806054795\\
83583.0783078308 0.00507904824834381\\
83591.7191719172 0.0050793158773776\\
83600.3600360036 0.00507958349258301\\
83609.00090009 0.00507985109396219\\
83617.6417641764 0.00508011868151727\\
83626.2826282628 0.0050803862552504\\
83634.9234923492 0.00508065381516372\\
83643.5643564356 0.00508092136125937\\
83652.2052205221 0.00508118889353949\\
83660.8460846085 0.00508145641200623\\
83669.4869486949 0.00508172391666172\\
83678.1278127813 0.00508199140750809\\
83686.7686768677 0.0050822588845475\\
83695.4095409541 0.00508252634778207\\
83704.0504050405 0.00508279379721395\\
83712.6912691269 0.00508306123284527\\
83721.3321332133 0.00508332865467816\\
83729.9729972997 0.00508359606271477\\
83738.6138613861 0.00508386345695723\\
83747.2547254726 0.00508413083740766\\
83755.895589559 0.00508439820406821\\
83764.5364536454 0.00508466555694101\\
83773.1773177318 0.0050849328960282\\
83781.8181818182 0.0050852002213319\\
83790.4590459046 0.00508546753285424\\
83799.099909991 0.00508573483059736\\
83807.7407740774 0.00508600211456339\\
83816.3816381638 0.00508626938475446\\
83825.0225022502 0.00508653664117269\\
83833.6633663366 0.00508680388382022\\
83842.304230423 0.00508707111269918\\
83850.9450945095 0.00508733832781169\\
83859.5859585959 0.00508760552915988\\
83868.2268226823 0.00508787271674588\\
83876.8676867687 0.00508813989057181\\
83885.5085508551 0.0050884070506398\\
83894.1494149415 0.00508867419695197\\
83902.7902790279 0.00508894132951045\\
83911.4311431143 0.00508920844831737\\
83920.0720072007 0.00508947555337484\\
83928.7128712871 0.00508974264468499\\
83937.3537353735 0.00509000972224995\\
83945.9945994599 0.00509027678607183\\
83954.6354635464 0.00509054383615275\\
83963.2763276328 0.00509081087249484\\
83971.9171917192 0.00509107789510021\\
83980.5580558056 0.00509134490397099\\
83989.198919892 0.0050916118991093\\
83997.8397839784 0.00509187888051725\\
84006.4806480648 0.00509214584819696\\
84015.1215121512 0.00509241280215055\\
84023.7623762376 0.00509267974238014\\
84032.403240324 0.00509294666888783\\
84041.0441044104 0.00509321358167576\\
84049.6849684968 0.00509348048074604\\
84058.3258325833 0.00509374736610077\\
84066.9666966697 0.00509401423774207\\
84075.6075607561 0.00509428109567206\\
84084.2484248425 0.00509454793989285\\
84092.8892889289 0.00509481477040655\\
84101.5301530153 0.00509508158721528\\
84110.1710171017 0.00509534839032114\\
84118.8118811881 0.00509561517972626\\
84127.4527452745 0.00509588195543272\\
84136.0936093609 0.00509614871744266\\
84144.7344734473 0.00509641546575817\\
84153.3753375338 0.00509668220038137\\
84162.0162016202 0.00509694892131437\\
84170.6570657066 0.00509721562855926\\
84179.297929793 0.00509748232211817\\
84187.9387938794 0.00509774900199319\\
84196.5796579658 0.00509801566818643\\
84205.2205220522 0.0050982823207\\
84213.8613861386 0.005098548959536\\
84222.502250225 0.00509881558469654\\
84231.1431143114 0.00509908219618372\\
84239.7839783978 0.00509934879399964\\
84248.4248424843 0.00509961537814642\\
84257.0657065707 0.00509988194862614\\
84265.7065706571 0.00510014850544091\\
84274.3474347435 0.00510041504859284\\
84282.9882988299 0.00510068157808402\\
84291.6291629163 0.00510094809391655\\
84300.2700270027 0.00510121459609254\\
84308.9108910891 0.00510148108461408\\
84317.5517551755 0.00510174755948327\\
84326.1926192619 0.00510201402070221\\
84334.8334833483 0.00510228046827299\\
84343.4743474347 0.00510254690219772\\
84352.1152115212 0.00510281332247849\\
84360.7560756076 0.0051030797291174\\
84369.396939694 0.00510334612211654\\
84378.0378037804 0.005103612501478\\
84386.6786678668 0.00510387886720389\\
84395.3195319532 0.00510414521929629\\
84403.9603960396 0.0051044115577573\\
84412.601260126 0.005104677882589\\
84421.2421242124 0.0051049441937935\\
84429.8829882988 0.00510521049137289\\
84438.5238523852 0.00510547677532925\\
84447.1647164716 0.00510574304566467\\
84455.8055805581 0.00510600930238125\\
84464.4464446445 0.00510627554548108\\
84473.0873087309 0.00510654177496624\\
84481.7281728173 0.00510680799083882\\
84490.3690369037 0.00510707419310091\\
84499.0099009901 0.0051073403817546\\
84507.6507650765 0.00510760655680198\\
84516.2916291629 0.00510787271824513\\
84524.9324932493 0.00510813886608613\\
84533.5733573357 0.00510840500032707\\
84542.2142214221 0.00510867112097004\\
84550.8550855085 0.00510893722801713\\
84559.495949595 0.0051092033214704\\
84568.1368136814 0.00510946940133195\\
84576.7776777678 0.00510973546760387\\
84585.4185418542 0.00511000152028822\\
84594.0594059406 0.0051102675593871\\
84602.700270027 0.00511053358490259\\
84611.3411341134 0.00511079959683675\\
84619.9819981998 0.00511106559519169\\
84628.6228622862 0.00511133157996946\\
84637.2637263726 0.00511159755117216\\
84645.904590459 0.00511186350880186\\
84654.5454545455 0.00511212945286064\\
84663.1863186319 0.00511239538335058\\
84671.8271827183 0.00511266130027375\\
84680.4680468047 0.00511292720363222\\
84689.1089108911 0.00511319309342809\\
84697.7497749775 0.00511345896966341\\
84706.3906390639 0.00511372483234027\\
84715.0315031503 0.00511399068146073\\
84723.6723672367 0.00511425651702688\\
84732.3132313231 0.00511452233904078\\
84740.9540954095 0.00511478814750451\\
84749.594959496 0.00511505394242014\\
84758.2358235824 0.00511531972378974\\
84766.8766876688 0.00511558549161539\\
84775.5175517552 0.00511585124589914\\
84784.1584158416 0.00511611698664308\\
84792.799279928 0.00511638271384927\\
84801.4401440144 0.00511664842751977\\
84810.0810081008 0.00511691412765667\\
84818.7218721872 0.00511717981426203\\
84827.3627362736 0.0051174454873379\\
84836.00360036 0.00511771114688637\\
84844.6444644465 0.00511797679290949\\
84853.2853285329 0.00511824242540934\\
84861.9261926193 0.00511850804438797\\
84870.5670567057 0.00511877364984745\\
84879.2079207921 0.00511903924178985\\
84887.8487848785 0.00511930482021723\\
84896.4896489649 0.00511957038513165\\
84905.1305130513 0.00511983593653517\\
84913.7713771377 0.00512010147442986\\
84922.4122412241 0.00512036699881778\\
84931.0531053105 0.00512063250970099\\
84939.6939693969 0.00512089800708155\\
84948.3348334833 0.00512116349096151\\
84956.9756975698 0.00512142896134295\\
84965.6165616562 0.00512169441822791\\
84974.2574257426 0.00512195986161846\\
84982.898289829 0.00512222529151665\\
84991.5391539154 0.00512249070792454\\
85000.1800180018 0.00512275611084419\\
85008.8208820882 0.00512302150027765\\
85017.4617461746 0.00512328687622698\\
85026.102610261 0.00512355223869424\\
85034.7434743474 0.00512381758768147\\
85043.3843384338 0.00512408292319074\\
85052.0252025202 0.00512434824522409\\
85060.6660666067 0.00512461355378359\\
85069.3069306931 0.00512487884887127\\
85077.9477947795 0.00512514413048921\\
85086.5886588659 0.00512540939863943\\
85095.2295229523 0.005125674653324\\
85103.8703870387 0.00512593989454497\\
85112.5112511251 0.00512620512230439\\
85121.1521152115 0.00512647033660431\\
85129.7929792979 0.00512673553744677\\
85138.4338433843 0.00512700072483382\\
85147.0747074707 0.00512726589876751\\
85155.7155715572 0.0051275310592499\\
85164.3564356436 0.00512779620628302\\
85172.99729973 0.00512806133986892\\
85181.6381638164 0.00512832646000965\\
85190.2790279028 0.00512859156670725\\
85198.9198919892 0.00512885665996377\\
85207.5607560756 0.00512912173978125\\
85216.201620162 0.00512938680616174\\
85224.8424842484 0.00512965185910728\\
85233.4833483348 0.00512991689861991\\
85242.1242124212 0.00513018192470167\\
85250.7650765077 0.00513044693735461\\
85259.4059405941 0.00513071193658077\\
85268.0468046805 0.00513097692238218\\
85276.6876687669 0.00513124189476089\\
85285.3285328533 0.00513150685371894\\
85293.9693969397 0.00513177179925836\\
85302.6102610261 0.0051320367313812\\
85311.2511251125 0.00513230165008949\\
85319.8919891989 0.00513256655538527\\
85328.5328532853 0.00513283144727058\\
85337.1737173717 0.00513309632574745\\
85345.8145814582 0.00513336119081792\\
85354.4554455446 0.00513362604248403\\
85363.096309631 0.00513389088074781\\
85371.7371737174 0.00513415570561129\\
85380.3780378038 0.00513442051707651\\
85389.0189018902 0.0051346853151455\\
85397.6597659766 0.00513495009982029\\
85406.300630063 0.00513521487110292\\
85414.9414941494 0.00513547962899542\\
85423.5823582358 0.00513574437349982\\
85432.2232223222 0.00513600910461815\\
85440.8640864086 0.00513627382235244\\
85449.504950495 0.00513653852670473\\
85458.1458145815 0.00513680321767703\\
85466.7866786679 0.00513706789527138\\
85475.4275427543 0.00513733255948981\\
85484.0684068407 0.00513759721033435\\
85492.7092709271 0.00513786184780701\\
85501.3501350135 0.00513812647190984\\
85509.9909990999 0.00513839108264485\\
85518.6318631863 0.00513865568001407\\
85527.2727272727 0.00513892026401953\\
85535.9135913591 0.00513918483466325\\
85544.5544554455 0.00513944939194725\\
85553.195319532 0.00513971393587356\\
85561.8361836184 0.0051399784664442\\
85570.4770477048 0.0051402429836612\\
85579.1179117912 0.00514050748752657\\
85587.7587758776 0.00514077197804235\\
85596.399639964 0.00514103645521054\\
85605.0405040504 0.00514130091903317\\
85613.6813681368 0.00514156536951226\\
85622.3222322232 0.00514182980664983\\
85630.9630963096 0.00514209423044789\\
85639.603960396 0.00514235864090848\\
85648.2448244824 0.0051426230380336\\
85656.8856885689 0.00514288742182527\\
85665.5265526553 0.00514315179228551\\
85674.1674167417 0.00514341614941634\\
85682.8082808281 0.00514368049321977\\
85691.4491449145 0.00514394482369782\\
85700.0900090009 0.0051442091408525\\
85708.7308730873 0.00514447344468582\\
85717.3717371737 0.00514473773519981\\
85726.0126012601 0.00514500201239647\\
85734.6534653465 0.00514526627627782\\
85743.2943294329 0.00514553052684587\\
85751.9351935194 0.00514579476410263\\
85760.5760576058 0.00514605898805012\\
85769.2169216922 0.00514632319869034\\
85777.8577857786 0.0051465873960253\\
85786.498649865 0.00514685158005702\\
85795.1395139514 0.0051471157507875\\
85803.7803780378 0.00514737990821875\\
85812.4212421242 0.00514764405235279\\
85821.0621062106 0.00514790818319161\\
85829.702970297 0.00514817230073723\\
85838.3438343834 0.00514843640499166\\
85846.9846984699 0.00514870049595689\\
85855.6255625563 0.00514896457363494\\
85864.2664266427 0.00514922863802781\\
85872.9072907291 0.00514949268913751\\
85881.5481548155 0.00514975672696604\\
85890.1890189019 0.0051500207515154\\
85898.8298829883 0.0051502847627876\\
85907.4707470747 0.00515054876078464\\
85916.1116111611 0.00515081274550852\\
85924.7524752475 0.00515107671696124\\
85933.3933393339 0.00515134067514481\\
85942.0342034203 0.00515160462006123\\
85950.6750675067 0.0051518685517125\\
85959.3159315932 0.00515213247010061\\
85967.9567956796 0.00515239637522757\\
85976.597659766 0.00515266026709538\\
85985.2385238524 0.00515292414570602\\
85993.8793879388 0.00515318801106151\\
86002.5202520252 0.00515345186316384\\
86011.1611161116 0.005153715702015\\
86019.801980198 0.00515397952761699\\
86028.4428442844 0.00515424333997182\\
86037.0837083708 0.00515450713908146\\
86045.7245724572 0.00515477092494792\\
86054.3654365437 0.0051550346975732\\
86063.0063006301 0.00515529845695928\\
86071.6471647165 0.00515556220310816\\
86080.2880288029 0.00515582593602183\\
86088.9288928893 0.00515608965570228\\
86097.5697569757 0.00515635336215151\\
86106.2106210621 0.00515661705537151\\
86114.8514851485 0.00515688073536427\\
86123.4923492349 0.00515714440213177\\
86132.1332133213 0.00515740805567602\\
86140.7740774077 0.00515767169599899\\
86149.4149414941 0.00515793532310267\\
86158.0558055806 0.00515819893698907\\
86166.696669667 0.00515846253766015\\
86175.3375337534 0.00515872612511792\\
86183.9783978398 0.00515898969936435\\
86192.6192619262 0.00515925326040143\\
86201.2601260126 0.00515951680823116\\
86209.900990099 0.0051597803428555\\
86218.5418541854 0.00516004386427646\\
86227.1827182718 0.00516030737249601\\
86235.8235823582 0.00516057086751614\\
86244.4644464446 0.00516083434933883\\
86253.1053105311 0.00516109781796606\\
86261.7461746175 0.00516136127339982\\
86270.3870387039 0.00516162471564208\\
86279.0279027903 0.00516188814469484\\
86287.6687668767 0.00516215156056006\\
86296.3096309631 0.00516241496323974\\
86304.9504950495 0.00516267835273584\\
86313.5913591359 0.00516294172905036\\
86322.2322232223 0.00516320509218526\\
86330.8730873087 0.00516346844214253\\
86339.5139513951 0.00516373177892414\\
86348.1548154816 0.00516399510253208\\
86356.795679568 0.00516425841296832\\
86365.4365436544 0.00516452171023483\\
86374.0774077408 0.00516478499433359\\
86382.7182718272 0.00516504826526658\\
86391.3591359136 0.00516531152303577\\
86400 0.00516557476764314\\
};
\end{axis}
\end{tikzpicture}%

%\caption{Infiltrazione cumulata ($m$) contro tempo ($s$) sulla cima della colonna di suolo}
%\label{fig:easy_infiltrazione_cumulata}
%\end{figure}

\newpage
\begin{figure}[H]
\begin{tikzpicture}

	\coordinate (P1) at (-3cm,0cm);
	\coordinate (P2) at (3cm,0cm);
	\coordinate (P3) at (3cm,6cm);
	\coordinate (P4) at (-3cm,6cm);
	\coordinate (P5) at (0cm,0cm);
	\coordinate (P6) at (0cm,6cm);

	\draw (P1) -- (P2) -- (P3) -- (P4) -- cycle;
	\draw[thick,dashed] (P5) -- (P6);

	\coordinate (P7) at (-4.5cm,0cm);
	\coordinate (P8) at (-4.5cm,6cm);	
	\coordinate (P9) at (4.5cm,6cm);
	\coordinate (P10) at (4.5cm,0cm);

	\filldraw[fill=gray] (P1) ;
	\draw -- (P7);
	[snake=zigzag] (P7) -- (P8);
	\draw  |- (P1);
	\fill[color=gray,opacity=.2] (P1) rectangle (P8) ;


	\coordinate (F11) at (-1.5cm,0cm);
	\coordinate (F21) at (-3cm,3cm);
	\coordinate (F31) at (-1.5cm,6cm);
	\coordinate (F12) at (1.5cm,6cm);
	\coordinate (F22) at (3cm,3cm);
	\coordinate (F32) at (1.5cm,0cm);	

	\coordinate (N1) at (-1.5cm,3cm);
	\coordinate (N2) at (1.5cm,3cm);

	\draw[thick] (N1) -- (N2);

	\foreach \i in {1,2,3}
	{
	  \draw[] (F\i1) circle (0.15em)
	    node[above right] {\tiny F\_{\i,1}};
	}
	\foreach \i in {1,2,3}
	{
	  \draw[] (F\i2) circle (0.15em)
	    node[above right] {\tiny F\_{\i,2}};
	}



	\draw[fill=red] (N1) circle (0.15em)
	    node[above right] {\tiny $\theta_{1},\psi_{1}$};
	\draw[fill=red] (N2) circle (0.15em)
	    node[above right] {\tiny $\theta_{2},\psi_{2}$};
\end{tikzpicture}



 
\end{figure}
 
 
\section{Conservare la massa: le iterazioni di Picard}
\label{appendix:picard} 
 
 
\section{Attacco alla variabilità: le SWRC}
\label{appendix:swrc}


\newpage
\thispagestyle{empty}
\printbibheading
\printbibliography[type=article,heading=subbibliography,title={Articoli}]
\printbibliography[type=book,heading=subbibliography,title={Libri}]
\printbibliography[type=software,heading=subbibliography,title={Software}]

\end{document}






