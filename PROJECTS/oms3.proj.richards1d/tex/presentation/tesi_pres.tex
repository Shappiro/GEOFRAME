\documentclass{beamer}
\usepackage[italian]{babel}
\usetheme[pageofpages=/,% String used between the current page and the
                         % total page count.
          bullet=circle,% Use circles instead of squares for bullets.
          titleline=true,% Show a line below the frame title.
          alternativetitlepage=true,% Use the fancy title page.
          titlepagelogo=logo_uni.png,% Logo for the first page.
         % watermark=./img/logo_uni.png,% Watermark used in every page.
          watermarkheight=80px,% Height of the watermark.
          watermarkheightmult=3,% The watermark image is 3 times bigger
                                % than watermarkheight.
          ]{Torino}

\usepackage{float}
\usepackage{tikz}
\usepackage{pgfplots}
\usepackage{subfig}
\usepackage{graphicx}
\usepackage{caption}
\usepackage{subcaption}
\usepackage[percent]{overpic}
\usepackage{listings}

\usepackage{amssymb}
\usepackage{amsmath}



\usepackage{paralist}
%\usepackage{tcolorbox}

\graphicspath{{./img/}}

\definecolor{coldef}{rgb}{0.0,0.80,0.00}
\definecolor{coldeff}{rgb}{0.1,0.95,0.20}
\setbeamercolor{postit}{fg=black,bg=coldef}
\setbeamercolor{postut}{fg=black,bg=coldeff}

\begin{document}
\lstset{language=Matlab}

\watermarkoff
\author{Aaron Iemma}
\title{{\LARGE Un problema di convergenza: implementazione di un metodo iterativo per la soluzione dell'equazione di Richards}}
\institute{\textsc{UNITN - Universit\`a degli Studi di Trento} \newline \textsl{Dipartimento di Ingegneria Ambientale}}
\date{\today}


\frame{\titlepage} 
%\frame{\frametitle{Indice}\tableofcontents} 


%\section{Introduzione} 




\section{Definizione del problema} 
\frame{
\frametitle{Definizione del problema}
\setbeamercovered{transparent}
Risoluzione dell'equazione di Richards in forma mista: \newline
\pause
\begin{equation}
\frac{\partial \theta(\psi)}{\partial t} = \nabla\cdot\lbrack \textbf{K}(\psi)\nabla(\psi + z) \rbrack + S
\end{equation}
\pause
\dots non lineare!
\begin{columns}
\begin{column}{0.5\textwidth}
\newline
Servono:
\begin{itemize}[<+->]
\pause
\item Metodo di linearizzazione
\item Relazioni costitutive $\psi - \theta$
\item Solutore per matrici definite positive
\item \emph{Framework} di implementazione
\end{itemize} 
\end{column}

\pause
\begin{column}{0.5\textwidth}
\begin{beamerboxesrounded}[upper=postit ,lower=postut ,shadow=true]{}
\begin{inparaenum}[\itshape a\upshape)] \item Metodo di Newton; \item SWRC - \emph{Soil Water Retention Curves}; \item CG - \emph{Conjugate gradient}; \item \texttt{OMS3}\end{inparaenum}
\end{beamerboxesrounded}
\end{column}
\end{columns}
} % --- FRAME END


\subsection{Il primo pezzo: un esempio semplice} 
\frame{
\frametitle{Il primo pezzo: un esempio semplice}
\setbeamercovered{transparent}

Un caso molto semplice: \emph{diffusione} (eh si!) monodimensionale con $\mathbf{K}(\psi)=K=cosst$ senza carico gravitativo 
\pause
\begin{equation}
\label{eq:rich_sim}
\begin{align}
\frac{\partial \theta(\psi)}{\partial t} & = & K \frac{\partial^{2} \theta}{\partial x^{2}} 
\end{align}   
\end{equation}
\pause
Lo schema numerico \`e presto creato:
\begin{columns}
\begin{column}{0.5\textwidth}
\begin{figure}[H]
 \centering 
   \begin{tikzpicture}[point/.style={circle,inner sep=0pt,minimum size=1pt,fill=black},scale=.8,every node/.style={transform shape}]
      \begin{scope}[>=latex]          
        \draw[-] (-5,0) --(3,0)node[midway,right] {};
        \draw[-] (-5,1.5) --(3,1.5)node[midway,right] {};
        \draw[dashed] (0.5,2)--(2,2)node[midway,right] {};
        \draw[dashed] (3.5,1.4)--(3.5,0.1)node[midway,right] {};
      \end{scope}

       % LOWER: j, i-2<=i=<i+2 
       \draw   (2,-0.3) node{$\theta_{i+2,j}$};
       \draw   (-4,-0.3) node{$\theta_{i-2,j}$};       
       \draw   (-4,0) node{$\bullet$};
       \draw   (-2.5,0) node{$\bullet$};
       \draw   (-2.5,-0.3) node{$\textcolor{red}{\theta_{i-1,j}}$};       
       \draw   (-1,0) node{$\bullet$};
       \draw   (-1,-0.3) node{$\textcolor{red}{\theta_{i,j}}$};       
       \draw   (0.5,0) node{$\bullet$};
       \draw   (0.5,-0.3) node{$\textcolor{red}{\theta_{i+1,j}}$};       
       \draw   (2,0) node{$\bullet$};

       % UPPER: j+1, i-2<=i=<i+2
       \draw   (-4,1.2) node{$\theta_{i-2,j+1}$};       
       \draw   (-4,1.5) node{{$\bullet$}};
       \draw   (-2.5,1.2) node{$\theta_{i-1,j+1}$};         
       \draw   (-2.5,1.5) node{{$\bullet$}};
       \draw   (-1,1.2) node{$\textcolor{red}{\theta_{i,j+1}}$};         
       \draw   (-1,1.5) node{$\bullet$};
       \draw   (0.5,1.2) node{$\theta_{i+1,j+1}$};         
       \draw   (0.5,1.5) node{$\bullet$};
       \draw   (2,1.2) node{$\theta_{i+2,j+1}$};         
       \draw   (2,1.5) node{$\bullet$};  
       
       % DX and DT
       \draw   (1.25,2.2) node{$\Delta x$}; 
       \draw   (3.8,0.79) node{$\Delta t$}; 
    \end{tikzpicture}
\end{figure}
\end{column}
\begin{column}{0.5\textwidth}
\begin{equation}
\begin{aligned}
 \theta_{i,j}  = \\ \lambda\theta_{i-1,j-1} + \\  
(1-2\lambda)\theta_{i,j-1} +\\
\lambda\theta_{i+1,j-1} \\
\text{con } \lambda = K\frac{\Delta t}{\Delta x^{2}} 
\end{aligned}
\end{equation}
\parbox{\textwidth}{\raggedright 
}
\end{column}
\end{columns}
} % --- END FRAME


\frame{  
\frametitle{Il primo pezzo: un esempio semplice}
\setbeamercovered{transparent}

\centering
La soluzione, e finora nessuna sorpresa
\begin{figure}[H]
\scalebox{.6}{
% This file was created by matlab2tikz v0.4.4 running on MATLAB 8.2.
% Copyright (c) 2008--2013, Nico Schlömer <nico.schloemer@gmail.com>
% All rights reserved.
% 
% The latest updates can be retrieved from
%   http://www.mathworks.com/matlabcentral/fileexchange/22022-matlab2tikz
% where you can also make suggestions and rate matlab2tikz.
% 
\begin{tikzpicture}
\tikzstyle{every node}=[font=\footnotesize]
\begin{axis}[%
width=4.40888888888889in,
height=3.47733333333333in,
scale only axis,
xmin=0,
xmax=0.1,
xlabel={Distanza dalla cima della colonna (m)},
ymin=0,
ymax=0.45,
ylabel={$\text{Contenuto d'acqua (}\Theta\text{)}$},
axis x line*=bottom,
axis y line*=left
]
\addplot [
color=blue,
solid,
forget plot
]
table[row sep=crcr]{
0 0.3\\
0.00101010101010101 0.1\\
0.00202020202020202 0.1\\
0.00303030303030303 0.1\\
0.00404040404040404 0.1\\
0.00505050505050505 0.1\\
0.00606060606060606 0.1\\
0.00707070707070707 0.1\\
0.00808080808080808 0.1\\
0.00909090909090909 0.1\\
0.0101010101010101 0.1\\
0.0111111111111111 0.1\\
0.0121212121212121 0.1\\
0.0131313131313131 0.1\\
0.0141414141414141 0.1\\
0.0151515151515152 0.1\\
0.0161616161616162 0.1\\
0.0171717171717172 0.1\\
0.0181818181818182 0.1\\
0.0191919191919192 0.1\\
0.0202020202020202 0.1\\
0.0212121212121212 0.1\\
0.0222222222222222 0.1\\
0.0232323232323232 0.1\\
0.0242424242424242 0.1\\
0.0252525252525253 0.1\\
0.0262626262626263 0.1\\
0.0272727272727273 0.1\\
0.0282828282828283 0.1\\
0.0292929292929293 0.1\\
0.0303030303030303 0.1\\
0.0313131313131313 0.1\\
0.0323232323232323 0.1\\
0.0333333333333333 0.1\\
0.0343434343434343 0.1\\
0.0353535353535354 0.1\\
0.0363636363636364 0.1\\
0.0373737373737374 0.1\\
0.0383838383838384 0.1\\
0.0393939393939394 0.1\\
0.0404040404040404 0.1\\
0.0414141414141414 0.1\\
0.0424242424242424 0.1\\
0.0434343434343434 0.1\\
0.0444444444444444 0.1\\
0.0454545454545455 0.1\\
0.0464646464646465 0.1\\
0.0474747474747475 0.1\\
0.0484848484848485 0.1\\
0.0494949494949495 0.1\\
0.0505050505050505 0.1\\
0.0515151515151515 0.1\\
0.0525252525252525 0.1\\
0.0535353535353535 0.1\\
0.0545454545454545 0.1\\
0.0555555555555556 0.1\\
0.0565656565656566 0.1\\
0.0575757575757576 0.1\\
0.0585858585858586 0.1\\
0.0595959595959596 0.1\\
0.0606060606060606 0.1\\
0.0616161616161616 0.1\\
0.0626262626262626 0.1\\
0.0636363636363636 0.1\\
0.0646464646464646 0.1\\
0.0656565656565657 0.1\\
0.0666666666666667 0.1\\
0.0676767676767677 0.1\\
0.0686868686868687 0.1\\
0.0696969696969697 0.1\\
0.0707070707070707 0.1\\
0.0717171717171717 0.1\\
0.0727272727272727 0.1\\
0.0737373737373737 0.1\\
0.0747474747474748 0.1\\
0.0757575757575758 0.1\\
0.0767676767676768 0.1\\
0.0777777777777778 0.1\\
0.0787878787878788 0.1\\
0.0797979797979798 0.1\\
0.0808080808080808 0.1\\
0.0818181818181818 0.1\\
0.0828282828282828 0.1\\
0.0838383838383839 0.1\\
0.0848484848484849 0.1\\
0.0858585858585859 0.1\\
0.0868686868686869 0.1\\
0.0878787878787879 0.1\\
0.0888888888888889 0.1\\
0.0898989898989899 0.1\\
0.0909090909090909 0.1\\
0.0919191919191919 0.1\\
0.0929292929292929 0.1\\
0.0939393939393939 0.1\\
0.094949494949495 0.1\\
0.095959595959596 0.1\\
0.096969696969697 0.1\\
0.097979797979798 0.1\\
0.098989898989899 0.1\\
0.1 0.1\\
};
\node[right, inner sep=0mm, text=black]
at (axis cs:0.025,0.1,0) {T =0 s};
\addplot [
color=blue,
solid,
forget plot
]
table[row sep=crcr]{
0 0.3\\
0.00101010101010101 0.284729050508919\\
0.00202020202020202 0.269597844226155\\
0.00303030303030303 0.254742291874003\\
0.00404040404040404 0.240290814204931\\
0.00505050505050505 0.226361023345884\\
0.00606060606060606 0.213056885360688\\
0.00707070707070707 0.200466476466858\\
0.00808080808080808 0.188660409265484\\
0.00909090909090909 0.177690965953928\\
0.0101010101010101 0.167591935754613\\
0.0111111111111111 0.158379116553197\\
0.0121212121212121 0.150051408473872\\
0.0131313131313131 0.142592401744706\\
0.0141414141414141 0.135972343948156\\
0.0151515151515152 0.130150363095629\\
0.0161616161616162 0.125076822676554\\
0.0171717171717172 0.120695692050317\\
0.0181818181818182 0.116946828924518\\
0.0191919191919192 0.113768088530389\\
0.0202020202020202 0.111097194665696\\
0.0212121212121212 0.108873329262368\\
0.0222222222222222 0.107038417965694\\
0.0232323232323232 0.105538108086718\\
0.0242424242424242 0.104322451263527\\
0.0252525252525253 0.103346315668391\\
0.0262626262626263 0.102569561411997\\
0.0272727272727273 0.101957018021516\\
0.0282828282828283 0.101478304850649\\
0.0292929292929293 0.101107534532983\\
0.0303030303030303 0.100822936725744\\
0.0313131313131313 0.100606435046074\\
0.0323232323232323 0.100443204881957\\
0.0333333333333333 0.100321234197787\\
0.0343434343434343 0.100230903984658\\
0.0353535353535354 0.100164599951858\\
0.0363636363636364 0.100116362634052\\
0.0373737373737374 0.100081579414558\\
0.0383838383838384 0.100056719072587\\
0.0393939393939394 0.100039107321407\\
0.0404040404040404 0.100026740340806\\
0.0414141414141414 0.100018132420617\\
0.0424242424242424 0.100012193410639\\
0.0434343434343434 0.100008131604947\\
0.0444444444444444 0.100005377873239\\
0.0454545454545455 0.100003527199956\\
0.0464646464646465 0.100002294231644\\
0.0474747474747475 0.100001479908819\\
0.0484848484848485 0.100000946730531\\
0.0494949494949495 0.100000600641216\\
0.0505050505050505 0.100000377924577\\
0.0515151515151515 0.100000235830876\\
0.0525252525252525 0.100000145950737\\
0.0535353535353535 0.100000089583267\\
0.0545454545454545 0.100000054534035\\
0.0555555555555556 0.100000032925618\\
0.0565656565656566 0.100000019716574\\
0.0575757575757576 0.100000011710249\\
0.0585858585858586 0.100000006898333\\
0.0595959595959596 0.100000004030619\\
0.0606060606060606 0.100000002335907\\
0.0616161616161616 0.100000001342772\\
0.0626262626262626 0.10000000076563\\
0.0636363636363636 0.100000000433024\\
0.0646464646464646 0.100000000242934\\
0.0656565656565657 0.100000000135194\\
0.0666666666666667 0.100000000074631\\
0.0676767676767677 0.100000000040868\\
0.0686868686868687 0.1000000000222\\
0.0696969696969697 0.100000000011963\\
0.0707070707070707 0.100000000006395\\
0.0717171717171717 0.100000000003391\\
0.0727272727272727 0.100000000001783\\
0.0737373737373737 0.10000000000093\\
0.0747474747474748 0.100000000000481\\
0.0757575757575758 0.100000000000246\\
0.0767676767676768 0.100000000000125\\
0.0777777777777778 0.100000000000063\\
0.0787878787878788 0.100000000000031\\
0.0797979797979798 0.100000000000015\\
0.0808080808080808 0.100000000000007\\
0.0818181818181818 0.100000000000003\\
0.0828282828282828 0.100000000000001\\
0.0838383838383839 0.1\\
0.0848484848484849 0.0999999999999999\\
0.0858585858585859 0.0999999999999999\\
0.0868686868686869 0.0999999999999999\\
0.0878787878787879 0.0999999999999999\\
0.0888888888888889 0.0999999999999999\\
0.0898989898989899 0.0999999999999999\\
0.0909090909090909 0.0999999999999999\\
0.0919191919191919 0.0999999999999999\\
0.0929292929292929 0.0999999999999999\\
0.0939393939393939 0.0999999999999999\\
0.094949494949495 0.0999999999999999\\
0.095959595959596 0.0999999999999999\\
0.096969696969697 0.0999999999999999\\
0.097979797979798 0.0999999999999999\\
0.098989898989899 0.0999999999999999\\
0.1 0.1\\
};
\addplot [
color=blue,
solid,
forget plot
]
table[row sep=crcr]{
0 0.3\\
0.00101010101010101 0.289193574443107\\
0.00202020202020202 0.278436687383833\\
0.00303030303030303 0.267778196382268\\
0.00404040404040404 0.257265612715489\\
0.00505050505050505 0.246944466716572\\
0.00606060606060606 0.236857717911955\\
0.00707070707070707 0.227045222653515\\
0.00808080808080808 0.217543270142902\\
0.00909090909090909 0.208384195637969\\
0.0101010101010101 0.199596077297757\\
0.0111111111111111 0.191202520654186\\
0.0121212121212121 0.183222532188431\\
0.0131313131313131 0.175670481029932\\
0.0141414141414141 0.168556145472294\\
0.0151515151515152 0.161884838889967\\
0.0161616161616162 0.155657607807537\\
0.0171717171717172 0.149871493369844\\
0.0181818181818182 0.144519846320434\\
0.0191919191919192 0.139592684835719\\
0.0202020202020202 0.135077084184137\\
0.0212121212121212 0.130957587169742\\
0.0222222222222222 0.127216624650511\\
0.0232323232323232 0.123834936054535\\
0.0242424242424242 0.120791980704508\\
0.0252525252525253 0.118066331849198\\
0.0262626262626263 0.11563604653325\\
0.0272727272727273 0.113479005756945\\
0.0282828282828283 0.111573220730705\\
0.0292929292929293 0.109897102364888\\
0.0303030303030303 0.108429692409502\\
0.0313131313131313 0.107150855833822\\
0.0323232323232323 0.106041435083805\\
0.0333333333333333 0.105083367754953\\
0.0343434343434343 0.104259769957739\\
0.0353535353535354 0.103554988227262\\
0.0363636363636364 0.102954623240814\\
0.0373737373737374 0.102445528864842\\
0.0383838383838384 0.102015790169289\\
0.0393939393939394 0.101654684039157\\
0.0404040404040404 0.101352625899067\\
0.0414141414141414 0.10110110586669\\
0.0424242424242424 0.10089261738535\\
0.0434343434343434 0.100720581074197\\
0.0444444444444444 0.100579266194313\\
0.0454545454545455 0.100463711776835\\
0.0464646464646465 0.100369649108655\\
0.0474747474747475 0.100293426933657\\
0.0484848484848485 0.100231940411751\\
0.0494949494949495 0.100182564590532\\
0.0505050505050505 0.100143092889388\\
0.0515151515151515 0.1001116808756\\
0.0525252525252525 0.100086795426762\\
0.0535353535353535 0.100067169223004\\
0.0545454545454545 0.100051760393792\\
0.0555555555555556 0.100039717055038\\
0.0565656565656566 0.100030346409331\\
0.0575757575757576 0.100023088041988\\
0.0585858585858586 0.100017491024583\\
0.0595959595959596 0.100013194432157\\
0.0606060606060606 0.10000991088718\\
0.0616161616161616 0.100007412759487\\
0.0626262626262626 0.100005520674163\\
0.0636363636363636 0.100004094006535\\
0.0646464646464646 0.100003023073017\\
0.0656565656565657 0.100002222757091\\
0.0666666666666667 0.100001627339993\\
0.0676767676767677 0.100001186334793\\
0.0686868686868687 0.100000861149891\\
0.0696969696969697 0.100000622433078\\
0.0707070707070707 0.100000447970076\\
0.0717171717171717 0.100000321031645\\
0.0727272727272727 0.100000229081137\\
0.0737373737373737 0.100000162769744\\
0.0747474747474748 0.100000115159874\\
0.0757575757575758 0.100000081128265\\
0.0767676767676768 0.100000056909807\\
0.0777777777777778 0.100000039750852\\
0.0787878787878788 0.100000027647172\\
0.0797979797979798 0.10000001914699\\
0.0808080808080808 0.100000013203726\\
0.0818181818181818 0.100000009066494\\
0.0828282828282828 0.100000006199118\\
0.0838383838383839 0.100000004220547\\
0.0848484848484849 0.100000002861256\\
0.0858585858585859 0.100000001931499\\
0.0868686868686869 0.100000001298321\\
0.0878787878787879 0.100000000868999\\
0.0888888888888889 0.100000000579164\\
0.0898989898989899 0.100000000384345\\
0.0909090909090909 0.100000000253952\\
0.0919191919191919 0.100000000167044\\
0.0929292929292929 0.100000000109345\\
0.0939393939393939 0.100000000071164\\
0.094949494949495 0.100000000045942\\
0.095959595959596 0.100000000029251\\
0.096969696969697 0.100000000018089\\
0.097979797979798 0.1000000000104\\
0.098989898989899 0.100000000004736\\
0.1 0.1\\
};
\addplot [
color=blue,
solid,
forget plot
]
table[row sep=crcr]{
0 0.3\\
0.00101010101010101 0.2911743492219\\
0.00202020202020202 0.28237568772727\\
0.00303030303030303 0.273630757278456\\
0.00404040404040404 0.26496580837769\\
0.00505050505050505 0.256406364011495\\
0.00606060606060606 0.247976994420183\\
0.00707070707070707 0.239701106200364\\
0.00808080808080808 0.231600748746594\\
0.00909090909090909 0.223696440676849\\
0.0101010101010101 0.216007018474928\\
0.0111111111111111 0.208549509132179\\
0.0121212121212121 0.201339028092745\\
0.0131313131313131 0.194388703313158\\
0.0141414141414141 0.187709625750777\\
0.0151515151515152 0.181310826108391\\
0.0161616161616162 0.175199277195685\\
0.0171717171717172 0.169379920832604\\
0.0181818181818182 0.163855717824246\\
0.0191919191919192 0.158627719189484\\
0.0202020202020202 0.153695156532236\\
0.0212121212121212 0.149055549209542\\
0.0222222222222222 0.144704825776986\\
0.0232323232323232 0.140637457080411\\
0.0242424242424242 0.136846598312329\\
0.0252525252525253 0.133324237359644\\
0.0262626262626263 0.130061346832262\\
0.0272727272727273 0.12704803727494\\
0.0282828282828283 0.124273709221115\\
0.0292929292929293 0.121727201940822\\
0.0303030303030303 0.11939693695775\\
0.0313131313131313 0.117271054655578\\
0.0323232323232323 0.115337542553398\\
0.0333333333333333 0.113584354097067\\
0.0343434343434343 0.111999517080792\\
0.0353535353535354 0.110571231074887\\
0.0363636363636364 0.109287953485935\\
0.0373737373737374 0.108138474109652\\
0.0383838383838384 0.107111978250853\\
0.0393939393939394 0.106198098675983\\
0.0404040404040404 0.105386956829652\\
0.0414141414141414 0.104669193886261\\
0.0424242424242424 0.10403599232081\\
0.0434343434343434 0.103479088769532\\
0.0444444444444444 0.102990779012243\\
0.0454545454545455 0.102563915945611\\
0.0464646464646465 0.102191901432023\\
0.0474747474747475 0.101868672904555\\
0.0484848484848485 0.101588685587298\\
0.0494949494949495 0.10134689115464\\
0.0505050505050505 0.101138713605596\\
0.0515151515151515 0.100960023072718\\
0.0525252525252525 0.100807108221863\\
0.0535353535353535 0.100676647831594\\
0.0545454545454545 0.100565682071316\\
0.0555555555555556 0.100471583927277\\
0.0565656565656566 0.100392031156985\\
0.0575757575757576 0.100324979086632\\
0.0585858585858586 0.100268634504008\\
0.0595959595959596 0.100221430841805\\
0.0606060606060606 0.100182004793832\\
0.0616161616161616 0.100149174459808\\
0.0626262626262626 0.100121919073252\\
0.0636363636363636 0.100099360331468\\
0.0646464646464646 0.100080745316746\\
0.0656565656565657 0.100065430973145\\
0.0666666666666667 0.100052870083575\\
0.0676767676767677 0.100042598676561\\
0.0686868686868687 0.100034224780971\\
0.0696969696969697 0.100027418439344\\
0.0707070707070707 0.100021902885985\\
0.0717171717171717 0.100017446794119\\
0.0727272727272727 0.100013857496749\\
0.0737373737373737 0.100010975087935\\
0.0747474747474748 0.100008667314728\\
0.0757575757575758 0.100006825174497\\
0.0767676767676768 0.100005359137687\\
0.0777777777777778 0.100004195921762\\
0.0787878787878788 0.100003275748138\\
0.0797979797979798 0.100002550019957\\
0.0808080808080808 0.10000197936459\\
0.0818181818181818 0.100001531990572\\
0.0828282828282828 0.100001182314216\\
0.0838383838383839 0.100000909816377\\
0.0848484848484849 0.100000698094653\\
0.0858585858585859 0.100000534080769\\
0.0868686868686869 0.100000407396887\\
0.0878787878787879 0.100000309828235\\
0.0888888888888889 0.100000234892644\\
0.0898989898989899 0.100000177490467\\
0.0909090909090909 0.100000133620828\\
0.0919191919191919 0.100000100152351\\
0.0929292929292929 0.100000074638365\\
0.0939393939393939 0.100000055168236\\
0.094949494949495 0.100000040247782\\
0.095959595959596 0.100000028702925\\
0.096969696969697 0.100000019601652\\
0.097979797979798 0.100000012190132\\
0.098989898989899 0.100000005839463\\
0.1 0.1\\
};
\node[right, inner sep=0mm, text=black]
at (axis cs:0.025,0.136846598312329,0) {T =25900 s};
\addplot [
color=blue,
solid,
forget plot
]
table[row sep=crcr]{
0 0.3\\
0.00101010101010101 0.292355791669729\\
0.00202020202020202 0.284729121171288\\
0.00303030303030303 0.277137405657463\\
0.00404040404040404 0.269597822306607\\
0.00505050505050505 0.262127191772357\\
0.00606060606060606 0.25474186569596\\
0.00707070707070707 0.2474576195339\\
0.00808080808080808 0.240289551869253\\
0.00909090909090909 0.233251991273032\\
0.0101010101010101 0.226358411664017\\
0.0111111111111111 0.219621356984325\\
0.0121212121212121 0.21305237586603\\
0.0131313131313131 0.206661966814196\\
0.0141414141414141 0.20045953427664\\
0.0151515151515152 0.194453355813504\\
0.0161616161616162 0.188650560423269\\
0.0171717171717172 0.183057117928833\\
0.0181818181818182 0.177677839180607\\
0.0191919191919192 0.172516386695431\\
0.0202020202020202 0.167575295222915\\
0.0212121212121212 0.162856001616308\\
0.0222222222222222 0.158358883284841\\
0.0232323232323232 0.154083304419856\\
0.0242424242424242 0.150027669118819\\
0.0252525252525253 0.146189480479935\\
0.0262626262626263 0.142565404705772\\
0.0272727272727273 0.139151339236685\\
0.0282828282828283 0.135942483933512\\
0.0292929292929293 0.13293341434304\\
0.0303030303030303 0.130118156108011\\
0.0313131313131313 0.127490259624718\\
0.0323232323232323 0.125042874103824\\
0.0333333333333333 0.122768820252548\\
0.0343434343434343 0.120660660866742\\
0.0353535353535354 0.11871076869816\\
0.0363636363636364 0.116911391043354\\
0.0373737373737374 0.115254710584549\\
0.0383838383838384 0.113732902097729\\
0.0393939393939394 0.112338184727583\\
0.0404040404040404 0.111062869611311\\
0.0414141414141414 0.109899402712394\\
0.0424242424242424 0.108840402800169\\
0.0434343434343434 0.107878694580363\\
0.0444444444444444 0.107007337045018\\
0.0454545454545455 0.106219647166818\\
0.0464646464646465 0.10550921911234\\
0.0474747474747475 0.104869939190993\\
0.0484848484848485 0.104295996791319\\
0.0494949494949495 0.103781891584018\\
0.0505050505050505 0.103322437291713\\
0.0515151515151515 0.102912762339508\\
0.0525252525252525 0.102548307708174\\
0.0535353535353535 0.102224822313905\\
0.0545454545454545 0.101938356235507\\
0.0555555555555556 0.101685252102358\\
0.0565656565656566 0.101462134944983\\
0.0575757575757576 0.101265900795391\\
0.0585858585858586 0.101093704306962\\
0.0595959595959596 0.100942945644327\\
0.0606060606060606 0.100811256872831\\
0.0616161616161616 0.100696488055479\\
0.0626262626262626 0.10059669324306\\
0.0636363636363636 0.100510116521006\\
0.0646464646464646 0.100435178254765\\
0.0656565656565657 0.100370461654363\\
0.0666666666666667 0.100314699758771\\
0.0676767676767677 0.100266762921735\\
0.0686868686868687 0.10022564686319\\
0.0696969696969697 0.100190461334256\\
0.0707070707070707 0.100160419429264\\
0.0717171717171717 0.100134827565237\\
0.0727272727272727 0.100113076137815\\
0.0737373737373737 0.100094630852714\\
0.0747474747474748 0.100079024723354\\
0.0757575757575758 0.100065850718256\\
0.0767676767676768 0.10005475503608\\
0.0777777777777778 0.100045430981621\\
0.0787878787878788 0.100037613412658\\
0.0797979797979798 0.100031073725073\\
0.0808080808080808 0.100025615342062\\
0.0818181818181818 0.100021069672407\\
0.0828282828282828 0.100017292502567\\
0.0838383838383839 0.100014160787677\\
0.0848484848484849 0.100011569807317\\
0.0858585858585859 0.100009430653007\\
0.0868686868686869 0.100007668015816\\
0.0878787878787879 0.10000621824399\\
0.0888888888888889 0.100005027642279\\
0.0898989898989899 0.100004050986365\\
0.0909090909090909 0.100003250227643\\
0.0919191919191919 0.10000259336534\\
0.0929292929292929 0.100002053464721\\
0.0939393939393939 0.100001607801743\\
0.094949494949495 0.100001237116044\\
0.095959595959596 0.100000924955559\\
0.096969696969697 0.100000657097283\\
0.097979797979798 0.100000421029798\\
0.098989898989899 0.100000205484102\\
0.1 0.1\\
};
\addplot [
color=blue,
solid,
forget plot
]
table[row sep=crcr]{
0 0.3\\
0.00101010101010101 0.293162291346467\\
0.00202020202020202 0.286337135083833\\
0.00303030303030303 0.279537014487306\\
0.00404040404040404 0.27277427523486\\
0.00505050505050505 0.266061058185837\\
0.00606060606060606 0.259409234029543\\
0.00707070707070707 0.252830340389786\\
0.00808080808080808 0.246335521940062\\
0.00909090909090909 0.239935474045999\\
0.0101010101010101 0.23364039040737\\
0.0111111111111111 0.227459915122164\\
0.0121212121212121 0.221403099540683\\
0.0131313131313131 0.215478364219279\\
0.0141414141414141 0.209693466222025\\
0.0151515151515152 0.204055471955358\\
0.0161616161616162 0.198570735656394\\
0.0171717171717172 0.193244883591254\\
0.0181818181818182 0.18808280395622\\
0.0191919191919192 0.183088642412771\\
0.0202020202020202 0.178265803128443\\
0.0212121212121212 0.173616955139668\\
0.0222222222222222 0.169144043801111\\
0.0232323232323232 0.164848307039045\\
0.0242424242424242 0.160730296084606\\
0.0252525252525253 0.156789900326607\\
0.0262626262626263 0.153026375893477\\
0.0272727272727273 0.149438377549844\\
0.0282828282828283 0.146023993475449\\
0.0292929292929293 0.142780782482558\\
0.0303030303030303 0.13970581322248\\
0.0313131313131313 0.136795704932264\\
0.0323232323232323 0.134046669278629\\
0.0333333333333333 0.131454552867441\\
0.0343434343434343 0.129014880003099\\
0.0353535353535354 0.126722895302554\\
0.0363636363636364 0.124573605792848\\
0.0373737373737374 0.12256182214841\\
0.0383838383838384 0.120682198754408\\
0.0393939393939394 0.118929272314483\\
0.0404040404040404 0.117297498754797\\
0.0414141414141414 0.115781288210728\\
0.0424242424242424 0.1143750379174\\
0.0434343434343434 0.113073162859843\\
0.0444444444444444 0.1118701240726\\
0.0454545454545455 0.110760454511524\\
0.0464646464646465 0.109738782451949\\
0.0474747474747475 0.108799852397105\\
0.0484848484848485 0.107938543508197\\
0.0494949494949495 0.107149885592863\\
0.0505050505050505 0.106429072711482\\
0.0515151515151515 0.10577147448104\\
0.0525252525252525 0.105172645173723\\
0.0535353535353535 0.104628330722286\\
0.0545454545454545 0.104134473756387\\
0.0555555555555556 0.103687216803639\\
0.0565656565656566 0.103282903796217\\
0.0575757575757576 0.102918080028504\\
0.0585858585858586 0.102589490713778\\
0.0595959595959596 0.10229407828833\\
0.0606060606060606 0.102028978609989\\
0.0616161616161616 0.101791516194972\\
0.0626262626262626 0.101579198632412\\
0.0636363636363636 0.101389710310198\\
0.0646464646464646 0.101220905578953\\
0.0656565656565657 0.101070801473361\\
0.0666666666666667 0.100937570101814\\
0.0676767676767677 0.100819530806626\\
0.0686868686868687 0.100715142188103\\
0.0696969696969697 0.100622994076616\\
0.0707070707070707 0.100541799527736\\
0.0717171717171717 0.100470386906524\\
0.0727272727272727 0.100407692118318\\
0.0737373737373737 0.100352751035004\\
0.0747474747474748 0.100304692157719\\
0.0757575757575758 0.100262729549468\\
0.0767676767676768 0.100226156064059\\
0.0777777777777778 0.100194336891355\\
0.0787878787878788 0.10016670343286\\
0.0797979797979798 0.100142747516366\\
0.0808080808080808 0.100122015953565\\
0.0818181818181818 0.100104105440292\\
0.0828282828282828 0.100088657795391\\
0.0838383838383839 0.100075355530966\\
0.0848484848484849 0.100063917744061\\
0.0858585858585859 0.100054096317543\\
0.0868686868686869 0.100045672416064\\
0.0878787878787879 0.100038453261489\\
0.0888888888888889 0.100032269170955\\
0.0898989898989899 0.100026970839865\\
0.0909090909090909 0.100022426851431\\
0.0919191919191919 0.100018521393968\\
0.0929292929292929 0.100015152166839\\
0.0939393939393939 0.100012228455845\\
0.094949494949495 0.100009669358807\\
0.095959595959596 0.100007402142167\\
0.096969696969697 0.100005360709503\\
0.097979797979798 0.100003484163022\\
0.098989898989899 0.100001715439216\\
0.1 0.1\\
};
\addplot [
color=blue,
solid,
forget plot
]
table[row sep=crcr]{
0 0.3\\
0.00101010101010101 0.293757737550715\\
0.00202020202020202 0.287525025722799\\
0.00303030303030303 0.281311371307766\\
0.00404040404040404 0.275126193772597\\
0.00505050505050505 0.268978782431856\\
0.00606060606060606 0.262878254611051\\
0.00707070707070707 0.256833515115154\\
0.00808080808080808 0.250853217302312\\
0.00909090909090909 0.244945726045832\\
0.0101010101010101 0.239119082847649\\
0.0111111111111111 0.233380973344054\\
0.0121212121212121 0.227738697419693\\
0.0131313131313131 0.222199142119102\\
0.0141414141414141 0.216768757516698\\
0.0151515151515152 0.211453535676496\\
0.0161616161616162 0.206258992802357\\
0.0171717171717172 0.201190154648575\\
0.0181818181818182 0.196251545229511\\
0.0191919191919192 0.191447178836175\\
0.0202020202020202 0.186780555337483\\
0.0212121212121212 0.182254658714726\\
0.0222222222222222 0.177871958749897\\
0.0232323232323232 0.173634415762317\\
0.0242424242424242 0.169543488263575\\
0.0252525252525253 0.165600143378604\\
0.0262626262626263 0.161804869860763\\
0.0272727272727273 0.158157693511386\\
0.0282828282828283 0.15465819479943\\
0.0292929292929293 0.151305528464711\\
0.0303030303030303 0.148098444878865\\
0.0313131313131313 0.14503531293149\\
0.0323232323232323 0.142114144205007\\
0.0333333333333333 0.139332618200499\\
0.0343434343434343 0.136688108378024\\
0.0353535353535354 0.134177708778594\\
0.0363636363636364 0.131798261000931\\
0.0373737373737374 0.129546381314133\\
0.0383838383838384 0.127418487697279\\
0.0393939393939394 0.12541082660857\\
0.0404040404040404 0.12351949929961\\
0.0414141414141414 0.121740487504664\\
0.0424242424242424 0.120069678349908\\
0.0434343434343434 0.118502888343646\\
0.0444444444444444 0.117035886324901\\
0.0454545454545455 0.115664415264523\\
0.0464646464646465 0.114384212829729\\
0.0474747474747475 0.113191030639691\\
0.0484848484848485 0.112080652156068\\
0.0494949494949495 0.111048909168257\\
0.0505050505050505 0.110091696848297\\
0.0515151515151515 0.109204987364763\\
0.0525252525252525 0.108384842058491\\
0.0535353535353535 0.107627422195463\\
0.0545454545454545 0.106928998323615\\
0.0555555555555556 0.106285958270623\\
0.0565656565656566 0.105694813828866\\
0.0575757575757576 0.105152206181704\\
0.0585858585858586 0.10465491013198\\
0.0595959595959596 0.104199837199262\\
0.0606060606060606 0.103784037656772\\
0.0616161616161616 0.103404701582327\\
0.0626262626262626 0.103059158999904\\
0.0636363636363636 0.102744879189777\\
0.0646464646464646 0.102459469245554\\
0.0656565656565657 0.102200671956025\\
0.0666666666666667 0.101966363088499\\
0.0676767676767677 0.101754548148445\\
0.0686868686868687 0.101563358687753\\
0.0696969696969697 0.101391048230944\\
0.0707070707070707 0.101235987885208\\
0.0717171717171717 0.101096661696378\\
0.0727272727272727 0.100971661808891\\
0.0737373737373737 0.100859683483507\\
0.0747474747474748 0.100759520022159\\
0.0757575757575758 0.100670057644843\\
0.0767676767676768 0.100590270358946\\
0.0777777777777778 0.100519214856956\\
0.0787878787878788 0.100456025474116\\
0.0797979797979798 0.100399909233316\\
0.0808080808080808 0.100350141000383\\
0.0818181818181818 0.100306058769022\\
0.0828282828282828 0.10026705909087\\
0.0838383838383839 0.100232592662636\\
0.0848484848484849 0.10020216007896\\
0.0858585858585859 0.100175307756571\\
0.0868686868686869 0.100151624032479\\
0.0878787878787879 0.10013073543633\\
0.0888888888888889 0.100112303134678\\
0.0898989898989899 0.1000960195428\\
0.0909090909090909 0.100081605097722\\
0.0919191919191919 0.100068805184434\\
0.0929292929292929 0.100057387205725\\
0.0939393939393939 0.100047137784756\\
0.094949494949495 0.100037860088315\\
0.095959595959596 0.100029371257697\\
0.096969696969697 0.100021499933327\\
0.097979797979798 0.100014083858514\\
0.098989898989899 0.100006967547191\\
0.1 0.1\\
};
\node[right, inner sep=0mm, text=black]
at (axis cs:0.025,0.169543488263575,0) {T =51800 s};
\addplot [
color=blue,
solid,
forget plot
]
table[row sep=crcr]{
0 0.3\\
0.00101010101010101 0.294220578323082\\
0.00202020202020202 0.288448736598409\\
0.00303030303030303 0.282692024958219\\
0.00404040404040404 0.276957934090231\\
0.00505050505050505 0.271253866002329\\
0.00606060606060606 0.265587105366577\\
0.00707070707070707 0.259964791627398\\
0.00808080808080808 0.254393892051793\\
0.00909090909090909 0.248881175890907\\
0.0101010101010101 0.243433189812234\\
0.0111111111111111 0.238056234750305\\
0.0121212121212121 0.232756344311103\\
0.0131313131313131 0.227539264851644\\
0.0141414141414141 0.222410437341547\\
0.0151515151515152 0.21737498109791\\
0.0161616161616162 0.212437679468787\\
0.0171717171717172 0.207602967524064\\
0.0181818181818182 0.202874921795823\\
0.0191919191919192 0.198257252093488\\
0.0202020202020202 0.193753295402354\\
0.0212121212121212 0.189366011857726\\
0.0222222222222222 0.185097982770901\\
0.0232323232323232 0.180951410667893\\
0.0242424242424242 0.176928121287181\\
0.0252525252525253 0.173029567469001\\
0.0262626262626263 0.169256834855976\\
0.0272727272727273 0.165610649313208\\
0.0282828282828283 0.162091385965511\\
0.0292929292929293 0.158699079740226\\
0.0303030303030303 0.155433437296177\\
0.0313131313131313 0.152293850212778\\
0.0323232323232323 0.149279409308084\\
0.0333333333333333 0.146388919950822\\
0.0343434343434343 0.143620918228922\\
0.0353535353535354 0.140973687836001\\
0.0363636363636364 0.138445277537388\\
0.0373737373737374 0.136033519078717\\
0.0383838383838384 0.133736045402651\\
0.0393939393939394 0.131550309043019\\
0.0404040404040404 0.129473600570269\\
0.0414141414141414 0.127503066967764\\
0.0424242424242424 0.125635729824817\\
0.0434343434343434 0.123868503239465\\
0.0444444444444444 0.122198211331694\\
0.0454545454545455 0.120621605275975\\
0.0464646464646465 0.119135379770567\\
0.0474747474747475 0.117736188869863\\
0.0484848484848485 0.116420661115044\\
0.0494949494949495 0.115185413907384\\
0.0505050505050505 0.114027067077566\\
0.0515151515151515 0.112942255613291\\
0.0525252525252525 0.111927641516153\\
0.0535353535353535 0.11097992476718\\
0.0545454545454545 0.110095853388505\\
0.0555555555555556 0.109272232596309\\
0.0565656565656566 0.10850593304735\\
0.0575757575757576 0.107793898188102\\
0.0585858585858586 0.107133150721689\\
0.0595959595959596 0.106520798213358\\
0.0606060606060606 0.105954037860279\\
0.0616161616161616 0.105430160455841\\
0.0626262626262626 0.104946553582437\\
0.0636363636363636 0.104500704069967\\
0.0646464646464646 0.104090199759922\\
0.0656565656565657 0.103712730616972\\
0.0666666666666667 0.103366089231558\\
0.0676767676767677 0.10304817075795\\
0.0686868686868687 0.102756972332817\\
0.0696969696969697 0.102490592019368\\
0.0707070707070707 0.102247227321823\\
0.0717171717171717 0.102025173314206\\
0.0727272727272727 0.101822820426386\\
0.0737373737373737 0.101638651928896\\
0.0747474747474748 0.101471241156405\\
0.0757575757575758 0.101319248507819\\
0.0767676767676768 0.101181418258926\\
0.0777777777777778 0.101056575221209\\
0.0787878787878788 0.100943621278154\\
0.0797979797979798 0.100841531827852\\
0.0808080808080808 0.100749352158231\\
0.0818181818181818 0.100666193778679\\
0.0828282828282828 0.100591230729272\\
0.0838383838383839 0.100523695886294\\
0.0848484848484849 0.100462877280235\\
0.0858585858585859 0.100408114440028\\
0.0868686868686869 0.100358794774926\\
0.0878787878787879 0.100314350003156\\
0.0888888888888889 0.100274252634338\\
0.0898989898989899 0.1002380125106\\
0.0909090909090909 0.100205173409404\\
0.0919191919191919 0.100175309709336\\
0.0929292929292929 0.10014802311841\\
0.0939393939393939 0.100122939462993\\
0.094949494949495 0.100099705534033\\
0.095959595959596 0.100077985986118\\
0.096969696969697 0.100057460283775\\
0.097979797979798 0.100037819688546\\
0.098989898989899 0.100018764279597\\
0.1 0.1\\
};
\addplot [
color=blue,
solid,
forget plot
]
table[row sep=crcr]{
0 0.3\\
0.00101010101010101 0.294593698974027\\
0.00202020202020202 0.289193602634086\\
0.00303030303030303 0.283805894305954\\
0.00404040404040404 0.278436714717437\\
0.00505050505050505 0.273092141004765\\
0.00606060606060606 0.267778166082672\\
0.00707070707070707 0.262500678494883\\
0.00808080808080808 0.257265442857828\\
0.00909090909090909 0.252078081005746\\
0.0101010101010101 0.246944053939727\\
0.0111111111111111 0.241868644676956\\
0.0121212121212121 0.236856942089338\\
0.0131313131313131 0.231913825812995\\
0.0141414141414141 0.227043952301878\\
0.0151515151515152 0.222251742089959\\
0.0161616161616162 0.217541368317356\\
0.0171717171717172 0.212916746566254\\
0.0181818181818182 0.208381526042814\\
0.0191919191919192 0.203939082131469\\
0.0202020202020202 0.199592510338147\\
0.0212121212121212 0.195344621629164\\
0.0222222222222222 0.191197939162901\\
0.0232323232323232 0.187154696401937\\
0.0242424242424242 0.183216836584195\\
0.0252525252525253 0.179386013522938\\
0.0262626262626263 0.175663593697141\\
0.0272727272727273 0.172050659586038\\
0.0282828282828283 0.168548014194406\\
0.0292929292929293 0.165156186708611\\
0.0303030303030303 0.161875439217524\\
0.0313131313131313 0.158705774427202\\
0.0323232323232323 0.155646944293751\\
0.0333333333333333 0.152698459495043\\
0.0343434343434343 0.149859599658969\\
0.0353535353535354 0.147129424263667\\
0.0363636363636364 0.144506784123676\\
0.0373737373737374 0.141990333375196\\
0.0383838383838384 0.1395785418736\\
0.0393939393939394 0.137269707916963\\
0.0404040404040404 0.135061971210672\\
0.0414141414141414 0.132953325990082\\
0.0424242424242424 0.130941634220652\\
0.0434343434343434 0.129024638797994\\
0.0444444444444444 0.127199976673741\\
0.0454545454545455 0.125465191837031\\
0.0464646464646465 0.123817748085669\\
0.0474747474747475 0.122255041525595\\
0.0484848484848485 0.120774412742117\\
0.0494949494949495 0.119373158591385\\
0.0505050505050505 0.118048543565772\\
0.0515151515151515 0.116797810692053\\
0.0525252525252525 0.115618191926612\\
0.0535353535353535 0.114506918017146\\
0.0545454545454545 0.113461227805597\\
0.0555555555555556 0.112478376952142\\
0.0565656565656566 0.111555646065064\\
0.0575757575757576 0.110690348226102\\
0.0585858585858586 0.109879835905502\\
0.0595959595959596 0.109121507265287\\
0.0606060606060606 0.108412811853387\\
0.0616161616161616 0.107751255695023\\
0.0626262626262626 0.107134405791248\\
0.0636363636363636 0.106559894037724\\
0.0646464646464646 0.106025420579633\\
0.0656565656565657 0.105528756621186\\
0.0666666666666667 0.105067746710333\\
0.0676767676767677 0.104640310521185\\
0.0686868686868687 0.104244444158171\\
0.0696969696969697 0.103878221007184\\
0.0707070707070707 0.103539792159918\\
0.0717171717171717 0.103227386438188\\
0.0727272727272727 0.10293931004542\\
0.0737373737373737 0.102673945872584\\
0.0747474747474748 0.102429752485689\\
0.0757575757575758 0.102205262821618\\
0.0767676767676768 0.101999082618486\\
0.0777777777777778 0.101809888605986\\
0.0787878787878788 0.101636426480248\\
0.0797979797979798 0.101477508686703\\
0.0808080808080808 0.10133201203328\\
0.0818181818181818 0.101198875154968\\
0.0828282828282828 0.101077095849458\\
0.0838383838383839 0.10096572830214\\
0.0848484848484849 0.100863880217312\\
0.0858585858585859 0.100770709870951\\
0.0868686868686869 0.100685423098937\\
0.0878787878787879 0.10060727023314\\
0.0888888888888889 0.100535542996333\\
0.0898989898989899 0.100469571365483\\
0.0909090909090909 0.100408720411608\\
0.0919191919191919 0.100352387123097\\
0.0929292929292929 0.100299997218156\\
0.0939393939393939 0.100251001950909\\
0.094949494949495 0.100204874914619\\
0.095959595959596 0.100161108844549\\
0.096969696969697 0.100119212422134\\
0.097979797979798 0.10007870708139\\
0.098989898989899 0.100039123817875\\
0.1 0.1\\
};
\addplot [
color=blue,
solid,
forget plot
]
table[row sep=crcr]{
0 0.3\\
0.00101010101010101 0.294902782643043\\
0.00202020202020202 0.289810765522128\\
0.00303030303030303 0.284729132959023\\
0.00404040404040404 0.279663037528097\\
0.00505050505050505 0.274617584384953\\
0.00606060606060606 0.269597815836213\\
0.00707070707070707 0.264608696228199\\
0.00808080808080808 0.259655097229924\\
0.00909090909090909 0.254741783583068\\
0.0101010101010101 0.249873399388291\\
0.0111111111111111 0.245054454993476\\
0.0121212121212121 0.240289314545288\\
0.0131313131313131 0.235582184260851\\
0.0141414141414141 0.230937101471338\\
0.0151515151515152 0.226357924484055\\
0.0161616161616162 0.221848323303978\\
0.0171717171717172 0.217411771250007\\
0.0181818181818182 0.213051537495207\\
0.0191919191919192 0.208770680554275\\
0.0202020202020202 0.204572042735338\\
0.0212121212121212 0.200458245567044\\
0.0222222222222222 0.196431686205764\\
0.0232323232323232 0.192494534821742\\
0.0242424242424242 0.188648732957058\\
0.0252525252525253 0.184895992842581\\
0.0262626262626263 0.181237797655545\\
0.0272727272727273 0.177675402694105\\
0.0282828282828283 0.174209837440294\\
0.0292929292929293 0.170841908478096\\
0.0303030303030303 0.167572203229097\\
0.0313131313131313 0.16440109446424\\
0.0323232323232323 0.161328745546673\\
0.0333333333333333 0.158355116357621\\
0.0343434343434343 0.15547996985449\\
0.0353535353535354 0.152702879208208\\
0.0363636363636364 0.150023235465005\\
0.0373737373737374 0.147440255676483\\
0.0383838383838384 0.144952991440916\\
0.0393939393939394 0.14256033779823\\
0.0404040404040404 0.140261042421068\\
0.0414141414141414 0.138053715044652\\
0.0424242424242424 0.13593683707893\\
0.0434343434343434 0.133908771347531\\
0.0444444444444444 0.131967771899536\\
0.0454545454545455 0.130111993841788\\
0.0464646464646465 0.128339503141519\\
0.0474747474747475 0.12664828635139\\
0.0484848484848485 0.125036260211521\\
0.0494949494949495 0.123501281085887\\
0.0505050505050505 0.122041154193301\\
0.0515151515151515 0.120653642596272\\
0.0525252525252525 0.119336475914163\\
0.0535353535353535 0.118087358730298\\
0.0545454545454545 0.116903978665934\\
0.0555555555555556 0.115784014097287\\
0.0565656565656566 0.114725141495096\\
0.0575757575757576 0.113725042369432\\
0.0585858585858586 0.112781409805637\\
0.0595959595959596 0.111891954580387\\
0.0606060606060606 0.111054410849837\\
0.0616161616161616 0.110266541404712\\
0.0626262626262626 0.109526142489899\\
0.0636363636363636 0.108831048188699\\
0.0646464646464646 0.108179134374308\\
0.0656565656565657 0.107568322233317\\
0.0666666666666667 0.106996581368102\\
0.0676767676767677 0.106461932486812\\
0.0686868686868687 0.105962449691377\\
0.0696969696969697 0.105496262375406\\
0.0707070707070707 0.105061556745159\\
0.0717171717171717 0.104656576977866\\
0.0727272727272727 0.104279626032591\\
0.0737373737373737 0.103929066129556\\
0.0747474747474748 0.10360331891442\\
0.0757575757575758 0.10330086532438\\
0.0767676767676768 0.103020245173196\\
0.0777777777777778 0.102760056472355\\
0.0787878787878788 0.102518954505491\\
0.0797979797979798 0.102295650673056\\
0.0808080808080808 0.102088911123885\\
0.0818181818181818 0.101897555189974\\
0.0828282828282828 0.101720453640244\\
0.0838383838383839 0.101556526768553\\
0.0848484848484849 0.101404742330568\\
0.0858585858585859 0.101264113343457\\
0.0868686868686869 0.101133695761628\\
0.0878787878787879 0.101012586041022\\
0.0888888888888889 0.100899918603699\\
0.0898989898989899 0.100794863213711\\
0.0909090909090909 0.100696622274483\\
0.0919191919191919 0.100604428057224\\
0.0929292929292929 0.100517539869144\\
0.0939393939393939 0.100435241169598\\
0.094949494949495 0.100356836641663\\
0.095959595959596 0.100281649226045\\
0.096969696969697 0.100209017123712\\
0.097979797979798 0.100138290773195\\
0.098989898989899 0.100068829808089\\
0.1 0.1\\
};
\node[right, inner sep=0mm, text=black]
at (axis cs:0.025,0.188648732957058,0) {T =77800 s};
\end{axis}
\end{tikzpicture}% 
}
\end{figure}
}

\end{document}
